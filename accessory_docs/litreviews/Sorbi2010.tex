%        File: litreview.tex
%     Created: Tue Jun 12 09:00 AM 2012 B
% Last Change: Tue Jun 12 09:00 AM 2012 B
%
\documentclass[a4paper]{report}
\begin{document}

\section{ Communication in a behaviour-based approach to target detection and tracking with Autonomous Underwater Vehicles \cite{Sorbi2010}}
\subsection{Biblio}
Sorbi, Toni, Dio De Capua, Rossi

TERA, Genova

\subsection{Grade}
1

\subsection{What is this work about?}
Collaborative target tracking behaviours
\subsection{What are the main findings of this work?}
No surprise; two AUVs with a comms link are better at tracking, and that tracking is easier when the targets are stupid/random.
Established tradeoff between tracking capability, and energy via communications link tradeoff
\subsection{What gap in our understanding does this work fill?}
Highlights MLO tracking operations, and provides insight into globally aware control systems
\subsection{What research tradition/approach/method is used?}
Stocastic analysis, utilising Artificial Potential Fields (for Motor Schema control) to factor in multiple behaviours (i.e. target tracking + obstacle avoidance.

Uses simple map maximisation to collaborate between nodes

\subsection{How is this work connected to the wider research field?}

Stojanovic (2008), Akyildiz/Pompolo (2006) regarding comms environment; Zorzi regarding routing considerations; Freitag (2005), and Deng (2008) regarding collabirative path planning for AUVs

\subsection{How is this work relevant to your assignment?}

More than I was expecting, as it presents the compromise between control and comms planes. Will be used as the basis of my simulation generation.

\subsection{What are the limitations of this work?}

Does not at all factor in trust into the collaborative generation of the global map, or have any kind of temporal backoff from the results of the global map. 

Assumes a maximum energy use comms model that is unrealistic and wasteful. 

\subsection{Useful snippit?}

NA, Its all useful
\end{document}
