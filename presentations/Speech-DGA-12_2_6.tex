

% ----------------------------------------------------------------------
%                   LATEX TEMPLATE FOR PhD THESIS
% ----------------------------------------------------------------------

% based on Harish Bhanderi's PhD/MPhil template, then Uni Cambridge
% http://www-h.eng.cam.ac.uk/help/tpl/textprocessing/ThesisStyle/
% corrected and extended in 2007 by Jakob Suckale, then MPI-CBG PhD programme
% and made available through OpenWetWare.org - the free biology wiki

%: Style file for Latex
% Most style definitions are in the external file PhDthesisPSnPDF.
% In this template package, it can be found in ./Latex/Classes/
\documentclass[oneside,11pt,a4paper]{Latex/Classes/PhDthesisPSnPDF}

%: Macro file for Latex
% Macros help you summarise frequently repeated Latex commands.
% Here, they are placed in an external file /Latex/Macros/MacroFile1.tex
% An macro that you may use frequently is the figuremacro (see introduction.tex)
\include{Latex/Macros/MacroFile1}


%: ----------------------------------------------------------------------
%:                  TITLE PAGE: name, degree,..
% ----------------------------------------------------------------------
% below is to generate the title page with crest and author name

%if output to PDF then put the following in PDF header
\ifpdf  
    \pdfinfo { /Title  (An Investigation into Trust and Reputation Frameworks for Autonomous Underwater Vehicles)
               /Creator (TeX)
               /Producer (pdfTeX)
               /Author (Andrew Bolster abolster01@qub.ac.uk)
               /CreationDate (D:YYYYMMDDhhmmss)  %format D:YYYYMMDDhhmmss
               /ModDate (D:YYYYMMDDhhmm)
               /Subject (xyz)
               /Keywords (add, your, keywords, here) }
    \pdfcatalog { /PageMode (/UseOutlines)
                  /OpenAction (fitbh)  }
\fi


\title{An Investigation into Trust and Reputation Frameworks for Autonomous Underwater Vehicles}



% ----------------------------------------------------------------------
% The section below defines www links/email for author and institutions
% They will appear on the title page of the PDF and can be clicked
\ifpdf
  \author{\href{mailto:me@andrewbolster.info}{Andrew Bolster}}
%  \cityofbirth{born in XYZ} % uncomment this if your university requires this
%  % If city of birth is required, also uncomment 2 sections in PhDthesisPSnPDF
%  % Just search for the "city" and you'll find them.
  \collegeordept{\href{http://www.ecit.qub.ac.uk}{Institute of Electronics, Communications and Information Technology (ECIT)}}
  \university{\href{http://www.qub.ac.uk}{Queen's University Belfast (QUB)}}

  % The crest is a graphics file of the logo of your research institution.
  % Place it in ./frontmatter/figures and specify the width
  \crest{\includegraphics[width=4cm]{frontmatter/figures/crest.png}}
  
% If you are not creating a PDF then use the following. The default is PDF.
\else
  \author{Andrew Bolster}
%  \cityofbirth{born in XYZ}
  \collegeordept{Institute of Electronics, Communications and Information Technology (ECIT)}
  \university{Queen's University Belfast (QUB)}
  \crest{\includegraphics[width=4cm]{frontmatter/figures/crest.png}}
\fi
%\renewcommand{\submittedtext}{change the default text here if needed}
\degree{Initial Speech to ECIT/RS on research progress}
\degreedate{2011 November}


% ----------------------------------------------------------------------
       
% turn of those nasty overfull and underfull hboxes
\hbadness=10000
\hfuzz=50pt

%: --------------------------------------------------------------
%:                  FRONT MATTER: dedications, abstract,..
% --------------------------------------------------------------

\begin{document}
\maketitle  % command to print the title page with above variables
\begin{doublespace}
\section{Document Preamble}
This document is intended to be verbally presented, as such its tone is
informal.
Subsection headings and \textbf{emboldened} are not intended to be spoken and
indicate slide transitions.
\section{Introduction}
\subsection{Title Slide} 
Good afternoon ladies and gentlemen, its an honour to be
invited to present our project here, and I hope you find it both intriguing and
enjoyable.

\section{AUVs}
\subsection{AUV Definition}
The general area of research selected is the
commercial, defence, and environmental use of groups of
Autonomous Underwater Vehicles, described as fleets. In this context, these 
groups could be numbered in the maximum of 10 or 12 individuals, acting more as 
a team than a 'swarm'.

\subsection{Predators} After over 20 years of exploratory research into 
Unmanned Vehicles on, above, and below sea level, the recent use of Unmanned 
Aerial Vehicles such as the USAF's Predator range, has proven the effectiveness 
of autonomous or semi-autonomous ordnance delivery and reconnaissance 
platforms, and this success has opened the field of research to a wider range
of applications.

\subsubsection{Predators + AUV}  One area of particular interest to the research
community has been the transfer of ideas and technologies proven in the skies to
the marine environment.

\subsection{AUV Examples}  Bringing these type of technologies into the Naval
space has led to applications such minefield detection and monitoring, submarine tracking,
long range reconnaissance, deep-sea surveying, and much more. 

\subsection{La Spezia} Earlier this year,
successful trials were performed by NATO Underwater Research Centre at La Spezia
in Northern Italy investigating the collective performance of AUV's in port-protection in a
counter-terrorism operational context.

\subsubsection{(Blank)} The major focus of academic research into autonomous
robotic collectives in marine environments has been the development of
individual and pre-defined group behaviour patterns within a fleet and how
those individuals and teams communicate. However, the project introduced here
addresses an important and emerging area; the incorporation of communications
'trust' to improve the operational effectiveness and reliability of such fleets.

\section{Trust and Reputation Management}
\subsection{Trust}  The question of trust is an important one; in this case it 
is a numerical assessment of the expected behaviour of a network node, an AUV 
in this case.

\subsubsection{Fleet Network}  Taking and example where a fleet of AUVs operate 
autonomously over an extended period, independently from any 'mother-ship'.
This fleet may need to occasionally communicate with friendly ships or relay 
bouy's to return data for analysis or simply to notify remote operators of
their status.  This raises the question of ``How do we assess the 
trustworthiness of these unknown entities such as the relay bouys or 'friendly' 
ships.

\subsection{Internal Network}
Additionally, the fleet will have significant internal communications between
individual AUVs, but in the marine environment, the energy cost of 
communication is high. Additionally, nodes may not always be within range of 
each other, so end-to-end communication can be impossible.

\subsubsection{Summary} If individuals can collaboratively predict
the probability that a given node will 'succeed or fail' in a given task, be that relaying a message
or providing useful data, this makes the total network much more robust and
secure. This collaborative behaviour requirement leads to the two basic trust
mechanisms; direct and indirect recommendations. 

\subsection{Example}
Direct trust comes from observation of one node by another, and indirect trust 
comes from a third entity, known to both.

These different trust assessments can also be propagated along
multiple chains of trust. These can be collected to generate trustworthiness
values across the local network.

\subsection{Summary Again}
This raises some important questions;
how does node, say node A, assess the trust of another entity, node B, that it 
hasn't inteacted with before?  What about if it receives a trust recommendation 
for node B from a node, node X, that it doesn't have any experience of? Given 
the multiple physical mediums used for communications, and their generally poor 
quality, how can you tell the difference between actual bad behaviour, and a 
poor communications channel?

\subsubsection{Loner}  Further, due to the extended mission times of such
fleets, and their planned isolation, they necessarily become vulnerable to 
attacks to compromise their mission.  This raises the question of reliably 
maintaining secure audit records, and returns us to the biggest question of 
assuring the identity of the members and associates of a fleet in a distributed 
fashion so that there is no single point of weakness.

\subsubsection{Healthy fleet}  Authentication should be
fairly simple; keep a record of some identifying factor, like an encryption 
key, and don't talk to anyone else. But in a practical sense, number and types 
of nodes in the fleet can change over time; with broken units 'dropping off' 
\textbf{(Dead Battery)-(fleet 4)} and fresh or specialised units being rolled 
out in-theatre \textbf{(fleet 6)} to join a particular fleet.

\subsubsection{Evil fleet}  This leaves them vulnerable to Evil twin style 
attacks.

\subsection{Blank}
It is these challenges that are the basis of our proposed research. To
date there has been little or no vulnerability analysis of fleets of AUVs.

\subsection{MANETS}  The threats experienced in AUV fleets are similar to those 
experiences by mobile ad-hoc networks, usually abbreviated as MANETs. A key 
initial area will be to investigate existing MANET threat models and attempt to 
adapt them for AUV operations. 

\subsubsection{Network Again} Assuming that existing trust structures in MANETs
could be applied to AUV operations, the metrics used to classify trust would 
have to be completly re-evaluated due to the unique communications environment.  
Normally, the types of metrics assessed involve packet loss rates, packet 
delay, and communications throughput, but additional physical layer metrics 
such as Received Signal Strength,and more behavioural metrics such as expected 
distance and deviations from expected motion paths, are being investigated to 
complement these MAC layer metrics.

The significant objects of work can be summarised as: 
\begin{itemize}
	\item\textbf{(DC1)}  1. An assessment of vulnerabilities to a fleet of AUVs
	during typical operations with an aim to develop a suitable threat model
	\item\textbf{(DC2)}  2. Identification of marine-specific metrics 
  suitable to discriminate between legitimate and suspicious AUV behaviour.
	\item\textbf{(DC3)}  3. Development of a distributed trust management framework
	incorporating reputation analysis schemes throughout a fleet of AUVs
	\item\textbf{(DC4)}  4. And finally to explore and assess the feasibility of
	the framework for use across a range of operational tasks with an aim to integrate
  into existing defence frameworks such as the MOD Architecture Framework aka	
  MODAF or the French AGATE
\end{itemize}

\subsection{Pools}  While the majority of the work is expected to be
theoretical development and simulation of threat models, the facilities afforded
by my co-supervisor, Professor Fontaine's robotics group (CRIIF, Centre de
Robotique Integree d'ile de France) at UPMC (Universite Pierre et Marie Curie)
in France, and his access to NURC in La Spezia will provide access to real AUV 
systems for testing.

\subsubsection{Defence Feedback}  In terms of outcomes of this research; this
work would allow AUVs to operate over a much wider
range of operations for much longer periods of time, in a much more secure and
reliable state.

The self-detection and classification of abnormal behaviour within a fleet in
the proposed distributed manner opens up the potential of a new range of secure
and self-learning distributed intrusion detection systems, with potential
applications both below and above the seas. If a general protocol could be
generated for this problem, this could be applied to self-driving cars,
environmental survey drones, satellite communications arrays,
internet Certificate Authority verification, and many more fields. But that
might have to be someone elses project to finish!

\section{Cheers}  Thank you for your time, and I'd be happy to answer any questions
you might have.
\end{doublespace}
\end{document}
