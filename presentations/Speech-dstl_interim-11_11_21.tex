

% ----------------------------------------------------------------------
%                   LATEX TEMPLATE FOR PhD THESIS
% ----------------------------------------------------------------------

% based on Harish Bhanderi's PhD/MPhil template, then Uni Cambridge
% http://www-h.eng.cam.ac.uk/help/tpl/textprocessing/ThesisStyle/
% corrected and extended in 2007 by Jakob Suckale, then MPI-CBG PhD programme
% and made available through OpenWetWare.org - the free biology wiki

%: Style file for Latex
% Most style definitions are in the external file PhDthesisPSnPDF.
% In this template package, it can be found in ./Latex/Classes/
\documentclass[oneside,11pt,a4paper]{Latex/Classes/PhDthesisPSnPDF}

%: Macro file for Latex
% Macros help you summarise frequently repeated Latex commands.
% Here, they are placed in an external file /Latex/Macros/MacroFile1.tex
% An macro that you may use frequently is the figuremacro (see introduction.tex)
\include{Latex/Macros/MacroFile1}


%: ----------------------------------------------------------------------
%:                  TITLE PAGE: name, degree,..
% ----------------------------------------------------------------------
% below is to generate the title page with crest and author name

%if output to PDF then put the following in PDF header
\ifpdf  
    \pdfinfo { /Title  (An Investigation into Trust and Reputation Frameworks for Autonomous Underwater Vehicles)
               /Creator (TeX)
               /Producer (pdfTeX)
               /Author (Andrew Bolster abolster01@qub.ac.uk)
               /CreationDate (D:YYYYMMDDhhmmss)  %format D:YYYYMMDDhhmmss
               /ModDate (D:YYYYMMDDhhmm)
               /Subject (xyz)
               /Keywords (add, your, keywords, here) }
    \pdfcatalog { /PageMode (/UseOutlines)
                  /OpenAction (fitbh)  }
\fi


\title{An Investigation into Trust and Reputation Frameworks for Autonomous Underwater Vehicles}



% ----------------------------------------------------------------------
% The section below defines www links/email for author and institutions
% They will appear on the title page of the PDF and can be clicked
\ifpdf
  \author{\href{mailto:me@andrewbolster.info}{Andrew Bolster}}
%  \cityofbirth{born in XYZ} % uncomment this if your university requires this
%  % If city of birth is required, also uncomment 2 sections in PhDthesisPSnPDF
%  % Just search for the "city" and you'll find them.
  \collegeordept{\href{http://www.ecit.qub.ac.uk}{Institute of Electronics, Communications and Information Technology (ECIT)}}
  \university{\href{http://www.qub.ac.uk}{Queen's University Belfast (QUB)}}

  % The crest is a graphics file of the logo of your research institution.
  % Place it in ./frontmatter/figures and specify the width
  \crest{\includegraphics[width=4cm]{frontmatter/figures/crest.png}}
  
% If you are not creating a PDF then use the following. The default is PDF.
\else
  \author{Andrew Bolster}
%  \cityofbirth{born in XYZ}
  \collegeordept{Institute of Electronics, Communications and Information Technology (ECIT)}
  \university{Queen's University Belfast (QUB)}
  \crest{\includegraphics[width=4cm]{frontmatter/figures/crest.png}}
\fi
%\renewcommand{\submittedtext}{change the default text here if needed}
\degree{Initial Speech to DSTL on research progress}
\degreedate{2011 November}


% ----------------------------------------------------------------------
       
% turn of those nasty overfull and underfull hboxes
\hbadness=10000
\hfuzz=50pt

%: --------------------------------------------------------------
%:                  FRONT MATTER: dedications, abstract,..
% --------------------------------------------------------------

\begin{document}
\maketitle  % command to print the title page with above variables
\begin{doublespace}
\section{Document Preamble}
This document is intended to be verbally presented, as such its tone is
informal.
Subsection headings and \textbf{emboldened} are not intended to be spoken and
indicate slide transitions.
\section{Introduction}
\subsection{Title Slide} 
Good afternoon ladies and gentlemen, its an honour to be
invited to present our project here, and I hope you find it both intriguing and
enjoyable.

\section{AUVs}
\subsection{AUV Definition}
The general area of research selected is the
commercial, defence, and environmental use of interdependent groups of
Autonomous Underwater Vehicles, hereafter described as fleets.

\subsection{Predators} After over 20 years of theoretical and exploratory
research into Unmanned Vehicles on, above, and below sea level, the recent
in-theatre use of Unmanned Aerial Vehicles such as the USAF's Predator range,
has proven the effectiveness of autonomous or semi-autonomous ordnance delivery and reconnaissance platforms,
and this success has opened the field of research to a much wider range of
applications and platforms.

\subsubsection{Predators + AUV}  One area of particular interest to the research
community has been the movement of ideas and technologies proven in the skies to the
marine environment.

\subsubsection{UAV Examples}  Three areas of particular interest to the research
community are swarming behaviours of small reconnaissance drones, analogous to
flocks of birds; human-guided operation, where by a remotely operated drone is 
flanked by a team of automated drones; aswell as the use of task 'checkpointing'
where by automated drones perform a particular task or series of tasks, and 
then notify a remote operator to take over manual control for specific roles.

\subsection{AUV Examples}  Bringing these type of technologies into the Naval
space has led to applications such minefield detection and monitoring, submarine tracking,
long range reconnaissance, deep-sea surveying, and much more. Earlier this year,
successful trials were performed by NURC at La Spezia in Northern Italy
investigating the collective performance of AUV's in port-protection in a
counter-terrorism operational context.

\subsubsection{(Blank)} The major focus of academic research into autonomous
robotic collectives in marine environments has been the development of
individual and pre-defined group behaviour patterns within a fleet. However,
the project introduced here addresses an important and emerging area; the
incorporation of communications 'trust' to improve the operational effectiveness
and reliability of such fleets.

\section{Trust and Reputation Management}
\subsection{Trust}  The question of trust is an important one; it is the quantitative
assessment of the expected behaviour of a network node, in this case an
individual AUV. 

\subsubsection{Summary} 
If individuals can collaboratively predict
the probability that a given node will 'succeed or fail' in a given task, be that relaying a message
or providing useful data, this makes the total network much more robust and
secure. This collaborative behaviour rebuirement leads to the two basic trust
mechanisms; direct and indirect recommendations. 
\subsubsection{Example}
For instance, Nicola, you can
communicate directly with Paul, and Paul can communicate directly with Alan, and
as such, Nicola can speak from direct experience to the trustworthiness of Paul,
and Paul of Alan. If Nicola wants to query Paul as to the trustworthiness of
Alan, \textbf{(Y)} Paul can respond with his trustworthiness of Alan.
\textbf{(X)} Nicola; you then factor in your trust of Paul to find an
\textbf{(Z)} indirect trust association of Alan, even if you've never interacted
with him. Further, implied reputation can be garnered by
association,\textbf{(Assoc)} so for instance I know Nicola you went to
Strathclyde, and I have friends at Strathclyde who tell me its brilliant, I can therefore imply some reputation from that. In the reverse, you may have knowledge  of Queen's, or of
Professors Marshall or Fontaine, and by my association with  them that may imply
some qualatative reputation of me.

All of these different trust assessments, taken from multiple chains of trust
can be used to generate trustworthiness values across many paths. This transfer
of trust knowledge is usually described as multipath propagation.

\subsection{Summary Again}
This raises an important question; how does one assess the trust of a
fleet-foreign entity of which no fleet-member has experience of?

\subsubsection{Fleet Network}  
Taking the example of independent task-oriented operation, whereby fleets of
AUVs operate autonomously over extended periods, independently from any
persistent 'mother-ship', this fleet may need to occasionally communicate with
friendly ships or relay bouy's to return data for analysis or simply to notify
task-commanders to their status. This raises the point just stated of how do we
assess the trustworthiness of these extra-fleet entities such as the relay bouys
or 'friendly' ships.


Additionally, the fleet will have significant internal communications between
individual AUVs. In the case of above-sea-level UAV communication systems, this
would simply be over free-space radio, but the marine environment creates
significant challenges to even localised communications, due to factors such as
low channel bandwidth, scattering, and long propagation delays.

\subsection{Physical}
As such, multiple technologies must be applied to enable underwater
communications, including free-space optical and \textbf{(acoustic)} acoustic
transmission systems, while \textbf{(radio)} radio is reserved for surface or
very short range applications. These conditions make it very difficult to assess the real trustworthiness of an entity when it is unclear if
its abnormal behaviour is due to these effects, or due to actual bad behaviour.

\subsubsection{Loner}  Further, due to the extended mission times of such
fleets, and their planned isolation from classical command and control structures, they
necessarily become vulnerable to attacks to subvert their objective; or to
obtain or otherwise compromise the information they intend to report.

\subsubsection{Healthy fleet}  Additionally, the size of the fleet can be
variable over time; with defective or otherwise incapacitated units 'dropping off' \textbf{(Dead
Battery)-(fleet 4)} and fresh or specialised units being rolled out in-theatre
\textbf{(fleet 6)} to join a particular fleet. This presents two major
vulnerabilities; enemy emulation of a friendly mother-ship, or of a fleet individual.

\subsubsection{Evil fleet}  These factors taken together present two key
challenges for a fleet; the ability to assess trust of a 'friendly' vessel, and to collectively
detect potentially abnormal behaviour of a fleet-member.

\subsection{Blank}
It is these challenges that are the driving force of the proposed research. To
date there has been little or no vulnerability analysis of fleets of AUVs. This
project initially seeks to identify a range of threat models for fleets of
autonomous AUVs; representative of real-mission events, for instance; passive
eave's-dropping of communications; fleet infiltration using cloned AUVs or
surface vessels; or partial fleet destruction to compromise mission
capabilities.

\subsection{MANETS}  These threats have similarities to those experiences by wireless
mobile ad-hoc and peer-to-peer networks, usually summarised as MANETs. Therefore
a key initial area will be to investigate the applicability of existing research
in those areas to the stated problem context. This would include research into
existing modelling techniques as well as any protocols and behaviours used to
counter these threats in the MANET space.

Any discussion of trust within this context leads to the need for a distributed
trust and reputation management framework, using past experience to gauge the
expected behaviour of a node in a distributed network. Most of these systems,
such as Objective Trust Management Framework or OTMF, \textbf{(OTMF Def)} use
only one physical or protocol measurement to make these trustworthiness decisions, but recent work has applied multi-parametric classification
techniques such as Grey Theory \textbf{(Grey Def)}
to not only detect abnormal behaviour but also to classify what type of attack
is being attempted.

\subsubsection{Network Again} 
Due to the previously stated marine effects on transmission,
even assuming that existing trust structures could be deemed applicable to AUV
operations, the metric-sets applied would have to be completely re-approached to
take account of the unique communications environment. Normally, the types of
metrics assessed involve packet loss rates, packet delay, and communications
throughput, but additional physical layer metrics such as Received Signal
Strength,and more behavioural metrics such as expected distance and deviations
from expected motion paths, are being investigated to complement these MAC layer
metrics.

The significant objects of work can be summarised as: 
\begin{itemize}
	\item\textbf{(DC1)}  1. An assessment of vulnerabilities to a fleet of AUVs
	during typical operations with an aim to develop a suitable threat model
	\item\textbf{(DC2)}  2. Identification of marine-specific metrics 
	suitable to discriminate between legitimate and suspicious AUV behaviour.
	\item\textbf{(DC3)}  3. Development of a distributed trust management framework
	incorporating reputation analysis schemes throughout a fleet of AUVs
	\item\textbf{(DC4)}  4. And finally to explore and assess the feasibility of
	the framework for use across a range of operational tasks with an aim to integrate
	into existing frameworks such as MODAF/AGATE.
\end{itemize}

\subsection{Pools}  While the majority of the work is expected to be
theoretical development and simulation of threat models, the facilities afforded by
Professor Fontaine's CIIRF group at UPMC in France will provide access to real
AUV systems. This experience is essential to determining real-world performance
of different metric-sets, and to evaluate the practicality of such schemes.

\section{Project Structure}  
\subsection{Team}
Both supervisors have extensive experience in defence
research, including research within the areas of Networking and AUV operations;
Professor Marshall has been involved in the assessment, development and implementation of reliable and secure wireless networks for over 12 years,
and Professor Fontaine is an established expert in the field of AUVs, including
involvement with a number of NATO initiatives in this area, and has ties to
research organisations such as the NATO Undersea Research Centre that will
provide essential practical resources to this project.

Considering the potential exploitation of the project, both supervisors have
experience in defence sponsored research and are aware of the relevant issues.
Additionally both have extensive track records in international publication.
Further, their associated institutions are already very well acquainted with
commercialising research. For example, Prof Marshall has founded a successful
wireless network security start-up which was spun-out from Queen's University.
Queen's was also named the UK Entrepreneur University of the year for 2009.
Professor Fontaine's involvement with NATO will also lead to further development
of the resultant techniques and findings from this research, potentially leading
to project-input to the forthcoming FP8 EU research call.

\subsection{Project Monitoring}  
Beyond internal team monitoring, the monitoring structures in
place within Queen's endeavour to keep PhD projects on track. Queen's institutes
two major review points within the first year of the project, called the 'Three Month
Report', and the Nine Month Differentiation.

\subsubsection{TMR}
Surprise Surprise, the Three Month Report is a report, delivered after three
months of work, summarising the researchers progress, and including;
\begin{itemize}
  \item Research Background and justifications
  \item Planned Research Objectives or key research questions (including
  justification on why these are important)
  \item A planned methodology including any ethical concerns or data-retention
  issues, as well as a general timetable
\end{itemize}


I'm currently in the process of drafting this report.

\subsubsection{Differentiation}
Six months later, there will a Differentiation report, mainly consisting of a
significant review of key field literature, and a descriptive analysis of the
intended methodology. Supplicant to this is a panel review. The panel then makes
the recommendation whether my research is good enough to be classed at PhD
level. Additionally, there is the option to 'retake' the differentiation. If
necessary.

Beyond the initial year, there are annual reviews of ongoing research projects,
and beyond Queen's, the expectation is that regular reviews will be undertaken
with DSTL/DGA, although I hope that like this presentation; these reviews
synchronise with Institution-mandated reviews so as to reduce my paperwork!

\subsection{Defence Context}  Moving on, the interest to defence for this
project is clear; within DSTL's already stated areas of strategic importance, this research falls into
at least four areas; namely Certification of Autonomous Systems; Shallow Oceanic
Zone and Coastal Environment; Security and Vulnerability; as well as Human
Operator Control of Unmanned Vehicles.

\subsubsection{Defence Feedback}  This work would provide the ability for a
fleet of AUVs to operate over a much wider range of operations for much longer
periods of time, in a much more secure and reliable state.

The self-detection and classification of abnormal behaviour within a fleet in
the proposed distributed manner opens up the potential of a new range of secure
and self-learning distributed intrusion detection systems, with potential
applications both below and above the seas.

\section{Cheers}  Thank you for your time, and I'd be happy to answer any questions
you might have.
\end{doublespace}
\end{document}