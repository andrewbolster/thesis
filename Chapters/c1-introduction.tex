% !TeX spellcheck = en_GB
\def\ChapterTitle{Introduction} % Write in your own chapter title

\chapter{\ChapterTitle}
\label{Chapter\thechapter}
\lhead{Chapter \thechapter.
\emph{\nameref{Chapter\thechapter}}} % Write in your own chapter title to set the page header

\todo{Lead into problem etc, mention trust, marine autonomy and applications}

\section{\glspl{manet}}\label{manets}

With the explosive growth in the use of mobile telephony and the increasing miniaturisation and efficiency gains of portable communications devices, the classical paradigm of a broadcast/receiver or server/client has given way to an increasing use of decentralised, ad-hoc networks that not only accommodate but take advantage of network mobility.

Whether these networks are decentralised cellular / RF / 802.11 WiFi networks for use in disaster relief areas \cite{Milliken2015} or biologically inspired wireless sensor networks for low-energy, low-maintenance environmental monitoring \cite{Nicholson2008}\cite{Selvakennedy2007}, \gls{manet} theory developed over the past 30 years has gone from it's first formal definition, emerging from DARPA's Packet Radio Network research \cite{Jubin1987}, to being an integral part of modern practical communications.

Minimally, a \gls{manet} consists of of a collection of mobile physical entities (nodes) that communicate cooperatively to collect, distribute, disseminate, and collate data and/or influence across an area.
In most cases \gls{manet} nodes incorporate bi-directional transceivers to send and receive data, however this bi-directionality is not a requirement (for example in the area of Wireless Sensor Networks \cite{Akyildiz2002}).
\glspl{manet} may utilise omnidirectional, static, or steerable communications antennae, and a selection of protocols such as WiFi, Bluetooth, GSM, UMTS, Optical or Acoustics, and may incorporate a range of mobilities across nodes, from static devices, terrestrial and marine surface platforms, and aerial and underwater platforms
A core characteristic of \glspl{manet} is the inclusion and integration of heterogeneous node collections, i.e where different nodes or groups of nodes in a network may have different capabilities in terms of propulsion, sensor apparatus, communications capability, etc.

\glspl{manet} may be totally independent with no external connections; include independent per-node communications backhauls (e.g. Cellular Modems in mobile phones as part of a Bluetooth Personal Area Network), or include static nodes that provide infrastructure based backhaul.
However, this multiplicity of variations and options presents several challenges to users and operators; the physical topology of \glspl{manet} can vary wildly over short periods of time.
A particular challenge to \gls{manet} operation is that given any node may operate as a routing / gateway node, if/when that node moves to a different region, network segments that had previously used that node as a path must renegotiate / re-establish their routes.
These situations, if not appropriately managed, lead to opportunities for subversion and selfishness.

The characteristics of \gls{manet}s as defined by Corson et al.\ are paraphrased in Table~\ref{tab:manet_characteristics}.

\begin{table}[h!]
  \hyphenpenalty=10000
\caption[Summary of Characteristics of \glspl{manet}]{Summary of Characteristics of \glspl{manet}\cite{Corson1999}}
\label{tab:manet_characteristics}
  \begin{tabularx}{\textwidth}{p{2cm}X}\toprule
    Dynamic Topologies & Nodes are free to move arbitrarily; thus, the typically multi-hop network topology may change randomly and rapidly at unpredictable times, and may consist of both bidirectional and unidirectional links.
\\
    Bandwidth Constrained, Varied Capacity & Wireless links will continue to have significantly lower capacity than their hardwired counterparts.
In addition, the realized throughput of wireless communications, after accounting for the effects of multiple access, fading, noise, and interference conditions, etc., is often much less than a radio's maximum transmission rate.
\par
One effect of the relatively low to moderate link capacities is that congestion is typically the norm rather than the exception, i.e.\  aggregate application demand will likely approach or exceed network capacity frequently.\\
    Energy Constrained Operation &  Some or all of the nodes in a \gls{manet} may rely on batteries or other exhaustible means for their energy.
For these nodes, the most important system design criteria for optimization may be energy conservation.\\
    Limited physical security & Mobile wireless networks are generally more prone to physical security threats than are fixed cable nets.
The increased possibility of eavesdropping, spoofing, and denial-of-service attacks should be carefully considered.\par
Existing link security techniques are often applied within wireless networks to reduce security threats.
As a benefit, the decentralized nature of network control in \glspl{manet} provides additional robustness against the single points of failure of more centralized approaches.\\\bottomrule
\end{tabularx}
\end{table}


\section{Node Density in \glspl{manet}}

One fundamental compromise in the operation of wireless \glspl{manet} is the trade-off between the number of hops required between source and destination nodes and the effective bandwidth available to the network overall\cite{Royer2001}.
This compromise is encapsulated in the relative density of a given network; that is, the number of nodes in a given node's one-hop locality, drawing direct links between wireless transmission strength / reception sensitivity, the environmental noise floor, environmental channel characteristics, the mobility of the nodes and the number of nodes deployed in a region.
\todo{Expand Node Density discussion to include examples of sparse, dense, long/deep, fully connected networks. This is graphics heavy and will take time}



\section{\glspl{manet} in Harsh Environments}

As \acrfullpl{manet} grow beyond the terrestrial arena, their operation and the protocols designed around them must be reviewed to assess their suitability to different communications environments, ensuring their continued security, reliability, and performance.

The distributed and dynamic nature of \glspl{manet} mean that it is difficult to maintain an evidence based trust system such as \gls{ttp}, Certificate Authorities or using \gls{pki}. 
In both cases, there is the assumption of a run-time canonical source of trust, i.e. a ``Master'' node or Certifying Authority that can objectively coordinate the security and trust of the network.
This single-point-of-failure is antithetical to \gls{manet} architectures, and given the normally limited transmission, storage, battery and computational power of \gls{manet} nodes, the overheads of true \gls{ttp} or \gls{pki} architectures have been out of the realms of practicality for most applications.
Therefore, a distributed, collaborative system must be applied to these networks.
\footnote{ \citet{Zouridaki} has demonstrated an intriguing low-power Distributed CA based \gls{manet} architecture, however given the soon-to-be-discussed assumptions about capable attackers(~\autoref{sec:capable_attackers}), this ``decentralised'' approach is less than ideal}
Such distributed trust management frameworks aim to detect, identify, and mitigate the impacts of malicious actors by distributing per-node assessments and opinions to collectively self-police behaviour.
As such, \glspl{tmf} can be used to predict and reason on the future interactions between entities in a system.

\glspl{tmf} provide information to assist the estimation of future states and actions of nodes within \glspl{manet}.
This information is used to optimize the performance of a network against malicious, selfish, or defective misbehaviour by one or more nodes.
Previous research has established the advantages of implementing \glspl{tmf} in 802.11 based \glspl{manet}, particularly in terms of preventing selfish operation in collaborative systems \cite{Li2007}, and maintaining throughput in the presence of malicious actors \cite{Buchegger2002}

\todo{Section on autonomy}

\section{Systems Approach to Trust and Trust Engineering}

\subsection{Trust vs Security vs Integrity}

Early attempts to secure and protect the integrity of \glspl{manet} have relied on various forms of strong-cryptography to protect information being transferred from tampering or malicious inspection.
While such approaches protect the integrity of individual pieces of data, the increased computation, and storage requirements of modern, strong, decentralised cryptographic systems presents a clear avenue for \gls{dos} attacks on \glspl{manet}.
This threat is particularly relevant in resource-constrained networks, where one or more aspects of the environment are limited, be it available power, mobility, data storage, onboard processing, bandwidth, and channel resources such as capacity and delay.
In such networks, where there is a requirement of security and/or integrity monitoring, strong-cryptographic methods present an entirely new opportunity to potential attackers.

One solution to the trade-off between \gls{dos}-protection, and security is the assessment of ``trustworthiness'' of nodes within a local network. 
``Trust'' in this case is an assessment of capability of a node based on previously observed behaviour.
Using this Trust to make simple routing decisions is significantly simpler and faster that strong-cryptographic methods, particularly in multi-hop networks or resource constrained networks\cite{Cordasco2008}.
With Trust being reliant on the near-real-time awareness of some behaviour, and cryptography on the pre-establishment of some entropy store and the repeated reinforcement of that numerical security, they represent two very different approaches to system integrity with very different costs/benefits and in practice, some elements of both methodologies will be used in different contexts and applications.

\subsection{Systemic Trust and Trusted Development}
As will be discussed further in \autoref{sec:trust_perspectives}, the ``Trust'' in the operation of a system is important well before a system is activated; the incubation, specification, design, development and testing of a system (particularly a system with some \gls{loa} or other non-deterministic operation) is critical to the trust that an end user can put into that system, and particularly, how much ``Trust'' can be exhibited within and between that systems individual components.

\section{Trust Operation Against Capable Attackers}\label{sec:capable_attackers}

\todo{Trust operation against capable attackers, something something russians}

\section{Contributions}

\todo{Contributions}

\section{Conclusion}

\todo{Introduction Conclusions}

\subsection{Thesis Layout}

\subsubsection{\autoref{Chapter2}}
In this chapter the current literature and research on the concepts, theory, and applications concerning Trust and Trust Management are explored, specifically leaning towards the applications of Trust within Autonomous \glspl{manet}.

In~\autoref{sec:trust_defs}, the abstract quantity of ``trust'' is explored,
In~\autoref{sec:trust_autonomy}, Autonomy and ``Trusted Operation'' of autonomous systems is investigated from a system architects and a system operators perspective.
In~\autoref{sec:trust_manets}, current use and applications of Trusted operation of \glspl{manet} is explored, including current \glspl{tmf}.

\subsubsection{\autoref{Chapter3}}
In \autoref{Chapter3}, the maritime context is investigated, particularly concerned with the mechanisms of maritime acoustic communications, including the opportunities and challenges of the marine acoustic channel and its modelling(\autoref{sec:marine_comms}).
Additionally, in \autoref{sec:marine_ops}, the application scope of \glspl{auv}, liteoral and sub-marine operations are explored to provide context to the problem.

\subsubsection{\autoref{Chapter4}}
In this chapter, the need for multi-metric trust assessment in \gls{uan} is demonstrated as an example of a harsh network environment.

The operation of a selection of traditional \gls{manet} \glspl{tmf} in this environment is investigated.
These challenges are characterised and results are presented that demonstrate a multi-metric approach to Trust greatly enhances the effectiveness of \glspl{tmf} in these environments.

In~\autoref{sec:initialsystemcharacterization} an experimental configuration for the marine space is established, and the scenarios and results presented in \cite{Guo11} are reviewed for comparison.
In~\autoref{sec:trustresultsanddiscussion} findings in trust establishment and malicious behaviour detection are presented and comparing with other current \glspl{tmf} (Hermes and \gls{otmf}) and the use of this multi-parameter approach to detecting malicious and selfish behaviour in autonomous marine networks is analysed.

The contributions of this chapter are the first study on the comparative operation and performance of \glspl{tmf} in marine acoustic networks, and a review of metric suitability for \glspl{tmf} in marine environments, informing future metric selection for experimenters and theorists.
Finally, a methodology to assess the usefulness of metrics in discriminating against misbehaviours in such constrained, delay-tolerant networks is demonstrated.
\todo{''Identify the need to generate trust in a new domain (physical)''}

Key parts of this chapter were presented at TrustCom-BigDataSE-ISPA 2015 as ``Single and Multi-Metric Trust Management Frameworks for use in Underwater Autonomous Networks.''\cite{Bolster2015}

\todo{Rewrite Ch4 Intro after rewriting Ch4/6}

\subsubsection{\autoref{Chapter5}}
This chapter proposes a new approach to trust in resource-constrained networks of autonomous systems based on their physical behaviour, using the motion of nodes within a team to detect and potentially identify malicious or failing operation within a cohort.
This is accomplished by looking specifically at operations within the three dimensions of the underwater space, based on kinematics of industry standard \glspl{auv}.
A series of composite metrics based on physical movement are presented and applied to the detection and discrimination of sample physical misbehaviours.
This approach opens the possibility of bringing information about both the physical and communications behaviours of autonomous \glspl{manet} together to strengthen and expand the application of Trust Management Frameworks in sparse and/or resource constrained environments.\citet{Bolster2016}

\todo{Need to include discussion of single-metric/vector/multi-domain}

\subsubsection{\autoref{Chapter6}}
In this chapter we establish an experimental configuration for the marine space, and review the scenarios and results presented in \citet{Guo11} in this context.
We present our findings in trust establishment and malicious behaviour detection, comparing with other current TMFs (Hermes and OTMF) and analyse the use of this multi-parameter approach to detecting malicious and selfish behaviour in autonomous marine networks.
\todo{Expand and elaborate on results}
\citet{Bolster2015}

\subsection{\autoref{Chapter7}}

In this chapter, a multi-domain trust management framework (MD-TMF) is demonstrated in collaborative marine \glspl{manet}
A methodology is demonstrated that applies Grey Sequence operations and Grey Generators to provide continuous trust assessment in a sparse, asynchronous metric space across multiple domains of trust.
By utilising information from multiple domains, it is demonstrated that trust assessment can be more accurate and consistent in identifying misbehaviour than in single-domain assessment.
Further, a methodology for assessing the usefulness of individual metrics in this cross-domain space is demonstrates, allowing for the elimination of redundant metrics, simplifying the runtime assessment process.\citet{Bolster2016a}\todo{Discussion}

\subsubsection{\autoref{Chapter8}}





%%%%%%%%%%%%%%%%%%%%%%%%%%%%%%%%%%%%%%%%%%%%%%%%%%%%%%%%%%%%%%%%%%%%%%%%%%%%%%%
