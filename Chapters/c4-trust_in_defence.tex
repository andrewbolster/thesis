

\chapter{Trust in Autonomous Systems of Systems for Maritime Defence Applications} 
\label{Chapter\thechapter}
\lhead{Chapter \thechapter. \emph{\nameref{Chapter\thechapter}}} 

\subsection{Trust in Marine Networks}\label{sec:trust_in_marine}

With demand for smaller, more decentralised marine survey and monitoring systems, and a drive towards lower per-unit cost, TMFs are going to be increasingly applied to the marine space, as the benefits they present are significant.
Beyond the constraints of the communications environment, knock on pressures are applying in battery capacity, on-board processing, and locomotion.
These pressures simultaneously present opportunities and incentives for malicious or selfish actors to appear to cooperate while not reciprocating, in order to conserve power for instance.
These multiple aspects of potential incentives, trust, and fairness do not directly fall under the scope of single metric trusts discussed above, and this context indicates that a multi-metric approach may be more appropriate.


