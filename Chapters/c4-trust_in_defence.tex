% !TeX spellcheck = en_GB
\def\ChapterTitle{Trust in Autonomous Systems of Systems for Maritime Defence Applications}

\ifx\ifthesis\undefined
\documentclass[a4paper, 11pt]{article}

% % Special Includes stolen from Thesis.cls
\usepackage{booktabs}
\usepackage{hyperref}
\usepackage{graphicx}
\usepackage{epstopdf}
\usepackage{subcaption}
\usepackage{rotating}
\usepackage{listings}
\usepackage{lstpatch}
\renewcommand{\arraystretch}{1.3}

% % % % % % % % % % % % % % % %

% PACKAGES
\usepackage[square, numbers, comma, sort&compress]{natbib}  % Use the "Natbib" style for the references in the Bibliography
\usepackage{verbatim,listings}  % Needed for the "comment" environment to make LaTeX comments
\usepackage{array}  % Needed for the "comment" environment to make LaTeX comments
\usepackage{vector}  % Allows "\bvec{}" and "\buvec{}" for "blackboard" style bold vectors in maths
\usepackage{amsmath,amsfonts,amsthm,color,psfrag,epsf, tabularx, multirow, longtable}
\usepackage{snapshot, todonotes}
% \usepackage{pstricks}
\usepackage{enumerate}
%\usepackage[lined,algonl,boxed]{algorithm2e}
\usepackage[ruled,linesnumbered,vlined]{algorithm2e}
\usepackage{float}
\usepackage{epigraph} % epigraph
\usepackage{tikz}
\usetikzlibrary{shapes.geometric, arrows}

\usepackage{setspace}
\doublespacing
% or:
%\onehalfspacing

% SETUP
\hypersetup{urlcolor=blue, colorlinks=false}  % Colours hyperlinks in blue, but this can be distracting if there are many links. 
\DeclareGraphicsExtensions{.pdf,.jpeg,.png}
\graphicspath{{../Figures/}{../../../Dropbox/Thesis_Figures/}}  % Location of the graphics files (set up for graphics to be in PDF format)

% NOTE THERES A FUCKUP IN TEX4HT http://tug.org/pipermail/tex4ht/2014q2/000944.html
% NEED TO MANUALLY CHANGE \def\pgfsys@svg@newline{{?nl}}
\ifdefined\HCode
\usepackage[compatibility=false]{caption}
\def\pgfsysdriver{pgfsys-tex4ht.def}
\else
\usepackage[]{caption}
\fi

% \SpecialCoor
\def\subsum{\mathit{\Sigma}}

%opening
\title{\ChapterTitle}
\author{Andrew Bolster}

\begin{document}

\maketitle

\else
\chapter{\ChapterTitle}
\label{Chapter\thechapter}
\lhead{Chapter \thechapter. \emph{\nameref{Chapter\thechapter}}} % Write in your own chapter title to set the page header
\fi

The aim of the chapter is to explore where trust is likely to impact on an indicative system (of systems) that contains autonomous elements.
To assist with scoping this, an indicative scenario is selected from the Maritime domain.  This scenario centres on autonomous Mine Counter  Measures and/or Hydrography, Capability (MCM/MHC) operations, incorporating Human Factors, Command and Control (C2) concerns, and Vehicle to vehicle (V2V) distributed communication, from the perspective of trusted and semi-trusted operation. 

With demand for smaller, more decentralised marine survey and monitoring systems, and a drive towards lower per-unit cost, TMFs are going to be increasingly applied to the marine space, as the benefits they present are significant.
Beyond the constraints of the communications environment, knock on pressures are applying in battery capacity, on-board processing, and locomotion.
These pressures simultaneously present opportunities and incentives for malicious or selfish actors to appear to cooperate while not reciprocating, in order to conserve power for instance.
These multiple aspects of potential incentives, trust, and fairness do not directly fall under the scope of single metric trusts discussed above, and this context indicates that a multi-metric approach may be more appropriate.

However, the implications of trust in autonomy beyond securing communications and data are an area in need of further research (BAE Systems, 2013. Maritime Autonomy Final Report - Combined Response,)
Of particular concern is the verification of autonomous behaviours.
Technology Readiness Level deficiencies were identified in the Maritime Capability Contribution of Unmanned Systems (MCCUS) Osprey Phase 1 report(Clark, H. et al., 2012. Maritime Capability Contribution of Unmanned Systems,), with a particular focus on failsafe behaviour. 
The addition of increased on-board autonomy in MUxS, properly understood and verified, would greatly improve this future capability, similar to recent developments in the UAS arena\cite{Cummings2010}.
Under the Osprey concept of operations, there is an opportunity for increased decentralisation and in-field collaboration(Walton, R., 2012. Maritime Autonomy PDR Pack.), however, difficulties in “Trust” between human operators and autonomous systems have already been clearly identified\cite{Chen2011b},and this has been demonstrated by the recent decision by the German government to renege on its €500M investment in the Euro Hawk programme, due to concerns about civil certification of the onboard autonomy\cite{Mehta2013}
In order for these new distributed structures to be relied upon to provide operational performance, reliability and to maintain in-field situational awareness, vulnerabilities to disruption, interruption, and subversion need to be understood and minimised.

In order to contextualise the discussions on trust in mixed and hybrid networks, an exemplar scenario is considered.
That scenario builds on existing Maritime Autonomy Framework (MAF) investigations(Mollet, J. et al., 2012. Osprey Task 37 Activity 8 - Unmanned Systems Operations: Technical Assurance Work Package - Security Issues and Mitigations - Final Report,)

While the initial assessment does not cover the MHPC PT CONUSE recommendations, it provides a starting point for future trust research in UxV operations.
In order to constrain the scope of this project, a single operational scenario will be analysed within documented MCHP CONUSE(Rudge, A., Chapman, K. \& Goddard, N., 2012. Information Management for MHPC: Research Strategy,), of Route/Area Survey within both peacetime and wartime contexts, with a Beyond Line of Sight (BLOS) operator.
This scenario will be a minimal MCM operation in a littoral area.
In field assets will consist of:
\begin{itemize}
  \item Two squads consisting of Three UUV’s, (tacitly modelled on the in-service REMUS 100 UUV), and a USV providing acoustic-RF relay capabilities per-squad
  \item	an UAV providing BLOS Comms
  \item	A remote human operator (MCMV / PJHQ / etc)
\end{itemize}

\missingfigure{Indicitive Future MCM Scenario}

The differential between the peacetime and wartime contexts will be an attempted capture of a UUV by a manned surface-based FIS asset.
Clearly, this paper has a limited scope and does not attempt to cover every aspect of a trustworthy system.

\section{Trust Perspectives}

In Human trust relationships it can be seen that there can be several perspectives of Trust for example organizational, sociological, interpersonal, psychological and neurological\cite{Lee2004}.
For the purposes of this work we can define two perspectives: Design and Operational.
These are summarised as follows:
\begin{itemize}
  \item Design Trust.
    When an autonomous system is under development a level of Trust is established in it through the manner in which it has been designed and tested.
    This is the same as conventional systems.
    The difference with systems that have high-levels of autonomy is that they are designed to behave adaptively to dynamic environments that are difficult to fully predict prior to operational deployment.
    For example, in a navigation system it is difficult to predict the dynamic environment it will need to adapt to.
    So Trust needs to be developed that the design and test of such systems are sufficient to predict that operational solutions will be, if not optimal, at least satisfactory.
  \item Operational Trust.
    Effectively, there are two aspects to this: trust that a system is operating as expected (which is inevitably tied in with, but distinct from) Design Trust; and trust that the interfaces between the operator and the system are as expected.
    This latter aspect covers issues such as physical links and interpretation of data.
\end{itemize}

Examples of roles that interact with a system from both of these trust perspectives are provided in Table 1 and Table 2 below.

\begin{table}[H]
  \label{tab:design_perspectives}
  \caption{Examples of Roles that require a Design Perspective of Trust in Autonomous Systems. }
  \begin{tabular}{|l|p{1.5in}|p{1.5in}|p{1.5in}|}
    \hline  & \multicolumn{3}{c}{\textbf{Role}}\\
    \hline  & \textbf{Designer} & \textbf{Acquirer} & \textbf{Disposer} \\
    \hline  \textbf{Definition} & Responsible for developing the system & Responsible for acquisition of the system & Responsible for the disposal of a system. \\ 
    \hline  \textbf{Level} & Organisation & Organisation & Organisation \\ 
    \hline  \textbf{Perspective} 
    & The designer of an Autonomous System develops trust through the application of known and trusted tools to well understood problems (e.g. a well-defined requirement set) using competent and trusted staff. \par The trust perspective therefore could be regarded as the \textbf{Design perspective}. 
    & The Acquirer of a System develops trust through prior experience of the vendor and similar products.  For any given product this is supplemented by the examination of engineering evidence provided by the Designer Organisation. \par Although there will be several trust aspects to the role, for the purposes of this paper this role can be seen as having a \textbf{Design perspective} since the Acquisition process needs to develop trust that the systems it is buying will be designed to be trustworthy in operation.  
    & System disposal does not necessarily indicate destruction.
    Where assets are passed to 3rd parties (e.g. though sale) the disposer must be confident that the autonomous behaviour can be reduced (where necessary) to a known and acceptable level.\par This perspective is therefore part of the \textbf{Design perspective} since there will be trust that (possibly advanced) behaviours can be prevented from being passed unwittingly to second user organisations; particularly since they may use the systems in a different context. \\ 
    \hline 
  \end{tabular} 
\end{table}
\begin{table}[H]
  \label{tab:operational_perspectives}
  \caption{Examples of Roles that require a Operational Perspective of Trust in Autonomous Systems. }
  \begin{tabular}{|l|p{1.5in}|p{1.5in}|p{1.5in}|}
    \hline  & \multicolumn{3}{c}{\textbf{Role}}\\
    \hline  & \textbf{Commander} & \textbf{Operator} & \textbf{User} \\
    \hline  \textbf{Definition} & Responsible for the system tactical activity (e.g. mission / activity setting) & Responsible for the ongoing control of the system when deployed on a particular mission / activity & An end user of the capabilities provided by the system. \\ 
    \hline  \textbf{Level} & Person & Person & Person/System/Org. \\ 
    \hline  \textbf{Perspective} 
    & The Commander places trust in the acquisition process to provide reliable assets.
    However, their trust perspective is \textbf{operational}.

    & An operator develops initial trust in a system through training and experience of similar systems.
    When interacting with a deployed system, the ongoing trust is maintained through correct and understandable system behaviour.
    This can be regarded as \textbf{Operational Trust }
    & A user of a System’s capability may not have any knowledge of the System itself but will need to develop trust in ability to provide trustworthy services.
    Again, this may be regarded as a form of \textbf{Operational Trust} \\ 
    \hline 
  \end{tabular} 
\end{table}

\subsection{Design Trust}

Five aspects of Design Trust have been identified:
\begin{enumerate}
  \item \textbf{Formal Specification of Dynamic Operation}: Autonomous Systems (AS) may be required to operate in complex, uncertain environments and as such their specification may need to reflect an ability to deal with unspecified circumstances.
    This includes engaging with dynamic systems of systems environments where an autonomous system may cooperate with a system not envisaged at design time. \textit{How can systems that are required to demonstrate that they meet their requirement be specified flexibly enough to permit adaptive behaviours}?

  \item \textbf{Security}: Any unmanned system has the potential to be used for illegitimate purposes by unscrupulous 3rd parties who could exploit security vulnerabilities to gain control of the system or sub-systems.
    Any system that has the potential to cause harm from such actions must have security designed in from the start to ensure that the system can be trusted to be resilient from cyber attack.
    Current accreditation schemes rely on a security assessment of a known architecture and there are mutual accreditation recognition schemes that could be encoded in dynamic discovery handshake protocols.
    This would produce a secure network assured through the accreditation of its component systems.  For example, the Multinational Security Accreditation Board (MSAB) deals with Combined Communications Electronics Board (CCEB) and NATO Accreditations to provide security assurance of internationally connected networks.
    Encoding such agreements into secure handshakes could enable dynamic accreditation of autonomous systems cooperating in a coalition environment.
    It is not known whether these have been demonstrated, so the question is: \textit{Can autonomous systems be designed to understand the security situation when interfacing with known or unknown systems?}

  \item \textbf{Verification and Validation of a Flexible Specification}: Following on from the description of a flexible specification, establish that the AS conforms and performs in accordance to the specification.  This has direct implication for the trust in the resultant system.
    How can systems demonstrate that they will behave acceptably when the environment is unknown?
  \item \textbf{Trust Modelling and Metrics}: This could be argued as part of the Verification and Validation of the system.
    However, models are increasingly being embedded into system design as a reference.
    Thus it is useful to consider this element separately.\textit{ How can trust be modelled sufficiently to span the space of most potential behaviours to help ensure that systems will be trusted when moved into operational environments?
    Can this be measured to allow comparison and minimum requirements set?}

  \item \textbf{Certification}: The certification requirements placed on specific systems will vary depending on domain and national approaches to certification.
    However, the common element in the requirement for certification is that a certified system is deemed as sufficiently trustworthy for use within its context of certification.
    Additionally Certification also relies on the predictability of a system.
    Because the aim of autonomous systems is to deal effectively with uncertain environments, \textit{can they (autonomous systems) be certified without being demonstrated in the environment within which they will adapt new behaviour?}
\end{enumerate}

Clearly existing military and commercial standards can play a significant role in demonstrating the trustworthiness of any systems’ design.
That is if a system has been designed to a Standard then it has known properties that have been accepted as good practice.
However, these do not address the issue of the five areas listed above.
The following sub section briefly outlines existing Standards for reference.

\subsubsection{Current Unmanned System Interface Standardisation}

There are three main organisations that are developing or have developed assurance standards for Unmanned Systems;
\begin{itemize}
  \item NATO Standardization Office (NSO)
  \item Society of Automotive Engineers (SAE)
  \item American Society of Testing and Materials (ASTM)
\end{itemize}

\paragraph{NATO Standardization Office}

Faced with the growing adoption of similar but disparate UAV systems within NATO territories and coalition nations, STANAG 4586\cite{STANAG4586} , promulgated in 2005, defined a logistic and interoperability framework to provide commonality in the C2 architecture and implementations of UAV/Ground station communications.

This included a particularly interesting development in the form of "Vehicle Specific Module" (VSM) interoperability, whereby existing systems could be grandfathered into 4586 compliance by the addition of a VSM to operate as a protocol translator.
This VSM could be mounted on the remote system, utilising a 4586 compliant Data Link Interface (DLI), or mounted on the UCS utilising a proprietary DLI to the remote system.
4586 described five Levels of Interoperability (LOI) for compliant UAV systems, shown in Table 3.
This structure has been criticised as being short sighted and at odds with the reality of modern and proposed autonomous vehicle operations \cite{Cummings2010}, specifically that in modern autonomous systems, there is no such thing as ‘direct control’ or ‘Operator-in-the-loop’, especially in the case of BLOS systems, and that in increasingly autonomous systems, operation is done as ‘Human Supervisory Control’ (HSC), or more commonly described as ‘Operator-on-the-loop’, whereby the operator interacts with the intermediate autonomous system and that autonomous system eventually performs that task on the hardware. 

\begin{table}
  \label{tab:levels_of_interoperability}
  \caption{Levels of Interoperability for STANAG 4586 Compliant UCS}
  \begin{tabularx}{\textwidth}{|l|X|}
    \hline  LOI &  \\ 
    \hline  1 &  Indirect receipt/transmission of UAV related payload data\\ 
    \hline  2 &  Direct receipt of Intelligence, Surveillance and Reconnaissance (ISR) data where “direct” covers reception of UAV payload data by the UCS when it has direct communication with the UAV\\ 
    \hline  3 &  Control and monitoring of the UAV payload in addition to direct receipt of ISR/other data\\ 
    \hline  4 &  Control and monitoring of the UAV, less launch and recovery\\
    \hline  5 &  Launch and Recovery in addition to LOI 4\\ 
    \hline 
  \end{tabularx}
\end{table}


\todo{SAE Levels of Autonomy possibly from \cite{Beer2014}}

Further, 4586 predominantly deals with a 1-to-1 mapping between operators and assets, when this is quite against the current state of the art; greater focus is being made in collective and collaborative assignment and having a single operator managing a task force of assets in-field, and handing off vehicle management responsibilities to the individual assets. 

\paragraph{Society of Automotive Engineers (SAE)}

The AS-4 steering group is responsible for the development and maintenance of the Joint Architecture for Unmanned System (JAUS) standards, which provide several service sets for Inter-System cooperation and interoperability, either in the form of a specified design language (JSIDL\footnote{JAUS Service Interface Definition Language}) or as a direct framework implementation, such as the JAUS Mobility, Mission Spooling, Environment Sensing, or Manipulator Service Sets\footnote{SAE AS6009, AS 6062, AS 6060, and AS 6057 respectively}.

This provides a stack-like interoperability model akin to the OSI inter-networking standard, providing logical connections between common levels across devices regardless of how subordinate layers are implemented.

Importantly, JAUS service models are open-sourced under the BSD-license, and a development toolkit is available for anyone to develop JAUS-compatible communications and control protocols\cite{JTS}. 

It is also important to note that JAUS is part funded, and heavily utilised by, US Army and Marine Robotic Systems Joint Project Office (RS-JPO), which manage the development, testing, and fielding of unmanned (ground) systems for those respective forces.
This includes now legacy M160 mine clearance platform and the highly popular (both with forces and their in-field operators) iRobot Packbot inspection and EOD clearance family of robots.


\paragraph{American Society of Testing and Materials (ASTM)}

The ASTM F38 committee has developed a LoS, single-asset-single-operator stove-piped framework for Unmanned Air Systems that is too constrained in scope for applicability to a more heterogeneous operating environment\cite{AmericanSocietyofTestingandMaterials2007}.
However, the F41 Committee, focused on Unmanned Maritime Vehicle Systems (UMVS) has collectively developed a range of interoperable standards, covering Communications, Autonomy and Control, Sensor Data Formats, and Mission Payload Interfacing.
Of particular interest is the Autonomy and Control standard \cite{AmericanSocietyofTestingandMaterials2006}, which highlighted a requirement on the vehicle system to be able to recognise an authorised client, be that a human operator or an additional collaborating vehicle.
Further, the standard states that the responsibility of the safety and integrity of any payload remains with the vehicle.
This standard was withdrawn in 2015 due to ASTM regulations requiring standards to be updated within 8 years of approval, and has no direct replacement within ASTM, but stands as a useful guiding perspective on autonomy standards within industry.

\missingfigure{ASTM F41 UMVS Architecture}

\subsection{Operational Trust}

This work is considering autonomous systems as entities of wider systems, we refer to these here as Autonomous Collaborative Systems.
As described earlier, Operational Trust has two main aspects, trust in the system to behave as expected and trust in the interfaces between systems (human/machine and machine/machine). 
Of all of the interfaces in an Autonomous Collaborative System, the most problematic is that arguably that between the System of Autonomous Systems (SoAS) and the human operator / team of operators. 
Cummings identified the main challenges to Human Supervisory Control (HSC), summarised below:\cite{Cummings2010}

\subsubsection{Information Overload}
Operator efficiency exhibits an optimum at moderate levels of cognitive engagement, above which cognitive ability is overloaded and performance drops (Otherwise known as the Yerkes-Dodson Law).
Additionally, in the case of under-engagement, operators can fall foul of boredom, and become desensitised to changing factors.
\textit{However, predicting this point of over-saturation is an open psychophysiological research problem.}

\subsubsection{Adaptive Automation}
Automation is well tailored to consistent levels of activity.
This is quite simply not the case in the military domain, characterised by long periods of ‘routine’ punctuated by high intensity, usually unpredictable, activity.
At those interfaces between ‘calm’ and ‘storm’, where SA and IA are imperative, temporary Information Overload is highly probable.
Adaptive Automation enables autonomous systems to increase their level of automation (LOA) based on specific events in the task environment, changes in operator performance or task loading, or physiological methods. 
It is taken as given that for routine operations, and increased LOA reduces operator workload, and vice versa.
However, this relationship is highly task dependent and can create severe problems in cases of LOA being greater, or indeed lesser, than is required. 
In the cases of overly-high LOA, operator skill is degraded, situational awareness is reduced as the operator is not as engaged, and the automated system may not be able to handle unexpected events, requiring the operator to take over, which, given the previous points, is a difficult prospect.
Alternatively, in sub-optimal LOA, Information Overload can result in the case of high intensity situations, but also the system can fall foul of overly-sensitive human cognitive biases, false positive pattern detection, boredom, and complacency in the case where less is going on.
Therefore, as a corollary to Information Overload challenges, there is a need to define the interrelationship between levels of situational activity (or risk) and appropriate levels of automation. 
\textit{Under what circumstances can AA be used to change the LOA of a system? Does the autonomous system or the human decide to change LOA? What LOAs are appropriate for what circumstances?}


\subsubsection{Distributed Decision Making}
In a modern, non-hierarchical, often distributed or cellular military management system (Network Centric Warfare doctrine for example), tools are increasingly being used to mitigate information asymmetry within C2.
A simple example of this is shared watch-logs in the Naval space, providing temporal collaboration between watch-teams separated in time.
The DoD Global Information Grid is another example of a spatial collaborative framework.
Recent work has demonstrated the power of collaborative analysis and human-machine shared sensing technologies even with low levels of training on the part of the operators providing superior results and resource efficiencies than either humans or machines alone in survey and search-and-rescue scenarios (Ahmed et al.2014).
As these temporal and spatial collaboration tools increase in complexity and ability, decisions that previously required SA that was only available at higher echelons within the standard hierarchy are available to commanders on the ground, or even to individual team members, enabling the potential for informed decisions to be taken faster and more effectively, enabled by automated strategies to present relevant information to teams based on the operational context.
However there are a range of operational, legal, psychological and technical challenges that need to be addressed before confidence in these distributed management structures can be established.
Studies into SA sharing techniques (telepresent table-top environments, video conferencing, and interactive whiteboards) have generally yielded positive results, however investigations into interruptive-communications (such as instant messaging chat) have demonstrated a negative impact on operational efficiency.
In short, the biggest problem with distributed decision making in the context of supervisory systems is that \textit{there is no consensus on whether it is advantageous or not, and what magnitude of operational delta is introduced, if any.}

\subsubsection{Complexity}
Beyond simple Information Overload, increasing complexity of information presented to operators is having a negative effect on operational efficiency. 
In HSC, displays are designed to reduce complexity, introducing abstractions with an aim to presenting the minimum amount of information to the operator required to maintain an accurate and up-to-date mental model of the environmental and operational state.
This has led to the development of many domain specific decision support interfaces, however, in academic research, there has been nothing but ‘mixed results’.
One commonly raised negative is the general bias on the ‘cool factor’ of interfaces.
Immersive 3D visual, aural, or haptic interfaces that at first appraisal seem to provide more approachable information to the operator, and are indeed tacitly preferred by operators in use. 
However, there has not been any evidence to demonstrate performance improvement when using these tools, and in-fact, \textit{improving the ‘fidelity’ of the interfaces has led to operators’ overly-relying on these representations of the environment rather than remaining engaged in the environment.}

\subsubsection{Cognitive Biases and Failing Heuristics}
The increasingly connected battlefield has massively increased the tempo of operations, with increasing requirements on commanders and operators to make rapid decisions with imperfect information.
However, Human decision making isn’t always rational (especially under pressure), and operators use personally derived heuristics to make ‘rational shortcuts’.
This is a double edged sword, where these heuristics can be employed to greatly reduce the normative cognitive load in a stressful situation, but also introduce destructive biases, where these shortcuts make assumptions that don’t bear out in reality.

For example, in the context of decision support systems, “Autonomy Bias” has been observed as a complement to the already well known “Confirmation Bias”\footnote{Confirmation Bias is the tendency for people to preferentially select from available information that information that supports pre-existing beliefs or hypotheses.}  and “Assimilation Bias”\footnote{Assimilation Bias is often thought of as a subset of Confirmation Bias, whereby it specifies that instead of seeking out information supporting of current views, any incoming data is interpreted as being supportive of a particular view without questioning that view, even if it appears contradictory.} , where operators that have been provided with a ‘correct’ answer by a decision support system do not look (or see, depending on your perspective) for any contradictory information, and will unquestionably follow, increasing error rates significantly. 

This behaviour isn’t only the reserve of decision support systems, but also in the generic allocation of operator attention; ‘scheduling heuristics’ are used to decide how much time tasks should be worked on, and time and again, humans are found to be far from optimal in this regard, especially in time-pressured scenarios where these heuristics are in even more demand.
Even when operators are given optimal scheduling rules, these quickly fall apart, often due to primary task efficiency degradation after interruption.
This highlights a critical interface in the adoption of complex autonomous systems that still demand ‘Man in the loop’ functionality; if a system is required to have full-time concentrated supervision (e.g. flying a UCAV), but also event-based reactive decision making (e.g. alerts from non-critical subsystems), both tasks are negatively impacted.
In an assessment of factors influencing trust in autonomous vehicles and medical diagnosis support systems, Carlson et al also identified that a major factor in an operator or users’ trust in a system was not only dependant on past performance and current accuracy but also on ‘soft factors’ such as the branding and reputation of the manufacture / designer.(Carlson et al.
2014)
Further, autonomous decision support / detection / classification systems have an ‘uncanny valley’ to overcome in terms of accuracy, in that there is a dangerous period when such systems are used but not perfect, but operators become complacent, causing an increased error rate, until such a time that those autonomous systems can match or exceed the detection rates of their human counterparts.

\subsubsection{Summary of Human Factors impacting Operational Trust in Defence Contexts}

When dealing with human supervision of autonomous or semi-autonomous systems, there is an inherent conflict between the expectations of the operator, the hopes of system architects.
System Architects aim to provide more and more information to the operator to justify a systems operation, and Operators in reality need less and less information to be efficient when things are going well, and responsive in a dynamic environment.
This places huge demands on Human Interface design and indeed on communications design to provide this timely, relevant, interactive connection between any autonomous system and the end operator(s).
Recent work has presented the idea of taking user interface (UI) inspiration from the entertainment sector, in terms of UI best practises developed over two decades of Real-Time Strategy game development \cite{Johnson2007}, and follow up work into automated mission debrief demonstrated that such operational support could improve causal situational awareness of an operator when compared to a human-baseline \cite{Johnson2011}.
In terms of the human factors challenges raised by Cummings, they are often contradictory in their direction, particularly when contrasting between Adaptive Automation and Cognitive Biases challenges.
This is a key part of the ‘soft-trust’ theory, where the operators and commanders need to be able to implicitly and explicitly trust the operation of a remote system with limited feed-back bandwidth, high latency, or long-term operation such that direct remote operation is infeasible or undesirable.
To be able to trust that systems’ ability to continue on a course, survey an area, notify on detection of an anomaly, etc.is going to be the corner stone of any autonomous systems justification in the future.

\section{Trust and Reputation in Autonomous Collaborative Systems}

In addition to the two perspectives of trust identified thus far, and for the purposes of this investigation, it is necessary to define and classify Operational Trust into two distinct but related sections, which we define as being

\begin{itemize}
  \item Hard Trust or technical trust, being the quantative measurement and communication of the expectation of an actor performing a certain task, based on historic performance and through consensus building within a networked system.
    Can be thought of as a de-risking strategy to measure the ability of a system to perform a task unsupervised.

  \item Soft Trust or common trust, being the qualitative assessment of the ability of an actor to perform a task or operation consistently and reliably based on social or experiential factors.
    This is the ‘natural’ form of trust and is the main motivational driver for the human-factors trust discussion.
    Can be rephrased as the level of confidence in an actor to perform a task unsupervised.
\end{itemize}

It is already clear that these two definitions are extremely close in their construction, but represent fundamentally different approaches to trust, one coming from a sociological perspective of person-to-person and person-to-group relationships from day to day life, and the other coming from a statistical appraisal of an activity by a system.
The difficulty with human supervisory controlled autonomous systems is that there is a need for both a hard and soft trust perspective, and that this interface can often create fundamental misunderstandings.


\section{Levels of Trust}
Trust relationships operate as part of a system architecture, and can quite often get confused.
As such, we constrain the focal domain as per \cite{Liu2006} into six constructs.
Sun\cite{Sun2008} suggests that within these there are two overarching forms of trust:
\begin{itemize}
  \item Behavioural: That one entity voluntarily depends on another entity in a specific situation
  \item Intentional: That one entity would be willing to depend on another entity
\end{itemize}

These concepts closely mirror the authors definitions of ‘Hard’ and ‘Soft’ trust respectively, one (Behavioural) being an invested dependency given certain parameters being satisfied, mirroring Hard Trust, and the other (Intentional) being the ‘capacity for belief’ in another entity, analogous to Soft Trust.
It is suggested that these overarching forms are supported by and indeed are drawn from four major constructs within social and networked environments:
\begin{itemize}
  \item Trusting Belief: the subjective belief within a system that the other trusted components are willing and able to act in each-others’ best interests
  \item Dispositional Trust: a general expectation of trustworthiness over time 
  \item Situational Decision Trust: in-situ risk assessment where the benefits of trust outweigh the negative outcomes of trust
  \item System Trust: the assurance that formal impersonal or procedural structures are in place to ensure successful operation.
\end{itemize}

While Sun argues that only System Trust and Behavioural Trust are relevant to trusted networking applications.
However, it is arguable that in any network where the operation of that network is not the only concern, or where that network has to interact with any operator, then all of these factors come into play.
Both System and Behavioural trust rely on what Sun calls a ‘Belief Formation Process’, or a trust assessment, while the other trust constructs deal with the interactions between trust and decision making against an internal assessment of network trustworthiness.

