\section{A Trust Management Framework for Detecting Malicious and Selfish Behaviour 
in Ad-Hoc Wireless Networks using Fuzzy Sets and Grey theory \citet*{Guo}}
\label{Guo}
\begin{itemize}
  \item Provides background on TMF
  \item References Policy Language, PKC\footnote{Public Key
  Cryptography}, Resurrecting Duckling Model and Distributed Trust Model
  \cite{Li2007}
  \item Raises need for TMF to mitigate selfishness
  \item Introduces Grey Theory From Deng Julong (Can't find any official
  citations for this, paper appears to be stuck in  Springerlink)
  \item Defines three types of trust relationship
  \begin{itemize}
    \item \emph{Direct}: Trust based on historical behaviour of node $B$ wrt
    node $A$.
    \item \emph{Indirect}: Trust transited through third-party entities, i.e
    $E$, $F$ wrt $B$ where neither communicate with $A$
    \item \emph{Recommendation}: a subjective trust transited through a common
    entity; i,e Trust($B\rightarrow C$) communicated to $A$ by $C$.
  \end{itemize}
  \item Highlights some of the potential attacks on TMF and their sources
  \cite{Sun2008}, \cite{Li2008}.
  \item Guo proposes a TMF leveraging the curve-fitting
  analytics of Grey Theory to allow in practical networks, multi-parametric
  based trust quantities, and to use this data and results to assess the type
  of selfish behaviour being exhibited by 'bad' nodes.
  \item Classical TMF only monitor single behaviours: probability of successful
  interactions (Bit Error Rate/Packet Loss Rate).
  The TMF suggested by Guo incorporates signal strength, data-rate and other
  physical factors in addition to PLR.
  \item The application of Grey Whitenisation and clustering allows for
  quantifiable Trustworthiness classification, i.e taking the multi-parametric
  measurements of behaviour and condensing this to a per-node wrt calculating
  node trust assessment.
  \item Includes worked simulation examples of a 6 node network
\end{itemize}

\subsection{Questions Raised}
\begin{itemize}
  \item Stated definitions of Direct, Indirect, and Recommendation trust do not
  suitably distinguish between Indirect/Recommendation. Hopefully after finding
  a copy of Julong this is be made clearer.
  \item Grey Theory seems mathematically intuitive but the fundamental question
    is the selection of \emph{distinguishing coefficient} values \cite{Cai2009}, and
  whether these values can be collaboratively 'learned' over time, almost like familial
  and social trust.
  \item Additionally, the derivation of Whitinisation functions is not at all
  clear.
  \item Generally: What is the implication for using this style of trust
  assessment against human operators, remotely or in physical nodes (eg
  mother-ships)? Will lapses in judgement be detrimentally held against an
  operator? Can a TMF as suggested be modified to accept human frailties (and
  should it!)?
\end{itemize}
