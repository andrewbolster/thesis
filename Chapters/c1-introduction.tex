% !TeX spellcheck = en_GB
\def\ChapterTitle{Introduction} % Write in your own chapter title

\chapter{\ChapterTitle}
\label{Chapter\thechapter}
\lhead{Chapter \thechapter.
\emph{\nameref{Chapter\thechapter}}} % Write in your own chapter title to set the page header

\section{Mobile Ad-hoc Networks (\acrshortpl{manet})}

With the explosive growth in the use of mobile telephony and the increasing miniaturisation and efficiency gains of portable communications devices, the classical paradigm of a broadcast/receiver or server/client has given way to an increasing use of decentralised, ad-hoc networks that not only accommodate but take advantage of network mobility.

Whether these networks are decentralised cellular / RF / 802.11 WiFi networks for use in disaster relief areas \cite{Milliken2015} or biologically inspired wireless sensor networks for low-energy, low-maintenance environmental monitoring \cite{Nicholson2008}\cite{Selvakennedy2007}, \gls{manet} theory developed over the past 30 years has gone from it's first formal definition, emerging from DARPA's Packet Radio Network research \cite{Jubin1987}, to being an integral part of modern practical communications.

Minimally, a \gls{manet} consists of of a collection of mobile physical entities (nodes) that communicate cooperatively to collect, distribute, disseminate, and collate data and/or influence across an area.
In many cases \gls{manet} nodes incorporate bi-directional transceivers to send and receive data, however this bi-directionality is not a requirement, for example in the area of Wireless Sensor Networks \cite{Akyildiz2002}.
\glspl{manet} may utilise omnidirectional, static, or steerable communications antennae, and a selection of protocols such as WiFi, Bluetooth, GSM, UMTS, Optical or Acoustics, and may incorporate a range of mobilities across nodes, from static devices, terrestrial and marine surface platforms, and aerial and underwater platforms
A core characteristic of the design of \glspl{manet} is the inclusion and integration of heterogeneous node collections, i.e where different nodes or groups of nodes in a network may have different capabilities in terms of propulsion, sensor apparatus, communications capability, etc.

\glspl{manet} may be totally independent with no external connections; include independent per-node communications backhauls (e.g. Cellular Modems in mobile phones as part of a Bluetooth Personal Area Network(PAN)), or include static nodes that provide infrastructure based backhaul.
However, this multiplicity of variations and options presents several challenges to users and operators; the physical topology of \glspl{manet} can vary wildly over short periods of time.
A particular challenge to \gls{manet} operation is that given any node may operate as a routing / gateway node, if/when that node moves to a different region, network segments that had previously used that node as a path must renegotiate / re-establish their routes.
These situations, if not appropriately managed, lead to opportunities for subversion and selfishness.

The characteristics of \gls{manet}s as defined by Corson et al.\ are paraphrased in Table~\ref{tab:manet_characteristics}.

\begin{table}[h!]
  \hyphenpenalty=10000
\caption[Summary of Characteristics of \gls{manet}s]{Summary of Characteristics of \gls{manet}s\cite{Corson1999}}
\label{tab:manet_characteristics}
  \begin{tabularx}{\textwidth}{p{2cm}X}\toprule
    Dynamic Topologies & Nodes are free to move arbitrarily; thus, the typically multi-hop network topology may change randomly and rapidly at unpredictable times, and may consist of both bidirectional and unidirectional links.
\\
    Bandwidth Constrained, Varied Capacity & Wireless links will continue to have significantly lower capacity than their hardwired counterparts.
In addition, the realized throughput of wireless communications, after accounting for the effects of multiple access, fading, noise, and interference conditions, etc., is often much less than a radio's maximum transmission rate.
\par
One effect of the relatively low to moderate link capacities is that congestion is typically the norm rather than the exception, i.e.\  aggregate application demand will likely approach or exceed network capacity frequently.\\
    Energy Constrained Operation &  Some or all of the nodes in a \gls{manet} may rely on batteries or other exhaustible means for their energy.
For these nodes, the most important system design criteria for optimization may be energy conservation.\\
    Limited physical security & Mobile wireless networks are generally more prone to physical security threats than are fixed cable nets.
The increased possibility of eavesdropping, spoofing, and denial-of-service attacks should be carefully considered.\par
Existing link security techniques are often applied within wireless networks to reduce security threats.
As a benefit, the decentralized nature of network control in \glspl{manet} provides additional robustness against the single points of failure of more centralized approaches.\\\bottomrule
\end{tabularx}
\end{table}

\section{Routing in \glspl{manet}}

Given the decentralised nature of \gls{manet} operations, routing protocols are an active area of research. 
This research is classified according to the strategies used for discovering, monitoring and updating routes within the network, and are usually grouped into three classes; proactive (or Table Driven), reactive (or On Demand) and hybrid.
A summary of the generalised characteristics of these classes is shown in \autoref{tab:routing_categories}.

\subsection{Proactive Routing}

In Proactive routing, protocols attempt to maintain a up-to-date, global topology awareness of the network, where every node at least knows how to make the best hop to contact any other node in the network.
This is extremely efficient for relatively static networks, with minimal storage and time requirements \cite{}<++>.
When the network topology is significantly modified by a shift in topology, either due to a node ``dropping out'' or moving, route renegotiation and optimisation is extrememly resource consuming, as this global state is converged upon in a distributed manner by nodes exchanging their local knowledge of the ``new'' topology.

\begin{table}\centering
  \caption[Selection of Proactive Routing Protocols]{Selection of Proactive Routing Protocols}
  \label{tab:proactive_routing_protocols}
  \begin{tabularx}{\textwidth}{p{1.25cm}|X}\toprule
    Name & Description \\ \midrule
    \gls{dsdv} & \acrlong{dsdv} is a loop free derivative of the Distributed Bellman-Ford algorithm where each node maintains two tables; one that attempts to maintain a globally accurate next-hop routing table for all destination nodes (routing table) and a route advertisement table, monitoring routes that the node itself can provide. These tables are updated both periodically and opportunistically. Loop-free status is maintained by monitoring the ``sequence number'', which guarantees that if a long-loop returned packet is observed, it is discarded in favour of a route with a higher sequence number (i.e.\ newer route)~\cite{Perkins1994}.\\
    \gls{olsr} & \acrlong{olsr} \\
    \gls{wrp} & \\
    \gls{tbrpf} & \\
    \gls{dream} & \\
    \bottomrule
  \end{tabularx}
\end{table}


\subsection{Reactive Routing}

\begin{table}\centering
  \caption[Selection of Reactive Routing Protocols]{Selection of Reactive Routing Protocols}
  \label{tab:reactive_routing_protocols}
  \begin{tabularx}{\textwidth}{p{1.25cm}|X}\toprule
    Name & Description \\ \midrule
    \gls{dsr} & \\
    \gls{aodv} & \\
    \gls{roam} & \\
    \gls{abr} & \\
    \gls{lar} & \acrlong{lar} incorporates location information (usually from \gls{gps}), and generates a heuristic based on either the distance from the current node \emph{towards} the destination location, or the distance from the current node \emph{away from} the original source, minimising and maximising this distance respectively.These methods limit control overheads and usually accurately determine the shortest path. However, in highly mobile networks this behaviour appears increasingly flood-like (similar to \gls{dsr} and \gls{aodv}), and the general requirement for highly accurate and timely positional information restricts the application of this protocol\\
    \gls{cbrp} & \acrlong{cbrp} uses a hierarchical topology where each cluster has a cluster-head which coordinates routing within that cluster. As only cluster-heads coordinate routing across clusters, transmission overheads are minimised compared to other route distribution methods. However, the negotiation and maintence overheads and propogation delays associated with hierarchical clustering make the network susceptible to temporary routing loops as nodes may have inconsistent residual routing information during cluster re-negotiation\\
    
    \bottomrule
  \end{tabularx}
\end{table}

\subsection{Hybrid Routing}

\begin{table}\centering
  \caption[Selection of Hybrid Routing Protocols]{Selection of Hybrid Routing Protocols}
  \label{tab:hybrid_routing_protocols}
  \begin{tabularx}{\textwidth}{p{1.25cm}|X}\toprule
    Name & Description \\ \midrule
    \gls{dst} & \\
    \gls{ddr} & \\
    \gls{zrp} & \\
    \gls{zhls} & \\
    \gls{slurp} & \\
    
    \bottomrule
  \end{tabularx}
\end{table}


\begin{table}\centering
  \caption[Comparison of Routing Strategy Classes]{Comparison of Routing Strategy Classes\cite{Abolhasan2004}}
  \label{tab:routing_categories}
  \begin{tabularx}{\textwidth}{p{2cm}|XXX}\toprule
    \diagbox[width=2cm, height=1.8cm]{Area}{Class} & Proactive & Reactive & Hybrid \\ \midrule
    Routing Structure & 
      Both flat and hierarchical structures are available &
      Mostly flat except \gls{cbrp} &
      Mostly hierarchical \\
    Route Availability &
      Always available if nodes are reachable &
      Determined when needed &
      Depends on the location of the destination \\
    Control Traffic Volume &
      Usually high, attempt at reduction is made. e.g. \gls{olsr}, \gls{tbrpf} &
      Lower than Global routing and further improved using \gls{gps}. e.g. \gls{lar} &
      Mostly lower than proactive and reactive \\
    Periodic Updating &
      Yes, some may be conditional e.g. \gls{star} &
      Not required, however some nodes may require periodic beacons. e.g. \glspl{abr} &
      Usually used within each zone or between gateway nodes \\
    Mobility Handling &
      Usually updates occur at fixed intervals. \gls{dream} alters periodic updates based on mobility &
      \gls{abr} uses localised broadcast queries, \gls{roam} uses threshold updates, \gls{aodv} routing uses local route discovery &
      Usually more than one path may be available. Single point of failures are reduces by working as a group\\
    Storage Requirements &
      High &
      Dependent on number of nodes kept or required; usually lower than proactive protocols &
      Usually depends on cluster or zone size; may become as large as proactive if clusters are big \\
    Delay Level &
      Short routes are predetermined &
      Higher than proactive &
      Short for destinations in the same zone/cluster as source. Interzone may be as large as Reactive protocols\\
    Scalability &
      Up to 100 nodes; \gls{ospf} and \gls{tbrpf} may scale higher &
      Source routing protocols; up to a few hundred nodes. Point-to-point may scale higher. Depends on level of traffic and levels of multihopping&
      Designed for up to or more than 1000 nodes \\
    \bottomrule
  \end{tabularx}
\end{table}

\section{Node Density in \glspl{manet}}

One fundamental compromise in the operation of wireless \glspl{manet} is the tradeoff between the number of hops required between source and destination nodes and the effective bandwidth available to the network overall\cite{Royer2001}.
This compromise is encapsulated in the relative density of a given network; that is, the number of nodes in a given node's one-hop locality, drawing direct links between wireless transmission strength / reception sensitivity, the environmental noise floor, envionmental channel characteristics, the mobility of the nodes and the number of nodes deployed in a region.




\section{\glspl{manet} in Harsh Environments}

As \acrfullpl{manet} grow beyond the terrestrial arena, their operation and the protocols designed around them must be reviewed to assess their suitability to different communications environments, ensuring their continued security, reliability, and performance.

The distributed and dynamic nature of \glspl{manet} mean that it is difficult to maintain a \gls{ttp} or evidence based trust system such as Certificate Authorities or using \gls{pki}.\todo{more background on the operation of TTP/CA/PKI?}
Therefore, a distributed, collaborative system must be applied to these networks.
Such distributed trust management frameworks aim to detect, identify, and mitigate the impacts of malicious actors by distributing per-node assessments and opinions to collectively self-police behaviour.
As such, \glspl{tmf} can be used to predict and reason on the future interactions between entities in a system.

\glspl{tmf} provide information to assist the estimation of future states and actions of nodes within \glspl{manet}.
This information is used to optimize the performance of a network against malicious, selfish, or defective misbehaviour by one or more nodes.
Previous research has established the advantages of implementing \glspl{tmf} in 802.11 based \glspl{manet}, particularly in terms of preventing selfish operation in collaborative systems \cite{Li2007}, and maintaining throughput in the presence of malicious actors \cite{Buchegger2002}

\section{Systems Approach to Trust and Trust Engineering}

\todo{Trust as Assurance}

\section{Trust Operation Against Capable Attackers}


\section{Contributions}


\section{Conclusion}
\subsection{Layout}

%%%%%%%%%%%%%%%%%%%%%%%%%%%%%%%%%%%%%%%%%%%%%%%%%%%%%%%%%%%%%%%%%%%%%%%%%%%%%%%
