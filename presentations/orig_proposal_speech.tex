

% ----------------------------------------------------------------------
%                   LATEX TEMPLATE FOR PhD THESIS
% ----------------------------------------------------------------------

% based on Harish Bhanderi's PhD/MPhil template, then Uni Cambridge
% http://www-h.eng.cam.ac.uk/help/tpl/textprocessing/ThesisStyle/
% corrected and extended in 2007 by Jakob Suckale, then MPI-CBG PhD programme
% and made available through OpenWetWare.org - the free biology wiki

%: Style file for Latex
% Most style definitions are in the external file PhDthesisPSnPDF.
% In this template package, it can be found in ./Latex/Classes/
\documentclass[oneside,9pt,a4paper]{Latex/Classes/PhDthesisPSnPDF}

%: Macro file for Latex
% Macros help you summarise frequently repeated Latex commands.
% Here, they are placed in an external file /Latex/Macros/MacroFile1.tex
% An macro that you may use frequently is the figuremacro (see introduction.tex)
\include{Latex/Macros/MacroFile1}



%: ----------------------------------------------------------------------
%:                  TITLE PAGE: name, degree,..
% ----------------------------------------------------------------------
% below is to generate the title page with crest and author name

%if output to PDF then put the following in PDF header
\ifpdf  
    \pdfinfo { /Title  (An Investigation into Trust and Reputation Frameworks for Autonomous Underwater Vehicles)
               /Creator (TeX)
               /Producer (pdfTeX)
               /Author (Andrew Bolster abolster01@qub.ac.uk)
               /CreationDate (D:YYYYMMDDhhmmss)  %format D:YYYYMMDDhhmmss
               /ModDate (D:YYYYMMDDhhmm)
               /Subject (xyz)
               /Keywords (add, your, keywords, here) }
    \pdfcatalog { /PageMode (/UseOutlines)
                  /OpenAction (fitbh)  }
\fi


\title{An Investigation into Trust and Reputation Frameworks for Autonomous Underwater Vehicles}



% ----------------------------------------------------------------------
% The section below defines www links/email for author and institutions
% They will appear on the title page of the PDF and can be clicked
\ifpdf
  \author{\href{mailto:me@andrewbolster.info}{Andrew Bolster}}
%  \cityofbirth{born in XYZ} % uncomment this if your university requires this
%  % If city of birth is required, also uncomment 2 sections in PhDthesisPSnPDF
%  % Just search for the "city" and you'll find them.
  \collegeordept{\href{http://www.ecit.qub.ac.uk}{Institute of Electronics, Communications and Information Technology (ECIT)}}
  \university{\href{http://www.qub.ac.uk}{Queen's University Belfast (QUB)}}

  % The crest is a graphics file of the logo of your research institution.
  % Place it in ./frontmatter/figures and specify the width
  \crest{\includegraphics[width=4cm]{frontmatter/figures/crest.png}}
  
% If you are not creating a PDF then use the following. The default is PDF.
\else
  \author{Andrew Bolster}
%  \cityofbirth{born in XYZ}
  \collegeordept{Institute of Electronics, Communications and Information Technology (ECIT)}
  \university{Queen's University Belfast (QUB)}
  \crest{\includegraphics[width=4cm]{frontmatter/figures/crest.png}}
\fi
%\renewcommand{\submittedtext}{change the default text here if needed}
\degree{Initial Speech to DSTL/DGA proposing research}
\degreedate{2011 June}


% ----------------------------------------------------------------------
       
% turn of those nasty overfull and underfull hboxes
\hbadness=10000
\hfuzz=50pt

%: --------------------------------------------------------------
%:                  FRONT MATTER: dedications, abstract,..
% --------------------------------------------------------------

\begin{document}
\maketitle  % command to print the title page with above variables
\section{Introduction}
\subsection{Title Slide} 
Good afternoon ladies and gentlemen, its an honour to be
invited to present our proposal here, and I hope you find it both intriguing and
enjoyable.

\section{AUVs}
\subsection{AUV Definition}
The general area of research selected is the
commercial, defence, and environmental use of independent groups of Autonomous Underwater
Vehicles, hereafter described as fleets.

\subsection{Predators} After over 20 years of theoretical and exploratory
research into Unmanned Vehicles on, above, and below sea level, the recent  in theatre use of
UAVs such as the USAF's Predator range, has proven the effectiveness of
autonomous or semi-autonomous delivery and reconnaissance platforms, and this
success has opened the field of research to a much wider range of possibilities.

\subsection{Predators + AUV}  One area of particular interest to the research community
has been the movement of ideas and technologies proven in the skies to the
marine environment.

\subsection{UAV Examples}  Of particular interest to the research community has research
into swarming behaviours of small reconnaissance drones, analogous to flocks of
birds; human-guided operation, where by a remotely operated drone is flanked by
a team of automated drones; and the use of task 'checkpointing' where by
automated drones perform a particular task or series of tasks, and then notify a
remote operator to take over manual control for specific roles.

\subsection{AUV Examples}  Bringing these type of technologies into the Naval space has
led to applications such minefield detection and monitoring, submarine tracking,
long range reconnaissance, deep-sea surveying, and much more. Earlier this year,
successful trials were performed by NURC at La Spezia in Northern Italy
investigating the collective performance of AUV's in a port-protection in a
counter-terrorism operational context.

The major foci of academic research into the use of autonomous robotics in a
marine environment have been the development of individual and collaborative
behaviour patterns within a fleet. However, the research proposed here addresses
an important and emerging area; the incorporation of 'trust' to improve the
operational effectiveness and reliability of such fleets.

\section{Trust and Reputation Management}
\subsection{Trust}  The question of trust is an important one; it is the quantitative
assessment of the expected behaviour of a network node, in this case an
individual AUV. This internal trust is generally robust and indeed secured the
node network further where by each node is sharing trust information about the
other nodes. This leads to the two basic trust mechanisms; direct and indirect
recommendations. For instance, Nicola, you can communicate directly with (Person
B), and (Person B) can communicate directly with (Person C), and as such, A can
establish a direct trust record of B, and B of C. If Nicola wants to query B as
to the trustworthiness of C, B can respond with a normalised trustworthiness of
C. This is an indirect trust recommendation. Further, implied reputation can be
garnered by association, so for instance I know Nicola you went to Strathclyde,
and I have friends at Strathclyde who tell me its brilliant, I can therefore
imply some reputation from that. In the reverse, you may have knowledge of
Queen's, or of Professors Marshall or Fontaine, and by my association with them
that may imply some qualatative reputatation of me.

This raises an important question; how does one assess the trust of a
fleet-foreign entity of which no fleet-member has experience of?

\subsection{fleet Network 1}  Taking the example of independent task-oriented operation,
whereby fleets of AUVs operate autonomously over extended periods, independently
from any persistent 'mother-ship', this fleet may need to occasionally
communicate with friendly ships or relay bouy's to return data for analysis or
simply to notify task-commanders to their status. This raises the point just
stated of how do we assess the trustworthiness of these extra-fleet entities
such as the relay bouys or 'friendly' ships.

Additionally, the fleet will have significant internal communications between
individual AUVs. In the case of above-sea-level UAV communication systems, this
would simply be over free-space radio, but the marine environment creates
significant challenges to even localised communications, due to factors such as
low channel bandwidth, scattering, and long propagation delays.

As such, multiple technologies must be applied to enable underwater
communications, including free-space optical and acoustic transmission systems,
while radio is reserved for surface applications. These conditions make it very
difficult to assess the real trustworthiness of an entity when it is unclear if
its abnormal behaviour is due to these effects, or due to actual bad behaviour.

\subsection{Loner}  Further, due to the extended mission times of such fleets, and their
planned isolation from classical command and control structures, they
necessarily become vulnerable to attacks to subvert their objective; or to
obtain or otherwise compromise the information they intend to report.

\subsection{Healthy fleet 5}  Additionally, the size of the fleet can be variable over
time; with defective or otherwise incapacitated units 'dropping off' (Dead
Battery)-(fleet 4) and fresh or specialised units being rolled out in-theatre
(fleet 6) to join a particular fleet. This presents two major vulnerabilities;
enemy emulation of a friendly mother-ship, or of a fleet individual.

\subsection{Evil fleet}  These factors taken together present two key challenges for a
fleet; the ability to assess trust of a 'friendly' vessel, and to collectively
detect potentially abnormal behaviour of a fleet-member.

It is these challenges that are the driving force of the proposed research. To
date there has been little or no vulnerability analysis of fleets of AUVs. This
proposal will initially seek to identify a range of threat models for fleets of
autonomous AUVs; representative of real-mission events, for instance; passive
eave's-dropping of communications; fleet infiltration using cloned AUVs or
surface vessels; or partial fleet destruction to compromise mission
capabilities.

\subsection{MANETS}  These threats have similarities to those experiences by wireless
mobile ad-hoc and peer-to-peer networks, usually summarised as MANETs. Therefore
a key initial area will be to investigate the applicability of existing research
in those areas to the stated problem context. This would include research into
existing modelling techniques as well as any protocols and behaviours used to
counter these threats in the MANET space.

Any discussion of trust within this context leads to the need for a distributed
trust and reputation management framework, using past experience to gauge the
expected behaviour of a node in a distributed network. Most of these systems,
such as Objective Trust Management Framework or OTMF, \subsection{OTMF Def}  use only
one physical or protocol measurement to make these trustworthiness decisions,
but recent work has applied multi-parametric classification techniques such as
Grey Theory \subsection{Grey Def}  to not only detect abnormal behaviour but also to
classify what type of attack is being attempted.

\subsection{Network Again}  Due to the previously stated marine effects on transmission,
even assuming that existing trust structures could be deemed applicable to AUV
operations, the metric-sets applied would have to be completely re-approached to
take account of the unique communications environment

A plan of work will be developed, including close feedback structures with
DSTL/DGA, that will involve analysis of the threats and vulnerabilities to such
fleets, followed by the development of suitable metric-sets and a trust
framework for a fleet.

\subsection{Pools}  While the majority of the work will be theoretical development and
simulation of threat models, the facilities afforded by Professor Fontaine's
CIIRF group at UPMC in France will provide access to real AUV systems. This
experience is essential to determining real-world performance of different
metric-sets, and to evaluate the practicality of such schemes.

The significant objects of work can be summarised as:\subsection{DC1}  1. An assessment
of vulnerabilities to a fleet of AUVs during typical operations with an aim to
develop a suitable threat model \subsection{DC2}  2. Identification of metrics and
authentication schemes suitable to discriminate between legitimate and
suspicious AUV behaviour.\subsection{DC3}  3. Development of a distributed trust
management framework incorporating reputation analysis schemes throughout a
fleet of AUVs \subsection{DC4}  4. And finally to explore and assess the feasibility of
the framework for use across a range of operational tasks with an aim to
integrate into existing frameworks such as MODAF/AGATE.


\section{Project Structure}  
\subsection{Team}
Both supervisors have extensive experience in defence
research, including naval research, and within this research area, have complementary
experience; Professor Marshall has been involved in the assessment, development
and implementation of reliable and secure wireless networks for over 12 years,
and Professor Fontaine is an established expert in the field of AUVs, including
involvement with a number of NATO initiatives in this area, and has ties to
research organisations such as the NATO Undersea Research Centre that can
provide essential practical resources to this project.

Both supervisors have extensive experience in the management of PhD students,
both independently and within industry and other organisations, and through
regular communication and progress updates between myself and these supervisors,
I am confident that should any problems arise, corrective action can and will be
suggested by them.

\subsection{Project Monitoring}  Beyond the internal monitoring of project development,
it is felt that it is essential to ensure that DSTL and DGA are involved in the
ongoing progress. This is projected to take the form of annual or semi-annual
progress reports, detailing not only the progress made, but whether this
progress is in line with development objectives, and to ensure that the findings
and direction of research is in line with expected stakeholder goals.

Considering the potential exploitation of the project, both supervisors have
experience in defence sponsored research and are aware of the relevant issues.
Additionally both have extensive track records in international publication.
Further, their associated institutions are already very well acquainted with
commercialising research. For example, Prof Marshall has founded a successful
wireless network security start-up which was spun-out from Queen's University.
Queen's was also named the UK Entrepreneur University of the year for 2009.
Professor Fontaine's involvement with NATO will also lead to further development
of the resultant techniques and findings from this research, potentially leading
to project-input to the forthcoming FP8 EU research call.

\subsection{Defence Context}  The interest to defence for this project is clear; within
DSTL's already stated  areas of strategic importance, this research falls into
at least four areas; namely Certification of Autonomous Systems; Shallow Oceanic
Zone and Coastal Environment; Security and Vulnerability; aswell as Human
Operator Control of Unmanned Vehicles.

\subsection{Defence Feedback}  If successful, this work would provide the ability for a
fleet of AUVs to operate over a much wider range of operations for much longer
periods of time, in a much more secure and reliable state.

The self-detection and classification of abnormal behaviour within a fleet in
the proposed distributed manner opens up the potential of a new range of secure
and self-learning distributed systems, with potential applications both below
and above the seas.

\section{Cheers}  Thank you for your time, and I'd be happy to answer any questions
you might have.

\end{document}