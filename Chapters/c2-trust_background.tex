
\def\ChapterTitle{Background on Trust and its Applications to MANETs} % Write in your own chapter title

\ifx\ifthesis\undefined
	\documentclass[a4paper, 11pt]{article}

% % Special Includes stolen from Thesis.cls
\usepackage{booktabs}
\usepackage{hyperref}
\usepackage{graphicx}
\usepackage{epstopdf}
\usepackage{subcaption}
\usepackage{rotating}
\usepackage{listings}
\usepackage{lstpatch}
\renewcommand{\arraystretch}{1.3}

% % % % % % % % % % % % % % % %

% PACKAGES
\usepackage[square, numbers, comma, sort&compress]{natbib}  % Use the "Natbib" style for the references in the Bibliography
\usepackage{verbatim,listings}  % Needed for the "comment" environment to make LaTeX comments
\usepackage{array}  % Needed for the "comment" environment to make LaTeX comments
\usepackage{vector}  % Allows "\bvec{}" and "\buvec{}" for "blackboard" style bold vectors in maths
\usepackage{amsmath,amsfonts,amsthm,color,psfrag,epsf, tabularx, multirow, longtable}
\usepackage{snapshot, todonotes}
% \usepackage{pstricks}
\usepackage{enumerate}
%\usepackage[lined,algonl,boxed]{algorithm2e}
\usepackage[ruled,linesnumbered,vlined]{algorithm2e}
\usepackage{float}
\usepackage{epigraph} % epigraph
\usepackage{tikz}
\usetikzlibrary{shapes.geometric, arrows}

\usepackage{setspace}
\doublespacing
% or:
%\onehalfspacing

% SETUP
\hypersetup{urlcolor=blue, colorlinks=false}  % Colours hyperlinks in blue, but this can be distracting if there are many links. 
\DeclareGraphicsExtensions{.pdf,.jpeg,.png}
\graphicspath{{../Figures/}{../../../Dropbox/Thesis_Figures/}}  % Location of the graphics files (set up for graphics to be in PDF format)

% NOTE THERES A FUCKUP IN TEX4HT http://tug.org/pipermail/tex4ht/2014q2/000944.html
% NEED TO MANUALLY CHANGE \def\pgfsys@svg@newline{{?nl}}
\ifdefined\HCode
\usepackage[compatibility=false]{caption}
\def\pgfsysdriver{pgfsys-tex4ht.def}
\else
\usepackage[]{caption}
\fi

% \SpecialCoor
\def\subsum{\mathit{\Sigma}}

%opening
\title{\ChapterTitle}
\author{Andrew Bolster}

\begin{document}

\maketitle

\else
	\chapter{\ChapterTitle}
	\label{Chapter\thechapter}
	\lhead{Chapter \thechapter. \emph{\nameref{Chapter\thechapter}}} % Write in your own chapter title to set the page header
\fi

\section{Trust}

In human trust relationships it is recognized that there can be several perspectives of Trust for example organizational, sociological, interpersonal, psychological and neurological \cite{Lee2004}.
For the purposes of this work we define two perspectives on trust for autonomous systems: Design and Operational. These are summarised as follows:

\begin{itemize}
  \item \emph{Design Trust}; When an autonomous system is under development a level of Trust is established in it through the manner in which it has been designed and tested.
    This is the same as conventional systems.
    The difference with systems that have high-levels of autonomy is that they are designed to behave adaptively to dynamic environments that are difficult to fully predict prior to operational deployment.
    For example, in a navigation system it is difficult to predict the dynamic environment it will need to adapt to.
    So Trust needs to be developed that th:e design and test of such systems are sufficient to predict that operation will be, if not optimal, at least satisfactory.

  \item \emph{Operational Trust}; Trust at runtime or in-situ that both the individual nodes within a system are operating as expected\footnote{Operational Trust is functionally derived from, but distinct from Design Trust}; and that the interfaces between the operator and the system are as expected.
    This latter aspect covers issues such as physical/wireless links and interpretation of data at each end of such a communication link.
\end{itemize}

In addition to the two perspectives of trust identified, it is necessary to define and classify Operational Trust into two distinct but related sections, which we define as being:
\begin{itemize}
  \item \emph{Hard Trust} or technical trust, being the quantitative measurement and communication of the expectation of an actor performing a certain task, based on historic performance and through consensus building within a networked system.
    Can be thought of as a de-risking strategy to measure and monitor the ability of a system, or another actor within a system, to perform a task unsupervised.
  \item \emph{Soft Trust} or common trust, being the qualitative assessment of the ability of an actor to perform a task or operation consistently and reliably based on social or experiential factors.
    This is the ‘natural’ form of trust and is the main motivational driver for the human-factors trust discussion.
    Can be rephrased as the level of confidence an operator has in an actor to perform a task unsupervised.
\end{itemize} 

It is already clear that these two definitions are extremely close in their construction, but represent fundamentally different approaches to trust, one coming from a sociological perspective of person-to-person and person-to-group relationships from day to day life, and the other coming from a statistical or formal appraisal of an activity by a system.
For the purposes of this work, we are concerned with the analytical establishment of hard trust within a topologically dynamic network of autonomous actors.


\section{Trust in MANETs}

As mobile ad-hoc networks (MANETs) grow beyond the terrestrial arena, their operation and the protocols designed around them must be reviewed to assess their suitability to different communications environments, ensuring their continued security, reliability, and performance.

Trust Management Frameworks (TMFs) provide information to assist the estimation of future states and actions of nodes within networks.
This information is used to optimize the performance of a network against malicious, selfish, or defective misbehaviour by one or more nodes.
Previous research has established the advantages of implementing TMFs in 802.11 based MANETs, particularly in terms of preventing selfish operation in collaborative systems \cite{Li2007}, and maintaining throughput in the presence of malicious actors \cite{Buchegger2002}

Most current TMFs use a single type of observed action to derive trust values, i.e. successfully forwarded packets. These observations then inform future decisions of individual nodes, for example, route selection \cite{Li2008}.

Recent work has demonstrated use of a number of metrics to form a ``vector'' of trust.
The Multi-parameter Trust Framework for MANETs (MTFM)\cite{Guo11}, uses a range of physical metrics beyond packet delivery/loss rate (PLR) to form a vector of trust.
This vectorized trust allows a system to detect and identify the tactics being used to undermine or subvert trust.
To date this work has been limited to terrestrial, RF based networks, however as autonomous underwater vehicles (AUVs) become more capable, and economical, they are being used in many applications requiring trust.
These applications are using the collective behaviour of teams or fleets of these AUVs to accomplish tasks \cite{Caiti2011}.
With this use being increasingly isolated from stable communications networks, the establishment of trust between nodes is essential for the reliability and stability of such teams.
As such, the use of trust methods developed in the terrestrial MANET space must be re-appraised for application within the challenging underwater communications channel.

The distributed and dynamic nature of MANETs mean that it is difficult to maintain a trusted third party (TTP) or evidence based trust system such as Certificate Authorities (CA) or Public Key Infrastructure (PKI).
Distributed trust management frameworks aim to detect, identify, and mitigate the impacts of malicious actors by distributing per-node assessments and opinions to collectively self-police behaviour.
Various models and algorithms for describing trust and developing trust management in distributed systems, P2P communities or wireless networks have been considered.
Taking some examples;

\begin{itemize}
  \item \emph{The Objective Trust Management Framework} takes a Bayesian Beta function to model per-link Packet Loss Rate (PLR) over time, combining ``Trust'' and ``Confidence of Assessment'' into a single value \cite{Li2008}.
    OTMF however does not appropriately combat multi-node-collusion in the network \cite{Cho2011}.
  \item \emph{Trust-based Secure Routing}\cite{Moe2008a} demonstrated an extension to Dynamic Source Routing (DSR), incorporating a Hidden Markov Model of next-hop network, reducing the efficacy of Byzantine attacks such as black-hole routing.
  \item \emph{CONFIDANT}\cite{Buchegger2002} presented an approach using a probabilistic estimation of PLR, similar to OTMF, also introducing a topology weighting scheme that also weighted trust assessments based on historical experience of the reporter.
  \item \emph{Fuzzy Trust-Based Filtering}; \cite{Luo2008} presents the use of Fuzzy Inference to adapt to malicious recommenders using conditional similarity to classify performance with overlapping Fuzzy Set Membership, filtering assessments across a network.
\end{itemize}

These TMFs can be generalised as single-value probabilistic estimation, based around using a binary input state and generating an probabilistic estimation of the future states of that input. This expectation value is $\text{beta}(p|\alpha,\beta) \to E(p) = \frac{\alpha}{\alpha+\beta}$ where $\alpha$ and $\beta$ represent the number of successful and unsuccessful interactions respectively.

These single metric TMFs provide malicious actors with a significant advantage if their activity is undetectable by that metric.
In the case where the attacker can subvert the TMF, the metric under assessment by that TMF does not cover the threat mounted by the attacker.
In turn, this causes a super-linearly negative effect in the efficiency of the network, as the TMF is assumed to have reduced the possible set of attacks when it has actually made it more advantageous to attack a different part of the networks operation.
An example of such a situation would be in a TMF focused on PLR where an attacker selectively delays packets going through it, reducing overall throughput but not dropping any packets.
Such behaviour would not be detected by the TMF.

There are also situations where the observed metrics will include significant noise and occur at irregular, sparse, intervals.
Conventional approaches such as probabilistic estimation do not produce trust values that reflect the underlying reality and context of the metrics available, as they require a-priori assumption that the trust value under exploration has an expected distribution, that distribution is mono-modal, and the input metrics are binary.
In scenarios with variable, sparse, noisy metrics, estimating the distribution is difficult to accomplish a-priori.


\subsection{Design Considerations}

There are five topics that are important to address in any MANETs trust model \cite{Kamvar2003}:

\begin{enumerate}
  \item The trust model should be without infrastructure. Because the network routing infrastructure is formed in an ad-hoc fashion, the trust management can not depend on, e.g., a trusted third party (TTP). There is no public key infrastructure (PKI), where some center nodes monitor the network, and publish illegal nodes periodically. In a MANET, there are no certification authorities (CA) or registration authorities (RA) with elevated privileges etc.
  \item The trust model should be anonymous because of the anonymity of mobile nodes in MANETs.
  \item The trust model should be robust. That is, it can be robust to all kinds of unfriendly attacks and the network itself should not be susceptible to attacks by unfriendly nodes. Moreover, in the presence of malicious nodes, they attempt to subvert the model in order to get the unfairly good trust value.
  \item The trust model should have minimal control overhead in accordance with computation, storage, and complexity.
  \item The trust model should be self-organized. MANETs are characterized to have dynamic, random, rapidly changing and multi-hop topologies composed of relatively bandwidth-constrained
\end{enumerate}

Trust is the level of confidence one agent has in another to perform a given action on request or in a certain context. Trust in the autonomous or semi-autonomous realm is the ability of a system to establish and maintain confidence in itself or another systems' operations. 
Managing this trust can be used to predict and reason on the future interactions between entities in a system, such as an autonomous mobile ad-hoc network (MANET).

The distributed and dynamic nature of MANETs mean that it is difficult to maintain a trusted third party (TTP) or evidence based trust system such as Certificate Authorities or using Public Key Infrastructures (PKI).
Therefore, a distributed, collaborative system must be applied to these networks.
Such distributed trust management frameworks aim to detect, identify, and mitigate the impacts of malicious actors by distributing per-node assessments and opinions to collectively self-police behaviour.

\subsection{Current Trust Management Frameworks}

Various models and algorithms for describing trust and developing trust management in distributed systems, P2P communities or wireless networks have been considered.
Taking some examples;

\begin{itemize}
  \item \emph{The Objective Trust Management Framework} takes a Bayesian approach and introduces the idea of applying a Beta function to changes in the per-link Packet Loss Rate (PLR) over time, combining ``Trust'' and ``Confidence of Assessment'' into a single value \cite{Li2008}.
    OTMF however does not appropriately combat multi-node-collusion in the network \cite{Cho2011}.
  \item \emph{Trust-based Secure Routing \cite{Moe2008a}} demonstrated an extension to Dynamic Source Routing (DSR), incorporating a Hidden Markov Model of the wider ad-hoc network, reducing the efficacy of Byzantine attacks, particularly black-hole attacks but is limited by focusing on single metric observation (PLR)\cite{Cho2011}.
  \item \emph{CONFIDANT}; \cite{Buchegger2002} presented an approach using a probabilistic estimation of normal observations, similar to OTMF. They also introduced a greedy topology weighting scheme that internally weighted incoming trust assessments based on historical experience of the reporter.
  \item \emph{Fuzzy Trust-Based Filtering}; \cite{Luo2008} presented a method using Fuzzy Inference to cope with imperfect or malicious recommendation based on a probabilistic estimation of performance using conditional similarity to classify performance using overlapping Fuzzy Set Membership functions to collaboratively filter reputations across a network.
\end{itemize}

OTMF, CONFIDANT, and Fuzzy Trust-Based Filtering can be generalised as single-value probabilistic estimation, based around a Bayesian idea of taking a binary input state and generating an idealised Beta Distribution (\ref{eq:beta}) of the future states of that input generated through an expectation value based on interactions (\ref{eq:beta_e}).
\begin{align}
  \label{eq:beta}
  \text{beta}(p|\alpha,\beta) = \frac{\Gamma(\alpha + \beta)}{\Gamma(\alpha)\Gamma(\beta)}p^{\alpha-1},\text{ where } 0 \leq p \leq 1; \alpha,\beta > 0\\
  \label{eq:beta_e}
  E(p) = \frac{\alpha}{\alpha + \beta}
\end{align}

Where $\alpha$ and $\beta$ represent the number of successful and unsuccessful interactions respectively.

These single metric TMFs provide malicious actors with a significant advantage if their activity is undetectable by that one assessed metric, especially if the attacker knows the metric in advance.

The objective of operating a TMF is to increase the confidence in, and efficiency of, a system by reducing the amount of undetectable negative operations an attacker can perform.
In the case where the attacker can subvert the TMF, the metric under assessment by that TMF does not cover the threat mounted by the attacker.
In turn, this causes a super-linearly negative effect in the efficiency of the network as the TMF is assumed to have reduced the possible set of attacks when in fact it has only made it more advantageous to attack a different aspect of the networks operation.
An example of such a behaviour would be the case in a TMF focused on PLR where an attacker selectively delays packets going through it, reducing the over all throughput of one or more virtual network routes.
Such behaviour would not be detected by the TMF.

Many trust systems operate on the basis of establishing closed system models based on noisy or perturbed information inputs, sourced by decentralised agents or nodes, with an aim to collaboratively establishing additional information about the expected states and behaviours of other agents within a system.\todo{Need to say somewhere that 'agent' and 'node' are used interchangeably in this document}
As such, trust systems can be described as fundamentally uncertain, particularly in the areas or reputation establishment and trust chaining.\cite{someone}.
Adding to this state the highly dynamic features of many aspects of trust theory applications (Ad Hoc Networks, Online Markets, etc.), we can generalise the sources of incomplete information from a single nodes perspective as being part of 4 cases.

\begin{itemize}
  \item Information on the system's boundary is incomplete
  \item Information about the range of system behaviours is incomplete
  \item Information about the structure of the system is incomplete or out of date
  \item Information about observed parameters (metrics) is incomplete or out of date.
\end{itemize}

These cases of incompleteness of information are closely mirrored by those for which grey theory was originally posited as a form of system modeling, putting information incompleteness at the centre of the assessment. 
While some work \cite{Guo11} has been done to apply grey theory to a trust context, it has not been fully explored.
Guo applies grey analysis to generate a ``trust vector'' from the grey whitenisation of independent or near-independent metrics. 
In this paper we demonstrate a methodology that applies Grey Sequence operations and Grey Generators (conceptually analogous to Sequential Bayesian Filtering``) to provide continuous trust assessment in a sparse, asynchronous metric space across multiple domains of trust.

\subsection{Trust as an incomplete system characteristic}

While application specific trust management frameworks are often based on a very limited space of available metrics, the problem of establishing trust in dynamical systems such as social, economic or autonomous systems have the opportunity to tap in to a wide range of potential metric spaces. 
Taking the example of Mobile Ad-Hoc Networks (MANET), the variable most applied to the assessment of trust is the packet error rate, or more generally, the number of successful and unsuccessful interactions between two agents within a system.
However, a wealth of other information is available within this example; for instance the delay in communications from one node to another; the total throughput of particular network links; and in the case of wireless networks, the strength of received signals.
Looking beyond the communications domain, within such a MANET, information is also usually available regarding the physical domain of a network; the relative positioning and motions of nodes within a network can also be used to inform the generation of trust assessments.


%Some groups use a reverse notation for trust, whereby $T_{1,0}$ would represent the trustworthiness of $n_1$ according to $n_0$; however, given the subjective nature of trustworthiness, it is more relatable to describe $T_{1,0}$ as ``The Trustworthiness of $n_1$ from the perspective of $n_0$. rather than an absolute statement on $n_1$'s trustworthiness.


Table~\ref{tab:uncertainty} provides a qualitative summary of the differences in use and application between Fuzzy, Probabilistic and Grey Systems of managing uncertainty.


\begin{table}[h]
  \caption{Comparison between selected methods of characterising uncertainty, adapted from \cite{Guo11} \cite{Liu2006} \cite{Ng1994} \cite{Wang2006} }
  \label{tab:uncertainty}
  \begin{center}
    \setlength{\tabcolsep}{8pt}
    \begin{tabular}{l|ccc}
      \toprule
        & Fuzzy Math & Bayesian Estimation & Grey Systems \\
      \midrule
      Objects & Cognitive Uncertainty & Distribution Refinement & Poor Information \\
      Set Style & Fuzzy Sets & Cantor Sets & Grey Hazy Sets \\
%      Methodology & Discerning Affiliation & Probability Distribution & Information Coverage \\
      Processes & Marginal Sampling & Frequency Distribution & Sequence Generation \\
      Requirement & Known Membership & Beta Distribution & Any Distribution \\
      Emphasis & Extension & Intension & Intension \\
%      Objective & Cognitive Expression & Stabilisation & Realism \\
      Characteristics & Experience & Large Samples & Small Samples\\
     \bottomrule
    \end{tabular}
    \setlength{\tabcolsep}{6pt}
  \end{center}
\end{table}

\section{Grey System Theory and Grey Trust Assessment}

\subsection{Grey numbers, operators and terminology}

Grey numbers are used to represent values where their discrete value is unknown, where that number may take its possible value within an interval of potential values, generally written using the symbol $\oplus$.
Taking $a$ and $b$ as the lower and upper bounds of the grey interval respectively, such that $\oplus \in [a,b] | a < b$ 
The ``field'' of $\oplus$ is the value space $[a,b]$.
There are several classifications of grey numbers based on the relationships between these bounds.\todo{don't think classification is the right word here}

Black and White numbers are the extremes of this classification; such that $\dot\oplus \in [-\infty, +\infty]$ and $\mathring\oplus \in [x, x] | x \in \mathbb{R}$ or $\oplus(x)$
It is clear that white numbers such as $\mathring\oplus$ have a field of zero while black numbers have an infinite field.

Grey numbers may represent partial knowledge about a system or metric, and as such can represent half-open concepts, by only defining a single bound; for example $\underline\oplus = \oplus(\underline x ) \in [x, +\infty]$ and $\overline\oplus = \oplus(\overline x) \in [-\infty, x]$.

Primary operations within this number system are as follows;

\begin{subequations}
\begin{align}
  \oplus_1 + \oplus_2      &\in [a_1+a_2,b_1+b_2] \label{eq:grey_add}\\
           -\oplus         &\in [-b,-a] \label{eq:grey_neg} \\
  \oplus_1 - \oplus_2      &= \oplus_1+(-\oplus) \label{eq:grey_sub}\\
  \oplus_1 \times \oplus_2 &\in \begin{aligned}[t]
    &[\min(a_1 a_2, a_1 b_2, b_1 a_2, b_2 a_2), \\
    & \max(a_1 a_2, a_1 b_2, b_1 a_2, b_2 a_2)]
  \end{aligned} \label{grey_mult}\\
  \oplus^{-1} &\in [b^{-1}, a^{-1}] \label{eq:grey_inv}\\
  \oplus_1 / \oplus_2 & = \oplus_1 \times \oplus_2^{-1} \label{grey_mult} \\
  \oplus \times k &\in [ka,kb] \label{eq:grey_times_scalar}\\
  \oplus^k &\in [a^k, b^k] \label{eq:grey_exp}
\end{align}
\end{subequations}

where $k$ is a scalar quantity.
  
\subsection{Whitenisation and the Grey Core}

The characterisation of grey numbers is based on the encapsulation of information in a grey system in terms of the grey numbers core ($\hat\oplus$) and it's degree of greyness ($g^\circ$).
If the distribution of a grey number field is unknown and continuous, $\hat\oplus = \frac{a + b}{2}$.

Non-essential grey numbers are those that can be represented by a white number obtained either through experience or particular method. \cite{Liu2011}
This white hissed value is represented by $\tilde\oplus$ or $\oplus(x)$ to represent grey numbers with $x$ as their whitenisation.
In some cases depending on the context of application, particular gray numbers may temporarily have no reasonable whitenisation value (for instance, a black number). Such numbers are said to be Essential grey numbers.

\subsection{Grey Sequence Buffers and Generators}

\todo{eqs of sequence buffers and partial derivs}

Given a fully populated value space, sequence buffer operations are used to provide abstractions over the dataspace.
These abstractions can be \emph{weakening} or \emph{strengthening}.
In the weakening case, these operations perform a level of smoothing on the volatility of a given input space, and strengthening buffers serve to highlight and 

A powerful tool in grey system theory is the use of grey incidence factors, comparing the ``likeness'' of one value against a cohort of values.
This usefulness applies particularly well in the case of multi-agent trust networks, where the aim is to detect and identify malicious or maladaptive behaviour, rather than an absolute assessment of ``trustworthiness''.

\subsection{Grey Trust}

Grey Theory performs cohort based normalization of metrics at runtime. 
This creates a more stable contextual assessment of trust, providing a ``grade'' of trust compared to other observed nodes in that interval, while maintaining the ability to reduce trust values down to a stable assessment range for decision support without requiring every environment entered into to be characterised.
Grey assessments are relative in both fairly and unfairly operating networks.
Nodes will receive mid-range trust assessments if there are no malicious actors as there is no-one else ``bad'' to compare against.

Guo\cite{Guo11} demonstrated the ability of Grey Relational Analysis (GRA)\cite{Zuo1995} to normalise and combine disparate traits of a communications link such as instantaneous throughput, received signal strength, etc. into a Grey Relational Coefficient, or a ``trust vector''.

In the case of the terrestrial communications network used in \cite{Guo11}, the observed metric set $X = {x_1,\dots,x_M}$ representing the measurements taken by each node of its neighbours at least interval, is defined as $X=[$packet loss rate, signal strength, data rate, delay, throughput$]$.
The trust vector is given as
%
\begin{align}
  \label{eq:grc}
  \theta_{k,j}^t = \frac{\min_k|a_{k,j}^t - g_j^t| + \rho \max_k|a_{k,j}^t-g_j^t|}{|a_{k,j}^t-g_j^t| + \rho \max_k|a_{k,j}^t-g_j^t|} \\
  \phi_{k,j}^t = \frac{\min_k|a_{k,j}^t - b_j^t| + \rho \max_k|a_{k,j}^t-b_j^t|}{|a_{k,j}^t-b_j^t| + \rho \max_k|a_{k,j}^t-b_j^t|} \notag 
\end{align}
%
where $a_{k,j}^t$ is the value of a observed metric $x_j$ for a given node $k$ at time $t$, $\rho$ is a distinguishing coefficient set to $0.5$, $g$ and $b$ are respectively the '``good'' and ``bad'' reference metric sequences from $\{a_{k,j}^t k=1,2\dots K\}$, e.g. $g_j=\max_k({a_{k,j}^t})$,  $b_j=\min_k({a_{k,j}^t})$ (where each metric is selected to be monotonically positive for trust assessment, e.g. higher throughput is always better). 

Weighting can be applied before generating a scalar value which allows the identification and classification of untrustworthy behaviours.

%
\begin{equation}
  \label{eq:metric_weighting}
  [\theta_k^t, \phi_k^t] = \left[\sum_{j=0}^M h_j \theta_{k,j}^t,\sum_{j=0}^M h_j \phi_{k,j}^t \right]
\end{equation}
Where $H=[h_0\dots h_M]$ is a metric weighting vector such that $\sum h_j = 1$, and in the basic case, $H=[\frac{1}{M},\frac{1}{M}\dots\frac{1}{M}]$ to treat all metrics evenly.
$\theta$ and $\phi$ are then scaled to $[0,1]$ using the mapping $y = 1.5 x - 0.5$.
The $[\theta,\phi]$ values are reduced into a scalar trust value by $T_k^t = ({1+{(\phi_k^t)^2}/{(\theta_k^t)^2}})^{-1}$.
This trust value minimises the uncertainties of belonging to either best ($g$) or worst ($b$) sequences in \eqref{eq:grc}.

MTFM combines this GRA with a topology-aware weighting scheme\eqref{eq:networkeffects} and a fuzzy whitenization model\eqref{eq:whitenization}. There are three classes of topological trust relationship used; Direct, Recommendation, and Indirect.
Where an observing node, $n_i$, assesses the trust of another, target, node, $n_j$; the Direct relationship is $n_i$'s own observations $n_j$'s behaviour.
In the Recommendation case, a node $n_k$, which shares Direct relationships with both $n_i$ and $n_j$, gives its assessment of $n_j$ to $n_i$.
The Indirect case, similar to the Recommendation case, the recommender $n_k$, does not have a direct link with the observer $n_i$ but $n_k$ has a Direct link with the target node, $n_j$.
These relationships give us node sets, $N_R$ and $N_I$ containing the nodes that have recommendation or indirect, relationships to the observing node respectively.
%
\begin{align}
  \label{eq:networkeffects}
  T_{i,j}^{MTFM}=\frac{1}{2} \cdot \max_s\{f_s(T_{i,j})\} T_{i,j}+&\frac{1}{2} \frac{2|N_R| }{2|N_R| + |N_I|}\sum_{n \in N_R} \max_s\{f_s(T_{i,n})\} T_{i,n}\\ \notag
  +&\frac{1}{2} \frac{|N_I| }{2|N_R| + |N_I|}\sum_{n \in N_I} \max_s\{f_s(T_{i,n})\} T_{i,n} 
\end{align}
 Where $T_{i,n}$ is the subjective trust assessment of $n_i$ by $n_n$, and $f_s = [ f_1,f_2, f_3]$ given as:
\begin{align}
  \label{eq:whitenization}
  f_1(x)&= -x+1\notag\\
  f_2(x)&= 
  \begin{cases}
    2x & \text{if }x\leq 0.5\\
    -2x+2 & \text{if }x>0.5
  \end{cases}\\
  f_3(x)&= x\notag
\end{align}














\subsection{PROSE: Whats the point}

Grey System Theory, by it's own authors admission, hasn't taken root in it's originally intended area of system modelling \cite{Liu11}.
However, given it's tentative application to MANET trust, taking a Grey approach on a per metric benefit has qualitative benefits that require investigation; the algebraic approach to uncertainty and the application of ``essential and non essential greyness'', whiteisation, and particularly grey buffer sequencing allow for the opportunity to generate continuous trust assessments from multiple domains asynchronously;

For a given metric set $X$ such that $X = {x_1,\dots,x_M}$ representing the $M$ different types of measurement generated by an observer. If these metrics are not synchronised, for instance if they are interrupt driven such as communications-based observations, generating more abstract measurements requires inherent assumptions about ``how to accumulate the data while you wait''. For instance, in \cite{Bolster2015}, we demonstrated a periodic trust assessment framework for autonomous marine environments, in such an environment, to establish useful, generalised, data, it was necessary to wait for a relatively long time to accumulate enough data to make assessments.
However, this left many 'smells'; data was being left in-buffer for a long time before being used to make decisions, and by the time the data was collated and processed, it could be wildly different from the reality. Further, while some periods could be extremely sparse or even empty, others could be extremely busy with many records having to be averaged down to provide a 'single period' response. 
Therefore, the implementation of a suitable sequence buffer version of the framework would be beneficial.

Such a sequence buffer framework would involve a tracking predictor that would provide best-guess estimates of an interpolated value for a metric between value updates, and a back-propagation algorithm to retroactively update historical assessments of that metrics so as to better inform any abstracted trust value predictor.

I had initially thought that such a back-propogator would be a total mess as I'd imagined that significant-model-breaking would potetially indicate untrustworthy behaviour, but this is stupid since the per-metric-model has the least information of anyone and is simply there to provide better intermediate values and has no / limited direct impact on the overall trust behaviour. 

This backpropogation will probably be a pain to implement as it'd require a retroactive reassessment of trust and could get really messy if it was interrupt driven, but it's better not to prematurly optimise.


\ifx\ifthesis\undefined
	%% ----------------------------------------------------------------
\label{Bibliography}
% \bibliographystyle{amsplain}
%\bibliographystyle{unsrtnat}  % Use the "unsrtnat" BibTeX style for formatting the Bibliography
\bibliographystyle{alpha}
\bibliography{../Thesis}  % The references (bibliography) information are stored in the file named "Thesis.bib"

\end{document}  % The End
%% ----------------------------------------------------------------
\else
\fi
