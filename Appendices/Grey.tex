% Appendix C - Grey Theory

\chapter{Grey System Theory and Grey Trust Assessmen}
\label{apx:grey}
\lhead{Appendix \thechapter. \emph{Grey System Theory and Grey Trust Assessment}}

\section{Grey numbers, operators and terminology}
Grey numbers are used to represent values where their discrete value is unknown, where that number may take its possible value within an interval of potential values, generally written using the symbol $\oplus$.
Taking $a$ and $b$ as the lower and upper bounds of the grey interval respectively, such that $\oplus \in [a,b] | a < b$ 
The ``field'' of $\oplus$ is the value space $[a,b]$.
There are several classifications of grey numbers based on the relationships between these bounds.\todo{don't think classification is the right word here}
Black and White numbers are the extremes of this classification; such that $\dot\oplus \in [-\infty, +\infty]$ and $\mathring\oplus \in [x, x] | x \in \mathbb{R}$ or $\oplus(x)$
It is clear that white numbers such as $\mathring\oplus$ have a field of zero while black numbers have an infinite field.

Grey numbers may represent partial knowledge about a system or metric, and as such can represent half-open concepts, by only defining a single bound; for example $\underline\oplus = \oplus(\underline x ) \in [x, +\infty]$ and $\overline\oplus = \oplus(\overline x) \in [-\infty, x]$.

Primary operations within this number system are as follows;
\begin{subequations}
  \begin{align}
    \oplus_1 + \oplus_2      &\in [a_1+a_2,b_1+b_2] \label{eq:grey_add}\\
    -\oplus         &\in [-b,-a] \label{eq:grey_neg} \\
    \oplus_1 - \oplus_2      &= \oplus_1+(-\oplus) \label{eq:grey_sub}\\
    \oplus_1 \times \oplus_2 &\in \begin{aligned}[t]
      &[\min(a_1 a_2, a_1 b_2, b_1 a_2, b_2 a_2), \\
      & \max(a_1 a_2, a_1 b_2, b_1 a_2, b_2 a_2)]
    \end{aligned} \label{eq:grey_mult}\\
    \oplus^{-1} &\in [b^{-1}, a^{-1}] \label{eq:grey_inv}\\
    \oplus_1 / \oplus_2 & = \oplus_1 \times \oplus_2^{-1} \label{eq:grey_div} \\
    \oplus \times k &\in [ka,kb] \label{eq:grey_times_scalar}\\
    \oplus^k &\in [a^k, b^k] \label{eq:grey_exp}
  \end{align}
\end{subequations}
where $k$ is a scalar quantity.

\section{Whitenisation and the Grey Core}
The characterisation of grey numbers is based on the encapsulation of information in a grey system in terms of the grey numbers core ($\hat\oplus$) and it's degree of greyness ($g^\circ$).
If the distribution of a grey number field is unknown and continuous, $\hat\oplus = \frac{a + b}{2}$.

Non-essential grey numbers are those that can be represented by a white number obtained either through experience or particular method. \cite{Liu2011}
This white value is represented by $\tilde\oplus$ or $\oplus(x)$ to represent grey numbers with $x$ as their whitenisation.
In some cases depending on the context of application, particular gray numbers may temporarily have no reasonable whitenisation value (for instance, a black number).
Such numbers are said to be Essential grey numbers.

\section{Grey Sequence Buffers and Generators}
\todo{eqs of sequence buffers and partial derivs}
Given a fully populated value space, sequence buffer operations are used to provide abstractions over the dataspace.
These abstractions can be \emph{weakening} or \emph{strengthening}.
In the weakening case, these operations perform a level of smoothing on the volatility of a given input space, and strengthening buffers serve to highlight and 
A powerful tool in grey system theory is the use of grey incidence factors, comparing the ``likeness'' of one value against a cohort of values.
This usefulness applies particularly well in the case of multi-agent trust networks, where the aim is to detect and identify malicious or maladaptive behaviour, rather than an absolute assessment of ``trustworthiness''.

\section{Grey Trust}
Grey Theory performs cohort based normalization of metrics at runtime.
This creates a more stable contextual assessment of trust, providing a ``grade'' of trust compared to other observed entities in that interval, while maintaining the ability to reduce trust values to a stable assessment range for decision support without requiring every environment entered into to be characterised.
Grey assessments are relative in both fairly and unfairly operating cohorts.
Entities will receive mid-range trust assessments if there are no malicious actors as there is no-one else ``bad'' to compare against.

Guo\cite{Guo11} demonstrated the ability of Grey Relational Analysis (GRA)\cite{Zuo1995} to normalise and combine disparate traits of a communications link such as instantaneous throughput, received signal strength, etc.\ into a Grey Relational Coefficient, or a ``trust vector''.

In \cite{Guo11}, the observed metric set $X = {x_1,\dots,x_M}$ representing the measurements taken by each node of its neighbours at least interval, is defined as $X=[$packet loss rate, signal strength, data rate, delay, throughput$]$.
The trust vector is given as
%
\begin{align}
  %\label{eq:grc}
  \theta_{k,j}^t = \frac{\min_k|a_{k,j}^t - g_j^t| + \rho \max_k|a_{k,j}^t-g_j^t|}{|a_{k,j}^t-g_j^t| + \rho \max_k|a_{k,j}^t-g_j^t|} \\
  \phi_{k,j}^t = \frac{\min_k|a_{k,j}^t - b_j^t| + \rho \max_k|a_{k,j}^t-b_j^t|}{|a_{k,j}^t-b_j^t| + \rho \max_k|a_{k,j}^t-b_j^t|} \notag 
\end{align}
%
where $a_{k,j}^t$ is the value of a observed metric $x_j$ for a given node $k$ at time $t$, $\rho$ is a distinguishing coefficient set to $0.5$, $g$ and $b$ are respectively the '``good'' and ``bad'' reference metric sequences from $\{a_{k,j}^t k=1,2\dots K\}$, e.g.\ $g_j=\max_k({a_{k,j}^t})$,  $b_j=\min_k({a_{k,j}^t})$ (where each metric is selected to be monotonically positive for trust assessment, e.g.\ higher throughput is always better).

Weighting can be applied before generating a scalar value which allows the identification and classification of untrustworthy behaviours.

%
\begin{equation}
  %\label{eq:metric_weighting}
  [\theta_k^t, \phi_k^t] = \left[\sum_{j=0}^M h_j \theta_{k,j}^t,\sum_{j=0}^M h_j \phi_{k,j}^t \right]
\end{equation}
Where $H=[h_0\dots h_M]$ is a metric weighting vector such that $\sum h_j = 1$, and in the basic case, $H=[\frac{1}{M},\frac{1}{M}\dots\frac{1}{M}]$ to treat all metrics evenly.
$\theta$ and $\phi$ are then scaled to $[0,1]$ using the mapping $y = 1.5 x - 0.5$.
The $[\theta,\phi]$ values are reduced into a scalar trust value by $T_k^t = ({1+{(\phi_k^t)^2}/{(\theta_k^t)^2}})^{-1}$.
This trust value minimises the uncertainties of belonging to either best ($g$) or worst ($b$) sequences in \eqref{eq:grc}.

\acrshort{mtfm} combines this GRA with a topology-aware weighting scheme\eqref{eq:networkeffects} and a fuzzy whitenization model\eqref{eq:whitenization}.
There are three classes of topological trust relationship used; Direct, Recommendation, and Indirect.
Where an observing node, $n_i$, assesses the trust of another, target, node, $n_j$; the Direct relationship is $n_i$'s own observations $n_j$'s behaviour.
In the Recommendation case, a node $n_k$, which shares Direct relationships with both $n_i$ and $n_j$, gives its assessment of $n_j$ to $n_i$.
The Indirect case, similar to the Recommendation case, the recommender $n_k$, does not have a direct link with the observer $n_i$ but $n_k$ has a Direct link with the target node, $n_j$.
These relationships give us node sets, $N_R$ and $N_I$ containing the nodes that have recommendation or indirect, relationships to the observing node respectively.
%
\begin{align}
  %\label{eq:networkeffects}
  T_{i,j}^{\acrshort{mtfm}}=\frac{1}{2} \cdot \max_s\{f_s(T_{i,j})\} T_{i,j}+&\frac{1}{2} \frac{2|N_R| }{2|N_R| + |N_I|}\sum_{n \in N_R} \max_s\{f_s(T_{i,n})\} T_{i,n}\\ \notag
  +&\frac{1}{2} \frac{|N_I| }{2|N_R| + |N_I|}\sum_{n \in N_I} \max_s\{f_s(T_{i,n})\} T_{i,n} 
\end{align}
Where $T_{i,n}$ is the subjective trust assessment of $n_i$ by $n_n$, and $f_s = [ f_1,f_2, f_3]$ given as:
\begin{align}
  %\label{eq:whitenization}
  f_1(x)&= -x+1\notag\\
  f_2(x)&= 
  \begin{cases}
    2x & \text{if }x\leq 0.5\\
    -2x+2 & \text{if }x>0.5
  \end{cases}\\
  f_3(x)&= x\notag
\end{align}

Grey System Theory, by it's own authors admission, hasn't taken root in it's originally intended area of system modelling \cite{Liu2011}.
However, given it's tentative application to \gls{manet} trust, taking a Grey approach on a per metric benefit has qualitative benefits that require investigation; the algebraic approach to uncertainty and the application of ``essential and non essential greyness'', whiteisation, and particularly grey buffer sequencing allow for the opportunity to generate continuous trust assessments from multiple domains asynchronously.

\todo{NORELEASE: Wrap up the grey discussion}

\section{UNEDITED PROSE: Real Time Grey Systems}
\emph{Incoming Train of Consciousness}


For a given metric set $X$ such that $X = {x_1,\dots,x_M}$ representing the $M$ different types of measurement generated by an observer. If these metrics are not synchronised, for instance if they are interrupt driven such as communications-based observations, generating more abstract measurements requires inherent assumptions about ``how to accumulate the data while you wait''. For instance, in \cite{Bolster2015}, we demonstrated a periodic trust assessment framework for autonomous marine environments, in such an environment, to establish useful, generalised, data, it was necessary to wait for a relatively long time to accumulate enough data to make assessments.
However, this left many 'smells'; data was being left in-buffer for a long time before being used to make decisions, and by the time the data was collated and processed, it could be wildly different from the reality. Further, while some periods could be extremely sparse or even empty, others could be extremely busy with many records having to be averaged down to provide a 'single period' response. 
Therefore, the implementation of a suitable sequence buffer version of the framework would be beneficial.

Such a sequence buffer framework would involve a tracking predictor that would provide best-guess estimates of an interpolated value for a metric between value updates, and a back-propagation algorithm to retroactively update historical assessments of that metrics so as to better inform any abstracted trust value predictor.

I had initially thought that such a back-propagator would be a total mess as I'd imagined that significant-model-breaking would potentially indicate untrustworthy behaviour, but this is stupid since the per-metric-model has the least information of anyone and is simply there to provide better intermediate values and has no / limited direct impact on the overall trust behaviour. 

This back propagation will probably be a pain to implement as it'd require a retroactive reassessment of trust and could get really messy if it was interrupt driven, but it's better not to prematurely optimise.


