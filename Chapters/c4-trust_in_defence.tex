% !TeX spellcheck = en_GB
\def\ChapterTitle{Trust in Autonomous Systems of Systems for Maritime Defence Applications}

\ifx\ifthesis\undefined
\documentclass[]{article}



% % Special Includes stolen from Thesis.cls
\usepackage{booktabs}
\usepackage{hyperref}


% % % % % % % % % % % % % % % %5
\usepackage[square, numbers, comma, sort&compress]{natbib}  % Use the "Natbib" style for the references in the Bibliography
\usepackage{verbatim,listings}  % Needed for the "comment" environment to make LaTeX comments
\usepackage{array}  % Needed for the "comment" environment to make LaTeX comments
\usepackage{vector}  % Allows "\bvec{}" and "\buvec{}" for "blackboard" style bold vectors in maths
\hypersetup{urlcolor=blue, colorlinks=false}  % Colours hyperlinks in blue, but this can be distracting if there are many links.

\usepackage{epigraph} % epigraph
%customize: \setlength{\epigraphwidth}{7cm}\setlength{\epigraphrule}{0pt}
%use: \epigraph{text}{reference}



% \documentclass[a4paper,12pt]{article}
%\usepackage[a4paper,vmargin={20mm,20mm},hmargin={20mm,20mm}]{geometry}
\usepackage{amsmath,amsfonts,amsthm,color,psfrag,epsf,graphicx, booktabs, tabularx, multirow, longtable}
\usepackage{snapshot, todo}
% \usepackage{pstricks}
\usepackage{enumerate,caption}
%\usepackage[lined,algonl,boxed]{algorithm2e}
\usepackage[ruled,linesnumbered,vlined]{algorithm2e}
\usepackage{float}
% \SpecialCoor
\def\subsum{\mathit{\Sigma}}


%opening
\title{\ChapterTitle}
\author{Andrew Bolster}

\begin{document}

\maketitle

\else
\chapter{\ChapterTitle}
\label{Chapter\thechapter}
\lhead{Chapter \thechapter. \emph{\nameref{Chapter\thechapter}}} % Write in your own chapter title to set the page header
\fi

With demand for smaller, more decentralised marine survey and monitoring systems, and a drive towards lower per-unit cost, TMFs are going to be increasingly applied to the marine space, as the benefits they present are significant.
Beyond the constraints of the communications environment, knock on pressures are applying in battery capacity, on-board processing, and locomotion.
These pressures simultaneously present opportunities and incentives for malicious or selfish actors to appear to cooperate while not reciprocating, in order to conserve power for instance.
These multiple aspects of potential incentives, trust, and fairness do not directly fall under the scope of single metric trusts discussed above, and this context indicates that a multi-metric approach may be more appropriate.


