
%%%%%%%%%%%%%%%%%%%%%%% file adhocnow15_bolster.tex %%%%%%%%%%%%%%%
%[Ad Hoc Now 2015](http://www.netmode.ntua.gr/adhocnow2015/index.html)
%
%# Dates
%* Deadline: 7/2/15
%* Acceptance: 7/3/15
%* Due: 28/3/15
%* Dates: 29/6/15-2/7/15
%
%# Topics of Interest
%* Access Control 
%* Ad Hoc Networks of Autonomous Intelligent Systems 
%* Algorithmic Issues
%* Analytic Methods and Modeling for Performance Evaluation 
%* Ad Hoc Network Applications and Architectures 
%* Delay-Tolerant Networking
%* Distributed Algorithms for Ad Hoc Networks 
%* Energy Efficiency 
%* Geometric Graphs
%* Location Discovery and Management 
%* Mobility Handling and Utilization 
%* Wireless Mesh Networks
%* Big Data Inspired Data Sensing
%* Mobile Ad Hoc Computing Platforms
%* Systems and Testbeds 
%* Mobile Social Networking 
%* Quality-of-Service 
%* Routing Protocols (Unicast, Multicast, etc.) 
%* Secure Services and Protocols 
%* Sensor Networks 
%* Self-Configuration 
%* Service Discovery 
%* Timing Synchronization 
%* Vehicular Networks 
%* Wireless Internet
%* Processing and Networking Technologies Complexity and Computational Issues
%* Prototype systems and real-world deployment experiences
%
%# Proposed Area of Focus
%Re-present Bellas Chap 4 work with extensions to explain and explore differences between Ad Hoc and Marine.
%
%
%%%%%%%%%%%%%%%%%%%%%%%%%%%%%%%%%%%%%%%%%%%%%%%%%%%%%%%%%%%%%%%%%%%


\documentclass[runningheads,a4paper]{llncs}

\usepackage{amssymb}
\setcounter{tocdepth}{3}
\usepackage{graphicx}

\usepackage{url}
\newcommand{\keywords}[1]{\par\addvspace\baselineskip
\noindent\keywordname\enspace\ignorespaces#1}

\begin{document}

\mainmatter  % start of an individual contribution

% first the title is needed
\title{Trust Framework Operation in Autonomous Marine Communications Environments}

% a short form should be given in case it is too long for the running head
\titlerunning{Trust Framework Operation in Marine Communications Environments}

% the name(s) of the author(s) follow(s) next
%
% NB: Chinese authors should write their first names(s) in front of
% their surnames. This ensures that the names appear correctly in
% the running heads and the author index.
%
\author{Andrew Bolster%
\thanks{Please note that the LNCS Editorial assumes that all authors have used
the western naming convention, with given names preceding surnames. This determines
the structure of the names in the running heads and the author index.}
, Alan Marshall, Ji Guo}
%
\authorrunning{Trust Framework Operation in Marine Communications Environments}
% (feature abused for this document to repeat the title also on left hand pages)

% the affiliations are given next; don't give your e-mail address
% unless you accept that it will be published
\institute{Advanced Networks Research Group, \\Department of Electrical Engineering \& Electronics,\\
University of Liverpool, UK\\
\url{{andrew.bolster,alan.marshall}@liv.ac.uk}\\
\url{http://www.anrg.liv.ac.uk/}}

\toctitle{Lecture Notes in Computer Science}
\tocauthor{Authors' Instructions}
\maketitle


\begin{abstract}
  As mobile ad-hoc networks (MANETs) grow beyond the terrestrial arena, their operation and the protocols designed around them must be reviewed to assess their suitability and optimality in different communications environments to ensure their continues security, reliability, and performance.

  This paper presents an overview of trust assessment within classical ad-hoc networking environments (Terrestrial MANET), including a critique of previous group work in this area utilitsing Fuzzy Sets and Gray Theory to construct a Trust Management Framework (TMF) for decentralised ad-hoc networks. We then present a comparative study on the operation and performance of such trust frameworks between the terrestrial and underwater communications environments.
\keywords{ad-hoc, MANET, trust, marine, underwater}
\end{abstract}


\section{Introduction}

Trust Management Frameworks (TMFs) provide information regarding the estimated future states and operations of nodes within networks. They are used to optimize the performance of a system of systems (i.e. collections of autonomous, semi-autonomous, and/or human systems) in the face of malicious, selfish, or defective behavior by one or more nodes within such a system. 
Previous research has established the advantages of implementing distributed TMFs in terrestrial, 802.11 based mobile ad-hoc networks (MANETs) \cite{Guo11}

Trust Management Frameworks (TMFs) provide information regarding the estimated future states and operations of nodes within networks. They are used to optimize the performance of a system of systems (i.e. collections of autonomous, semi-autonomous, and/or human systems) in the face of malicious, selfish, or defective behavior by one or more nodes within such a system. Previous research has established the potential advantages of implementing distributed TMFs in mobile ad-hoc networks (MANETs) \cite{Li2007}

Current TMFs generally use a single type of observed action to derive trust metrics, e.g. successfully forwarded packets. These historical observations then inform future decisions of individual nodes, for example, the selection of a forward router with the highest previous forwarding success rate \cite{Li2008}.

Recent work has demonstrated the use of a number of metrics together, forming a ‘vector’ of trust; in the case of \cite{Guo2012}, metrics related to inter-node communications. This vectorized trust allows a system to detect anomalous behavior and identify the tactics used to undermine or subvert trust.

\subsection{Trust in MANETs}\label{sec:trustinmanets}

In Human trust relationships it can be seen that there can be several perspectives of Trust for example organizational, sociological, interpersonal, psychological and neurological \cite{Lee2004}. 

For the purposes of this work we can define two perspectives: Design and Operational. These are summarised as follows:

\begin{itemize}
  \item Design Trust. When an autonomous system is under development a level of Trust is established in it through the manner in which it has been designed and tested. This is the same as conventional systems. The difference with systems that have high-levels of autonomy is that they are designed to behave adaptively to dynamic environments that are difficult to fully predict prior to operational deployment. For example, in a navigation system it is difficult to predict the dynamic environment it will need to adapt to. So Trust needs to be developed that the design and test of such systems are sufficient to predict that operational solutions will be, if not optimal, at least satisfactory.

  \item Operational Trust. Effectively, trust that both the individual nodes withing system are operating as expected (which is inevitably tied in with, but distinct from Design Trust); and that the interfaces between the operator and the system are as expected. This latter aspect covers issues such as physical/wireless links and interpretation of data at each end of such a communication link.
\end{itemize}

In addition to the two perspectives of trust identified, it is necessary to define and classify Operational Trust into two distinct but related sections, which we define as being:
\begin{itemize}
  \item Hard Trust or technical trust, being the quantative measurement and communication of the expectation of an actor performing a certain task, based on historic performance and through consensus building within a networked system. Can be thought of as a de-risking strategy to measure the ability of a system to perform a task unsupervised.
  \item Soft Trust or common trust, being the qualitative assessment of the ability of an actor to perform a task or operation consistently and reliably based on social or experiential factors. This is the ‘natural’ form of trust and is the main motivational driver for the human-factors trust discussion. Can be rephrased as the level of confidence in an actor to perform a task unsupervised.
\end{itemize} 

It is already clear that these two definitions are extremely close in their construction, but represent fundamentally different approaches to trust, one coming from a sociological perspective of person-to-person and person-to-group relationships from day to day life, and the other coming from a statistical or formal appraisal of an activity by a system.

\subsection{Existing Trust Management Frameworks}

Recently, various models and algorithms for describing trust and developing trust management in distributed systems, P2P communities or wireless networks have been considered.

\begin{itemize}
  \item \emph{The Objective Trust Management Framework} takes a Bayesian network approach and introduces the idea of applying a Beta function as an encapsulation method, combining ``Trust'' and ``Confidence of Assessment'' into a single value (Li et al. 2008). OTMF however does not appropriately combat multi-node-collusion in the network (Cho, Swarmi and Chen 2011). 

  \item \emph{Trust-based Secure Routing (Moe, Helvik and Knapskog 2008)} demonstrated an extension to Dynamic Source Routing (DSR), incorporating a Hidden Markov Model of the wider ad-hoc network, reducing the efficacy of Byzantine attacks, particularly black-hole attacks but, along with many more TMFs surveyed in (Cho, Swarmi and Chen 2011), falls under the same limitation of focusing on single metric observation. 
\end{itemize}

These single metric TMFs provide malicious actors with a significant advantage if their activity is undetectable by that one assessed metric, especially if the attacker knows the metric in advance. 
The objective of operating a TMF is to increase the confidence in, and efficiency of, a system by reducing the amount of undetectable negative operations an attacker can perform. 
This space of potential attacks can be described as the ‘Threat Surface’. 
In the case where the attacker can subvert the TMF, the metric under assessment by that TMF does not cover the threat mounted by the attacker. 
In turn, this causes a super-linearly negative effect in the efficiency of the network. 
The TMF is assumed to have reduced the threat surface when in fact it has simply made it more advantageous to attack a different part of it.  
(Haung, Hong and Gerla 2010) also raised the need for a more expanded view of trust but did so with a domain-partitioning approach rather than combining trust assessments from multiple domains within networks. 

Guo ('we'?) demonstrated the ability of Grey Relational Analysis (GRA) to normalize and operationally combine disparate traits of a communications link into a single comparable value, a ‘trust vector” (Guo 2012).
For applications involving low fidelity, temporally sparse metrics with unknown statistical distributions, GRA is a more stable comparative analysis, providing an interval of potential trust values rather than fuzzy-logic or the Bayesian-Beta distributions found in current TMFs (Liu 2006). 
It is this work that is being expanded upon in paper.

\subsection{Systematic Constraints in Marine Acoustic Networks}

In this section, we review selected features of the underwater communications channel, highlighting particular challenges and differentials against terrestrial equivalents. 

The key challenges of underwater acoustic communications are centred around the impact of slow and differential propogation of energy (RF, Optical, Acoustic) through water, and it's interfaces with the seabed / air.
The resultant challenges include; long delays due to propogation, significant inter-symbol interference and Doppler spreading, fast and slow fading due to environmental effects (aquatic flora/fauna; surface weather), carrier-frequency dependent signal attenuation, multipath caused by the medium interfaces at the surface and seabed, variations in propogation speed due to depth dependant effects (salinity, pressure, gaseous concentrations), and subscequent beam-lensing due to that same propogation variation. 
This final effect in combination with with the multipath nature of the medium result in supposedly ``line of sight'' propogations being extremely unreliable for estimating distances to targets, as the first arriving beam has bent in the medium, and commonly has bounced between the surface/seabed before arriving at a receiver. 

The attenuation that occurs in an underwater acoustic channel over a distance $d$ for a signal about frequency $f$ in linear and $dB$ forms respectivly is given by

\begin{equation}
  \label{eq:acoattenuation}
  A(d,f) = A_0d^ka(f)^d
\end{equation}
\begin{equation}
  \label{eq:acoattenuationdb}
  10 \log A(d,f)/A_0 = k \cdot 10 \log d + d \cdot 10 \log a(f)
\end{equation}

where $A_0$ is a unit-normalising constant, $k$ is a spreading factor (commonly taken as 1.5), and $a(f)$ is the absorption coefficient, expressed empiracally using Thorp's formula\ref{eq:thorp}


\begin{equation}
  \label{eq:thorp}
  10 \log a(f) = 0.11 \cdot \frac{f^2}{1+f^2} + 44\cdot\frac{f^2}{4100+f^2}+ 2.75\times10^{-4} f^2 + 0.003
\end{equation}



\subsection{Trust Requirement in Marine Networks}

In this section we establish the requirement for communications trust in acoustic marine networks, extending and expanding on the generic assessment given in \ref{sec:trustinmanets}

\begin{table}[h]
  \caption{Comparison of Base-Simulation constraints as applied between Terrestrial and Marine communications} \label{tab:simconstraints}
  \begin{center}
    \begin{tabular}{l l l l}

      Parameter & Unit & Terrestrial & Marine \\
      \hline
      Simulated Duration & $s$ & 300 & 36000\\
      Simulated Area & $km^2$ & 0.7 & 0.7 \\
      Transmission Range & $km$ & 0.25 & 1.5 \\
      Number of Nodes & & 6 & 6 \\
      Comms Medium & & RF(802.11) & Acoustic(CSMA)\\
      Propogation Speed& $m/s$ & $3\times10^8$ & 1490\\
      Center Frequency& $Hz$ & $2.6\times10^9$ & $10^3$ \\
      Bandwidth& $Hz$ & $22\times10^6$ & $10^3$\\
      Routing Protocol & & DSDV & FBR \\
      Mobility & & Various & Various \\
      Max Speed & $ms^{-1}$ & 5 & 1.25 \\
      Data Rate & $bps$ & $10^6$ & 300 \\
      Packet Size & bits & 4096 & 9600 \\
      Destination Selection & & Random & Random\\
      Single Transmission Duration & $s$ & 10 & 32 \\
      Single Transmission Size & bits & $10^7$ & $9600$ \\
    \end{tabular}
  \end{center}
\end{table}


\section{Previous Work}

\subsection{Experimental Scenarios}

Four Mobility scenarios were used to explore the trust-behaviour, covering the majority of MANET operational requirements; 

\begin{itemize}
  \item All Nodes Static
  \item Central node performing a random walk with leaf-nodes static
  \item Leaf-nodes randomly walking with central node static
  \item All nodes randomly walking
\end{itemize}

Nodes the six nodes were arranged in the form of a flattened pentagon with the 'central' node places near the geometric middle. This imperfect placement was to <make the experiment easier to write with round numbers?>

\section{}<++>

\subsubsection*{Acknowledgments.} The heading should be treated as a
subsubsection heading and should not be assigned a number.

\section{The References Section}\label{references}

%\bibliographystyle{amsplain} % UNCOMMENT BEFORE PUB!!!!!!!!!!!!!!!!!!!!!!!!!!!!!!!!!!!!!!!!!!!!!!1
\bibliographystyle{apalike}
\bibliography{refs}
% 
% \begin{thebibliography}{4}
% 
% \bibitem{jour} Smith, T.F., Waterman, M.S.: Identification of Common Molecular
% Subsequences. J. Mol. Biol. 147, 195--197 (1981)
% 
% \bibitem{lncschap} May, P., Ehrlich, H.C., Steinke, T.: ZIB Structure Prediction Pipeline:
% Composing a Complex Biological Workflow through Web Services. In: Nagel,
% W.E., Walter, W.V., Lehner, W. (eds.) Euro-Par 2006. LNCS, vol. 4128,
% pp. 1148--1158. Springer, Heidelberg (2006)
% 
% \bibitem{book} Foster, I., Kesselman, C.: The Grid: Blueprint for a New Computing
% Infrastructure. Morgan Kaufmann, San Francisco (1999)
% 
% \bibitem{proceeding1} Czajkowski, K., Fitzgerald, S., Foster, I., Kesselman, C.: Grid
% Information Services for Distributed Resource Sharing. In: 10th IEEE
% International Symposium on High Performance Distributed Computing, pp.
% 181--184. IEEE Press, New York (2001)
% 
% \bibitem{proceeding2} Foster, I., Kesselman, C., Nick, J., Tuecke, S.: The Physiology of the
% Grid: an Open Grid Services Architecture for Distributed Systems
% Integration. Technical report, Global Grid Forum (2002)
% 
% \bibitem{url} National Center for Biotechnology Information, \url{http://www.ncbi.nlm.nih.gov}
% 
% \end{thebibliography}


\begin{itemize}
\settowidth{\leftmargin}{{\Large$\square$}}\advance\leftmargin\labelsep
\itemsep8pt\relax
\renewcommand\labelitemi{{\lower1.5pt\hbox{\Large$\square$}}}

\item The final \LaTeX{} source files
\item A final PDF file
\item A copyright form, signed by one author on behalf of all of the
authors of the paper.
\item A readme giving the name and email address of the
corresponding author.
\end{itemize}
\end{document}
