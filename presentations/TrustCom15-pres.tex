\documentclass{beamer}
\usepackage[T1]{fontenc}

%\documentclass[aspectratio=169]{beamer}
%\usetheme{Madrid} % My favorite!
%\usetheme{Boadilla} % Pretty neat, soft color.
%\usetheme{default}
%\usetheme{Warsaw}
%\usetheme{Bergen} % This template has nagivation on the left
%\usetheme{Frankfurt} % Similar to the default 
%\usetheme{Copenhagen}
\usetheme{Goettingen}
%with an extra region at the top.
\usecolortheme{seahorse} % Simple and clean template
%\usetheme{Darmstadt} % not so good
% Uncomment the following line if you want %
% page numbers and using Warsaw theme%
% \setbeamertemplate{footline}[page number]
%\setbeamercovered{transparent}
\setbeamercovered{invisible}
\setbeamersize{text margin right=3.5mm, text margin left=7.5mm}  % text margin
\setbeamertemplate{caption}[numbered]


% To remove the navigation symbols from 
% the bottom of slides%
%
\usepackage{graphicx}

\usepackage[
backend=biber,
natbib=true,
style=numeric,
sorting=none]{biblatex}

\addbibresource{TrustCom15.bib}
\usepackage{tikz}
\usepackage{calc}


\usepackage{amsmath,amsthm, amssymb, latexsym}
\usepackage{booktabs}
\usepackage{colortbl}
\usepackage{hyperref}
\usepackage{todonotes}


\usepackage[caption=false,font=scriptsize]{subfig}
\usepackage[english]{babel}
\addto\captionsenglish{\renewcommand{\figurename}{Fig.}}

\def\checkmark{\tikz\fill[scale=0.4](0,.35) -- (.25,0) -- (1,.7) -- (.25,.15) -- cycle;} 
\def\scalecheck{\resizebox{\widthof{\checkmark}*\ratio{\widthof{x}}{\widthof{\normalsize x}}}{!}{\checkmark}}
%that's defined it - now for a test

\makeatletter
\newcommand*{\minuscellcolor}{}
\def\minuscellcolor\ignorespaces{%
	% \ignorespaces not really needed, because \@ifnextchar gobbles spaces
	
	\@ifnextchar-{\cellcolor[HTML]{FFAAAA}}{}
}
\newcolumntype{L}{>{\minuscellcolor}l}
\newcolumntype{C}{>{\minuscellcolor}c}
\newcolumntype{R}{>{\minuscellcolor}r}
\makeatother

\graphicspath{{../Figures/}{../posters/PDW-15/figures/},{./img/}}
\DeclareGraphicsExtensions{.pdf,.png,.jpg}
%\usepackage{bm}         % For typesetting bold math (not \mathbold)
%\logo{\includegraphics[height=0.6cm]{yourlogo.eps}}
%

\title[Multi-Metric Trust in UANs]{Single and Multi-Metric Trust Management Frameworks for use in Underwater Autonomous Networks}

\author[Bolster, A \& Marshall A]{Andrew Bolster and Alan Marshall}
\institute[UoL]
{
University of Liverpool \\
\medskip
{\emph{\{andrew.bolster,alan.marshall\}@liv.ac.uk}}\\
\vspace{0.3in}
\includegraphics[width=0.5\textwidth]{img/livuni}%
}
\date[TrustCom RATSP 2015]{Recent Advances of Trust, Security and Privacy in Computing Communications (RATSP)}
% \today will show current date. 
% Alternatively, you can specify a date.
%

% If you have a file called "university-logo-filename.xxx", where xxx
% is a graphic format that can be processed by latex or pdflatex,
% resp., then you can add a logo as follows:

% \pgfdeclareimage[height=0.5cm]{university-logo}{university-logo-filename}
% \logo{\pgfuseimage{university-logo}}



% Delete this, if you do not want the table of contents to pop up at
% the beginning of each subsection:
\AtBeginSubsection[]
{
  \begin{frame}<beamer>{Outline}
    \tableofcontents[currentsection,currentsubsection]
  \end{frame}
}


\begin{document}

\begin{frame}
  \titlepage
\end{frame}

\frame{\tableofcontents}
%

\section{Motivation}


\begin{frame}{Summary}
  % - A title should summarize the slide in an understandable fashion
  %   for anyone how does not follow everything on the slide itself.

  \begin{itemize}
  \item
    Methods developed for establishing Communications Trust in the MANET space increasingly applied to other arenas such as the  underwater realm.
  \item
    These Trust Management Frameworks (TMFs) must be reassessed with respect to the sparse, noisy and contested marine communications environment.
  \item 
    Most MANET TMFs rely on one\footnotemark type of observation (metric); recent work (MTFM \cite{Guo11}) introduces the use of multiple types of continuous metrics for assessment.
  \item 
    How do these Single and Multi-Metric Frameworks perform in the challenging marine communications environment?
  \item
    What metrics are suitable for use underwater?
  \end{itemize}
  \footnotetext{Packet Loss Rate (PLR) or other binary success observation}
\end{frame}



\subsection{Related Work}

\begin{frame}{Trust in Conventional MANETS}
  \begin{itemize}
    \item 
      TMFs provide information to assist the estimation of future states and actions of nodes within networks.
    \item
      Centralised methods (CA/TTP/PKI) unsuitable for dynamic decentralised networks\cite{Caiti2011}.
    \item 
      Need to detect, identify, \& mitigate threats in a distributed fashion.
  \end{itemize}
\end{frame}

\begin{frame}[allowframebreaks]{Single-Metric TMFs}
  \begin{itemize}
    \item \emph{Hermes} \cite{Zouridaki2005} - Bayesian estimation based on PLR; encapsulates both ``Trust'' and ``Confidence'')
    \item \emph{OTMF} \cite{Li2008} - Collaborative Assessments of Bayesian Trust, PLR.
    \item \emph{TSR} \cite{Moe2008a} - Builds HMM into Dynamic Source Routing (DSR), Session Loss Rate.
    \item \emph{CONFIDANT} \cite{Buchegger2002} - Probablistic PLR assessment, includes some topology and reputational weighting.
    \item \emph{Fuzzy Trust-Based Filtering} \cite{Luo2008} - Fuzzy classification on the \emph{nature} of packet delivery (eg. ``late'', ``unreliable'', ``unknown'', etc.)
  \end{itemize}
  Most can be generalised as single-value estimations of PLR/Successful Routes, with the incorporation of some \emph{meta}-observations eg Topology
  \framebreak
  \begin{itemize}
  \item
    Single Metric TMFs present opportunities for malicious actors to undermine the operation of a network if their attack does not directly impact packet delivery. 
  \item 
    Not an issue in networks where Comms. is the primary operating concern, but is significant in resource constrained environments (eg power, mobility, channel occupancy, physical location)
  \end{itemize}

\end{frame}

\begin{frame}[allowframebreaks]{Multi-Metric TMF}
  \begin{itemize} 
    \item \emph{Multi-metric Trust For MANETS (MTFM)} \cite{Guo11} - Uses additional metrics such as Power, Throughput, Delay, etc. in addition to PLR to assess trust, as well as incorporating topological and metric weighting.
    \item Use of multiple metrics allows classification of behaviours through dynamic metric weighting.
    \item Use of Grey Relational Grading to provide dynamic runtime normalisation, assessing \emph{comparative} trust within a cohort of actors.
  \end{itemize}
  \framebreak
\begin{align}
  \label{eq:grcg}
  \theta_{k,j}^t &= \frac{\min_k|a_{k,j}^t - g_j^t| + \rho \max_k|a_{k,j}^t-g_j^t|}{|a_{k,j}^t-g_j^t| + \rho \max_k|a_{k,j}^t-g_j^t|} \\
  \label{eq:grcb}
  \phi_{k,j}^t &= \frac{\min_k|a_{k,j}^t - b_j^t| + \rho \max_k|a_{k,j}^t-b_j^t|}{|a_{k,j}^t-b_j^t| + \rho \max_k|a_{k,j}^t-b_j^t|}  \\
  \label{eq:grc}
  [\theta_k^t, \phi_k^t]& = \left[\sum_{j=0}^M h_j \theta_{k,j}^t,\sum_{j=0}^M h_j \phi_{k,j}^t \right]\\
  \label{eq:grcT}
  T_k^t &= ({1+{(\phi_k^t)^2}/{(\theta_k^t)^2}})^{-1}
\end{align}
Where  $a_{k,j}^t$ is the value of an observed metric $x_j$ for a given node $k$ at time $t$,  $g$ and $b$ are respectively the ``good'' and ``bad'' reference metric sequences from $\{a_{k,j}^t k=1,2\dots K\}$, $H=[h_0\dots h_M]$ is a metric weighting vector such that $\sum h_j = 1$

  \framebreak

  This Grey Trust value is then combined\footnotemark with the shared assessments from other actors in the network weighted based on their relative topology to provide a final value; $T_{i,j}^{MTFM}$
    \begin{figure}[h]
      \centering
      \includegraphics[width=.6\textwidth]{node_relationships}
      \label{fig:node_relationships}
    \end{figure}
    \footnotetext{\hyperlink{eq:networkeffects}{\beamergotobutton{Details}}}

\end{frame}

\begin{frame}{Multi-Metric Compared to Single in MANETs}
  Guo et al.\cite{Guo11} demonstrated that MTFM operates favourably in 802.11 based terrestrial MANETs against OTMF and Hermes, and can accurately detect, identify, \& characterise misbehaviours within a group of six nodes, with $n_0$ as the primary observer and $n_1$ as the misbehavor.

  \begin{figure}[h]
    \begin{center}
      \includegraphics[width=0.6\textwidth]{s1_layout}
    \end{center}
    \caption{Initial Node Layouts in \cite{Guo11}}
    \label{fig:node_layout}
  \end{figure}
  \end{frame}

\subsection{Challenges to Trust in Underwater Networks}

\begin{frame}[allowframebreaks]{Communications Channel Considerations}
  Key Characteristics of the Marine Acoustic Channel \cite{Urick1983,Partan2006,Stojanovic2007,Stefanov2011}:
  \begin{itemize}
    \item Slow propogation ($~1400ms^{-1}$) incurring long delays
    \item Inter-symbol interference
    \item Doppler Spreading
    \item Non-Linear propocation due to refraction
    \item Fast \& Slow fades from environmental factors (flora/fauna/surface and seabed conditions)
    \item Freq. dependant attenuation
    \item Sigificant destructive multipath effects
  \end{itemize}
  
  \framebreak

  The attenuation that occurs in an underwater acoustic channel over a distance $d$ for a signal about frequency $f$ in linear power is given as $A_{\text{aco}}(d,f) = A_0d^ka(f)^d$ and in $dB$ form as;
  %
  \begin{equation}
    \label{eq:acoattenuationdb}
    10 \log A_{\text{aco}}(d,f)/A_0 = k \cdot 10 \log d + d \cdot 10 \log a(f)
  \end{equation}
  %
  where $A_0$ is a normalising constant, $k$ is a spreading factor (commonly taken as 1.5  \cite{Stojanovic2007}), and $a(f)$ is the absorption coefficient, approximated using Thorp's formula \cite{Stefanov2011}
  %
  \begin{equation}
    \label{eq:thorp}
    10 \log a(f) = \frac{0.11 \cdot f^2}{1+f^2} + \frac{44\cdot f^2}{4100+f^2}+ 2.75\times10^{-4} f^2 + 0.003
  \end{equation}

  \framebreak

  Compared to RF Free space PL: $(A_{\text{RF}}(d,f) \approx \left( \frac{4\pi d f}{c} \right)^2)$
  \begin{itemize}
    \item Exponential in $d$: $A_{\text{aco}} \propto f^{2d}$ vs $A_{\text{RF}} \propto (df)^2$
    \item Quadratic $f$ factor four orders higher in $f\propto A_{\text{aco}}$ vs $f\propto A_{\text{RF}}$

  \end{itemize}
  
\end{frame}




\section{Our Contribution}

\subsection{Experimental Context}
\begin{frame}{Operational Considerations: Collaborative AUV Survey}
  \begin{columns}
    \begin{column}{0.5\textwidth}
      Context:
      \begin{itemize}
        \item Fleets of up to 16 collaborating Autonomous Underwater Vehicles(AUVs)
        \item Constrained in Power, Mobility, Processing, Storage Capacity
        \item Tasked to perform ongoing survey of an area
      \end{itemize}

    Communications Efficiency is not the only operational asset at risk from malicious exploitation
      
    \end{column}
    \begin{column}{0.5\textwidth}
      \begin{figure}[h]
        \begin{center}
          \includegraphics[width=\linewidth]{remus100cmre}
        \end{center}
        \caption{REMUS 100 AUV as deployed at NATO CMRE La Spezia}
        \label{fig:remus100cmre}
      \end{figure}
      
    \end{column}
  \end{columns}
\end{frame}

\begin{frame}{Scale Considerations}
  \begin{itemize}
    \item Simulations based on SimPy \cite{Mueller2003SimPy}, Network stack using AUVNetSim \cite{Miquel2008} and channel constraints based on Stojaovic and Stefanov \cite{Stojanovic2007,Stefanov2011}\hyperlink{tab:sysconstraints}{\beamergotobutton{Details}}
    \item Established a safe operating zone in terms of communications rate and node distances to optimise for delay/throughput at 0.015pps and avg. init. range 300m \hyperlink{eq:networkeffects}{\beamergotobutton{Details}}
    \item Six per-link communications metrics: TX/RX Throughput/Power, Delay and PLR, lacking the 802.11 Data Rate metric from \cite{Guo11}  
    
  \end{itemize}

\end{frame}

\begin{frame}{Misbehaviour Specification}
  Two misbehaviours developed:
  \begin{itemize}
    \item \emph{Malicious Power Control}(MPC) - attacker $n_1$ aims to make $n_0$ appear selfish by increasing power to all nodes except to/from $n_0$
    \item \emph{Selfish Target Selection}(STS) - $n_1$ preferentially communicates with nodes close to it, to conserve its own power.
  \end{itemize}

\end{frame}

\subsection{Main Results}

\begin{frame}{Principal Aims}
\begin{itemize}
  \item Operation of MTFM in the Marine Environment
  \item 
\end{itemize}
\end{frame}

\begin{frame}[allowframebreaks]{Multi-Metric Operation}
  \setcounter{subfigure}{0}% Reset subfigure counter
  \begin{figure}[htp]
    \centering
    \subfloat[Fair Static]{\includegraphics[width=0.33\linewidth]{trust_bella_static_fair} \label{fig:trust_static}}\hfil
    \subfloat[Malicious Static]{\includegraphics[width=0.33\linewidth]{trust_bella_static_malicious} \label{fig:trust_static_mal}}\hfil
    \subfloat[Selfish Static]{\includegraphics[width=0.33\linewidth]{trust_bella_static_selfish} \label{fig:trust_static_sel}}\hfil

    \subfloat[Fair Mobile]{\includegraphics[width=0.33\linewidth]{trust_bella_all_mobile_fair}  \label{fig:trust_all_mobile}}\hfil
    \subfloat[Malicious Mobile]{\includegraphics[width=0.33\linewidth]{trust_bella_all_mobile_malicious}  \label{fig:trust_all_mobile_mal}}\hfil
    \subfloat[Selfish Mobile]{\includegraphics[width=0.33\linewidth]{trust_bella_all_mobile_selfish}  \label{fig:trust_all_mobile_sel}}\hfil
    \caption{Observations of $n_1$ ($T_{1,X}$), showing Direct, Recommender and Indirect relationships and $T_{MTFM}$ and $T_{AVG}$\hyperlink{fig:trust_mobility_closeup}{\beamergotobutton{Closeup}}}
    \label{fig:trust_mobility}
  \end{figure}
%\end{frame}
\framebreak
Key Observations: 
%\begin{frame}{Multi-Metric Operation - Key Observations}
  %
  \begin{itemize}
    \item Mobility greatly increases variation in instantenously observed trust
    \item $T_{MTFM}$ remains more stable in both mobility cases when compared to either single-node assessments or $T_{Avg}$
    \item Raw $T_{MTGM}$ isn't perfect; in Fig~\ref{fig:trust_all_mobile_mal} demonstrates huge variability in Direct assessment ($T_{1,0}$) that isn't reflected in $T_{MTFM}$. Partially expected in this directed attack.
    \item Larger general variability in observations in ``Fair'' case compared to misbehaviours
  \end{itemize}
\end{frame}

\begin{frame}[allowframebreaks,t]{Blind Comparison of Single/Multi-metric TMFs}
  \vspace{-24pt}%
  \begin{columns}
    \begin{column}[T]{0.5\textwidth}
      \begin{figure}[t]
        \centering
        \subfloat[Fair Scenario]{\includegraphics[width=\linewidth]{trust_beta_otmf_fair} \label{fig:all_mobile_fair_beta}}\hfil
        \renewcommand{\thesubfigure}{c}% New fixed/manual numbering
        \subfloat[Selfish Target Selection Scenario]{\includegraphics[width=\linewidth]{trust_beta_otmf_selfish} \label{fig:all_mobile_selfish_beta}}
        \label{fig:otmf_beta_comparison}
      \end{figure}%
    \end{column}
    \begin{column}[T]{0.5\textwidth}
      \begin{figure}[t]
        \centering
        \vspace{0pt}%
        \renewcommand{\thesubfigure}{b}% New fixed/manual numbering
        \subfloat[Malicious Power Control Scenario]{\includegraphics[width=\linewidth]{trust_beta_otmf_malicious} \label{fig:all_mobile_badmouthing_beta}}
        \label{fig:otmf_beta_comparison}
      \end{figure}
      $T_{1,0}$ for Hermes, OTMF and MTFM assessment values for fair and malicious behaviours in the fully mobile scenario (mean of MTFM also shown)

    \end{column}
  \end{columns}

  \framebreak

  Key Observations:
  \begin{itemize}
    \item 
  \end{itemize}

\end{frame}

\begin{frame}{Metric Significance Assessment}
\end{frame}

\subsection{Current Work}{Current Work and Paths to Proof/Implementation}

\begin{frame}{Make Titles Informative.}
\end{frame}



\section*{Summary}

\begin{frame}{Summary}

  % Keep the summary *very short*.
  \begin{itemize}
  \item
    The \alert{first main message} of your talk in one or two lines.
  \item
    The \alert{second main message} of your talk in one or two lines.
  \item
    Perhaps a \alert{third message}, but not more than that.
  \end{itemize}
  
  % The following outlook is optional.
  \vskip0pt plus.5fill
  \begin{itemize}
  \item
    Outlook
    \begin{itemize}
    \item
      Something you haven't solved.
    \item
      Something else you haven't solved.
    \end{itemize}
  \end{itemize}
\end{frame}

\begin{frame}[t,allowframebreaks]
  \frametitle{References}
  \printbibliography[title=References]% [nottype=video]}
\end{frame}

\begin{frame}
  \centerline{The End}
\end{frame}

\begin{frame}[allowframebreaks]{Grey Trust Equs}
  \begin{align}
    \label{eq:networkeffects}
    T_{i,j}^{MTFM}=&\frac{1}{2} \cdot \max_s\{f_s(T_{i,j})\} T_{i,j}\\ \notag
    +&\frac{1}{2} \frac{2|N_R| }{2|N_R| + |N_I|}\sum_{n \in N_R} \max_s\{f_s(T_{i,n})\} T_{i,n}\\ \notag
    +&\frac{1}{2} \frac{|N_I| }{2|N_R| + |N_I|}\sum_{n \in N_I} \max_s\{f_s(T_{i,n})\} T_{i,n} 
  \end{align}

  Where $T_{i,n}$ is the subjective trust assessment of $n_i$ by $n_n$, and $f_s = [ f_1,f_2, f_3]$ given as...

  \framebreak

  \begin{align}
    \label{eq:whitenization}
    f_1(x)&= -x+1\notag\\
    f_2(x)&= 
    \begin{cases}
      2x & \text{if }x\leq 0.5\\
      -2x+2 & \text{if }x>0.5
    \end{cases}\\
    f_3(x)&= x\notag
  \end{align}
\end{frame}

\begin{frame}[allowframebreaks]{Comms Scaling Graphs}

  \begin{figure}[h]
    \centering
    \subfloat[][All Nodes Static]{\includegraphics[width=0.35\linewidth]{2d_ratio_static.pdf}}
    \subfloat[][$n_1$ Random Walk]{\includegraphics[width=0.35\linewidth]{2d_ratio_single_mobile.pdf}}\\
    \subfloat[][All nodes but $n_1$ Random Walk]{\includegraphics[width=0.35\linewidth]{2d_ratio_allbut1.pdf}}
    \subfloat[][All nodes Random Walk]{\includegraphics[width=0.35\linewidth]{2d_ratio_all_mobile.pdf}}
    \label{fig:CommsThroughputRatios}
  \end{figure}

\end{frame}

\begin{frame}[shrink]{System Model Constraints}
\centering
\begin{table}[h]
  \caption{Comparison of system model constraints as applied between Terrestrial and Marine communications} \label{tab:sysconstraints}
  \begin{center}
    \setlength{\tabcolsep}{8pt}
    \begin{tabular}{lccc}
      \toprule
      Parameter & Unit & Terrestrial & Marine \\
      \midrule
      Simulated Duration & $s$ & 300 & 18000\\
      Trust Sampling Period & $s$ & 1 & 600 \\
      Simulated Area & $km^2$ & 0.7 & 0.7-4 \\
      Transmission Range & $km$ & 0.25 & 1.5 \\
      Physical Layer & & RF(802.11) & Acoustic\\
      Propagation Speed& $m/s$ & $3\times10^8$ & 1490\\
      Center Frequency& $Hz$ & $2.6\times10^9$ & $2 \times 10^4$ \\
      Bandwidth& $Hz$ & $22\times10^6$ & $1\times10^4$\\
      MAC Type & & CSMA/DCF & CSMA/CA\\
      Routing Protocol & & DSDV & FBR \\
      Max Speed & $ms^{-1}$ & 5 & 1.5 \\
      Max Data Rate & $bps$ & $5\times10^6$ & $\approx 240$ \\
      Packet Size & bits & 4096 &  9600 \\
      Single Transmission Duration & $s$ & 10 & 32 \\
      Single Transmission Size & bits & $10^7$ & $9600$ \\
      \bottomrule
    \end{tabular}
    \setlength{\tabcolsep}{6pt}
  \end{center}
\end{table}

\end{frame}

\begin{frame}{MTFM Operation Detail}
%
\setcounter{subfigure}{0}% Reset subfigure counter
\label{fig:trust_mobility_closeup}%
\begin{figure}[h]
  \subfloat[Fair Static]{\includegraphics[width=1.0\linewidth]{trust_bella_static_fair} \label{fig:trust_static}}
\end{figure}
\end{frame}
\begin{frame}{MTFM Operation Detail}
\begin{figure}[h]
  \subfloat[Malicious Static]{\includegraphics[width=1.0\linewidth]{trust_bella_static_malicious} \label{fig:trust_static_mal}}
\end{figure}
\end{frame}
\begin{frame}{MTFM Operation Detail}
\begin{figure}[h]
  \subfloat[Selfish Static]{\includegraphics[width=1.0\linewidth]{trust_bella_static_selfish} \label{fig:trust_static_sel}}
\end{figure}
\end{frame}
\begin{frame}{MTFM Operation Detail}
\begin{figure}[h]
  \subfloat[Fair Mobile]{\includegraphics[width=1.0\linewidth]{trust_bella_all_mobile_fair}  \label{fig:trust_all_mobile}}
\end{figure}
\end{frame}
\begin{frame}{MTFM Operation Detail}
\begin{figure}[h]
  \subfloat[Malicious Mobile]{\includegraphics[width=1.0\linewidth]{trust_bella_all_mobile_malicious}  \label{fig:trust_all_mobile_mal}}
\end{figure}
\end{frame}
\begin{frame}{MTFM Operation Detail}
\begin{figure}[h]
  \subfloat[Selfish Mobile]{\includegraphics[width=1.0\linewidth]{trust_bella_all_mobile_selfish}  \label{fig:trust_all_mobile_sel}}
\end{figure}
\end{frame}

\end{document}
