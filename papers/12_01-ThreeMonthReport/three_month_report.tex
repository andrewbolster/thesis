
% ----------------------------------------------------------------------
%                   LATEX TEMPLATE FOR PhD THESIS
% ----------------------------------------------------------------------

% based on Harish Bhanderi's PhD/MPhil template, then Uni Cambridge
% http://www-h.eng.cam.ac.uk/help/tpl/textprocessing/ThesisStyle/
% corrected and extended in 2007 by Jakob Suckale, then MPI-CBG PhD programme
% and made available through OpenWetWare.org - the free biology wiki


%: Style file for Latex
% Most style definitions are in the external file PhDthesisPSnPDF.
% In this template package, it can be found in ./Latex/Classes/
% Move this to individual files
\documentclass[twoside,10pt,a4paper]{Latex/Classes/PhDthesisPSnPDF}


%: Macro file for Latex
% Macros help you summarise frequently repeated Latex commands.
% Here, they are placed in an external file /Latex/Macros/MacroFile1.tex
% An macro that you may use frequently is the figuremacro (see introduction.tex)
\include{Latex/Macros/MacroFile1}
\ifpdf
  \usepackage{helvet}
\fi
  %bugfix for mathml conversion to oolatex
  \let\columnlines\empty

%: ----------------------------------------------------------------------
%:                  TITLE PAGE: name, degree,..
% ----------------------------------------------------------------------
% below is to generate the title page with crest and author name

%if output to PDF then put the following in PDF header
\ifpdf  
    \pdfinfo { /Title  (An Investigation into Trust and Reputation Frameworks for Autonomous Underwater Vehicles)
               /Creator (TeX)
               /Producer (pdfTeX)
               /Author (Andrew Bolster abolster01@qub.ac.uk)
               /CreationDate (D:YYYYMMDDhhmmss)  %format D:YYYYMMDDhhmmss
               /ModDate (D:YYYYMMDDhhmm)
               /Subject (xyz)
               /Keywords (add, your, keywords, here) }
    \pdfcatalog { /PageMode (/UseOutlines)
                  /OpenAction (fitbh)  }
\fi


\title{An Investigation into Trust and Reputation Frameworks for Autonomous Underwater Vehicles}



% ----------------------------------------------------------------------
% The section below defines www links/email for author and institutions
% They will appear on the title page of the PDF and can be clicked
\ifpdf
  \author{\href{mailto:me@andrewbolster.info}{Andrew Bolster}}
%  \cityofbirth{born in XYZ} % uncomment this if your university requires this
%  % If city of birth is required, also uncomment 2 sections in PhDthesisPSnPDF
%  % Just search for the "city" and you'll find them.
  \collegeordept{\href{http://www.ecit.qub.ac.uk}{Institute of Electronics, Communications and Information Technology (ECIT)}}
  \university{\href{http://www.qub.ac.uk}{Queen's University Belfast (QUB)}}

  % The crest is a graphics file of the logo of your research institution.
  % Place it in ./frontmatter/figures and specify the width
  \crest{\includegraphics[width=4cm]{frontmatter/figures/crest.png}}
  
% If you are not creating a PDF then use the following. The default is PDF.
\else
  \author{Andrew Bolster}
%  \cityofbirth{born in XYZ}
  \collegeordept{Institute of Electronics, Communications and Information Technology (ECIT)}
  \university{Queen's University Belfast (QUB)}
  \crest{\includegraphics[width=4cm]{frontmatter/figures/crest}}
\fi

%\renewcommand{\submittedtext}{change the default text here if needed}
\degree{Three Month Report}
\degreedate{2012 January}


% ----------------------------------------------------------------------
       
% turn of those nasty overfull and underfull hboxes
\hbadness=10000
\hfuzz=50pt

%: --------------------------------------------------------------
%:                  FRONT MATTER: dedications, abstract,..
% --------------------------------------------------------------

\begin{document}

%\language{english}

% sets line spacing
\renewcommand\baselinestretch{1}
%\baselineskip=12pt


%: ----------------------- generate cover page ------------------------

\maketitle  % command to print the title page with above variables

%: ----------------------- abstract ------------------------

% Your institution may have specific regulations if you need an abstract and where it is to be placed in the document. The default here is just after title.

\begin{abstracts}        %this creates the heading for the abstract page
As Autonomous underwater vehicles (AUVs) become more technically capable, and
fiscally more economical, they are being used in a many defence, commercial and
environmental applications. Increasingly, these applications are tending towards
utilising  independent collective behaviour of teams or fleets of these
platforms. With this use being increasingly independent of classical command and
control structures, the accurate and timely establishment of mutual and
distributed communications trust between nodes within such fleets is essential
for the reliability and stability of such systems, and to the secure integration
of such systems into larger management systems-of-systems. 

This report aims to initially demonstrate that there is a gap in currently
theorised and implemented technologies within the scope of distributed trust,
theorise as to what potential avenues of research would produce such a
distributed trust management framework (DTMF) and to demonstrate that such a
framework is essential to the performance, longevity, and scalability of such
technologies, specifically within the marine defence space, but also within the
general field of autonomous vehicle operation (UxV). 
\end{abstracts}

% The original template provides and abstractseparate environment, if your institution requires them to be separate. I think it's easier to print the abstract from the complete thesis by restricting printing to the relevant page.
% \begin{abstractseparate}
%   As \glspl{auv} become more technically capable and economically feasible, they are being increasingly used in a great many areas of defence, commercial and environmental applications. 
These applications are tending towards using independent, autonomous, ad-hoc, collaborative behaviour of teams or fleets of these \gls{auv} platforms.
This convergence of research experiences in the \gls{uan} and \gls{manet} fields, along with the increasing \gls{loa} of such platforms, creates unique challenges to secure the operation and communication of these networks.

The question of security and reliability of operation in networked systems has usually been resolved by having a centralised coordinating agent to manage shared secrets and monitor for misbehaviour.
However, in the sparse, noisy and constrained communications environment of \glspl{uan}, the communications overheads and single-point-of-failure risk of this model is challenged (particularly when faced with capable attackers).

As such, more lightweight, distributed, experience based\footnote{rather than ``Evidence based'' in the case of shared keys, \gls{pki} etc.} systems of ``Trust'' have been proposed to dynamically model and evaluate the ``trustworthiness'' of nodes within a \gls{manet} across the network to prevent or isolate the impact of malicious, selfish, or faulty misbehaviour. 
Previously, these models have monitored actions purely within the communications domain. 
Moreover, the vast majority rely on only one type of observation (metric) to evaluate trust; successful packet forwarding.
In these cases, motivated actors may use this limited scope of observation to either perform unfairly without repercussions in other domains/metrics, or to make another, fair, node appear to be operating unfairly.

This thesis is primarily concerned with the use of terrestrial-\gls{manet} trust frameworks to the \gls{uan} space. 
Considering the massive theoretical and practical difference in the communications environment, these frameworks must be reassessed for suitability to the marine realm. 
We find that current single-metric \glspl{tmf} do not perform well in a best-case scaling of the marine network, due to sparse and noisy observation metrics, and while basic multi-metric communications-only frameworks perform better than their single-metric forms, this performance is still not at a reliable level. 
We propose, demonstrate (through simulation) and integrate the use of physical observational metrics for trust assessment, in tandem with metrics from the communications realm, improving the safety, security, reliability and integrity of autonomous \glspl{uan}.

Three main novelties are demonstrated in this work:
Trust evaluation using metrics from the physical domain (movement/distribution/etc.), demonstration of the failings of Communications-based Trust evaluation in sparse, noisy, delayful and non-linear \gls{uan} environments, and the deployment of trust assessment across multiple domains, e.g.\ the physical and communications domains.
The latter contribution includes the generation and optimisation of cross-domain metric composition or``synthetic domains'' as a performance improvement method.

% \end{abstractseparate}

%: ----------------------- contents ------------------------

\setcounter{secnumdepth}{3} % organisational level that receives a numbers
\setcounter{tocdepth}{1}    % print table of contents for level 3
\tableofcontents            % print the table of contents
% levels are: 0 - chapter, 1 - section, 2 - subsection, 3 - subsection

%: --------------------------------------------------------------
%:                  MAIN DOCUMENT SECTION
% --------------------------------------------------------------

% the main text starts here with the introduction, 1st chapter,...
\mainmatter

\renewcommand{\chaptername}{} % uncomment to print only "1" not "Chapter 1"

% --------------------------------------------------------------
%:                  INTRODUCTION
% --------------------------------------------------------------
\chapter{Introduction} After over 20 years of theoretical and exploratory
research into Unmanned Vehicles, the recent
in-theatre use of Unmanned Aerial Vehicles (UAVs)\footnote{Such as the US Air
Force's use of the Predator range of reconnaissance aircraft in Iraq, 
Afghanistan, and Bosnia 1995-onwards 
http://www.fas.org/irp/program/collect/predator.htm} has proven the 
effectiveness and reliability of individual
autonomous or semi-autonomous ordnance delivery and reconnaissance platforms.
Meanwhile, the use of AUVs in Oceanographic Surveying has demonstrated the
viability of semi-autonomous remotely operated vehicles (ROVs) within this 
harsh environment. 
	
This has lead to significant research into applying the technologies proven in
the skies to the marine environment. One of the primary problems of which is 
establishing reliable communications between AUV 'nodes' or between an AUV node 
and a remote operator \cite{Partan2006}.

Research into terrestrial Mobile wireless Ad-hoc Networks (MANETs) has shown 
that significant communicative power reductions can be made by using
many nodes together to collaboratively 'pass' messages over short hops, instead
of investing in high power transmissions over wide areas \cite{Royer1999}.
Additionally, research into Delay-Tolerant Networking (DTN) introduces the
ability for nodes to act as repeaters for other spatially and temporally
separated nodes \cite{Fall2003}.

These multiple fields of research all at least partly rely on the
trustworthiness of each and every node within the network. I have taken this 
trust to imply not only 'intention-trust (i.e `Is this node trying to disrupt
the network by design') but also wellness-trust (i.e `Is this node faulty
in some way, making it behave in a non-optimal way'). These terms were arrived 
upon by me as there appears to me no better terminology for the perceived 
difference.

It is the intention of this project to establish a framework for establishing
quantitative trust within a spatially and temporally distributed mobile network
in a marine medium.

% --------------------------------------------------------------
%:                  REVIEW BODY: Intro, Lit summary, Lit Plan
% --------------------------------------------------------------
\chapter{Summary of Reviewed Literature}

See Appendix A for Completed and Planned Reading lists for each of the 
following sections, and Appendix B for Completed Literature Reviews .

\section{Marine Operational and Communications Considerations}

The Marine environment is a terrible medium in which to have high speed
communications; RF has too high attenuation to accomplish more than a few metres
of transmission, Acoustic has the advantage of being carried fairly well in
terms of distance, but frequency dependant attenuation effects such as delays
and Doppler Effects render stable high-speed communications difficult, and
free-space optical/IR links have numerous constraints making them very unstable
even in advantageous situations \cite{Partan2006} \cite{Chen}
 \cite{Akyildiz2005}. While the physical reasons for these acoustic effects is
not explained in the reviewed work, it is justified in other work
 \cite{Catipovic1990}.

This presents the general compromise between Underwater communications rate,
stability, and range.

Within a WSN or a network of AUVs, an additional concern is added, especially if
the security of communications is a factor; the use of a centralised gateway
node through which communications is coordinated, suggested as a solution to
some operational situations \cite{Partan2006} \cite{Caiti}, introduces a central
point of vulnerability to the network as a whole. Given the general defence
context of this project, and the management framework set out in
 \cite{Banks2010}, the concept of ``centrality/security versus resource
constraint'' will be an important aspect to the subsequent research in this
project.

\section{Threat Modelling}
At this point, no direct research into threat modelling has been performed, 
beyond the tangential mentions in the more domain specific literature. 

The kinds of potential attacks discussed and evaluated across the reviewed
literature ( \cite{Liu2006} \cite{Guo}) include;
\begin{itemize}
  \item \textbf{Bad-mouthing:}~where a malicious node provides dishonest
    recommendations about other nodes
  \item \textbf{Sybil attack:}~where a malicious node uses multiple 
    pseudonymous
    entities to 'take blame' for bad behaviour while maintaining a good
    reputation with the nodes real identity
  \item \textbf{Newcomer attack:}~similar to the Sybil attack, where a 
    malicious
    node will periodically assume a 'fresh' identity with the network.
  \item \textbf{Grey-hole attack:}~where a node (or collaborating collection of
    malicious nodes) selectively drops messages to or from certain (or all)
    nodes, implicitly reducing the apparent trustworthiness of those attacked
    nodes, while lowering overall network throughput
\end{itemize}

\section{Trust and Reputation Management in MANET/DTN}

The major learning points so far have been the theoretical establishment of
trust \cite{Liu2006}, the practical evaluation of a current TMF (OTMF) 
contrasted against alternative frameworks (Trust Establishment Framework 
\cite{Buchegger2002}, and Reputation-Based Frameworks \cite{Marti2000}), 
showing the importance of both collaborative trust establishment within a local 
network, and experiential reputation in making a network resilient to malicious 
or faulty nodes \cite{Li2007a}. The leads to the extension of these 
single-metric (i.e.  the probability of a successful interaction) trust into a 
multi-parametric TMF with the application of Grey Whitenisation and clustering 
to detect not only \emph{if} malicious behaviour is happening, but to classify 
\emph{what} malicious behaviour is causing interruption. \cite{Guo}.

\section{Types of Trust}

The subjective nature of ``trust'' is evident in the range of different usages. 
The most common form of trust discussed is that of direct, indirect, and 
recommendation trust, and takes is root from the network/graph theoretical 
relationships between the nodes that are assessing, being assessed, and passing 
on trust.

In \citet{Guo}, such trusts are categorised as;
\begin{itemize}
    \item \emph{Direct}: Trust based on historical behaviour of node $B$ w.r.t.
    node $A$.
    \item \emph{Indirect}: Trust transited through third-party entities, i.e
    $E$, $F$ w.r.t. $B$ where neither communicate with $A$
    \item \emph{Recommendation}: a subjective trust transited through a common
    entity; i,e Trust($B\rightarrow C$) communicated to $A$ by $C$.
\end{itemize}

These constructions only comment on the method of acquisition of a subjective 
trust value from a third party (or in the case of 'Direct', an objective 
estimate of future trust given observation).

I would add that there is an additional level of trust-classification; that 
being a causal separation between gains or breaks in trust records due to 
either intentional, malicious, behaviour, or alternatively, partial or complete 
failure in a system/sub-system on a node (either side of the trust record) 
causing what appears to be malicious behaviour. I term these categories 
'intention-trust' and 'wellness-trust' respectively.

\chapter{Intended Direction of Subsequent Research}
\section{Direction of Research}
While catch-up work needs to be done in the area of Threat modelling, it's
important to start producing something worthwhile in parallel with ongoing
generalist research. 

I will be undertaking at least one paper submission before
differentiation\footnote{On the subject of accelerating DSL DSM algorithms;
following on from my MEng work in that area}.

Additionally, regardless of what specific area of this currently reviewed
research is focused on, establishing a AUV/Underwater Sensor Network simulation
framework will be essential. An investigation will be undertaken as to the
applicability of the NS3 software package. This is discussed further below.

Beyond that, the aim is to study the practical operation of AUV/Marine Sensor
Networks/Human Operators in an effort to establish reliable physical, MAC, and 
behavioural level metrics in current distributed marine networks that can be 
used with existing (and potentially, novel) trust and reputation management 
frameworks.

\section{Specific Direction}
A potential future, more esoteric, direction of research, would be to 
investigate the practical (and ethical/legal) issues surrounding the 
integration of human operators into multi-parametric trust networks. This 
raises important questions such as 'If a fleet demonstrably cannot trust a 
human operator, should it rebel?' etc.  

However, the specific direction I intend to pursue is to investigate the 
media-level
metrics that can be used to establish trust(i.e. those characteristics found 
within the Network, Data link, and Physical OSI Model layers, ranging from 
Doppler Shift effects on signals, to the path that a conversation takes across 
the network), and to integrate these metrics,
through a parallel process of simulation and physical experimentation, into a
practical DTMF, and to evaluate such a systems performance in both simulation 
and real-world implementations.

\section{Major Researchers/Trends within the Field}
While the major researchers and research trends are detailed in Appendices A 
and B, it is worth highlighting a few particular examples. 

\begin{itemize}
  \item \textbf{Prof James Kurose}: of the University of Massachusetts, 
    Amherst, co-author of  ``Computer Networking: A Top Down Approach'', is a 
    leader in the field of computer networks, more recently focusing on MANETs 
    and ad-hoc network provisioning, and has covered a wide range of
    subjects relevant to this project; most usefully, his collaboration with 
    Partan and Levine \cite{Partan2006}
  \item \textbf{Prof Andrea Caiti}: of Universita di Pisa, is a leading 
    researcher in the area of Control Systems for Oceanographic applications, 
    specifically cooperative behaviours in Underwater Sensor Networks utilising 
    Acoustic communications \cite{Caiti} \cite{Caitia} \cite{Caiti2011}
  \item \textbf{Prof K. J. Ray Liu}: of the University of Maryland, is an 
    expert in the application of information and game-theoretic practices to 
    communications and networking, having authored eight textbooks in the 
    field, and authored/co-authored hundreds of papers in the subject.  
    \cite{Liu2006} \cite{Sun2008}
  \item \textbf{Kevin Fall Ph.D} is a Principal Engineer at the Intel Research 
    laboratory in Berkeley, California, USA.  His research interests include 
    computer networks, operations in remote and difficult environments, 
    security, and network science. Current work focuses on Delay Tolerant 
    Networking, building on Vint Cerf's work on IPN. \cite{Fall2003}
\end{itemize}

\section{Simulation Considerations}
NS2 is a popular discrete-event network simulator that emerged and was developed
by the academic community. Its spiritual successor, NS3, has emerged in a
similar fashion with a brand new core architecture, aimed towards improving APIs for model
integration, and has been 'operational' since mid-2008.

NS3 is not directly backwards compatible with NS2, but the majority of the most 
used models have been converted.

Since I have almost-zero experience in the NS2 framework, and the integration
systems in place for NS3 (C++/Python for both models and simulation scripts), I
feel that it is worthwhile discarding NS2 as a framework for this project.
Never-the-less; an investigation will be undertaken to make sure that NS3 can
handle the task (and I can handle NS3) to implement a basic MANET with
off-the-shelf marine physical characteristics.

Regardless of simulation platform adopted, it is essential that simulation is 
verified by physical experimentation on real platforms.

\section{Publication Plans}
From previous work with Prof Marshall and Dr McKinley, I would put together at
least one (possibly two) papers for submission on the subjects of: Memoisation
of Power Spectral Density (PSD) calculations for Level 2 Dynamic Spectrum Management
(DSM) applications for DSL networks, and possibly a paper on GPU acceleration
small-matrix solvers. 

Further it is expected that some papers should be produced on the topics of:
Integration of physical measurements into NS3 simulated environments,
Performance analysis of different Trust Management Systems, and ideally moving
towards a standardised protocol for integrating real-world Systems-of-systems
into a Dynamic Trust Management Framework, although it is not expected that 
this work will be completed in the first year.

\section{Potential Field Contributions}
The development of a robust, distributed trust management framework within the 
domain of AUV-fleet operations would provide the ability for such fleets to 
operate over much wider ranges of operation, for greatly extended periods of 
time than previously safe. 

The self-detection and classification of abnormal behaviour within a fleet in
the proposed distributed manner opens up the potential of a new range of secure
and self-learning distributed intrusion detection systems, with potential
applications both in marine and terrestrial contexts; If a general protocol 
could be generated for this problem, this could be applied to self-driving 
cars, environmental survey drones, satellite communications arrays,
Internet Certificate Authority verification, Distributed Computing 
applications, and many more fields.

\section{Project Schedule}
\includegraphics[scale=0.9]{projectdefinition/figures/Gannt_tmr.png}
\subsection{Detail on Selected Tasks}
\begin{itemize}
  \item \textbf{2:~Induction with DSTL Stakeholders} took place at DSTL Porton 
    Down in November, where the details, intentions, and political context of 
    the UK/FR PhD Programme were further outlaid. It also included 
    project-centred discussion and scope-clarification with the relevant 
    stakeholders.
  \item \textbf{3:~IRISSCON}; took place in Dublin, Ireland, and included in 
    depth discussion on the business and legal impacts of having data stored 
    and acted upon in a distributed fashion with third parties.
  \item \textbf{8:~IET UAV Conference} will take place in London, and will 
    cover a range of topics, mostly focused on the security and safety of 
    civilian UAV platforms and their communications, and the integration of 
    such systems into existing technical and regulatory frameworks
  \item \textbf{15,16:~Differentiation} will encompass a report on the progress 
    made during the first year of the project, including a review of relevant 
    literature, and more tightly specify the direction of continuing research.
\end{itemize}


% --------------------------------------------------------------
%:                  BACK MATTER: appendices, refs,..
% --------------------------------------------------------------

% the back matter: appendix and references close the thesis
\pagebreak
\appendix
\addtocontents{toc}{\protect\setcounter{tocdepth}{0}}
\addappheadtotoc            % Add appendix header to TOC (From appendix pkg)
\chapter{Reading Lists}
\section{Fully Reviewed}
See Appendix B for full reviews
\begin{itemize}
  \item ``A Survey of Practical Issues in Underwater Networks'' - \citet*{Partan2006}
  \item ``A Delay-Tolerant Network Architecture for Challenged Internets'' - \citet*{Fall2003}
  \item ``A Trust Management Framework for Detecting Malicious and Selfish
  Behaviour in Ad-Hoc Wireless Networks using Fuzzy Sets and Grey Theory'' - \citet*{Guo}
  \item ``Information Theoretic Framework of Trust Modeling and Evaluation for
  Ad-Hoc Networks'' - \citet*{Liu2006}
\end{itemize}

\section{Partly Reviewed}
These are documents that have been read but not written up yet, either due to time constraints or their contents not being applicable\footnote{Such as derivative works or exhibiting ``common knowledge''}.
\begin{itemize}
  \item ``Future Trust Management Frameworks for Mobile Ad-Hoc Networks''
  - \citet*{Li2008}
  \item ``An Entropy-based Trust Modeling and Evaluation for Wireless Sensor
  Networks'' - \citet*{Hongjun2008}
  \item ``A Review of Current Routing Protocols for Ad-Hoc Mobile Wireless 
  \item ``Classification and Review of Security Schemes in Mobile Computing'' 
    - \citet*{Kumar2010}
  \item ``Underwater Acoustic Sensor Networks: Research Challenges''
  - \citet*{Akyildiz2005}
  Communications using Acoustic Directional Transducers'' - \citet*{Chen}
  \item ``An Entropy-based Trust Modeling and Evaluation for Wireless Sensor
    Networks'' - \citet*{Hongjun2008}
  \item ``A Framework of Requirements for the Design and Management of 
    Dependable Network Enabled Capability System of Systems'' 
    - \citet*{Banks2010}
\end{itemize}

\section{Planned Reading}
\subsection{Marine Operational and Communications Considerations}

\begin{itemize}
  \item ``The Importance of Trust Between Operator and AUV: Crossing the
  Human/Computer Language Barrier'' - \citet*{Johnson2007}
  \item ``Cooperative Distributed Behaviours of an AUV Network for Asset
  Protection with Communication Constraints'' - \citet*{Caiti2011}
  \item ``Performance Limitations in underwater acoustic telemetry''
  - \citet*{Catipovic1990}
  \item ``A Multimedia Cross-Layer Protocol for Underwater Acoustic Sensor
    Networks''- \citet*{Pompili2010}
  \item ``Secure Cooperation of Mobile Sensors in an Underwater Acoustic
  Network'' - \citet*{Caiti}
  \item ``Slotted FAMA: a MAC protocol for underwater acoustic networks''
  - \citet*{Molins2006}
\end{itemize}

Potential Reading: \cite{Johnson2007} \cite{Freitag2004}

\subsection{Threat Modelling}

\begin{itemize}
  \item ``Denial of Service Attacks in Wireless Networks: The Case of Jammers'' - \citet*{Pelechrinis2011}
  \item ``Cross-Layer Based Anomaly Detection in Wireless Mesh Networks'' - \citet*{Wang2009}
  \item ``Evaluation of Intrusion Detection Systems Under a Resource Constraint'' - \citet*{Ryu2008}
\end{itemize}

\subsection{Trust and Reputation Management in MANET/DTN}

\begin{itemize}
  \item ``Trust Management in Distributed Systems'' - \citet*{Li2007}
  \item ``Introduction to Grey System Theory'' - \citet*{Deng1989}
  \item ``A Game-Theoretic Analysis of Trust Management in P2P Systems'' - \citet*{Tuan2006}
  \item ``Defense of trust management vulnerabilities in distributed networks'' - \citet*{Sun2008}
  \item ``A Framework of Requirements for the Design and Management of
    Dependable  Network Enabled Capability System of Systems'' - \citet*{Banks2010}
\end{itemize}

Potential Reading: \cite{Banks2010}, \cite{Fung2011}, \cite{Guan2008}, 
 \cite{Cai2009}

\ifpdf
  \chapter{Literature Reviews}
  \section{A Survey of Practical Issues in Underwater Networks
 \citet*{Partan2006}}
\label{Partan2006_gen}
\begin{itemize}
  \item Predominantly Medium Access Control (MAC) considerations of marine
  networks, technological, and economical.
  \item Classifies (marine) networks into 4 (or 5, counting null network)
  regimens; 
  \begin{itemize}
    \item Null Networks (Nodes too distant and immobile for communication)
    \item Disruption Tolerant Networks (DTN, sparse but sufficiently
    mobile networks)
    \item Unpartitioned, multi-hop networks (overlapping chains of TDMA/CDMA 
    clusters with MACA or S-FAMA \cite{Molins2006})
    \item Single-Hop TDMA networks (fully connected network coverage, with
    limited contention)
    \item Dense Single-Hop Network (ok for CSMA, not much else, bandwidth
    contention)
  \end{itemize}
  \item Summarises marine channel characterise, generally summed up as 'bad'
  \begin{itemize}
    \item \textbf{Acoustic}: Experiences Propagation Delays and
    Doppler Effects\footnote{Speed of sound in water is 1500m/s (depth
    variant)}, as well as limited bandwidth due to frequency dependant
    attenuation. Additionally, high \acro{BER} due to Phase and Amplitude
    variations, requiring \acro{FEC}. See \cite{Catipovic1990}
    \item \textbf{RF}: Poor but possible (122 kHZ, 6-10m, 1-8kbits/sec
    )
    \item \textbf{Optical}: Very poor but possible in \textit{extremely} clear
    environments (490-500 nm, \textless 100m, several Mbits/s)
    \item \textbf{IR}: potential to use OTS IrDA TX/RX for low cost,
    short range links (1-2m, 57.6kbits/s)
  \end{itemize}
  \item Includes several Operational Examples (4.1, 5.1, 5.2, 5.5) that will be
  useful later.
  \item Introduces LBL\footnote{Long Baseline, an acoustic positioning method
  conceptually similar to GPS} localisation and potential contention between
  Navigational and data comms (5.2)
  \item Introduces a variety of DTN packet exchange techniques for mobile
  networks (5.3,5.4)
\end{itemize}

\subsection{Questions Raised}
\begin{itemize}
  \item It is not explained or justified what causes the indicated phase and
  amplitude variations that lead to high \acro{BER}. Possibly covered in
  \cite{Catipovic1990}.
  \item In one operational example (MCM, 4.1), a central gateway buoy was used
  as a uplink service and hub for a star-network of AUVs. Depending on how
  DSTL/DGA want to pursue this, this demonstrates a necessary compromise
  between centrality for power and communications economy, and decentralised security and
  reliability.
\end{itemize}

  \section{A Delay-Tolerant Network Architecture for
Challenged Internets \citet*{Fall2003}}
\label{Fall2003_gen}
\begin{itemize}
  \item This paper deals primarily with the description of Challenged
  network environments, but proposes a solution that is unsuitable for the
  desired application; i.e. centralised DTN Bundle Gateway interconnects
  between networks.
  \item Such an implementation \emph{could} be factored into the generation of a
  truly decentralised system.
  \item Investigates in some detail the characteristics and difficulties of
  Challenged Networks; summarised here.
  \begin{itemize}
    \item Path and Link Characteristics
    \begin{itemize}
      \item \emph{High-Latency, Low Data Rate}
      \item \emph{Disconnection} caused by fault (power loss, tx failure) or
      non-fault (motion or low-duty-cycle) sources. Motion can be predictable or
      opportunistic.
      \item \emph{Long Queuing Times} caused by multi-hop with unknown and
      dynamic topology rendering source-initiated retransmission
      extremely expensive, requiring long packet retention times to negate.
	\end{itemize}
    \item Network Architecture Concerns
    \begin{itemize}
      \item \emph{Interoperability Considerations}; individual CN's not designed
      to a standard 'stack', and thus may fail to implement 'abstractions \ldots
      supporting layered protocol families'.
      \item \emph{Security}; E2E\footnote{End-to-End} security, usually requiring challenge/key
      exchange, is undesirable due to the long latencies and delay prone nature
      of the networks. Additionally, its a waste of data to carry it all the way
      to the destination for authentication/access only to be denied.
	\end{itemize}
    \item End-System Characteristics
    \begin{itemize}
      \item \emph{Limited Longevity}: Message transit may outlive their
      source-node, due to power, strategic, or environmental considerations.
      Fall et all suggest that message reliability monitoring should be
      delegated to a currently (and hopefully, future) operating node.
      \item \emph{Low-Duty-Cycle Operation}: Highlighted in low-power/long-life
      networks where nodes are regularly 'listening' and/or recording data to be
      relayed, and periodically (but more rarely) transmit. The performance of
      this kind of operation and network is dependant on predictable paths
      (leading to efficient time-dependant path selection) and a-priori
      scheduling.
      \item \emph{Limited Resources}: Highlights the decisions to be made with
      regards to memory resources, factoring in RTT, expected retransmissions,
      and maintaining a store of in-transit data.
	\end{itemize}
  \end{itemize}
  \item Additionally, this paper covers the potential use of Proxy or Gateway
  agents at network interface points, E-Mail-like async messaging as
  solutions to the above, before detailing a bundle-gateway based DTN
  architecture based on regional and sub-regional addressing between challenged
  networks, with QoS-like behaviour based on Postal Service conventions (return
  receipt/delivery record/'handle with care')
  \item On the subject of path selection this paper refers to another 2003
  paper\cite{Alonso2003a} which should be reviewed, but is quite technical from
  a cursory skim, so should be delayed.
  \item Introduces the idea of Persistent and Non Persistent Gateways, and
  discussed the effect on network reliability and architecture between these
  nodes.
  \item Deals with network convergence; not entirely relevant to the project at
  hand.
  \item Highlights the congestion difficulties of challenged networks (4.9), eg
  'contacts may not arrive for some time in the future' and 'adopted' packets,
  where a node is delegated custody the ensure that packets reception, which
  cannot be deleted except under extreme conditions. Currently tackled using a
  shared priority queue, but this introduces exploitable behaviour i.e. priority
  inversion / head-of-line blocking.
\end{itemize}

\subsection{Questions Raised}
\begin{itemize}
  \item Fall et al state that the problem of \emph{Security} wastage (i.e.
  carrying excess authentication data to a destination only to be denied) remains an open
  problem on the Internet. I believe that within the general scope of this
  project, such an authentication scheme would be decentralised and
  N2N \footnote{Node-to-node} rather than E2E. This would be an example of
  collaborative delegation, as discussed below.
  \item On the subject of \emph{Limited Longevity}, Fall et al. suggest message
  delivery acknowledgement be delegated to a 'surviving' node. This implies a
  centralised approach which is counter to the project aims. One could envisage
  a collaborative delegation system whereby a subset of nodes within a fleet
  (i.e, the local neighbours N\{\ldots\} of node X upon X's transmission of
  message M), such that any of N can authoritatively accept a delivery
  acknowledgement (although personally I don't think ARQ is suitable for this
  application due to data overheads and network disruption). Doing so would also
  alleviate the issues raised in \emph{Limited Resources}, as once the
  transmission has been spread to its neighbours, N can realistically drop that
  information and assume it will makes its way across the network. This N2N
  approach is tangentially covered in (4.5)
\end{itemize}

  \section{A Trust Management Framework for Detecting Malicious and Selfish Behaviour 
in Ad-Hoc Wireless Networks using Fuzzy Sets and Grey theory \citet*{Guo}}
\label{Guo}
\begin{itemize}
  \item Provides background on TMF
  \item References Policy Language, PKC\footnote{Public Key
  Cryptography}, Resurrecting Duckling Model and Distributed Trust Model
  \cite{Li2007}
  \item Raises need for TMF to mitigate selfishness
  \item Introduces Grey Theory From Deng Julong (Can't find any official
  citations for this, paper appears to be stuck in  Springerlink)
  \item Defines three types of trust relationship
  \begin{itemize}
    \item \emph{Direct}: Trust based on historical behaviour of node $B$ wrt
    node $A$.
    \item \emph{Indirect}: Trust transited through third-party entities, i.e
    $E$, $F$ wrt $B$ where neither communicate with $A$
    \item \emph{Recommendation}: a subjective trust transited through a common
    entity; i,e Trust($B\rightarrow C$) communicated to $A$ by $C$.
  \end{itemize}
  \item Highlights some of the potential attacks on TMF and their sources
  \cite{Sun2008}, \cite{Li2008}.
  \item Guo proposes a TMF leveraging the curve-fitting
  analytics of Grey Theory to allow in practical networks, multi-parametric
  based trust quantities, and to use this data and results to assess the type
  of selfish behaviour being exhibited by 'bad' nodes.
  \item Classical TMF only monitor single behaviours: probability of successful
  interactions (Bit Error Rate/Packet Loss Rate).
  The TMF suggested by Guo incorporates signal strength, data-rate and other
  physical factors in addition to PLR.
  \item The application of Grey Whitenisation and clustering allows for
  quantifiable Trustworthiness classification, i.e taking the multi-parametric
  measurements of behaviour and condensing this to a per-node wrt calculating
  node trust assessment.
  \item Includes worked simulation examples of a 6 node network
\end{itemize}

\subsection{Questions Raised}
\begin{itemize}
  \item Stated definitions of Direct, Indirect, and Recommendation trust do not
  suitably distinguish between Indirect/Recommendation. Hopefully after finding
  a copy of Julong this is be made clearer.
  \item Grey Theory seems mathematically intuitive but the fundamental question
    is the selection of \emph{distinguishing coefficient} values \cite{Cai2009}, and
  whether these values can be collaboratively 'learned' over time, almost like familial
  and social trust.
  \item Additionally, the derivation of Whitinisation functions is not at all
  clear.
  \item Generally: What is the implication for using this style of trust
  assessment against human operators, remotely or in physical nodes (eg
  mother-ships)? Will lapses in judgement be detrimentally held against an
  operator? Can a TMF as suggested be modified to accept human frailties (and
  should it!)?
\end{itemize}

  \section{Poster Abstract: Secure Cooperation of Mobile Sensors in an Underwater 
Acoustic Network \citet*{Caiti}}
\label{Caiti}
\begin{itemize}
  \item Covers methodologies and algorithms for protection-oriented operation of an asset by a team of AUVs in a UAN.
  \item References \cite{Curtin1993} regarding goal oriented operation.
  \item References \cite{Akyildiz2005}/\cite{Caitia} regarding channel state
  \item Takes an Application-level approach to communication; i.e. AUVs guided by an interest function which factors in the protection of an asset with maintaining stable comms.
  \item Refers to ReKeying service for group key revocation and generation (\cite{Dini}).
  \item Refers to CTS Encryption, maintaining ciphertext/plaintext size.
  \item Highlights DoS danger within Marine channel, states that derived behaviour will drive any nodes 'under attack' closer to their protection target naturally.
\end{itemize}

\subsection{Questions Raised}
\begin{itemize}
  \item Does not comment on how RKS revocations are triggered. How are attacks recognised?
  \item Can similar emergant application level (i.e motive) behaviours be applied to the physical/logical levels so as to implement a few very simple rules that, when scaled, provide stable, robust behaviours?
\end{itemize}

  \section{Information theoretic framework of trust modeling and evaluation for
ad hoc networks \citet*{Liu2006}}
\label{Liu2006_gen}
\begin{itemize}
  \item This paper develops an information theoretic understanding of single and
    multi-path trust modelling, providing a solid grounding in the mathematical
    concept of trust from axiomatic foundations.
  \item As this is the first Journal paper read, it's taken a while to get my
    head around it. As such, the review is separated by chapter.
  \item 
    \begin{enumerate}
      \item \emph{Introduction}
        \begin{itemize}
          \item Highlights the levels of challenges within the field, particularly
            the axiomatic \emph{Trust Definition} (Reputation of entity, Trust-Of-Opinion,
            Probability of Action), the generation of \emph{Trust Metrics}
            (linguistic applications such as PGP, PolicyMaker, DTM, TPL, SPKI/SDSI, or
            numerical solutions, demonstrated in
            \cite{Abdul-Rahman1997}
            %\cite{Jøsang1999}
            \cite{Theodorakopoulos2004}\cite{Maurer1996})
            and the generation of \emph{Quantative Trust Models}
            (\emph{Subjective Logics} from linguistics, \emph{Fuzzy Logic}, or
            \emph{Bayesian modeling})
          \item From \cite{Gambetta2000}, states that ``Trust is a level of likelihood with which an
            agent will perform a particular action before such action can be monitored
            and in a context in which it affects our own actions''
          \item Leads on to propose that trust is a measure of uncertainty, and as
            such, can be measured by entropy.
          \item Aim to develop a distributed scheme to build, maintain, and update
            trust records in ad hoc networks, where trust records are used to assist
            dynamic route selection and to perform malicious node detection.
          \item Aim to highlight potential attack vectors, including presenting a new
            strategy.
        \end{itemize}
      \item \emph{Basic Axioms}\label{liu2006_basicaxioms}
        \begin{enumerate}
          \item \label{Axiom1} \emph{Axiom 1} \textbf{Uncertainty is a Measure of Trust} : ``Let \(T\{S:A,a\}\)
            denote the trust value for relationship \(\{S:A,a\}\)\footnote{\(S\mapsto subject,
            A\mapsto Agent, a\mapsto Action\)} and \(P\{S:A,a\}\) denote the
            probability that the agent will perform the action from the subjects point
            of view''. From information theory; entropy is the nature measure of
            uncertainty, and that leads to an entropy-based trust value.  

            \begin{equation}
              T\{S:A,a\} =\begin{array}{cc} 
                1 - H(p), & \mbox{for } 0.5\leq p \leq 1 \\
                H(p) - 1, & \mbox{for } 0\leq p \le 0.5 \\  	  
              \end{array}
              \label{liu2006_entropy_based_trust_value-equ}
            \end{equation}

            Where \(H(p) = -p \log_2(p) - (1-p)\log_2(1-p)\) and \(p\mapsto
            P\{S:A,a\}\)

            \begin{wrapfigure}{R}{0.45\textwidth}
              % GNUPLOT: LaTeX picture
\setlength{\unitlength}{0.240900pt}
\ifx\plotpoint\undefined\newsavebox{\plotpoint}\fi
\sbox{\plotpoint}{\rule[-0.200pt]{0.400pt}{0.400pt}}%
\begin{picture}(750,450)(0,0)
\sbox{\plotpoint}{\rule[-0.200pt]{0.400pt}{0.400pt}}%
\put(171.0,131.0){\rule[-0.200pt]{4.818pt}{0.400pt}}
\put(151,131){\makebox(0,0)[r]{-1}}
\put(669.0,131.0){\rule[-0.200pt]{4.818pt}{0.400pt}}
\put(171.0,201.0){\rule[-0.200pt]{4.818pt}{0.400pt}}
\put(151,201){\makebox(0,0)[r]{-0.5}}
\put(669.0,201.0){\rule[-0.200pt]{4.818pt}{0.400pt}}
\put(171.0,271.0){\rule[-0.200pt]{4.818pt}{0.400pt}}
\put(151,271){\makebox(0,0)[r]{ 0}}
\put(669.0,271.0){\rule[-0.200pt]{4.818pt}{0.400pt}}
\put(171.0,340.0){\rule[-0.200pt]{4.818pt}{0.400pt}}
\put(151,340){\makebox(0,0)[r]{ 0.5}}
\put(669.0,340.0){\rule[-0.200pt]{4.818pt}{0.400pt}}
\put(171.0,410.0){\rule[-0.200pt]{4.818pt}{0.400pt}}
\put(151,410){\makebox(0,0)[r]{ 1}}
\put(669.0,410.0){\rule[-0.200pt]{4.818pt}{0.400pt}}
\put(171.0,131.0){\rule[-0.200pt]{0.400pt}{4.818pt}}
\put(171,90){\makebox(0,0){ 0}}
\put(171.0,390.0){\rule[-0.200pt]{0.400pt}{4.818pt}}
\put(301.0,131.0){\rule[-0.200pt]{0.400pt}{4.818pt}}
\put(301,90){\makebox(0,0){ 0.25}}
\put(301.0,390.0){\rule[-0.200pt]{0.400pt}{4.818pt}}
\put(430.0,131.0){\rule[-0.200pt]{0.400pt}{4.818pt}}
\put(430,90){\makebox(0,0){ 0.5}}
\put(430.0,390.0){\rule[-0.200pt]{0.400pt}{4.818pt}}
\put(560.0,131.0){\rule[-0.200pt]{0.400pt}{4.818pt}}
\put(560,90){\makebox(0,0){ 0.75}}
\put(560.0,390.0){\rule[-0.200pt]{0.400pt}{4.818pt}}
\put(689.0,131.0){\rule[-0.200pt]{0.400pt}{4.818pt}}
\put(689,90){\makebox(0,0){ 1}}
\put(689.0,390.0){\rule[-0.200pt]{0.400pt}{4.818pt}}
\put(171.0,271.0){\rule[-0.200pt]{124.786pt}{0.400pt}}
\put(171.0,131.0){\rule[-0.200pt]{0.400pt}{67.211pt}}
\put(171.0,131.0){\rule[-0.200pt]{124.786pt}{0.400pt}}
\put(689.0,131.0){\rule[-0.200pt]{0.400pt}{67.211pt}}
\put(171.0,410.0){\rule[-0.200pt]{124.786pt}{0.400pt}}
\put(30,270){\makebox(0,0){$T(S:A,a)$}}
\put(430,29){\makebox(0,0){$P(S:A,a)$}}
\put(176,142){\usebox{\plotpoint}}
\multiput(176.59,142.00)(0.477,0.933){7}{\rule{0.115pt}{0.820pt}}
\multiput(175.17,142.00)(5.000,7.298){2}{\rule{0.400pt}{0.410pt}}
\multiput(181.59,151.00)(0.482,0.581){9}{\rule{0.116pt}{0.567pt}}
\multiput(180.17,151.00)(6.000,5.824){2}{\rule{0.400pt}{0.283pt}}
\multiput(187.59,158.00)(0.477,0.710){7}{\rule{0.115pt}{0.660pt}}
\multiput(186.17,158.00)(5.000,5.630){2}{\rule{0.400pt}{0.330pt}}
\multiput(192.59,165.00)(0.477,0.599){7}{\rule{0.115pt}{0.580pt}}
\multiput(191.17,165.00)(5.000,4.796){2}{\rule{0.400pt}{0.290pt}}
\multiput(197.59,171.00)(0.477,0.599){7}{\rule{0.115pt}{0.580pt}}
\multiput(196.17,171.00)(5.000,4.796){2}{\rule{0.400pt}{0.290pt}}
\multiput(202.00,177.59)(0.599,0.477){7}{\rule{0.580pt}{0.115pt}}
\multiput(202.00,176.17)(4.796,5.000){2}{\rule{0.290pt}{0.400pt}}
\multiput(208.59,182.00)(0.477,0.599){7}{\rule{0.115pt}{0.580pt}}
\multiput(207.17,182.00)(5.000,4.796){2}{\rule{0.400pt}{0.290pt}}
\multiput(213.00,188.60)(0.627,0.468){5}{\rule{0.600pt}{0.113pt}}
\multiput(213.00,187.17)(3.755,4.000){2}{\rule{0.300pt}{0.400pt}}
\multiput(218.00,192.59)(0.487,0.477){7}{\rule{0.500pt}{0.115pt}}
\multiput(218.00,191.17)(3.962,5.000){2}{\rule{0.250pt}{0.400pt}}
\multiput(223.00,197.60)(0.774,0.468){5}{\rule{0.700pt}{0.113pt}}
\multiput(223.00,196.17)(4.547,4.000){2}{\rule{0.350pt}{0.400pt}}
\multiput(229.00,201.60)(0.627,0.468){5}{\rule{0.600pt}{0.113pt}}
\multiput(229.00,200.17)(3.755,4.000){2}{\rule{0.300pt}{0.400pt}}
\multiput(234.00,205.60)(0.627,0.468){5}{\rule{0.600pt}{0.113pt}}
\multiput(234.00,204.17)(3.755,4.000){2}{\rule{0.300pt}{0.400pt}}
\multiput(239.00,209.60)(0.627,0.468){5}{\rule{0.600pt}{0.113pt}}
\multiput(239.00,208.17)(3.755,4.000){2}{\rule{0.300pt}{0.400pt}}
\multiput(244.00,213.60)(0.627,0.468){5}{\rule{0.600pt}{0.113pt}}
\multiput(244.00,212.17)(3.755,4.000){2}{\rule{0.300pt}{0.400pt}}
\multiput(249.00,217.61)(1.132,0.447){3}{\rule{0.900pt}{0.108pt}}
\multiput(249.00,216.17)(4.132,3.000){2}{\rule{0.450pt}{0.400pt}}
\multiput(255.00,220.61)(0.909,0.447){3}{\rule{0.767pt}{0.108pt}}
\multiput(255.00,219.17)(3.409,3.000){2}{\rule{0.383pt}{0.400pt}}
\multiput(260.00,223.61)(0.909,0.447){3}{\rule{0.767pt}{0.108pt}}
\multiput(260.00,222.17)(3.409,3.000){2}{\rule{0.383pt}{0.400pt}}
\multiput(265.00,226.61)(0.909,0.447){3}{\rule{0.767pt}{0.108pt}}
\multiput(265.00,225.17)(3.409,3.000){2}{\rule{0.383pt}{0.400pt}}
\multiput(270.00,229.61)(1.132,0.447){3}{\rule{0.900pt}{0.108pt}}
\multiput(270.00,228.17)(4.132,3.000){2}{\rule{0.450pt}{0.400pt}}
\multiput(276.00,232.61)(0.909,0.447){3}{\rule{0.767pt}{0.108pt}}
\multiput(276.00,231.17)(3.409,3.000){2}{\rule{0.383pt}{0.400pt}}
\multiput(281.00,235.61)(0.909,0.447){3}{\rule{0.767pt}{0.108pt}}
\multiput(281.00,234.17)(3.409,3.000){2}{\rule{0.383pt}{0.400pt}}
\put(286,238.17){\rule{1.100pt}{0.400pt}}
\multiput(286.00,237.17)(2.717,2.000){2}{\rule{0.550pt}{0.400pt}}
\put(291,240.17){\rule{1.300pt}{0.400pt}}
\multiput(291.00,239.17)(3.302,2.000){2}{\rule{0.650pt}{0.400pt}}
\multiput(297.00,242.61)(0.909,0.447){3}{\rule{0.767pt}{0.108pt}}
\multiput(297.00,241.17)(3.409,3.000){2}{\rule{0.383pt}{0.400pt}}
\put(302,245.17){\rule{1.100pt}{0.400pt}}
\multiput(302.00,244.17)(2.717,2.000){2}{\rule{0.550pt}{0.400pt}}
\put(307,247.17){\rule{1.100pt}{0.400pt}}
\multiput(307.00,246.17)(2.717,2.000){2}{\rule{0.550pt}{0.400pt}}
\put(312,249.17){\rule{1.300pt}{0.400pt}}
\multiput(312.00,248.17)(3.302,2.000){2}{\rule{0.650pt}{0.400pt}}
\put(318,251.17){\rule{1.100pt}{0.400pt}}
\multiput(318.00,250.17)(2.717,2.000){2}{\rule{0.550pt}{0.400pt}}
\put(323,252.67){\rule{1.204pt}{0.400pt}}
\multiput(323.00,252.17)(2.500,1.000){2}{\rule{0.602pt}{0.400pt}}
\put(328,254.17){\rule{1.100pt}{0.400pt}}
\multiput(328.00,253.17)(2.717,2.000){2}{\rule{0.550pt}{0.400pt}}
\put(333,256.17){\rule{1.100pt}{0.400pt}}
\multiput(333.00,255.17)(2.717,2.000){2}{\rule{0.550pt}{0.400pt}}
\put(338,257.67){\rule{1.445pt}{0.400pt}}
\multiput(338.00,257.17)(3.000,1.000){2}{\rule{0.723pt}{0.400pt}}
\put(344,258.67){\rule{1.204pt}{0.400pt}}
\multiput(344.00,258.17)(2.500,1.000){2}{\rule{0.602pt}{0.400pt}}
\put(349,260.17){\rule{1.100pt}{0.400pt}}
\multiput(349.00,259.17)(2.717,2.000){2}{\rule{0.550pt}{0.400pt}}
\put(354,261.67){\rule{1.204pt}{0.400pt}}
\multiput(354.00,261.17)(2.500,1.000){2}{\rule{0.602pt}{0.400pt}}
\put(359,262.67){\rule{1.445pt}{0.400pt}}
\multiput(359.00,262.17)(3.000,1.000){2}{\rule{0.723pt}{0.400pt}}
\put(365,263.67){\rule{1.204pt}{0.400pt}}
\multiput(365.00,263.17)(2.500,1.000){2}{\rule{0.602pt}{0.400pt}}
\put(370,264.67){\rule{1.204pt}{0.400pt}}
\multiput(370.00,264.17)(2.500,1.000){2}{\rule{0.602pt}{0.400pt}}
\put(375,265.67){\rule{1.204pt}{0.400pt}}
\multiput(375.00,265.17)(2.500,1.000){2}{\rule{0.602pt}{0.400pt}}
\put(380,266.67){\rule{1.445pt}{0.400pt}}
\multiput(380.00,266.17)(3.000,1.000){2}{\rule{0.723pt}{0.400pt}}
\put(391,267.67){\rule{1.204pt}{0.400pt}}
\multiput(391.00,267.17)(2.500,1.000){2}{\rule{0.602pt}{0.400pt}}
\put(386.0,268.0){\rule[-0.200pt]{1.204pt}{0.400pt}}
\put(401,268.67){\rule{1.204pt}{0.400pt}}
\multiput(401.00,268.17)(2.500,1.000){2}{\rule{0.602pt}{0.400pt}}
\put(396.0,269.0){\rule[-0.200pt]{1.204pt}{0.400pt}}
\put(427,269.67){\rule{1.445pt}{0.400pt}}
\multiput(427.00,269.17)(3.000,1.000){2}{\rule{0.723pt}{0.400pt}}
\put(406.0,270.0){\rule[-0.200pt]{5.059pt}{0.400pt}}
\put(454,270.67){\rule{1.204pt}{0.400pt}}
\multiput(454.00,270.17)(2.500,1.000){2}{\rule{0.602pt}{0.400pt}}
\put(433.0,271.0){\rule[-0.200pt]{5.059pt}{0.400pt}}
\put(464,271.67){\rule{1.204pt}{0.400pt}}
\multiput(464.00,271.17)(2.500,1.000){2}{\rule{0.602pt}{0.400pt}}
\put(459.0,272.0){\rule[-0.200pt]{1.204pt}{0.400pt}}
\put(474,272.67){\rule{1.445pt}{0.400pt}}
\multiput(474.00,272.17)(3.000,1.000){2}{\rule{0.723pt}{0.400pt}}
\put(480,273.67){\rule{1.204pt}{0.400pt}}
\multiput(480.00,273.17)(2.500,1.000){2}{\rule{0.602pt}{0.400pt}}
\put(485,274.67){\rule{1.204pt}{0.400pt}}
\multiput(485.00,274.17)(2.500,1.000){2}{\rule{0.602pt}{0.400pt}}
\put(490,275.67){\rule{1.204pt}{0.400pt}}
\multiput(490.00,275.17)(2.500,1.000){2}{\rule{0.602pt}{0.400pt}}
\put(495,276.67){\rule{1.445pt}{0.400pt}}
\multiput(495.00,276.17)(3.000,1.000){2}{\rule{0.723pt}{0.400pt}}
\put(501,277.67){\rule{1.204pt}{0.400pt}}
\multiput(501.00,277.17)(2.500,1.000){2}{\rule{0.602pt}{0.400pt}}
\put(506,279.17){\rule{1.100pt}{0.400pt}}
\multiput(506.00,278.17)(2.717,2.000){2}{\rule{0.550pt}{0.400pt}}
\put(511,280.67){\rule{1.204pt}{0.400pt}}
\multiput(511.00,280.17)(2.500,1.000){2}{\rule{0.602pt}{0.400pt}}
\put(516,281.67){\rule{1.445pt}{0.400pt}}
\multiput(516.00,281.17)(3.000,1.000){2}{\rule{0.723pt}{0.400pt}}
\put(522,283.17){\rule{1.100pt}{0.400pt}}
\multiput(522.00,282.17)(2.717,2.000){2}{\rule{0.550pt}{0.400pt}}
\put(527,285.17){\rule{1.100pt}{0.400pt}}
\multiput(527.00,284.17)(2.717,2.000){2}{\rule{0.550pt}{0.400pt}}
\put(532,286.67){\rule{1.204pt}{0.400pt}}
\multiput(532.00,286.17)(2.500,1.000){2}{\rule{0.602pt}{0.400pt}}
\put(537,288.17){\rule{1.100pt}{0.400pt}}
\multiput(537.00,287.17)(2.717,2.000){2}{\rule{0.550pt}{0.400pt}}
\put(542,290.17){\rule{1.300pt}{0.400pt}}
\multiput(542.00,289.17)(3.302,2.000){2}{\rule{0.650pt}{0.400pt}}
\put(548,292.17){\rule{1.100pt}{0.400pt}}
\multiput(548.00,291.17)(2.717,2.000){2}{\rule{0.550pt}{0.400pt}}
\put(553,294.17){\rule{1.100pt}{0.400pt}}
\multiput(553.00,293.17)(2.717,2.000){2}{\rule{0.550pt}{0.400pt}}
\multiput(558.00,296.61)(0.909,0.447){3}{\rule{0.767pt}{0.108pt}}
\multiput(558.00,295.17)(3.409,3.000){2}{\rule{0.383pt}{0.400pt}}
\put(563,299.17){\rule{1.300pt}{0.400pt}}
\multiput(563.00,298.17)(3.302,2.000){2}{\rule{0.650pt}{0.400pt}}
\put(569,301.17){\rule{1.100pt}{0.400pt}}
\multiput(569.00,300.17)(2.717,2.000){2}{\rule{0.550pt}{0.400pt}}
\multiput(574.00,303.61)(0.909,0.447){3}{\rule{0.767pt}{0.108pt}}
\multiput(574.00,302.17)(3.409,3.000){2}{\rule{0.383pt}{0.400pt}}
\multiput(579.00,306.61)(0.909,0.447){3}{\rule{0.767pt}{0.108pt}}
\multiput(579.00,305.17)(3.409,3.000){2}{\rule{0.383pt}{0.400pt}}
\multiput(584.00,309.61)(1.132,0.447){3}{\rule{0.900pt}{0.108pt}}
\multiput(584.00,308.17)(4.132,3.000){2}{\rule{0.450pt}{0.400pt}}
\multiput(590.00,312.61)(0.909,0.447){3}{\rule{0.767pt}{0.108pt}}
\multiput(590.00,311.17)(3.409,3.000){2}{\rule{0.383pt}{0.400pt}}
\multiput(595.00,315.61)(0.909,0.447){3}{\rule{0.767pt}{0.108pt}}
\multiput(595.00,314.17)(3.409,3.000){2}{\rule{0.383pt}{0.400pt}}
\multiput(600.00,318.61)(0.909,0.447){3}{\rule{0.767pt}{0.108pt}}
\multiput(600.00,317.17)(3.409,3.000){2}{\rule{0.383pt}{0.400pt}}
\multiput(605.00,321.61)(1.132,0.447){3}{\rule{0.900pt}{0.108pt}}
\multiput(605.00,320.17)(4.132,3.000){2}{\rule{0.450pt}{0.400pt}}
\multiput(611.00,324.60)(0.627,0.468){5}{\rule{0.600pt}{0.113pt}}
\multiput(611.00,323.17)(3.755,4.000){2}{\rule{0.300pt}{0.400pt}}
\multiput(616.00,328.60)(0.627,0.468){5}{\rule{0.600pt}{0.113pt}}
\multiput(616.00,327.17)(3.755,4.000){2}{\rule{0.300pt}{0.400pt}}
\multiput(621.00,332.60)(0.627,0.468){5}{\rule{0.600pt}{0.113pt}}
\multiput(621.00,331.17)(3.755,4.000){2}{\rule{0.300pt}{0.400pt}}
\multiput(626.00,336.60)(0.627,0.468){5}{\rule{0.600pt}{0.113pt}}
\multiput(626.00,335.17)(3.755,4.000){2}{\rule{0.300pt}{0.400pt}}
\multiput(631.00,340.60)(0.774,0.468){5}{\rule{0.700pt}{0.113pt}}
\multiput(631.00,339.17)(4.547,4.000){2}{\rule{0.350pt}{0.400pt}}
\multiput(637.00,344.59)(0.487,0.477){7}{\rule{0.500pt}{0.115pt}}
\multiput(637.00,343.17)(3.962,5.000){2}{\rule{0.250pt}{0.400pt}}
\multiput(642.00,349.60)(0.627,0.468){5}{\rule{0.600pt}{0.113pt}}
\multiput(642.00,348.17)(3.755,4.000){2}{\rule{0.300pt}{0.400pt}}
\multiput(647.59,353.00)(0.477,0.599){7}{\rule{0.115pt}{0.580pt}}
\multiput(646.17,353.00)(5.000,4.796){2}{\rule{0.400pt}{0.290pt}}
\multiput(652.00,359.59)(0.599,0.477){7}{\rule{0.580pt}{0.115pt}}
\multiput(652.00,358.17)(4.796,5.000){2}{\rule{0.290pt}{0.400pt}}
\multiput(658.59,364.00)(0.477,0.599){7}{\rule{0.115pt}{0.580pt}}
\multiput(657.17,364.00)(5.000,4.796){2}{\rule{0.400pt}{0.290pt}}
\multiput(663.59,370.00)(0.477,0.599){7}{\rule{0.115pt}{0.580pt}}
\multiput(662.17,370.00)(5.000,4.796){2}{\rule{0.400pt}{0.290pt}}
\multiput(668.59,376.00)(0.477,0.710){7}{\rule{0.115pt}{0.660pt}}
\multiput(667.17,376.00)(5.000,5.630){2}{\rule{0.400pt}{0.330pt}}
\multiput(673.59,383.00)(0.482,0.581){9}{\rule{0.116pt}{0.567pt}}
\multiput(672.17,383.00)(6.000,5.824){2}{\rule{0.400pt}{0.283pt}}
\multiput(679.59,390.00)(0.477,0.933){7}{\rule{0.115pt}{0.820pt}}
\multiput(678.17,390.00)(5.000,7.298){2}{\rule{0.400pt}{0.410pt}}
\put(469.0,273.0){\rule[-0.200pt]{1.204pt}{0.400pt}}
\put(171.0,131.0){\rule[-0.200pt]{0.400pt}{67.211pt}}
\put(171.0,131.0){\rule[-0.200pt]{124.786pt}{0.400pt}}
\put(689.0,131.0){\rule[-0.200pt]{0.400pt}{67.211pt}}
\put(171.0,410.0){\rule[-0.200pt]{124.786pt}{0.400pt}}
\end{picture}

              \caption{Plot of equ (\protect
              \ref{liu2006_entropy_based_trust_value-equ}), demonstrating Entropy Based
              Trust}
            \end{wrapfigure}
          \item \label{Axiom2} \emph{Axiom 2} \textbf{Concatenation Propagation of Trust does not increase Trust} :
            From information theory; information cannot be increased through
            propagation. Mathematically summarised in
            \equref{liu2006_concatenation_propagation-equ};
            \begin{equation}
              |T_{AC}| \leq \min(|R_{AB}|,|T_{BC}|)
              \label{liu2006_concatenation_propagation-equ}
            \end{equation}
            Where 
            \begin{equation}
              \begin{array}{l} T_{AC}\mapsto T\{A:C, action\},\\
                R_{AB}\mapsto T\{A:B, recommend\},\\ 
                T_{BC}\mapsto T\{B:C, action\}
              \end{array}
            \end{equation}

            Or Graphically in \figref{liu2006_concatenation_propagation-dot}.
            \begin{figure}[c!]
              \input{litreviews/figures/liu2006_concatenation_propagation.dot.tex}
              \caption{Graph of three node network showing Concatenation Propagation of
              Trust}
              \label{liu2006_concatenation_propagation-dot}
            \end{figure}
          \item \label{Axiom3} \emph{Axiom 3} \textbf{Multipath Propagation of Trust does not reduce Trust} :
            Multipath recommendations will not increase the uncertainty of a
            resultant trust metric. In a simple network, the establishment of
            trust between node \(A\) and node \(C\) can be through either a
            single concatenation (see
            \figref{liu2006_multipath_concatenation-dots_A}) or through two
            paths (see \figref{liu2006_multipath_concatenation-dots_B})

            Letting \(T_{A_1C_1}=T\{A_1 : C_1, action\}\), \(T_{A_2C_2}=T\{A_2:C_2,
            action\}\), \(R_{2_B}=R_{2_D}\), and \(T_{2_B}=T_{2_D}\) the constraint on multipath uncertainty can be shown in
            \equref{liu2006_multipath_concatenation-equ}.

            \begin{equation}
              \begin{array}{l}
                T_{A_2C_2} \geq T_{A_1C_1} \geq 0, \mbox{ for } R_1 > 0 \mbox{ and
                } T_{2} \geq 0 \\
                T_{A_2C_2} \leq T_{A_1C_1} \leq 0, \mbox{ for } R_1 > 0 \mbox{ and
                } T_{2} \leq 0
              \end{array}
              \label{liu2006_multipath_concatenation-equ}.
            \end{equation}
            \begin{figure}[H!]
              \subfloat[Single Recommendation Path]{
              \label{liu2006_multipath_concatenation-dots_A}
              \input{litreviews/figures/liu2006_multipath_concatenation_A.dot.tex}
              }
              \subfloat[Multipath Recommendation]{
              \label{liu2006_multipath_concatenation-dots_B}
              \input{litreviews/figures/liu2006_multipath_concatenation_B.dot.tex}
              }
              \caption{Combining  multiple recommendation paths}
              \label{liu2006_multipath_concatenation-dots}
            \end{figure}
          \item \label{Axiom4} \emph{Axiom 4} \textbf{Trust based on multiple
            recommendations from a single source should not be higher than that
            from independent sources}: The counterpart to Axiom 3 (\ref{Axiom3}), which
            sets a lower limit on multi-path recommendation, this states that
            the ``trust built on \ldots correlated recommendations [from a
            single source] should not be higher than the trust built upon
            recommendations from independent sources''.
            Letting \(T_{A_1C_1}=T\{A_1 : C_1, action\}\), \(T_{A_2C_2}=T\{A_2:C_2,
            action\} \), the constraint on multipath uncertainty is shown in
            \equref{liu2006_multipath_propagation-equ}.

            \begin{equation}
              \begin{array}{l}
                T_{A_2C_2} \geq T_{A_1C_1} \geq 0, \mbox{ if } T_{A_1C_1} \geq 0 \\
                T_{A_2C_2} \leq T_{A_1C_1} \leq 0, \mbox{ if } T_{A_1C_1} < 0
              \end{array}
              \label{liu2006_multipath_propagation-equ}.
            \end{equation}

            \begin{figure}[H!]
              \subfloat[Recommendation from single correlated source]{
              \label{liu2006_multipath_propagation-dots_A}
              \input{litreviews/figures/liu2006_multipath_propagation_A.dot.tex}
              }
              \subfloat[Recommendation from independent sources]{
              \label{liu2006_multipath_propagation-dots_B}
              \input{litreviews/figures/liu2006_multipath_propagation_B.dot.tex}
              }
              \caption{Single entity providing multiple recommendation paths}
              \label{liu2006_multipath_propagation-dots}
            \end{figure}
        \end{enumerate}
      \item \emph{Trust Models}
      Identifies, derives, and proves that Entropy and Probability based trust
      models satisfy the stated axioms. Summarising based on
      \figref{liu2006_multipath_concatenation-dots} and
      \figref{liu2006_multipath_propagation-dots}.
      \begin{enumerate}
        \item \emph{Entropy-Based}:
        \begin{itemize}
          \item Single-Chain Trust (as in
            \figref{liu2006_multipath_concatenation-dots_A})
            \begin{equation}
              T_{ABC}=R_{AB}T_{BC}
            \end{equation}
          \item Multiple Paths (as in
            \figref{liu2006_multipath_concatenation-dots_B}), use maximal ratio
            combination.
            \begin{equation}
              T\{A:C,action\}= w_1(R_{AB}T_{BC})+w_2(R_{AD}T_{DC})
            \end{equation}
            Where
            \begin{equation}
              w_1=\frac{R_{AB}}{R_{AB}+R_{AD}}
              \mbox{ and } w_2=\frac{R_{AD}}{R_{AB}+R_{AD}}
            \end{equation}
          \item Independent Paths (as in
            \figref{liu2006_multipath_propagation-dots_B}),
            % TODO Need to read \cite{Hongjun2008} first
        \end{itemize}
      \end{enumerate}




      \item \emph{Trust Establishment Based On Observation}
      \begin{itemize}
        \item Presents two approaches; using a minimum variance unbiased estimation for $\theta$ = $P{A:X,a}_{i+1}$\footnote{The probability that a requested action will be performed the next time it is ordered} and one to estimate $Pr(V(N+1)=1|n(N)=k)$. 
        \begin{equation}
          \hat{\theta}=\frac{k}{N}
        \end{equation}
        This simple version does not require knowledge of the distribution of $\theta$, but does not represent the uncertainty of the trust of $A$ (i.e where $N$ is very small).
        \item The second approach (derived from Bayesian Probability Theory) presents a much more representative observation for very small cases (i.e k=2,N=3, $\theta$=1/2)
        \begin{equation}
          Pr(V(N+1)=1|n(N)=k) = \frac{k+1}{N+2}
        \end{equation}
        \item This is expanded to include a time series analysis of historical data (i.e integrating a 'remembering factor' $\beta$) 
        \begin{equation}
          Pr(V(N+1)=1|n(N)=k)=\frac{1+ \sum_{j=1}^I \beta^{t_c-t_j} k_j}{2+ \sum_{j=1}^I \beta^{t_c-t_j} N_j}
        \end{equation}
        where $t_c$ is the current time, $t_j$ is the timecode of the $j$th
        observation. $beta$ should be selected with knowledge of the expected
        speed of behaviour change. 
      \end{itemize}
      \item \emph{Security In Ad Hoc Network Routing}
          %TODO could be useful later to go into more detail
          The paper then goes on to detail proposed schemes for Sending, and Responding to TRRs\footnote{Trust Recommendation Requests}, the detail of which is not massively relevant at this stage.
          Also details a scheme for updating internal trust records based on three 'databases', namely the 'trust record', 'recommendation buffer' and 'observation buffer'.

          \emph{The Trust Record} for node $A$ contains entries in the format $\{\text{subject, agent, action, T, P, }t_{est}\}$, where the subject field is always $A$, and as such represents $A$'s trust of the nodes around it based \emph{solely} on direct interaction.

          \emph{The Recommendation Buffer} has the same format as the Trust Record, but is based on other nodes interactions with agents, i.e the $\text{subject}$ field is the node reporting the recommendation to $A$. 

          \emph{The Observation Buffer} simply buffers observations until they can be processed and the Trust Record can be updated.

      \item \emph{Simulations}
        Demonstrates:
        \begin{itemize}
          \item The proposed scheme can identify malicious nodes, highlighting that this identification is not in the form of tight clusters due in part that the simulated malicious nodes had a range of behaviours, and more importantly, that \emph{relatively slight} changes in the trust records mean that the malicious nodes are interacted with much less that 'good' nodes, so those 'good' nodes have less 'bad' experiences (which is naturally desirable).
          \item Selectively Malicious behaviour can 'taint' the trust (of recommendation, not action) between otherwise good nodes, but not nearly to an extent as to whitewash the malicious behaviour of the nodes. Interestingly, the 'swarm' adapts to this coordinated attack (i.e the malicious nodes performed grey-hole attacks on 50\% of the swarm) so as to maintain overall throughput.
          \item The proposed scheme is adaptive to changing behaviour; i.e. Malicious 'infection' of otherwise good nodes over time.
          \item Both the Probability and Entropy based models perform very
            similarly, maintaining a average packer drop ratio around 40\% of the baseline, even before malicious node detection has 'activated' (i.e. sufficiently clustered malicious nodes within the trust-space)
        \end{itemize}
        \emph{Implementation Notes}: Nodes move randomly within a 1km sq area using the random waypoint model with random pausing; packet arrival time is modeled as a Poisson Process; system simulated as an 802.11 DCF MAC layer with Dynamic Source Routing.
      \item \emph{Discussion And Conclusion}
      The proposed system is robust against bad-mouthing attacks, as;
      \begin{enumerate}
        \item Recommendation trust is only established through good recommendations
        \item Only recommendations from trusted entities propagate across the network
        \item The Fundamental Axioms\ref{liu2006_basicaxioms} limit the recommendation power of entities with low recommendation trust
      \end{enumerate}
      On the subject of overhead and mobility;
      \begin{itemize}
        \item Nodal distance of propagation of TRRs $\varpropto P_p$\footnote{$P_p$ represents the probability of establishing a trust propagation path between two defined nodes}, but exponential effect on communications overhead as more nodes must process and pass on the requests/responses. TRR's given TTL/Max Transit values moderate this.
        \item Mobility requires higher overheads due to the recurring need for local recommenders within the dynamic system; as a node moves around and between locales, trust must be established within the new local network. Further, mobile nodes are more likely to be 'well known' to the network and thus are more straight forward to attain recommendations for.\footnote{i.e if $A$ is a requester, $X_s$ is a stationary target, and $X_m$ is a mobile target, $P_p(X_m)>P_p(X_s)$}. On the other hand, high node mobility can hinder malicious node detection, as fast moving 'honest' nodes exhibit a high packet drop rate, leading to higher false alarm rates within packet-forwarding trust.
      \end{itemize}
      

    \end{enumerate}
\end{itemize}

\subsection{Questions Raised}
\begin{itemize}
  \item In Section [V-A] (Obtaining Trust Recommendations) There appears to be an information leak towards a malicious attacker, where upon the malicious node ($X$) sending a Trust Recommendation Request (TRR) does not receive responses from certain nodes within $Z$, it can assume that those nodes trust of $X$ is less than some boundary. The question is whether this could be used as an attack monitoring vector, so as to behave in a way that is malicious, but can stop if $Z$ are overly suspicious of $X$
\end{itemize}

\fi

%: ----------------------- bibliography ------------------------

% The section below defines how references are listed and formatted
% The default below is 2 columns, small font, complete author names.
% Entries are also linked back to the page number in the text and to external URL if provided in the BibTex file.

% PhDbiblio-url2 = names small caps, title bold & hyperlinked, link to page 
\begin{multicols}{2} % \begin{multicols}{ # columns}[ header text][ space]
\begin{tiny} % tiny(5) < scriptsize(7) < footnotesize(8) < small (9)

\bibliographystyle{plainnat} % calls style file plainnat.bst
%\bibliographystyle{Latex/Classes/PhDbiblio-url2} % Title is link if provided
\renewcommand{\bibname}{References} % changes the header; default: Bibliography

\bibliography{library} % adjust this to fit your BibTex file

\end{tiny}
\end{multicols}

% --------------------------------------------------------------
% Various bibliography styles exit. Replace above style as desired.

% in-text refs: (1) (1; 2)
% ref list: alphabetical; author(s) in small caps; initials last name; page(s)
%\bibliographystyle{Latex/Classes/PhDbiblio-case} % title forced lower case
%\bibliographystyle{Latex/Classes/PhDbiblio-bold} % title as in bibtex but bold
%\bibliographystyle{Latex/Classes/PhDbiblio-url} % bold + www link if provided

%\bibliographystyle{Latex/Classes/jmb} % calls style file jmb.bst
% in-text refs: author (year) without brackets
% ref list: alphabetical; author(s) in normal font; last name, initials; page(s)

% in-text refs: author (year) without brackets
% (this works with package natbib)


% --------------------------------------------------------------

% according to Dresden med fac summary has to be at the end
%As \glspl{auv} become more technically capable and economically feasible, they are being increasingly used in a great many areas of defence, commercial and environmental applications. 
These applications are tending towards using independent, autonomous, ad-hoc, collaborative behaviour of teams or fleets of these \gls{auv} platforms.
This convergence of research experiences in the \gls{uan} and \gls{manet} fields, along with the increasing \gls{loa} of such platforms, creates unique challenges to secure the operation and communication of these networks.

The question of security and reliability of operation in networked systems has usually been resolved by having a centralised coordinating agent to manage shared secrets and monitor for misbehaviour.
However, in the sparse, noisy and constrained communications environment of \glspl{uan}, the communications overheads and single-point-of-failure risk of this model is challenged (particularly when faced with capable attackers).

As such, more lightweight, distributed, experience based\footnote{rather than ``Evidence based'' in the case of shared keys, \gls{pki} etc.} systems of ``Trust'' have been proposed to dynamically model and evaluate the ``trustworthiness'' of nodes within a \gls{manet} across the network to prevent or isolate the impact of malicious, selfish, or faulty misbehaviour. 
Previously, these models have monitored actions purely within the communications domain. 
Moreover, the vast majority rely on only one type of observation (metric) to evaluate trust; successful packet forwarding.
In these cases, motivated actors may use this limited scope of observation to either perform unfairly without repercussions in other domains/metrics, or to make another, fair, node appear to be operating unfairly.

This thesis is primarily concerned with the use of terrestrial-\gls{manet} trust frameworks to the \gls{uan} space. 
Considering the massive theoretical and practical difference in the communications environment, these frameworks must be reassessed for suitability to the marine realm. 
We find that current single-metric \glspl{tmf} do not perform well in a best-case scaling of the marine network, due to sparse and noisy observation metrics, and while basic multi-metric communications-only frameworks perform better than their single-metric forms, this performance is still not at a reliable level. 
We propose, demonstrate (through simulation) and integrate the use of physical observational metrics for trust assessment, in tandem with metrics from the communications realm, improving the safety, security, reliability and integrity of autonomous \glspl{uan}.

Three main novelties are demonstrated in this work:
Trust evaluation using metrics from the physical domain (movement/distribution/etc.), demonstration of the failings of Communications-based Trust evaluation in sparse, noisy, delayful and non-linear \gls{uan} environments, and the deployment of trust assessment across multiple domains, e.g.\ the physical and communications domains.
The latter contribution includes the generation and optimisation of cross-domain metric composition or``synthetic domains'' as a performance improvement method.


%: Declaration of originality
%\include{backmatter/declaration}



\end{document}
