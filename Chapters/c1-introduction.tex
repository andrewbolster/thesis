% !TeX spellcheck = en_GB
\def\ChapterTitle{Introduction} % Write in your own chapter title

\ifx\ifthesis\undefined
\documentclass[]{article}



% % Special Includes stolen from Thesis.cls
\usepackage{booktabs}
\usepackage{hyperref}


% % % % % % % % % % % % % % % %5
\usepackage[square, numbers, comma, sort&compress]{natbib}  % Use the "Natbib" style for the references in the Bibliography
\usepackage{verbatim,listings}  % Needed for the "comment" environment to make LaTeX comments
\usepackage{array}  % Needed for the "comment" environment to make LaTeX comments
\usepackage{vector}  % Allows "\bvec{}" and "\buvec{}" for "blackboard" style bold vectors in maths
\hypersetup{urlcolor=blue, colorlinks=false}  % Colours hyperlinks in blue, but this can be distracting if there are many links.

\usepackage{epigraph} % epigraph
%customize: \setlength{\epigraphwidth}{7cm}\setlength{\epigraphrule}{0pt}
%use: \epigraph{text}{reference}



% \documentclass[a4paper,12pt]{article}
%\usepackage[a4paper,vmargin={20mm,20mm},hmargin={20mm,20mm}]{geometry}
\usepackage{amsmath,amsfonts,amsthm,color,psfrag,epsf,graphicx, booktabs, tabularx, multirow, longtable}
\usepackage{snapshot, todo}
% \usepackage{pstricks}
\usepackage{enumerate,caption}
%\usepackage[lined,algonl,boxed]{algorithm2e}
\usepackage[ruled,linesnumbered,vlined]{algorithm2e}
\usepackage{float}
% \SpecialCoor
\def\subsum{\mathit{\Sigma}}


%opening
\title{\ChapterTitle}
\author{Andrew Bolster}

\begin{document}

\maketitle

\else
\chapter{\ChapterTitle}
\label{Chapter\thechapter}
\lhead{Chapter \thechapter.
\emph{\nameref{Chapter\thechapter}}} % Write in your own chapter title to set the page header
\fi

\section{Mobile Ad-hoc Networks (\acrshortpl{manet})}

With the explosive growth in the use of mobile telephony and the increasing miniaturisation and efficiency gains of portable communications devices, the classical paradigm of a broadcast/receiver or server/client has given way to an increasing use of decentralised, ad-hoc networks that not only accommodate but take advantage of network mobility.

Wether these networks are decentralised cellular / RF / 802.11 WiFi networks for use in disaster relief areas \cite{Milliken2015} or biologically inspired wireless sensor networks for low-energy, low-maintenance environmental monitoring \cite{Bhargavi2015},\todo{Inappropriate Citation} \gls{manet} theory developed over the past 30 years has gone from it's first formal definition, emerging from DARPA's Packet Radio Network research\cite{Jubin1987}, to being an integral part of modern practical communications.

Minimally, a \gls{manet} consists of of a collection of mobile physical entities (nodes) with some form of communications, processing/data collection, and power systems.\todo{rephrase this paragraph}
Similarly in terms of communications capability, while in many cases \gls{manet} nodes incorporate bi-directional transceivers to send and receive data, this bi-directionality is also not a limiting factor on inclusion within the \gls{manet} field, particularly in the area of Wireless Sensor Networks \cite{Akyildiz2002}.
Nor is the capability or mix of communications technologies used; omnidirectional, static, or steerable communications antennae, utilising a range of technologies such as WiFi, Bluetooth, GSM, UMTS, Optical or Acoustics.
A core characteristic of the design of \glspl{manet} is the inclusion and integration of heterogeneous node collections, i.e where different nodes or groups of nodes in a network have difference capabilities, whether this be in terms of propulsion, sensor apparatus, communications capability, etc.

These networks may be totally independent with no external connections, include independent per-node communications backhauls (e.g.
Cellular Modems in mobile phones as part of a Bluetooth Personal Area Network(PAN)), or include static nodes that provide infrastructure based backhaul.
However, this multiplicity of variations and options presents several challenges to users and operators; the physical topology of \glspl{manet} can vary wildly over short periods of time.
A particular challenge to \gls{manet} operation is that given any node may operate as a routing / gateway node, if/when that node moves to a different region, network segments that had previously used that node as a path must renegotiate / reestablish their routes.
These situations, if not appropriatly managed, lead to opportunities for subversion and selfishness.

The characteristics of \gls{manet}s as defined by Corson et al.\ are paraphrased in Table~\ref{tab:manet_characteristics}.

\begin{table}
  \hyphenpenalty=10000
\caption[Summary of Characteristics of \gls{manet}s]{Summary of Characteristics of \gls{manet}s\cite{Corson1999}}
\label{tab:manet_characteristics}
  \begin{tabularx}{\textwidth}{p{2cm}X}\toprule
    Dynamic Topologies & Nodes are free to move arbitrarily; thus, the typically multihop network topology may change randomly and rapidly at unpredictable times, and may consist of both bidirectional and unidirectional links.
\\
    Bandwidth Constrained, Varied Capacity & Wireless links will continue to have significantly lower capacity than their hardwired counterparts.
In addition, the realized throughput of wireless communications, after accounting for the effects of multiple access, fading, noise, and interference conditions, etc., is often much less than a radio's maximum transmission rate.
\par
One effect of the relatively low to moderate link capacities is that congestion is typically the norm rather than the exception, i.e.\  aggregate application demand will likely approach or exceed network capacity frequently.\\
    Energy Constrained Operation &  Some or all of the nodes in a \gls{manet} may rely on batteries or other exhaustible means for their energy.
For these nodes, the most important system design criteria for optimization may be energy conservation.\\
    Limited physical security & Mobile wireless networks are generally more prone to physical security threats than are fixed cable nets.
The increased possibility of eavesdropping, spoofing, and denial-of-service attacks should be carefully considered.\par
Existing link security techniques are often applied within wireless networks to reduce security threats.
As a benefit, the decentralized nature of network control in \gls{manet}s provides additional robustness against the single points of failure of more centralized approaches.\\\bottomrule
\end{tabularx}
\end{table}

\todo{Sparsity / Density of MANETS}
\section{\glspl{manet} in Harsh Environments}

As mobile ad-hoc networks (\gls{manet}s) grow beyond the terrestrial arena, their operation and the protocols designed around them must be reviewed to assess their suitability to different communications environments, ensuring their continued security, reliability, and performance.

The distributed and dynamic nature of \glspl{manet} mean that it is difficult to maintain a trusted \gls{ttp} or evidence based trust system such as Certificate Authorities or using \gls{pki}.\todo{possibly worthwhile doing more background on the operation of these}
Therefore, a distributed, collaborative system must be applied to these networks.
Such distributed trust management frameworks aim to detect, identify, and mitigate the impacts of malicious actors by distributing per-node assessments and opinions to collectively self-police behaviour.
As such, \glspl{tmf} can be used to predict and reason on the future interactions between entities in a system.

\glspl{tmf} provide information to assist the estimation of future states and actions of nodes within \glspl{manet}.
This information is used to optimize the performance of a network against malicious, selfish, or defective misbehaviour by one or more nodes.
Previous research has established the advantages of implementing TMFs in 802.11 based \glspl{manet}, particularly in terms of preventing selfish operation in collaborative systems \cite{Li2007}, and maintaining throughput in the presence of malicious actors \cite{Buchegger2002}

\section{Systems Approach to Trust and Trust Engineering}
\todo{Trust as Assurance}

\section{Trust operation against Capable Attackers}


\section{Contributions}



\section{Conclusion}
\subsection{Layout}

%%%%%%%%%%%%%%%%%%%%%%%%%%%%%%%%%%%%%%%%%%%%%%%%%%%%%%%%%%%%%%%%%%%%%%%%%%%%%%%
\ifx\ifthesis\undefined
	%% ----------------------------------------------------------------
\label{Bibliography}
% \bibliographystyle{amsplain}
%\bibliographystyle{unsrtnat}  % Use the "unsrtnat" BibTeX style for formatting the Bibliography
\bibliographystyle{alpha}
\bibliography{../Thesis}  % The references (bibliography) information are stored in the file named "Thesis.bib"

\end{document}  % The End
%% ----------------------------------------------------------------
\fi
