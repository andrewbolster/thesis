
%% bare_conf.tex
%% V1.3
%% 2007/01/11
%% by Michael Shell
%% See:
%% http://www.michaelshell.org/
%% for current contact information.
%%
%% This is a skeleton file demonstrating the use of IEEEtran.cls
%% (requires IEEEtran.cls version 1.7 or later) with an IEEE conference paper.
%%
%% Support sites:
%% http://www.michaelshell.org/tex/ieeetran/
%% http://www.ctan.org/tex-archive/macros/latex/contrib/IEEEtran/
%% and
%% http://www.ieee.org/

%%*************************************************************************
%% Legal Notice:
%% This code is offered as-is without any warranty either expressed or
%% implied; without even the implied warranty of MERCHANTABILITY or
%% FITNESS FOR A PARTICULAR PURPOSE! 
%% User assumes all risk.
%% In no event shall IEEE or any contributor to this code be liable for
%% any damages or losses, including, but not limited to, incidental,
%% consequential, or any other damages, resulting from the use or misuse
%% of any information contained here.
%%
%% All comments are the opinions of their respective authors and are not
%% necessarily endorsed by the IEEE.
%%
%% This work is distributed under the LaTeX Project Public License (LPPL)
%% ( http://www.latex-project.org/ ) version 1.3, and may be freely used,
%% distributed and modified. A copy of the LPPL, version 1.3, is included
%% in the base LaTeX documentation of all distributions of LaTeX released
%% 2003/12/01 or later.
%% Retain all contribution notices and credits.
%% ** Modified files should be clearly indicated as such, including  **
%% ** renaming them and changing author support contact information. **
%%
%% File list of work: IEEEtran.cls, IEEEtran_HOWTO.pdf, bare_adv.tex,
%%                    bare_conf.tex, bare_jrnl.tex, bare_jrnl_compsoc.tex
%%*************************************************************************

% *** Authors should verify (and, if needed, correct) their LaTeX system  ***
% *** with the testflow diagnostic prior to trusting their LaTeX platform ***
% *** with production work. IEEE's font choices can trigger bugs that do  ***
% *** not appear when using other class files.                            ***
% The testflow support page is at:
% http://www.michaelshell.org/tex/testflow/



% Note that the a4paper option is mainly intended so that authors in
% countries using A4 can easily print to A4 and see how their papers will
% look in print - the typesetting of the document will not typically be
% affected with changes in paper size (but the bottom and side margins will).
% Use the testflow package mentioned above to verify correct handling of
% both paper sizes by the user's LaTeX system.
%
% Also note that the "draftcls" or "draftclsnofoot", not "draft", option
% should be used if it is desired that the figures are to be displayed in
% draft mode.
%
\documentclass[conference]{IEEEtran}
% Add the compsoc option for Computer Society conferences.
%
% If IEEEtran.cls has not been installed into the LaTeX system files,
% manually specify the path to it like:
% \documentclass[conference]{../sty/IEEEtran}





% Some very useful LaTeX packages include:
% (uncomment the ones you want to load)


% *** MISC UTILITY PACKAGES ***
%
%\usepackage{ifpdf}
% Heiko Oberdiek's ifpdf.sty is very useful if you need conditional
% compilation based on whether the output is pdf or dvi.
% usage:
% \ifpdf
%   % pdf code
% \else
%   % dvi code
% \fi
% The latest version of ifpdf.sty can be obtained from:
% http://www.ctan.org/tex-archive/macros/latex/contrib/oberdiek/
% Also, note that IEEEtran.cls V1.7 and later provides a builtin
% \ifCLASSINFOpdf conditional that works the same way.
% When switching from latex to pdflatex and vice-versa, the compiler may
% have to be run twice to clear warning/error messages.






% *** CITATION PACKAGES ***
%
%\usepackage{cite}
% cite.sty was written by Donald Arseneau
% V1.6 and later of IEEEtran pre-defines the format of the cite.sty package
% \cite{} output to follow that of IEEE. Loading the cite package will
% result in citation numbers being automatically sorted and properly
% "compressed/ranged". e.g., [1], [9], [2], [7], [5], [6] without using
% cite.sty will become [1], [2], [5]--[7], [9] using cite.sty. cite.sty's
% \cite will automatically add leading space, if needed. Use cite.sty's
% noadjust option (cite.sty V3.8 and later) if you want to turn this off.
% cite.sty is already installed on most LaTeX systems. Be sure and use
% version 4.0 (2003-05-27) and later if using hyperref.sty. cite.sty does
% not currently provide for hyperlinked citations.
% The latest version can be obtained at:
% http://www.ctan.org/tex-archive/macros/latex/contrib/cite/
% The documentation is contained in the cite.sty file itself.






% *** GRAPHICS RELATED PACKAGES ***
%
\ifCLASSINFOpdf
  \usepackage[pdftex]{graphicx}
  % declare the path(s) where your graphic files are
  \graphicspath{{../pdf/}{../jpeg/}}
  % and their extensions so you won't have to specify these with
  % every instance of \includegraphics
  \DeclareGraphicsExtensions{.pdf,.jpeg,.png}
\else
  % or other class option (dvipsone, dvipdf, if not using dvips). graphicx
  % will default to the driver specified in the system graphics.cfg if no
  % driver is specified.
  \usepackage[dvips]{graphicx}
  % declare the path(s) where your graphic files are
  \graphicspath{{../eps/}}
  % and their extensions so you won't have to specify these with
  % every instance of \includegraphics
  \DeclareGraphicsExtensions{.eps}
\fi
% graphicx was written by David Carlisle and Sebastian Rahtz. It is
% required if you want graphics, photos, etc. graphicx.sty is already
% installed on most LaTeX systems. The latest version and documentation can
% be obtained at: 
% http://www.ctan.org/tex-archive/macros/latex/required/graphics/
% Another good source of documentation is "Using Imported Graphics in
% LaTeX2e" by Keith Reckdahl which can be found as epslatex.ps or
% epslatex.pdf at: http://www.ctan.org/tex-archive/info/
%
% latex, and pdflatex in dvi mode, support graphics in encapsulated
% postscript (.eps) format. pdflatex in pdf mode supports graphics
% in .pdf, .jpeg, .png and .mps (metapost) formats. Users should ensure
% that all non-photo figures use a vector format (.eps, .pdf, .mps) and
% not a bitmapped formats (.jpeg, .png). IEEE frowns on bitmapped formats
% which can result in "jaggedy"/blurry rendering of lines and letters as
% well as large increases in file sizes.
%
% You can find documentation about the pdfTeX application at:
% http://www.tug.org/applications/pdftex





% *** MATH PACKAGES ***
%
\usepackage[cmex10]{amsmath}
% A popular package from the American Mathematical Society that provides
% many useful and powerful commands for dealing with mathematics. If using
% it, be sure to load this package with the cmex10 option to ensure that
% only type 1 fonts will utilized at all point sizes. Without this option,
% it is possible that some math symbols, particularly those within
% footnotes, will be rendered in bitmap form which will result in a
% document that can not be IEEE Xplore compliant!
%
% Also, note that the amsmath package sets \interdisplaylinepenalty to 10000
% thus preventing page breaks from occurring within multiline equations. Use:
\interdisplaylinepenalty=2500
% after loading amsmath to restore such page breaks as IEEEtran.cls normally
% does. amsmath.sty is already installed on most LaTeX systems. The latest
% version and documentation can be obtained at:
% http://www.ctan.org/tex-archive/macros/latex/required/amslatex/math/

% ADDED BY BOLSTER
\usepackage{amssymb}





% *** SPECIALIZED LIST PACKAGES ***
%
%\usepackage{algorithmic}
% algorithmic.sty was written by Peter Williams and Rogerio Brito.
% This package provides an algorithmic environment fo describing algorithms.
% You can use the algorithmic environment in-text or within a figure
% environment to provide for a floating algorithm. Do NOT use the algorithm
% floating environment provided by algorithm.sty (by the same authors) or
% algorithm2e.sty (by Christophe Fiorio) as IEEE does not use dedicated
% algorithm float types and packages that provide these will not provide
% correct IEEE style captions. The latest version and documentation of
% algorithmic.sty can be obtained at:
% http://www.ctan.org/tex-archive/macros/latex/contrib/algorithms/
% There is also a support site at:
% http://algorithms.berlios.de/index.html
% Also of interest may be the (relatively newer and more customizable)
% algorithmicx.sty package by Szasz Janos:
% http://www.ctan.org/tex-archive/macros/latex/contrib/algorithmicx/




% *** ALIGNMENT PACKAGES ***
%
%\usepackage{array}
% Frank Mittelbach's and David Carlisle's array.sty patches and improves
% the standard LaTeX2e array and tabular environments to provide better
% appearance and additional user controls. As the default LaTeX2e table
% generation code is lacking to the point of almost being broken with
% respect to the quality of the end results, all users are strongly
% advised to use an enhanced (at the very least that provided by array.sty)
% set of table tools. array.sty is already installed on most systems. The
% latest version and documentation can be obtained at:
% http://www.ctan.org/tex-archive/macros/latex/required/tools/


\usepackage{mdwmath}
\usepackage{mdwtab}
% Also highly recommended is Mark Wooding's extremely powerful MDW tools,
% especially mdwmath.sty and mdwtab.sty which are used to format equations
% and tables, respectively. The MDWtools set is already installed on most
% LaTeX systems. The lastest version and documentation is available at:
% http://www.ctan.org/tex-archive/macros/latex/contrib/mdwtools/


% IEEEtran contains the IEEEeqnarray family of commands that can be used to
% generate multiline equations as well as matrices, tables, etc., of high
% quality.


%\usepackage{eqparbox}
% Also of notable interest is Scott Pakin's eqparbox package for creating
% (automatically sized) equal width boxes - aka "natural width parboxes".
% Available at:
% http://www.ctan.org/tex-archive/macros/latex/contrib/eqparbox/





% *** SUBFIGURE PACKAGES ***
\usepackage[tight,footnotesize]{subfigure}
% subfigure.sty was written by Steven Douglas Cochran. This package makes it
% easy to put subfigures in your figures. e.g., "Figure 1a and 1b". For IEEE
% work, it is a good idea to load it with the tight package option to reduce
% the amount of white space around the subfigures. subfigure.sty is already
% installed on most LaTeX systems. The latest version and documentation can
% be obtained at:
% http://www.ctan.org/tex-archive/obsolete/macros/latex/contrib/subfigure/
% subfigure.sty has been superceeded by subfig.sty.



\usepackage[caption=false]{caption}
\usepackage[font=footnotesize]{subfig}
% subfig.sty, also written by Steven Douglas Cochran, is the modern
% replacement for subfigure.sty. However, subfig.sty requires and
% automatically loads Axel Sommerfeldt's caption.sty which will override
% IEEEtran.cls handling of captions and this will result in nonIEEE style
% figure/table captions. To prevent this problem, be sure and preload
% caption.sty with its "caption=false" package option. This is will preserve
% IEEEtran.cls handing of captions. Version 1.3 (2005/06/28) and later 
% (recommended due to many improvements over 1.2) of subfig.sty supports
% the caption=false option directly:
%\usepackage[caption=false,font=footnotesize]{subfig}
%
% The latest version and documentation can be obtained at:
% http://www.ctan.org/tex-archive/macros/latex/contrib/subfig/
% The latest version and documentation of caption.sty can be obtained at:
% http://www.ctan.org/tex-archive/macros/latex/contrib/caption/




% *** FLOAT PACKAGES ***
%
%\usepackage{fixltx2e}
% fixltx2e, the successor to the earlier fix2col.sty, was written by
% Frank Mittelbach and David Carlisle. This package corrects a few problems
% in the LaTeX2e kernel, the most notable of which is that in current
% LaTeX2e releases, the ordering of single and double column floats is not
% guaranteed to be preserved. Thus, an unpatched LaTeX2e can allow a
% single column figure to be placed prior to an earlier double column
% figure. The latest version and documentation can be found at:
% http://www.ctan.org/tex-archive/macros/latex/base/



%\usepackage{stfloats}
% stfloats.sty was written by Sigitas Tolusis. This package gives LaTeX2e
% the ability to do double column floats at the bottom of the page as well
% as the top. (e.g., "\begin{figure*}[!b]" is not normally possible in
% LaTeX2e). It also provides a command:
%\fnbelowfloat
% to enable the placement of footnotes below bottom floats (the standard
% LaTeX2e kernel puts them above bottom floats). This is an invasive package
% which rewrites many portions of the LaTeX2e float routines. It may not work
% with other packages that modify the LaTeX2e float routines. The latest
% version and documentation can be obtained at:
% http://www.ctan.org/tex-archive/macros/latex/contrib/sttools/
% Documentation is contained in the stfloats.sty comments as well as in the
% presfull.pdf file. Do not use the stfloats baselinefloat ability as IEEE
% does not allow \baselineskip to stretch. Authors submitting work to the
% IEEE should note that IEEE rarely uses double column equations and
% that authors should try to avoid such use. Do not be tempted to use the
% cuted.sty or midfloat.sty packages (also by Sigitas Tolusis) as IEEE does
% not format its papers in such ways.





% *** PDF, URL AND HYPERLINK PACKAGES ***
%
\usepackage{url}
% url.sty was written by Donald Arseneau. It provides better support for
% handling and breaking URLs. url.sty is already installed on most LaTeX
% systems. The latest version can be obtained at:
% http://www.ctan.org/tex-archive/macros/latex/contrib/misc/
% Read the url.sty source comments for usage information. Basically,
% \url{my_url_here}.





% *** Do not adjust lengths that control margins, column widths, etc. ***
% *** Do not use packages that alter fonts (such as pslatex).         ***
% There should be no need to do such things with IEEEtran.cls V1.6 and later.
% (Unless specifically asked to do so by the journal or conference you plan
% to submit to, of course. )


% correct bad hyphenation here
\hyphenation{op-tical net-works semi-conduc-tor}

%%%%%%%%%%%%%%%%%%%%%%%%%%%%%%%%%%%
% BOLSTER PACKAGES FOR DEV PERIOD %
%%%%%%%%%%%%%%%%%%%%%%%%%%%%%%%%%%%

\usepackage[textsize=footnotesize]{todonotes}
\presetkeys{todonotes}{inline}{}



\begin{document}
%
% paper title
% can use linebreaks \\ within to get better formatting as desired
\title{Applicability of Trust Management Frameworks in Underwater Autonomous Networks \\ {\large DRAFT for 9pg submission to MASS 15 - 3 April}}


% author names and affiliations
% use a multiple column layout for up to three different
% affiliations
\author{\IEEEauthorblockN{Andrew Bolster}
\IEEEauthorblockA{Department of Electical Engineering and Electronics\\
University of Liverpool\\
Liverpool, UK\\
Email: bolster@liv.ac.uk}
\and
\IEEEauthorblockN{Alan Marshall}
\IEEEauthorblockA{Department of Electical Engineering and Electronics\\
University of Liverpool\\
Liverpool, UK\\
Email: alan.marshall@liv.ac.uk}
}
% conference papers do not typically use \thanks and this command
% is locked out in conference mode. If really needed, such as for
% the acknowledgment of grants, issue a \IEEEoverridecommandlockouts
% after \documentclass

% for over three affiliations, or if they all won't fit within the width
% of the page, use this alternative format:
% 
%\author{\IEEEauthorblockN{Michael Shell\IEEEauthorrefmark{1},
%Homer Simpson\IEEEauthorrefmark{2},
%James Kirk\IEEEauthorrefmark{3}, 
%Montgomery Scott\IEEEauthorrefmark{3} and
%Eldon Tyrell\IEEEauthorrefmark{4}}
%\IEEEauthorblockA{\IEEEauthorrefmark{1}School of Electrical and Computer Engineering\\
%Georgia Institute of Technology,
%Atlanta, Georgia 30332--0250\\ Email: see http://www.michaelshell.org/contact.html}
%\IEEEauthorblockA{\IEEEauthorrefmark{2}Twentieth Century Fox, Springfield, USA\\
%Email: homer@thesimpsons.com}
%\IEEEauthorblockA{\IEEEauthorrefmark{3}Starfleet Academy, San Francisco, California 96678-2391\\
%Telephone: (800) 555--1212, Fax: (888) 555--1212}
%\IEEEauthorblockA{\IEEEauthorrefmark{4}Tyrell Inc., 123 Replicant Street, Los Angeles, California 90210--4321}}

% use for special paper notices
%\IEEEspecialpapernotice{(Invited Paper)}




% make the title area
\maketitle


\begin{abstract}
%\boldmath
In this paper, we demonstrate the need for a different approach towards metric selection and trust-timing in Underwater Autonomous Networks and a performance comparison with a selection of current Trust Management Framework assessment processes.

\end{abstract}
% IEEEtran.cls defaults to using nonbold math in the Abstract.
% This preserves the distinction between vectors and scalars. However,
% if the conference you are submitting to favors bold math in the abstract,
% then you can use LaTeX's standard command \boldmath at the very start
% of the abstract to achieve this. Many IEEE journals/conferences frown on
% math in the abstract anyway.

% no keywords




% For peer review papers, you can put extra information on the cover
% page as needed:
% \ifCLASSOPTIONpeerreview
% \begin{center} \bfseries EDICS Category: 3-BBND \end{center}
% \fi
%
% For peerreview papers, this IEEEtran command inserts a page break and
% creates the second title. It will be ignored for other modes.
\IEEEpeerreviewmaketitle



\section{Introduction}\label{sec:introduction}
% no \IEEEPARstart
% You must have at least 2 lines in the paragraph with the drop letter
% (should never be an issue)

As mobile ad-hoc networks (MANETs) grow beyond the terrestrial arena, their operation and the protocols designed around them must be reviewed to assess their suitability to different communications environments, ensuring their continued security, reliability, and performance.

Trust Management Frameworks (TMFs) provide information to assist the estimation of future states and actions of nodes within networks.
This information is used to optimize the performance of a network against malicious, selfish, or defective misbehaviour by one or more nodes.
Previous research has established the advantages of implementing TMFs in 802.11 based MANETs, particularly in terms of preventing selfish operation in collaborative systems \cite{Li2007}, and maintaining throughput in the presence of malicious actors \cite{Buchegger2002}

Most current TMFs use a single type of observed action to derive trust values, i.e. successfully forwarded packets. These observations then inform future decisions of individual nodes, for example, route selection \cite{Li2008}.

Recent work has demonstrated use of a number of metrics to form a ``vector'' of trust.
The Multi-parameter Trust Framework for MANETs (MTFM)\cite{Guo11}, uses a range of physical metrics beyond packet delivery/loss rate (PLR) to form a vector of trust.
This vectorized trust allows a system to detect and identify the tactics being used to undermine or subvert trust.
To date this work has been limited to terrestrial, RF based networks, however as autonomous underwater vehicles (AUVs) become more capable, and economical, they are being used in many applications requiring trust\cite{Caiti2011}.
With this use being increasingly isolated from stable communications networks, the establishment of trust between nodes is essential for the reliability and stability of such teams.
As such, the use of trust methods developed in the terrestrial MANET space must be re-appraised for application within the challenging underwater communications channel.

The paper is laid out as follows.
In Section \ref{sec:trustandtmfs} we discuss Trust and Trust Management Frameworks, defining our terminology and reviewing the justifications for the use and development of Trust Management Frameworks in marine acoustic networks.
In Section \ref{sec:marineacousticnetworks}, we review selected features of the underwater communications channel, highlighting particular challenges and differentials against terrestrial equivalents.
In Section \ref{sec:initialsystemcharacterization}, we establish an experimental configuration for the marine space, and review the scenarios and results presented in \cite{Guo11}.
In Section \ref{sec:trustresultsanddiscussion}, we present our findings in trust establishment and malicious behaviour detection, comparing with current TMFs such as OTMF and Beta.

The contributions of this paper are: A Trust Management Framework applicable to Underwater MANETs, a study on the comparative operation and performance between terrestrial and underwater MANETs using TMFs, and a review of metric suitability for Trust Management Frameworks in marine environments, informing future metric selection for experimenters and theorists.

\section{Trust and Trust Management Frameworks}\label{sec:trustandtmfs}

\subsection{Trust in MANETs}\label{sec:trustinmanets}

Trust is the level of confidence one agent has in another to perform an action. 
The distributed and dynamic nature of MANETs mean that it is difficult to maintain a trusted third party (TTP) or evidence based trust system such as Certificate Authorities (CA) or Public Key Infrastructure (PKI).
Distributed trust management frameworks aim to detect, identify, and mitigate the impacts of malicious actors by distributing per-node assessments and opinions to collectively self-police behaviour.
Various models and algorithms for describing trust and developing trust management in distributed systems, P2P communities or wireless networks have been considered.
Taking some examples;

\begin{itemize}
  \item \emph{The Objective Trust Management Framework} takes a Bayesian Beta function to model per-link Packet Loss Rate (PLR) over time, combining ``Trust'' and ``Confidence of Assessment'' into a single value \cite{Li2008}.
    OTMF however does not appropriately combat multi-node-collusion in the network \cite{Cho2011}.
  \item \emph{Trust-based Secure Routing}\cite{Moe2008a} demonstrated an extension to Dynamic Source Routing (DSR), incorporating a Hidden Markov Model of next-hop network, reducing the efficacy of Byzantine attacks such as black-hole routing.
  \item \emph{CONFIDANT}\cite{Buchegger2002} presented an approach using a probabilistic estimation of PLR, similar to OTMF, also introducing a topology weighting scheme that also weighted trust assessments based on historical experience of the reporter.
  \item \emph{Fuzzy Trust-Based Filtering}; \cite{Luo2008} presents the use of Fuzzy Inference to adapt to malicious recommenders using conditional similarity to classify performance with overlapping Fuzzy Set Membership, filtering assessments across a network.
\end{itemize}

These TMFs can be generalised as single-value probabilistic estimation, based around using a binary input state and generating an probabilistic estimation of the future states of that input. This expectation value is $\text{beta}(p|\alpha,\beta) \to E(p) = \frac{\alpha}{\alpha+\beta}$ where $\alpha$ and $\beta$ represent the number of successful and unsuccessful interactions respectively.

These single metric TMFs provide malicious actors with a significant advantage if their activity is undetectable by that metric.
In the case where the attacker can subvert the TMF, the metric under assessment by that TMF does not cover the threat mounted by the attacker.
In turn, this causes a super-linearly negative effect in the efficiency of the network, as the TMF is assumed to have reduced the possible set of attacks when it has actually made it more advantageous to attack a different part of the networks operation.
An example of such a situation would be in a TMF focused on PLR where an attacker selectively delays packets going through it, reducing overall throughput but not dropping any packets.
Such behaviour would not be detected by the TMF.

There are also situations where the observed metrics will include significant noise and occur at irregular, sparse, intervals.
Conventional approaches such as probabilistic estimation do not produce trust values that reflect the underlying reality and context of the metrics available, as they require a-priori assumption that the trust value under exploration has an expected distribution, that distribution is mono-modal, and the input metrics are binary.
In scenarios with variable, sparse, noisy metrics, estimating the distribution is difficult to accomplish a-priori.

\subsection{Grey Theory and MTFM}

Grey Theory performs cohort based normalization of metrics at runtime. 
This creates a more stable contextual assessment of trust, providing a ``grade'' of trust compared to other observed nodes in that interval, while maintaining the ability to reduce trust values down to a stable assessment range for decision support without requiring every environment entered into to be characterised.
This presents a stark difference between the Grey and Probabilistic approaches.
Grey assessments are relative in both fairly and unfairly operating networks.
Nodes will receive mid-range trust assessments if there are no malicious actors as there is no-one else ``bad'' to compare against, and variations in assessment will be primarily driven by topological and environmental factors.

Guo\cite{Guo11} demonstrated the ability of Grey Relational Analysis (GRA)\cite{Zuo1995} to normalise and combine disparate traits of a communications link such as instantaneous throughput, received signal strength, etc. into a Grey Relational Coefficient, or a ``trust vector''.

In the case of the terrestrial communications network used in \cite{Guo11}, the observed metric set $X = {x_1,\dots,x_M}$ representing the measurements taken by each node of its neighbours at least interval, is defined as $X=[$packet loss rate, signal strength, data rate, delay, throughput$]$.
The trust vector is given as
%
\begin{align}
  \label{eq:grc}
  \theta_{k,j}^t = \frac{\min_k|a_{k,j}^t - g_j^t| + \rho \max_k|a_{k,j}^t-g_j^t|}{|a_{k,j}^t-g_j^t| + \rho \max_k|a_{k,j}^t-g_j^t|} \\
  \phi_{k,j}^t = \frac{\min_k|a_{k,j}^t - b_j^t| + \rho \max_k|a_{k,j}^t-b_j^t|}{|a_{k,j}^t-b_j^t| + \rho \max_k|a_{k,j}^t-b_j^t|} \notag 
\end{align}
%
where $a_{k,j}^t$ is the value of a observed metric $x_j$ for a given node $k$ at time $t$, $\rho$ is a distinguishing coefficient set to $0.5$, $g$ and $b$ are respectively the '``good'' and ``bad'' reference metric sequences from $\{a_{k,j}^t k=1,2\dots K\}$, e.g. $g_j=\max_k({a_{k,j}^t})$,  $b_j=\min_k({a_{k,j}^t})$ (where each metric is selected to be monotonically positive for trust assessment, e.g. higher throughput is always better). 

Weighting can be applied before generating a scalar value which allows the identification and classification of untrustworthy behaviours.

%
\begin{equation}
  \label{eq:metric_weighting}
  [\theta_k^t, \phi_k^t] = \left[\sum_{j=0}^M h_j \theta_{k,j}^t,\sum_{j=0}^M h_j \phi_{k,j}^t \right]
\end{equation}
Where $H=[h_0\dots h_M]$ is a metric weighting vector such that $\sum h_j = 1$, and in the basic case, $H=[\frac{1}{M},\frac{1}{M}\dots\frac{1}{M}]$ to treat all metrics evenly.
$\theta$ and $\phi$ are then scaled to $[0,1]$ using the mapping $y = 1.5 x - 0.5$.
The $[\theta,\phi]$ values are reduced into a scalar trust value by $T_k^t = ({1+{(\phi_k^t)^2}/{(\theta_k^t)^2}})^{-1}$.
This trust value minimises the uncertainties of belonging to either best ($g$) or worst ($b$) sequences in \eqref{eq:grc}.

MTFM combines this GRA with a topology-aware weighting scheme\eqref{eq:networkeffects} and a fuzzy whitenization model\eqref{eq:whitenization}. There are three classes of topological trust relationship used; Direct, Recommendation, and Indirect.
Where an observing node, $n_i$, assesses the trust of another, target, node, $n_j$; the Direct relationship is $n_i$'s own observations $n_j$'s behaviour.
In the Recommendation case, a node $n_k$, which shares Direct relationships with both $n_i$ and $n_j$, gives its assessment of $n_j$ to $n_i$.
The Indirect case, similar to the Recommendation case, the recommender $n_k$, does not have a direct link with the observer $n_i$ but $n_k$ has a Direct link with the target node, $n_j$.
These relationships give us node sets, $N_R$ and $N_I$ containing the nodes that have recommendation or indirect, relationships to the observing node respectively.
%
\begin{align}
  \label{eq:networkeffects}
  T_{i,j}^{MTFM}=\frac{1}{2} \cdot \max_s\{f_s(T_{i,j})\} T_{i,j}+&\frac{1}{2} \frac{2|N_R| }{2|N_R| + |N_I|}\sum_{n \in N_R} \max_s\{f_s(T_{i,n})\} T_{i,n}\\ \notag
  +&\frac{1}{2} \frac{|N_I| }{2|N_R| + |N_I|}\sum_{n \in N_I} \max_s\{f_s(T_{i,n})\} T_{i,n} 
\end{align}
 Where $T_{i,n}$ is the subjective trust assessment of $n_i$ by $n_n$, and $f_s = [ f_1,f_2, f_3]$ given as:
\begin{align}
  \label{eq:whitenization}
  f_1(x)&= -x+1\notag\\
  f_2(x)&= 
  \begin{cases}
    2x & \text{if }x\leq 0.5\\
    -2x+2 & \text{if }x>0.5
  \end{cases}\\
  f_3(x)&= x\notag
\end{align}
%
\section{Marine Acoustic Networks}\label{sec:marineacousticnetworks}

The key challenges of underwater acoustic communications are centred around the impact of slow and differential propagation of energy (RF, Optical, Acoustic) through water, and it's interfaces with the seabed / air.
The resultant challenges include; long delays due to propagation, significant inter-symbol interference and Doppler spreading, fast and slow fading due to environmental effects (aquatic flora/fauna; surface weather), carrier-frequency dependent signal attenuation, multipath caused by the medium interfaces, variations in propagation speed due to depth dependant effects (salinity, temperature, and pressure), and subsequent refractive spreading and lensing due to that same propagation variation\cite{Partan2006}.

The attenuation that occurs in an underwater acoustic channel over a distance $d$ for a signal about frequency $f$ in linear power as $A_{\text{aco}}(d,f) = A_0d^ka(f)^d$ and in $dB$ form is given as 
%
\begin{equation}
  \label{eq:acoattenuationdb}
  10 \log A_{\text{aco}}(d,f)/A_0 = k \cdot 10 \log d + d \cdot 10 \log a(f)
\end{equation}
%
where $A_0$ is a unit-normalising constant, $k$ is a spreading factor (commonly taken as 1.5), and $a(f)$ is the absorption coefficient, expressed empirically using Thorp's formula \eqref{eq:thorp} from \cite{Stojanovic2007}
%
\begin{equation}
  \label{eq:thorp}
  10 \log a(f) = 0.11 \cdot \frac{f^2}{1+f^2} + 44\cdot\frac{f^2}{4100+f^2}+ 2.75\times10^{-4} f^2 + 0.003
\end{equation}
%
Refractive lensing and the multipath nature of the medium result in supposedly line of sight propagation being extremely unreliable for estimating distances to targets.
The first arriving beam has as the very least bent in the medium, and commonly has reflected off the surface/seabed before arriving at a receiver, creating secondary paths that are sometimes many times longer than the first arrival path, generating symbol spreading over orders of seconds depending on the ranges and depths involved.
Extensive Forward Error Correction coding is used on such channels to minimise packet losses.

Comparing $A_{aco}(d,f)$ with the RF Free-Space Path Loss model $A_{\text{RF}}(d,f) \approx \left( \frac{4\pi d f}{c} \right)^2$, the impact of range on signal power is exponential underwater, rather than quadratic in RF space ($A_{\text{aco}} \propto f^{2d}$ vs $A_{\text{RF}} \propto (df)^2$). 
While both frequency dependant factors are quadratic, approximating the factors in \eqref{eq:thorp}, $f\propto A_{\text{aco}}$ is at least 4 orders of magnitude higher than $f\propto A_{\text{RF}}$


\subsection{Trust in Marine Networks}\label{sec:trust_in_marine}

With demand for smaller, more decentralised marine survey and monitoring systems, and a drive towards lower per-unit cost, TMFs are going to be increasingly applied to the marine space, as the benefits they present are significant.
Beyond the constraints of the communications environment, knock on pressures are applying in battery capacity, on-board processing, and locomotion.
These pressures simultaneously present opportunities and incentives for malicious or selfish actors to appear to cooperate while not reciprocating, in order to conserve power for instance.
These multiple aspects of potential incentives, trust, and fairness do not directly fall under the scope of single metric trusts discussed above, and this context indicates that a multi-metric approach may be more appropriate.


\section{Initial System Model Characterization}\label{sec:initialsystemcharacterization}

\subsection{Mobility, Topology, and Communications}

Four mobility scenarios were used in \cite{Guo11} to explore trust behaviour; all nodes static, a central node $n_1$ performing a random walk with other nodes remaining static, all nodes but the central node ($n_1$) randomly walking, and all nodes randomly walking. From these we select the all static and all mobile cases for presentation.
The reason for this is that giving a malicious node special privilege or capabilities will skew the results of trust assessment, as the behaviours of the static and mobile nodes will be significantly different regardless of malice.

The six nodes are placed as in \cite{Guo11}, as per Fig.~\ref{fig:s1_layout}, with each node on average 100m from each other, as per \cite{Guo11}.
The use of six nodes and the particular layout enables the investigation of the three trust relationships based on minimum path topologies, such that the node generating the trust assessments, $n_0$ has Direct, Recommendation, and Indirect trust assessments of $n_1$ available to it from itself, $[n_2,n_3]$, and $[n_4,n_5]$ respectively.

In all of the scenarios, each link from $n_i \rightarrow n_j$ periodically sent 10 second bursts of Constant Bit Rate (CBR) style traffic.
%
\begin{figure}[h]
  \centering
  \includegraphics[width=0.3\textwidth]{img/s1_layout.pdf}
  \caption{Initial layout with nodes spaced an average of 100m apart}
  \label{fig:s1_layout}
\end{figure}
%
Guo demonstrated that when compared against OTMF and Beta trust assessment, MTFM provided increased variation in trust assessment over time, providing more information about the nodes behaviour than packet delivery probability. 
By weighting the metrics used in MTFM, it was shown that the trust assessments could be used to identify the style of misbehaviour being performed within the network and by who.
We present a corollary method to investigate and apply this work to the Marine MANET field.

\subsection{Simulation Background}

Simulations were conducted using a Python based simulation framework, SimPy\cite{Mueller2003SimPy}, with a network stack built upon AUVNetSim\cite{Miquel2008}, with transmission parameters (Table \ref{tab:sysconstraints}) taken from and validated against \cite{Stojanovic2007} and \cite{Stefanov2011}.

Given the differences in delay and propagation between RF and marine networks, it is natural that the same application rates (e.g. packet emission rates or throughput) and node separations should not be assumed to be equivalent.
Therefore, we characterise an operational zone of performance within which the network can operate stably.
%
\begin{table}[h]
  \caption{Comparison of system model constraints as applied between Terrestrial and Marine communications} \label{tab:sysconstraints}
  \begin{center}
    \setlength{\tabcolsep}{8pt}
    \begin{tabular}{lccc}
      \toprule
      Parameter & Unit & Terrestrial & Marine \\
      \midrule
      Simulated Duration & $s$ & 300 & 18000\\
      Trust Sampling Period & $s$ & 1 & 600 \\
      Simulated Area & $km^2$ & 0.7 & 0.7-4 \\
      Transmission Range & $km$ & 0.25 & 1.5 \\
      Physical Layer & & RF(802.11) & Acoustic\\
      Propagation Speed& $m/s$ & $3\times10^8$ & 1490\\
      Center Frequency& $Hz$ & $2.6\times10^9$ & $2 \times 10^4$ \\
      Bandwidth& $Hz$ & $22\times10^6$ & $1\times10^4$\\
      MAC Type & & CSMA/DCF & CSMA/CA\\
      Routing Protocol & & DSDV & FBR \\
      Max Speed & $ms^{-1}$ & 5 & 1.5 \\
      Max Data Rate & $bps$ & $5\times10^6$ & $\approx 240$ \\
      Packet Size & bits & 4096 &  9600 \\
      Single Transmission Duration & $s$ & 10 & 32 \\
      Single Transmission Size & bits & $10^7$ & $9600$ \\
      \bottomrule
    \end{tabular}
    \setlength{\tabcolsep}{6pt}
  \end{center}
\end{table}
%
\subsection{Scaling Considerations between Terrestrial and Underwater Environments}

In this section we characterise the simulated communications environment, establishing an optimal packet emission rate for comparison against \cite{Guo11}.

We establish a appropriate safe operating zone for marine communications by looking at the communications rate and physical distribution factors across the two selected mobility scenarios.
In scaling the physical distribution of the nodes, we also scale the environment in which the nodes are restricted to, which has a significant impact on the number of potential runtime topologies, with nodes getting increasingly isolated as the environment space increases.
This leads to increasing delays as routes are constantly broken, re-advertised and re-established. 
From Table~\ref{tab:sysconstraints}, the operating transmission range of this model of acoustic communications is $\approx 6$ times further than that of 802.11, indicating that a suitable operating environment will have an area $\approx \sqrt{6}$ times the area of the 802.11 case.
However, it was recognised in Section~\ref{sec:marineacousticnetworks} that the relationship between attenuation and distance is exponential underwater, so this would represent an upper bound of performance, where nodes begin approximately 400$m$ apart. 

As the separation is increased, the emission rate at which the network becomes saturated decreases, reducing overall throughput. 
This throughput degradation is tightly coupled with the mobility.
For instance, in Fig.~\ref{fig:throughput_static}, where all nodes are static, we do not see significant drops in saturation rates until we approach 800m, nearly double our initial estimate. 
However, in Fig. ~\ref{fig:throughput_all_mobile}, where all the nodes are randomly walking, the saturation point collapses from 0.025pps at 300m to 0.015pps at 400m.
These results indicate that the best area to continue operating in for a range of node separations is at 0.015pps, and that a reasonable position scaling is from 100m to 300m, beyond which communication becomes increasingly unstable, especially in terms of end to end delay (not shown) 
%
\begin{figure}[h]
\begin{subfigure}{.5\textwidth}
  \centering
  \includegraphics[width=.9\linewidth]{img/throughput_sep_lines_static.pdf}
  \caption{Static}
  \label{fig:throughput_static}
\end{subfigure}%
\begin{subfigure}{.5\textwidth}
\centering
  \includegraphics[width=.9\linewidth]{img/throughput_sep_lines_all_mobile.pdf}
  \caption{Mobile}
  \label{fig:throughput_all_mobile}
\end{subfigure}
\caption{Throughput Characteristics for varying node separations across increasing packet emission rates}
\label{fig:scenario_throughputs_plain}
\end{figure}
%
\section{Trust}\label{sec:trustresultsanddiscussion}

Having established a safe operating range for comparison, at 300m separation and an emission rate of 0.015pps, we repeat the static and mobile scenarios presented in \cite{Guo11}. We select an assessment period of 10 mins for a 5 hour mission to scale in comparison to relative bitrates experienced (1Mbps vs $\approx15$bps).

Metrics used for Grey assessment are transmitted and received throughput and power, delay, and packet loss rate as calculated by aborted, unacknowledged, transmissions.
Compared to \cite{Guo11}, this metric set lacks a data rate quantity as the network is not dynamically adjusting bandwidth.
In context of Grey Relational Coefficient generation \eqref{eq:grc}, the best sequence $g$ was selected using the lowest PLR, delay, and powers, and the highest throughputs, with the worst sequence, $b$ the inverse of these metrics.

The particular factors under discussion are the relative performance of MTFM against OTMF and Beta with respect to statistical stability across mobilities and in responsiveness to changing network behaviour. 
We establish a similar result set by initially tracking the resultant trust values established by MTFM in the pair of mobility scenarios, shown in Fig.\ref{fig:trust_mobility}.
For simplicity, we are primarily concerned with the observational trust relationship between $n_0$ and $n_1$, i.e. $n_0$'s assessment of the trustworthiness of $n_1$, or $T_{1,0}$.
We are also concerned with the opinions of $n_1$ provided to $n_0$ by other nodes, where $[T_{1,2},T_{1,3}]$ and $[T_{1,4},T_{1,5}]$ denote the sets of recommendation and indirect trust assessment respectively.
We also include aggregate assessments; $T_{1,\text{Avg}}$, the flat average of direct trust assessments of $n_1$, $T_{1,\text{Net}}$, that weights assessments according to the network topology from \eqref{eq:networkeffects}, without the whitenization factor $f_s$, and $T_{1,\text{MTFM}}$, the final MTFM trust assessment value based on both network topology and whitenization from \eqref{eq:whitenization}.

The variability in assessment is loosely tied to the mobility; in the static case (Fig.~\ref{fig:trust_static}), we see that the nodes close to $n_1$ ($[n_0,n_2,n_3]$) have reasonably consistent distributions, and as the range increases out to $[n_4,n_5]$, this variability increases.
In the full mobility case, shown in Fig.~\ref{fig:trust_all_mobile}, this subjective variability is greatly increased. 
As the topology is highly dynamic, delays due to re-establishing routes can be very large, perturbing the trust value.
The aggregate trust values using topology information ($iT_{1,\text{Net}},T_{1,\text{MTFM}}$) display a decreased variation than those of the individual subjective observations in both cases.
%
\begin{figure}[h]
\begin{subfigure}{.5\textwidth}
  \centering
  \includegraphics[width=\linewidth]{img/trust_bella_static.pdf}
  \caption{Static}
  \label{fig:trust_static}
\end{subfigure}%
\begin{subfigure}{.5\textwidth}
\centering
  \includegraphics[width=\linewidth]{img/trust_bella_all_mobile.pdf}
  \caption{Mobile}
  \label{fig:trust_all_mobile}
\end{subfigure}
\caption{MTFM Trust assessments of $n_1$ ($T_{1,X}$)\protect\footnotemark}
\label{fig:trust_mobility}
\end{figure}
%
\subsection{Comparison under malicious behaviour}

Guo introduces a range of malicious actors, including modification of the packet loss rate of routing nodes and limiting throughput on a per-link basis as well as a selection of combined misbehaviours. 

Given that the established links are already heavily constrained, heavy handed attacks such as introducing selective PLR and adding to the already extreme and hugely variable delays would severely impact the general performance of the network beyond the scope of simple selfishness, effectively triggering saturation collapses in regions that the network should be stable.
Therefore, we select a Selfish Power Control behaviour, where $n_1$ increases it's transmission power by 20\% for all nodes \emph{except} communications with $n_0$.

In order to assess this experiment, parallel simulations were performed where there was no malicious behaviour, the ``fair'' scenario. 
From this, we apply a sequence of metric vectors that preferentially weight each metric during \eqref{eq:metric_weighting}.
For an arbitrary metric weight vector $H$, where the metric $m_j$ is emphasised as being twice as important as the other metrics, we form an initial weighting vector $H'=[h_i...h_M]$ such that $h_i = 1 \forall i \ne j; h_j=2$. We then scale that vector $H'$ such that $\sum H = 1$ by $H= \frac{H'}{\sum H'}$.
%Footnote for fig:trust_all_mobile
\footnotetext{Box plots centres indicate the median, bounds indicate the 25\%-75\% range, and whiskers represent the points within $\pm2\sigma$}
Using this process we can extract and highlight the primary aspects of an attack by comparing against the deviation from the 'fairness' set; 

\begin{figure}[h]
\begin{subfigure}{0.5\textwidth}
  \centering
  \includegraphics[width=.95\linewidth]{img/trust_bella_static_emph_ADelay_BadMouthingPowerControl.pdf}
  \caption{Delay}
  \label{fig:static_badmouthing_delay}
\end{subfigure}
\begin{subfigure}{0.5\textwidth}
\centering
  \includegraphics[width=.95\linewidth]{img/trust_bella_static_emph_ATXP_BadMouthingPowerControl.pdf}
  \caption{TX Power}
  \label{fig:static_badmouthing_txp}
\end{subfigure}
\begin{subfigure}{0.5\textwidth}
\centering
  \includegraphics[width=.95\linewidth]{img/trust_bella_static_emph_RXThroughput_BadMouthingPowerControl.pdf}
  \caption{RX Throughput}
  \label{fig:static_badmouthing_rxthroughput}
\end{subfigure}
\begin{subfigure}{0.5\textwidth}
\centering
  \includegraphics[width=.95\linewidth]{img/trust_bella_static_emph_TXThroughput_BadMouthingPowerControl.pdf}
  \caption{TX Throughput}
  \label{fig:static_badmouthing_txthroughput}
\end{subfigure}
\caption{$T_{1,MTFM}$ in the All Static case for the Selective Power Control behaviour, emphasising selected metrics and showing the mean and $\pm \sigma$ of $T_{1,MTFM}$ in the same 'fair' scenario}
\label{fig:static_badmouthing}
\end{figure}
%
\begin{figure}
\begin{subfigure}{0.5\textwidth}
  \centering
  \includegraphics[width=.95\linewidth]{img/trust_bella_all_mobile_emph_ARXP_BadMouthingPowerControl.pdf}
  \caption{RX Power}
  \label{fig:all_mobile_badmouthing_rxp}
\end{subfigure}
\begin{subfigure}{0.5\textwidth}
\centering
  \includegraphics[width=.95\linewidth]{img/trust_bella_all_mobile_emph_ATXP_BadMouthingPowerControl.pdf}
  \caption{TX Power}
  \label{fig:all_mobile_badmouthing_txp}
\end{subfigure}
\begin{subfigure}{0.5\textwidth}
\centering
  \includegraphics[width=.95\linewidth]{img/trust_bella_all_mobile_emph_RXThroughput_BadMouthingPowerControl.pdf}
  \caption{RX Throughput}
  \label{fig:all_mobile_badmouthing_rxthroughput}
\end{subfigure}
\begin{subfigure}{0.5\textwidth}
\centering
  \includegraphics[width=.95\linewidth]{img/trust_bella_all_mobile_emph_TXThroughput_BadMouthingPowerControl.pdf}
  \caption{TX Throughput}
  \label{fig:all_mobile_badmouthing_txthroughput}
\end{subfigure}
\caption{$T_{1,MTFM}$ in the All Mobile case for the Selective Power Control behaviour}
\label{fig:all_mobile_badmouthing}
\end{figure}
%
From Fig.~\ref{fig:static_badmouthing} we can see that the selfish node is consistently outside the $\pm\sigma$ envelope of the fair comparison, particularly TX Power, with smaller impacts on RX/TX Throughput, as would be expected for a power related selfish behaviour. 
However, the impact on delay is minimal to insignificant, occasionally breaching the envelope for a short period. 
This was to be expected in a contention-based medium access network operating close to its saturation point; it can be observed that the delay deviance appears to increase as simulation time progresses. 
This indicates that the variation in delay could be caused not by a malicious behaviour but simple congestion.
In the mobile case (Fig.~\ref{fig:all_mobile_badmouthing}) we observe a similar pattern, however it should be noted that the deviation envelop is greatly increased compared to the static case due to the underlying variations in topology and configuration in this scenario.

A significant factor of trust assessment in such a constrained environment is that there may be long periods where two edge nodes (for instance, $n0 \to n_5$) may not interact at all. 
This can be due to a range of factors beyond potential malicious behaviour including simple random scheduling coincidence, and intermediate or neighbouring nodes collectively causing long back-off or contention periods.
This disconnection hinders trust assessment in two ways; assessing nodes that do not receive timely recommendations may make decisions based on very old data, and malicious nodes have a long dwelling time where they can operate under a reasonable certainty that the TMF will not detect it (especially if the node itself is behaving disruptively).
One potential solution to this would be to move from a stepping-window of trust periods to a continuous trust log, updated on packet reception rather than waiting for a number of packets to arrive.

In comparison to \cite{Guo11}, these results are qualitatively similar, however in this case the weighted deltas are significantly less clear than in the comparable terrestrial space, where Guo shows the same type of malicious behaviour and demonstrates a weighted delta from $\approx$ 0.4 to $\approx$ 0.9 across the simulation period, compared to our maximum delta in TX Power of $\approx$ 0.3 for an inconsistent interval.


\subsection{Comparison to OTMF and Beta}

The same experiments were also performed utilising OTMF and Beta assessment as well as MTFM, providing like-for-like comparison of assessment at runtime.
%
\begin{figure}[h]
  \begin{subfigure}{0.5\textwidth}
    \centering
    \includegraphics[width=.95\linewidth]{img/trust_beta_otmf_fair.pdf}
    \caption{Fair Scenario}
    \label{fig:all_mobile_fair_beta}
  \end{subfigure}
  \begin{subfigure}{0.5\textwidth}
    \centering
    \includegraphics[width=.95\linewidth]{img/trust_beta_otmf_malicious.pdf}
    \caption{Selective Power Control scenario}
    \label{fig:all_mobile_badmouthing_beta}
  \end{subfigure}
\caption{$T_{1,0}$ for Beta, OTMF and MTFM assessment values for fair and selfish behaviours in the fully mobile scenario}
\label{fig:otmf_beta_comparison}
\end{figure}
%
The use of Forward Beam Routing and a CSMA/CA MAC scheme from AUVNetSim\cite{Miquel2008} in our simulation mitigates a significant number of packet losses through collision avoidance, and contention handling, leading to the situation that the only genuinely lost packets occur when a node moves completely out of range of any other node and times out in route discovery rather than transmission.
As such, confirmed packet losses are extremely rare and in a delaying network like this, it is difficult to set a differentiating time-out between packets that are in the network but queued, and packets that are actually 'lost'.
This renders OTMF and Beta assessment at best uninformative and at worst misleading; consistently providing nodes a high trust assessment as they have very little information to extract trust from. 
The single metric TMFs used in Terrestrial MANETs require regular and constant streams of positive and negative validation to shape and adjust their evaluations, which for a network with significant delays such as this, is not practical.

Fig.~\ref{fig:otmf_beta_comparison} shows a comparison between the unweighted response of MTFM compared to OTMF and Beta assessment functions on the same data for the fair and selfish behaviours respectively.
It is important to note a distinction between the expectations of MTFM compared to other TMFs; MTFM is primarily concerned with the identification of differences in the behaviours of nodes in a network, and is relative rather than absolute.
That is to say that under MTFM, agents are compared against the worst current performances across metrics of other nodes and graded against them, rather than the absolute (objective) approach taken by many TMFs.
This relative versus absolute difference is particularly clear when comparing mobility models. 
In this case, particularly since the method of attack was not directly related to PLR, OTMF and Beta have not registered significant activity in the correct behaviour.

While the MTFM value does not display any immediate difference between the two behaviours, we have shown that by exploring the metric space by weight variation, the existence and nature of the malicious behaviour can be discovered.
Another difference is that computationally, MTFM is significantly more intensive than the relatively simple Beta / OTMF algorithm, and the repeated metric matrix re-weighting required for real time behaviour detection is an area that requires optimization. 
As such, a hybrid system could be implemented, that used OTMF as a 'trigger' to detect potentially selfish or malicious behaviour, and allow MTMF weight matrix execution to be triggered at less regular intervals.

\section{Conclusions and Future Work}
We have demonstrated that existing MANET Trust Management Frameworks cannot be directly applied to the contentious and dynamic underwater medium.
We presented a comparison between trust establishment in Terrestrial MANET and in the underwater space, demonstrating that in order to have any reasonable expectation of performance, throughput and delay responses must be characterised before implementing trust in such environments. 
We demonstrated initial, unfiltered Grey Trust assessment using all available metrics (transmitted and received throughput, delay, received signal strength, transmitted power, and packet loss rate), as well as the application of multiple weighting vectors to iteratively emphasise different aspects of trust operation to expose and identify misbehaviour on the network.
However, with significant delays (order from seconds to hours), in a fading, refractive medium with varying propagation characteristics, the environment is not as predictable or performant as classical MANET TMF deployment environments.
We show that, without significant adaptation, single metric probabilistic estimation based TMFs are ineffective in such an environment.
We have shown that existing frameworks are overly optimistic about the nature and stability of the communications channel, and can overlook characteristics of the channel that are useful for assessing the behaviour of nodes in the network. 
This indicates that there is a good case, particularly within constrained MANETs such as this, for multi-vector, and even multi-domain trust assessment, where metrics about the communications network and topology would be brought together with information about the physical behaviours and operations of nodes to assess trust.

Future work will investigate the stability of GRA under multi-node collusion, the development of real-time outlier detection and filtering for metrics (e.g differentiating between a very long delay that was an 'accident' and a malicious router), and the introduction of physical metrics and sensing capabilities into the trust management context.



















\section{Trust as an incomplete system characteristic}

While application specific trust management frameworks are often based on a very limited space of available metrics, the problem of establishing trust in dynamical systems such as social, economic or autonomous systems have the opportunity to tap in to a wide range of potential metric spaces. 
Taking the example of Mobile Ad-Hoc Networks (MANET), the variable most applied to the assessment of trust is the packet error rate, or more generally, the number of successful and unsuccessful interactions between two agents within a system.
However, a wealth of other information is available within this example; for instance the delay in communications from one node to another; the total throughput of particular network links; and in the case of wireless networks, the strength of received signals.
Looking beyond the communications domain, within such a MANET, information is also usually available regarding the physical domain of a network; the relative positioning and motions of nodes within a network can also be used to inform the generation of trust assessments.

\section{Grey System Theory}

\subsection{Grey numbers, operators and terminology}

Grey numbers are used to represent values where their discrete value is unknown, where that number may take its possible value within an interval of potential values, generally written using the symbol $\oplus$.
Taking $a$ and $b$ as the lower and upper bounds of the grey interval respectively, such that $\oplus \in [a,b] | a < b$ 
The ``field'' of $\oplus$ is the value space $[a,b]$.
There are several classifications of grey numbers based on the relationships between these bounds.\todo{don't think classification is the right word here}

Black and White numbers are the extremes of this classification; such that $\dot\oplus \in [-\infty, +\infty]$ and $\mathring\oplus \in [x, x] | x \in \mathbb{R}$ or $\oplus(x)$
It is clear that white numbers such as $\mathring\oplus$ have a field of zero while black numbers have an infinite field.

Grey numbers may represent partial knowledge about a system or metric, and as such can represent half-open concepts, by only defining a single bound; for example $\underline\oplus = \oplus(\underline x ) \in [x, +\infty]$ and $\overline\oplus = \oplus(\overline x) \in [-\infty, x]$.

Primary operations within this number system are as follows;

\begin{subequations}
\begin{align}
  \oplus_1 + \oplus_2      &\in [a_1+a_2,b_1+b_2] \label{eq:grey_add}\\
           -\oplus         &\in [-b,-a] \label{eq:grey_neg} \\
  \oplus_1 - \oplus_2      &= \oplus_1+(-\oplus) \label{eq:grey_sub}\\
  \oplus_1 \times \oplus_2 &\in \begin{aligned}[t]
    &[\min(a_1 a_2, a_1 b_2, b_1 a_2, b_2 a_2), \\
    & \max(a_1 a_2, a_1 b_2, b_1 a_2, b_2 a_2)]
  \end{aligned} \label{grey_mult}\\
  \oplus^{-1} &\in [b^{-1}, a^{-1}] \label{eq:grey_inv}\\
  \oplus_1 / \oplus_2 & = \oplus_1 \times \oplus_2^{-1} \label{grey_mult} \\
  \oplus \times k &\in [ka,kb] \label{eq:grey_times_scalar}\\
  \oplus^k &\in [a^k, b^k] \label{eq:grey_exp}
\end{align}
\end{subequations}

where $k$ is a scalar quantity.
  
\subsection{Whitenisation and the Grey Core}

The characterisation of grey numbers is based on the encapsulation of information in a grey system in terms of the grey numbers core ($\hat\oplus$) and it's degree of greyness ($g^\circ$).
If the distribution of a grey number field is unknown and continuous, $\hat\oplus = \frac{a + b}{2}$.

Non-essential grey numbers are those that can be represented by a white number obtained either through experience or particular method. \cite{Liu2011}
This white hissed value is represented by $\tilde\oplus$ or $\oplus(x)$ to represent grey numbers with $x$ as their whitenisation.
In some cases depending on the context of application, particular gray numbers may temporarily have no reasonable whitenisation value (for instance, a black number). Such numbers are said to be Essential grey numbers.

\subsection{Grey Sequence Buffers and Generators}

\todo{eqs of sequence buffers and partial derivs}

Given a fully populated value space, sequence buffer operations are used to provide abstractions over the dataspace.
These abstractions can be \emph{weakening} or \emph{strengthening}.
In the weakening case, these operations perform a level of smoothing on the volatility of a given input space, and strengthening buffers serve to highlight and 

A powerful tool in grey system theory is the use of grey incidence factors, comparing the ``likeness'' of one value against a cohort of values.
This usefulness applies particularly well in the case of multi-agent trust networks, where the aim is to detect and identify malicious or maladaptive behaviour, rather than an absolute assessment of ``trustworthiness''.

\subsection{Grey Trust}

Grey Theory performs cohort based normalization of metrics at runtime. 
This creates a more stable contextual assessment of trust, providing a ``grade'' of trust compared to other observed nodes in that interval, while maintaining the ability to reduce trust values down to a stable assessment range for decision support without requiring every environment entered into to be characterised.
Grey assessments are relative in both fairly and unfairly operating networks.
Nodes will receive mid-range trust assessments if there are no malicious actors as there is no-one else ``bad'' to compare against.

Guo\cite{Guo11} demonstrated the ability of Grey Relational Analysis (GRA)\cite{Zuo1995} to normalise and combine disparate traits of a communications link such as instantaneous throughput, received signal strength, etc. into a Grey Relational Coefficient, or a ``trust vector''.

In the case of the terrestrial communications network used in \cite{Guo11}, the observed metric set $X = {x_1,\dots,x_M}$ representing the measurements taken by each node of its neighbours at least interval, is defined as $X=[$packet loss rate, signal strength, data rate, delay, throughput$]$.
The trust vector is given as
%
\begin{align}
  \label{eq:grc}
  \theta_{k,j}^t = \frac{\min_k|a_{k,j}^t - g_j^t| + \rho \max_k|a_{k,j}^t-g_j^t|}{|a_{k,j}^t-g_j^t| + \rho \max_k|a_{k,j}^t-g_j^t|} \\
  \phi_{k,j}^t = \frac{\min_k|a_{k,j}^t - b_j^t| + \rho \max_k|a_{k,j}^t-b_j^t|}{|a_{k,j}^t-b_j^t| + \rho \max_k|a_{k,j}^t-b_j^t|} \notag 
\end{align}
%
where $a_{k,j}^t$ is the value of a observed metric $x_j$ for a given node $k$ at time $t$, $\rho$ is a distinguishing coefficient set to $0.5$, $g$ and $b$ are respectively the '``good'' and ``bad'' reference metric sequences from $\{a_{k,j}^t k=1,2\dots K\}$, e.g. $g_j=\max_k({a_{k,j}^t})$,  $b_j=\min_k({a_{k,j}^t})$ (where each metric is selected to be monotonically positive for trust assessment, e.g. higher throughput is always better). 

\subsection{PROSE: Whats the point}

Grey System Theory, by it's own authors admission, hasn't taken root in it's originally intended area of system modelling \cite{Liu11}.
However, given it's tentative application to MANET trust, taking a Grey approach on a per metric benefit has qualitative benefits that require investigation; the algebraic approach to uncertainty and the application of ``essential and non essential greyness'', whiteisation, and particularly grey buffer sequencing allow for the opportunity to generate continuous trust assessments from multiple domains asynchronously;

For a given metric set $X$ such that $X = {x_1,\dots,x_M}$ representing the $M$ different types of measurement generated by an observer. If these metrics are not synchronised, for instance if they are interrupt driven such as communications-based observations, generating more abstract measurements requires inherent assumptions about ``how to accumulate the data while you wait''. For instance, in \cite{Bolster2015}, we demonstrated a periodic trust assessment framework for autonomous marine environments, in such an environment, to establish useful, generalised, data, it was necessary to wait for a relatively long time to accumulate enough data to make assessments.
However, this left many 'smells'; data was being left in-buffer for a long time before being used to make decisions, and by the time the data was collated and processed, it could be wildly different from the reality. Further, while some periods could be extremely sparse or even empty, others could be extremely busy with many records having to be averaged down to provide a 'single period' response. 
Therefore, the implementation of a suitable sequence buffer version of the framework would be beneficial.

Such a sequence buffer framework would involve a tracking predictor that would provide best-guess estimates of an interpolated value for a metric between value updates, and a back-propagation algorithm to retroactively update historical assessments of that metrics so as to better inform any abstracted trust value predictor.

I had initially thought that such a back-propogator would be a total mess as I'd imagined that significant-model-breaking would potetially indicate untrustworthy behaviour, but this is stupid since the per-metric-model has the least information of anyone and is simply there to provide better intermediate values and has no / limited direct impact on the overall trust behaviour. 

This backpropogation will probably be a pain to implement as it'd require a retroactive reassessment of trust and could get really messy if it was interrupt driven, but it's better not to prematurly optimise.


% An example of a floating figure using the graphicx package.
% Note that \label must occur AFTER (or within) \caption.
% For figures, \caption should occur after the \includegraphics.
% Note that IEEEtran v1.7 and later has special internal code that
% is designed to preserve the operation of \label within \caption
% even when the captionsoff option is in effect. However, because
% of issues like this, it may be the safest practice to put all your
% \label just after \caption rather than within \caption{}.
%
% Reminder: the "draftcls" or "draftclsnofoot", not "draft", class
% option should be used if it is desired that the figures are to be
% displayed while in draft mode.
%
%\begin{figure}[!t]
%\centering
%\includegraphics[width=2.5in]{myfigure}
% where an .eps filename suffix will be assumed under latex, 
% and a .pdf suffix will be assumed for pdflatex; or what has been declared
% via \DeclareGraphicsExtensions.
%\caption{Simulation Results}
%\label{fig_sim}
%\end{figure}

% Note that IEEE typically puts floats only at the top, even when this
% results in a large percentage of a column being occupied by floats.


% An example of a double column floating figure using two subfigures.
% (The subfig.sty package must be loaded for this to work.)
% The subfigure \label commands are set within each subfloat command, the
% \label for the overall figure must come after \caption.
% \hfil must be used as a separator to get equal spacing.
% The subfigure.sty package works much the same way, except \subfigure is
% used instead of \subfloat.
%
%\begin{figure*}[!t]
%\centerline{\subfloat[Case I]\includegraphics[width=2.5in]{subfigcase1}%
%\label{fig_first_case}}
%\hfil
%\subfloat[Case II]{\includegraphics[width=2.5in]{subfigcase2}%
%\label{fig_second_case}}}
%\caption{Simulation results}
%\label{fig_sim}
%\end{figure*}
%
% Note that often IEEE papers with subfigures do not employ subfigure
% captions (using the optional argument to \subfloat), but instead will
% reference/describe all of them (a), (b), etc., within the main caption.


% An example of a floating table. Note that, for IEEE style tables, the 
% \caption command should come BEFORE the table. Table text will default to
% \footnotesize as IEEE normally uses this smaller font for tables.
% The \label must come after \caption as always.
%
%\begin{table}[!t]
%% increase table row spacing, adjust to taste
%\renewcommand{\arraystretch}{1.3}
% if using array.sty, it might be a good idea to tweak the value of
% \extrarowheight as needed to properly center the text within the cells
%\caption{An Example of a Table}
%\label{table_example}
%\centering
%% Some packages, such as MDW tools, offer better commands for making tables
%% than the plain LaTeX2e tabular which is used here.
%\begin{tabular}{|c||c|}
%\hline
%One & Two\\
%\hline
%Three & Four\\
%\hline
%\end{tabular}
%\end{table}


% Note that IEEE does not put floats in the very first column - or typically
% anywhere on the first page for that matter. Also, in-text middle ("here")
% positioning is not used. Most IEEE journals/conferences use top floats
% exclusively. Note that, LaTeX2e, unlike IEEE journals/conferences, places
% footnotes above bottom floats. This can be corrected via the \fnbelowfloat
% command of the stfloats package.


% conference papers do not normally have an appendix


% use section* for acknowledgement
\section*{Acknowledgment}

 The Authors would like to thank the DSTL/DGA UK/FR PhD Programme for their support during this project.


% trigger a \newpage just before the given reference
% number - used to balance the columns on the last page
% adjust value as needed - may need to be readjusted if
% the document is modified later
%\IEEEtriggeratref{8}
% The "triggered" command can be changed if desired:
%\IEEEtriggercmd{\enlargethispage{-5in}}

% references section

% can use a bibliography generated by BibTeX as a .bbl file
% BibTeX documentation can be easily obtained at:
% http://www.ctan.org/tex-archive/biblio/bibtex/contrib/doc/
% The IEEEtran BibTeX style support page is at:
% http://www.michaelshell.org/tex/ieeetran/bibtex/
\bibliographystyle{IEEEtran}
% argument is your BibTeX string definitions and bibliography database(s)
\bibliography{IEEEabrv,refs}
%
% <OR> manually copy in the resultant .bbl file
% set second argument of \begin to the number of references
% (used to reserve space for the reference number labels box)
%\begin{thebibliography}{1}
%
%\bibitem{IEEEhowto:kopka}
%H.~Kopka and P.~W. Daly, \emph{A Guide to \LaTeX}, 3rd~ed.\hskip 1em plus
%  0.5em minus 0.4em\relax Harlow, England: Addison-Wesley, 1999.
%
%\end{thebibliography}




% that's all folks
\end{document}


