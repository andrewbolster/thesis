\begin{tabular}{
*{2}{@{\hspace{1em}}r@{\hspace{1em}}}
*{3}{@{\hspace{1em}}p{0.1\textwidth} @{\hspace{1em}}}  }
\toprule
 Separation(m) &  Delay(s) &  Probability of Arrival &  RTS/Data Ratio &  Ideal Delivery Time(s) \\
\midrule
           100 &     21.88 &                    0.99 &            1.43 &                    1.03 \\
           150 &     23.23 &                    0.99 &            1.43 &                    1.07 \\
           200 &     21.78 &                    0.99 &            1.39 &                    1.10 \\
           250 &     28.94 &                    0.98 &            1.50 &                    1.14 \\
           300 &     30.31 &                    0.98 &            1.49 &                    1.17 \\
           350 &     36.41 &                    0.98 &            1.51 &                    1.21 \\
           400 &     55.63 &                    0.98 &            1.61 &                    1.25 \\
           450 &    131.67 &                    0.97 &            1.72 &                    1.28 \\
           500 &    163.58 &                    0.97 &            1.71 &                    1.32 \\
           550 &    240.06 &                    0.96 &            1.80 &                    1.35 \\
           600 &    433.12 &                    0.95 &            1.85 &                    1.39 \\
           650 &    901.09 &                    0.92 &            1.96 &                    1.42 \\
           700 &    847.05 &                    0.92 &            1.92 &                    1.46 \\
           750 &   1253.23 &                    0.87 &            2.27 &                    1.50 \\
           800 &   1522.80 &                    0.84 &            2.34 &                    1.53 \\
           850 &   1805.63 &                    0.82 &            2.44 &                    1.57 \\
           900 &   2125.33 &                    0.76 &            3.05 &                    1.60 \\
           950 &   2276.76 &                    0.76 &            3.00 &                    1.64 \\
          1000 &   2231.69 &                    0.73 &            3.46 &                    1.67 \\
\bottomrule
\end{tabular}
