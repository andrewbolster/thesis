\section{A Delay-Tolerant Network Architecture for
Challenged Internets \citet*{Fall2003}}
\label{Fall2003_gen}
\begin{itemize}
  \item This paper deals primarily with the description of Challenged
  network environments, but proposes a solution that is unsuitable for the
  desired application; i.e. centralised DTN Bundle Gateway interconnects
  between networks.
  \item Such an implementation \emph{could} be factored into the generation of a
  truly decentralised system.
  \item Investigates in some detail the characteristics and difficulties of
  Challenged Networks; summarised here.
  \begin{itemize}
    \item Path and Link Characteristics
    \begin{itemize}
      \item \emph{High-Latency, Low Data Rate}
      \item \emph{Disconnection} caused by fault (power loss, tx failure) or
      non-fault (motion or low-duty-cycle) sources. Motion can be predictable or
      opportunistic.
      \item \emph{Long Queuing Times} caused by multi-hop with unknown and
      dynamic topology rendering source-initiated retransmission
      extremely expensive, requiring long packet retention times to negate.
	\end{itemize}
    \item Network Architecture Concerns
    \begin{itemize}
      \item \emph{Interoperability Considerations}; individual CN's not designed
      to a standard 'stack', and thus may fail to implement 'abstractions \ldots
      supporting layered protocol families'.
      \item \emph{Security}; E2E\footnote{End-to-End} security, usually requiring challenge/key
      exchange, is undesirable due to the long latencies and delay prone nature
      of the networks. Additionally, its a waste of data to carry it all the way
      to the destination for authentication/access only to be denied.
	\end{itemize}
    \item End-System Characteristics
    \begin{itemize}
      \item \emph{Limited Longevity}: Message transit may outlive their
      source-node, due to power, strategic, or environmental considerations.
      Fall et all suggest that message reliability monitoring should be
      delegated to a currently (and hopefully, future) operating node.
      \item \emph{Low-Duty-Cycle Operation}: Highlighted in low-power/long-life
      networks where nodes are regularly 'listening' and/or recording data to be
      relayed, and periodically (but more rarely) transmit. The performance of
      this kind of operation and network is dependant on predictable paths
      (leading to efficient time-dependant path selection) and a-priori
      scheduling.
      \item \emph{Limited Resources}: Highlights the decisions to be made with
      regards to memory resources, factoring in RTT, expected retransmissions,
      and maintaining a store of in-transit data.
	\end{itemize}
  \end{itemize}
  \item Additionally, this paper covers the potential use of Proxy or Gateway
  agents at network interface points, E-Mail-like async messaging as
  solutions to the above, before detailing a bundle-gateway based DTN
  architecture based on regional and sub-regional addressing between challenged
  networks, with QoS-like behaviour based on Postal Service conventions (return
  receipt/delivery record/'handle with care')
  \item On the subject of path selection this paper refers to another 2003
  paper\cite{Alonso2003a} which should be reviewed, but is quite technical from
  a cursory skim, so should be delayed.
  \item Introduces the idea of Persistent and Non Persistent Gateways, and
  discussed the effect on network reliability and architecture between these
  nodes.
  \item Deals with network convergence; not entirely relevant to the project at
  hand.
  \item Highlights the congestion difficulties of challenged networks (4.9), eg
  'contacts may not arrive for some time in the future' and 'adopted' packets,
  where a node is delegated custody the ensure that packets reception, which
  cannot be deleted except under extreme conditions. Currently tackled using a
  shared priority queue, but this introduces exploitable behaviour i.e. priority
  inversion / head-of-line blocking.
\end{itemize}

\subsection{Questions Raised}
\begin{itemize}
  \item Fall et al state that the problem of \emph{Security} wastage (i.e.
  carrying excess authentication data to a destination only to be denied) remains an open
  problem on the Internet. I believe that within the general scope of this
  project, such an authentication scheme would be decentralised and
  N2N \footnote{Node-to-node} rather than E2E. This would be an example of
  collaborative delegation, as discussed below.
  \item On the subject of \emph{Limited Longevity}, Fall et al. suggest message
  delivery acknowledgement be delegated to a 'surviving' node. This implies a
  centralised approach which is counter to the project aims. One could envisage
  a collaborative delegation system whereby a subset of nodes within a fleet
  (i.e, the local neighbours N\{\ldots\} of node X upon X's transmission of
  message M), such that any of N can authoritatively accept a delivery
  acknowledgement (although personally I don't think ARQ is suitable for this
  application due to data overheads and network disruption). Doing so would also
  alleviate the issues raised in \emph{Limited Resources}, as once the
  transmission has been spread to its neighbours, N can realistically drop that
  information and assume it will makes its way across the network. This N2N
  approach is tangentially covered in (4.5)
\end{itemize}
