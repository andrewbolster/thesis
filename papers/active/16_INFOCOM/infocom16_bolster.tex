\newcommand*{\PaperRoot}{../../} %
\input{\PaperRoot/templates/ieeetran/bare_conf_header.tex}
\usepackage{todonotes}
\begin{document}
%
% paper title
% can use linebreaks \\ within to get better formatting as desired
\title{Multi-Domain Trust Frameworks for Harsh Environments}


% author names and affiliations
% use a multiple column layout for up to three different
% affiliations
\author{\IEEEauthorblockN{Andrew Bolster}
\IEEEauthorblockA{Department of Electrical Engineering and Electronics\\
University of Liverpool\\
Liverpool, UK\\
Email: bolster@liv.ac.uk}
\and
\IEEEauthorblockN{Alan Marshall}
\IEEEauthorblockA{Department of Electrical Engineering and Electronics\\
University of Liverpool\\
Liverpool, UK\\
Email: alan.marshall@liv.ac.uk}
}
% conference papers do not typically use \thanks and this command
% is locked out in conference mode. If really needed, such as for
% the acknowledgment of grants, issue a \IEEEoverridecommandlockouts
% after \documentclass

% for over three affiliations, or if they all won't fit within the width
% of the page, use this alternative format:
% 
%\author{\IEEEauthorblockN{Michael Shell\IEEEauthorrefmark{1},
%Homer Simpson\IEEEauthorrefmark{2},
%James Kirk\IEEEauthorrefmark{3}, 
%Montgomery Scott\IEEEauthorrefmark{3} and
%Eldon Tyrell\IEEEauthorrefmark{4}}
%\IEEEauthorblockA{\IEEEauthorrefmark{1}School of Electrical and Computer Engineering\\
%Georgia Institute of Technology,
%Atlanta, Georgia 30332--0250\\ Email: see http://www.michaelshell.org/contact.html}
%\IEEEauthorblockA{\IEEEauthorrefmark{2}Twentieth Century Fox, Springfield, USA\\
%Email: homer@thesimpsons.com}
%\IEEEauthorblockA{\IEEEauthorrefmark{3}Starfleet Academy, San Francisco, California 96678-2391\\
%Telephone: (800) 555--1212, Fax: (888) 555--1212}
%\IEEEauthorblockA{\IEEEauthorrefmark{4}Tyrell Inc., 123 Replicant Street, Los Angeles, California 90210--4321}}




% use for special paper notices
%\IEEEspecialpapernotice{(Invited Paper)}




% make the title area
\maketitle


\begin{abstract}
%\boldmath
With the increasing application of autonomy in cyber-physical systems, Trust Management Frameworks (TMFs) are being used to improve the efficiency, security, and reliability of decentralized and distributed autonomous systems. 
In such systems, subtle misbehaviors can significantly impact the operation and performance of the system as a whole. Techniques have been developed for high-speed, uncontended environments such as terrestrial 802.11 MANETs. 
However, these do not perform well in sparse / harsh environments such as those found in Underwater Acoustic Networks (UANs), where network nodes experience significant and variable delays, comparatively low data rates, large contention periods, and considerable multi-path artifacts. 
In such sparse networks, trust establishment based on statistical observations of success/failure events become unstable and ineffective in detecting or identifying misbehaviors. 
Additionally, these methodologies focus solely on the communications actions of entities and do not incorporate the physical domain. 
In this paper we demonstrate the use and operation of a multi-domain trust management framework (MD-TMF) for collaborative mobile autonomous networks (CMANs), using UANs as an exemplar application. 
We also present a machine learning methodology for assessing the relative and collective performance of varying metric sets and subsets in detection and differentiation of a range of communications and physical misbehaviors.

\end{abstract}
% IEEEtran.cls defaults to using nonbold math in the Abstract.
% This preserves the distinction between vectors and scalars. However,
% if the conference you are submitting to favors bold math in the abstract,
% then you can use LaTeX's standard command \boldmath at the very start
% of the abstract to achieve this. Many IEEE journals/conferences frown on
% math in the abstract anyway.

% no keywords




% For peer review papers, you can put extra information on the cover
% page as needed:
% \ifCLASSOPTIONpeerreview
% \begin{center} \bfseries EDICS Category: 3-BBND \end{center}
% \fi
%
% For peerreview papers, this IEEEtran command inserts a page break and
% creates the second title. It will be ignored for other modes.
\IEEEpeerreviewmaketitle



\section{Introduction}
% no \IEEEPARstart
With the increasing application of autonomy in cyber-physical systems, Trust Management Frameworks (TMFs) are increasingly being applied to assist the efficiency, security, and reliability of decentralised and distributed autonomous systems, from highway-bound autonomous vehicles to aerial battlefield drones. 
Classical applications of trust management in Mobile Ad-Hoc Networks (MANETs) have focused solely on observations from the communications domain to make trust assessments. However, these methods are not as effective in applications exhibiting sparse, delayed, or otherwise challenged communications environments\cite{Pavan2015}. MD-TMF expands this paradigm to include relevant physical factors and movements to increase the threat area covered the trust framework. An example area of application is the underwater marine environment, where extreme challenges to communications present themselves (propagation delays, frequency dependent attenuation, fast and slow fading, refractive multi-path distortion, etc.).
In addition to the communications challenges, other considerations such as command and control isolation, as well as power and locomotive limitations, drive towards the use of teams of smaller and cheaper autonomous underwater vehicles (AUVs). These increasingly decentralised applications present unique threats against trust management. In underwater environments, communications is both sparse and noisy.
Therefore the observations about the communications processes that are used to generate the trust metrics, occur much less frequently, with much greater error (noise) and delay than is experienced in terrestrial RF MANETS.
Trust Management Frameworks (TMFs) provide information to assist the estimation of future states and actions of nodes within networks. This information is used to optimize the performance of a network against malicious, selfish, or defective misbehaviour by one or more nodes. Previous research has established the advantages of implementing TMFs in 802.11 based MANETs, particularly in terms of preventing selfish operation in collaborative systems \cite{Li2007}, and maintaining throughput in the presence of malicious actors \cite{Buchegger2002}. Most current TMFs use a single type of observed action to derive trust values, typically successfully delivered or forwarded packets. These observations then inform future decisions of individual nodes, for example, route selection \cite{Li2008}.
Recent work has demonstrated the use of a number of metrics to form a ``vector'' of trust. The Multi-parameter Trust Framework for MANETs (MTFM) \cite{Guo11}, uses a range of communications metrics beyond packet delivery/loss rate (PLR) to assess trust. This vectorized trust also allows a system to detect and identify the tactics being used to undermine or subvert trust. The authors have previously applied this method to the marine space, comparing against a selection of existing communications TMFs \cite{Bolster2015b} showing that MTFM is more effective at detecting misbehaviours in sparse environments. This paper continues and extends that work to encompass physical as well as communications observations in the establishment of trust and the detection and classification of misbehaviours across both physical and communications domains. 
This paper is laid out as follows; in section 2 we discuss Trust and TMFs, defining out terminology and reviewing the justifications for the use and development of TMFs in harsh environments such as UANs. In section 3 we review selected features of the underwater communications channel, highlighting particular challenges against terrestrial equivalents. In section 4 we review the findings of \cite{Bolster2015b} and establish experimental parameters and simulated behaviours under assessment. In section 5 we present our analysis pipeline for assessing misbehaviour using MTFM, and intermediate results of the independent detection of physical and communications misbehaviours using single-domain observations. In section 6 we demonstrate results from multi-domain MTFM and discuss the significance of these findings in terms of detection and classification of cross-domain misbehaviour sets.

\section{Trust and Trust Management Frameworks}

\subsubsection{Trust in Networked Systems}
\todo[inline]{Insert Generic Discussion of Trust Here: ¼ pg}

\subsection{Trust Management in Conventional MANETs}
The distributed and dynamic nature of MANETs mean that it is difficult to maintain a trusted third party (TTP) or evidence based trust system such as Certificate Authorities (CA) or Public Key Infrastructure (PKI). Distributed trust management frameworks aim to detect, identify, and mitigate the impacts of malicious actors by distributing per-node assessments and opinions to collectively police behaviour. Various models and algorithms for describing trust and developing trust management in distributed systems, P2P communities or wireless networks have been considered.
Hermes Trust Establishment Framework uses a Bayesian Beta function to model per-link Packet Loss Rate (PLR) over time, combining ``Trust'' and ``Confidence of Assessment'' into a single value \cite{Zouridaki2005}. Objective Trust Management Framework (OTMF) builds upon Hermes and distributes node observations across the network \cite{Li2008}, however does not appropriately combat multi-node-collusion in the network \cite{Cho2011}. Trust-based Secure Routing demonstrated an extension to Dynamic Source Routing (DSR), incorporating a Hidden Markov Model of sub-networks, reducing the efficacy of Byzantine attacks such as black-hole routing  \cite{Moe2008a}. CONFIDANT presented an approach using a probabilistic estimation of PLR, similar to OTMF, also introducing a topology aware weighting scheme and also weighting trust assessments based on historical experience of the reporter \cite{Buchegger2002}. Fuzzy Trust-Based Filtering uses Fuzzy Inference to adapt to malicious recommenders using conditional similarity to classify performance with overlapping fuzzy set membership, filtering assessments across a network \cite{Luo2008}. 
These TMFs can be generalised as single-value estimation based on a binary input state (success or failure of packet delivery) and generating a probabilistic estimation of the future states of that input. These single metric TMFs provide malicious actors with a significant advantage if their activity does not impact that metric.In the case where the attacker can subvert the TMF, the metric under assessment by that TMF does not cover the threat mounted by the attacker. This causes a significant negative effect on the efficiency of the network, as the TMF is assumed to have reduced the possible set of attacks when it has actually made it more advantageous to attack a different part of the networks operation.
An example of such a situation would be in a TMF focused on PLR where an attacker selectively delays packets going through it, reducing overall throughput but not dropping any packets. Such behaviour would not be detected by the TMF.
Multi-Parameter Trust Framework for MANETS (MTFM) extends this single-parameter approach, applying Grey Relational Analysis \cite{Zuo1995} to provide cohort based normalization of a range of disparate metrics at runtime, providing a “grade” of trust compared to other observed nodes, while maintaining the ability to reduce trust valued down to a stable assessment range for decision support without requiring a-priori environmental or metric characterisation. This presents a stark difference between the previously discussed probabilistic approaches. Grey assessments are relative in both fairly and unfairly operating networks. All nodes will receive mid-range trust assessments if there are no malicious actors as there is nothing ``bad'' to compare against, and variations in assessment will be primarily driven by topological and environmental factors.
Guo et al. \cite{Guo11} demonstrated the ability of grey relational analysis (GRA) \cite{Zuo1995} to normalise and combine disparate traits of a communications link such as instantaneous throughput, received signal strength, etc. into a grey relational coefficient (GRC), or a ``trust vector'' in this instance.

\todo[inline]{Really not sure how much MTFM detail to go in to here}

\section{Marine Acoustic Communications}

The key challenges of underwater acoustic communications are centred around the impact of slow and differential propagation of energy (RF, Optical, Acoustic) through water, and its interfaces with the seabed / air.
The resultant challenges include; long propagation delays, significant inter-symbol interference and Doppler spreading, fast and slow fading due to environmental effects (aquatic flora/fauna, surface weather), carrier-frequency dependent signal attenuation, multi-path caused by reflective medium interfaces, variations in propagation speed due to depth dependant effects (salinity, temperature, and pressure), and subsequent refractive spreading and lensing due to that same propagation variation \cite{Partan2006}.
Refractive lensing and the multi-path nature of the medium result in line of sight propagation being extremely unreliable for estimating distances to targets.
The first arriving acoustic signal has as the very least curved in the medium, and commonly has reflected off the surface/seabed before arriving at a receiver, creating secondary paths that are sometimes many times longer than the first arrival path, generating symbol spreading over orders of seconds depending on the ranges and depths involved.
Forward Error Correction coding is used on such channels to minimise packet losses.
\todo[inline]{I’ve dropped the equations from in here for discussion}

\section{Existing Work}
\todo[inline]{
<summarise Bolster2015b, highlight major take-aways of: 
1.	Classical MANET Trust not directly suitable to harsh environment due to sparsity of observations leading to more statistical noise than signal in terms of detection of misbehaviour
2.	Power of weight variation on the detection and characterisation of misbehaviours
3.	Open Questions (not all of them will be answered here)
}

We have demonstrated that existing MANET Trust Management Frameworks are not directly suitable to the sparse, noisy, and dynamic underwater medium.
We presented a comparison between trust establishment in MANETs in a simulated underwater environment, demonstrating that in order to have any reasonable expectation of performance, throughput and delay responses must be characterised before implementing trust in such environments.  While the MTFM value does not display any immediate difference between the two behaviours, we have shown that by exploring the metric space by weight variation, the existence and nature of the malicious behaviour can be discovered. Another difference is that MTFM is significantly more computationally intensive than the relatively simple Hermes / OTMF algorithms. The repeated metric re-weighting required for real time behaviour detection is therefore an area that requires optimization. 
We demonstrated initial, unfiltered Grey Trust assessment using all available metrics (transmitted and received throughput, delay, received signal strength, transmitted power, and packet loss rate), as well as the application of multiple weighting vectors to iteratively emphasise different aspects of trust operation to expose and identify misbehaviour on the network. With significant delays (from seconds to many minutes), in a fading, refractive medium with varying propagation characteristics, the environment is not as predictable or performant as classical MANET TMF deployment environments. We show that, without significant adaptation, single metric probabilistic estimation based TMFs are ineffective in such an environment.
We have shown that existing frameworks are overly optimistic about the nature and stability of the communications channel, and can overlook characteristics that are useful for assessing the behaviour of nodes in the network. 
This indicates that there is a good case, particularly within constrained MANETs as this, for multi-vector, and even multi-domain trust assessment, where metrics about the communications network and topology would be brought together with information about the physical behaviours and operations of nodes to assess trust. Also, a significant factor of trust assessment in such a constrained environment, is that there may be long periods where two edge nodes (for instance, $n_0 \to n_5$) may not interact at all. 
This can be due to a range of factors beyond malicious behaviour, including simple random scheduling coincidence and intermediate or neighbouring nodes collectively causing long back-off or contention periods.
This disconnection hinders trust assessment in two ways; assessing nodes that do not receive timely recommendations may make decisions based on very old data, and malicious nodes have a long dwelling time where they can operate under a reasonable certainty that the TMF will not detect it (especially if the node itself is behaving disruptively).
\todo{Summarise Experimental setup and selected behaviours}

\section{Analysis Design and Per-Domain Results}

<Formal step-through from simulation results (i.e. position/packet logs) through trust assessment through metric weight exploration with appropriate visual aids. This will mostly be equations / algorithms with some explanations and graphs>
<Finish with presentation of Physical Only / Comms Only Metric/Behaviour results, along with maybe a small discussion on computational complexity of expanding the metric set and a discussion on strategies for mixing/pre-mixing intermediate trust results vs “throw em all in a bag”>

\section{Cross-Domain Results and Discussion}

<Centre behaviour is the “fair” case that everything else is being compared against, and is thus, zero, left are comms behaviours, right are physical behaviours. I’m working on better ways to present this but basically, each behaviour has a ‘signature’ that’s fairly clear that maximises the ‘outlier’ state. In reality, this is maximisation problem, so now that the ‘brute force’ case has been worked, there’s ample opportunity for putting together a greedy algo to work through these cases in much closer to real time. Depending on how successful my experiments are this weekend, I may slip that into this, or may leave for AAMAS in Nov> 

% An example of a floating figure using the graphicx package.
% Note that \label must occur AFTER (or within) \caption.
% For figures, \caption should occur after the \includegraphics.
% Note that IEEEtran v1.7 and later has special internal code that
% is designed to preserve the operation of \label within \caption
% even when the captionsoff option is in effect. However, because
% of issues like this, it may be the safest practice to put all your
% \label just after \caption rather than within \caption{}.
%
% Reminder: the "draftcls" or "draftclsnofoot", not "draft", class
% option should be used if it is desired that the figures are to be
% displayed while in draft mode.
%
%\begin{figure}[!t]
%\centering
%\includegraphics[width=2.5in]{myfigure}
% where an .eps filename suffix will be assumed under latex, 
% and a .pdf suffix will be assumed for pdflatex; or what has been declared
% via \DeclareGraphicsExtensions.
%\caption{Simulation Results}
%\label{fig_sim}
%\end{figure}

% Note that IEEE typically puts floats only at the top, even when this
% results in a large percentage of a column being occupied by floats.


% An example of a double column floating figure using two subfigures.
% (The subfig.sty package must be loaded for this to work.)
% The subfigure \label commands are set within each subfloat command, the
% \label for the overall figure must come after \caption.
% \hfil must be used as a separator to get equal spacing.
% The subfigure.sty package works much the same way, except \subfigure is
% used instead of \subfloat.
%
%\begin{figure*}[!t]
%\centerline{\subfloat[Case I]\includegraphics[width=2.5in]{subfigcase1}%
%\label{fig_first_case}}
%\hfil
%\subfloat[Case II]{\includegraphics[width=2.5in]{subfigcase2}%
%\label{fig_second_case}}}
%\caption{Simulation results}
%\label{fig_sim}
%\end{figure*}
%
% Note that often IEEE papers with subfigures do not employ subfigure
% captions (using the optional argument to \subfloat), but instead will
% reference/describe all of them (a), (b), etc., within the main caption.


% An example of a floating table. Note that, for IEEE style tables, the 
% \caption command should come BEFORE the table. Table text will default to
% \footnotesize as IEEE normally uses this smaller font for tables.
% The \label must come after \caption as always.
%
%\begin{table}[!t]
%% increase table row spacing, adjust to taste
%\renewcommand{\arraystretch}{1.3}
% if using array.sty, it might be a good idea to tweak the value of
% \extrarowheight as needed to properly center the text within the cells
%\caption{An Example of a Table}
%\label{table_example}
%\centering
%% Some packages, such as MDW tools, offer better commands for making tables
%% than the plain LaTeX2e tabular which is used here.
%\begin{tabular}{|c||c|}
%\hline
%One & Two\\
%\hline
%Three & Four\\
%\hline
%\end{tabular}
%\end{table}


% Note that IEEE does not put floats in the very first column - or typically
% anywhere on the first page for that matter. Also, in-text middle ("here")
% positioning is not used. Most IEEE journals/conferences use top floats
% exclusively. Note that, LaTeX2e, unlike IEEE journals/conferences, places
% footnotes above bottom floats. This can be corrected via the \fnbelowfloat
% command of the stfloats package.



\section{Conclusion}
The conclusion goes here.




% conference papers do not normally have an appendix


% use section* for acknowledgement
\section*{Acknowledgment}

The Authors would like to thank the DSTL/DGA UK/FR PhD Programme for their support during this project, as well as NATO CMRE for their advice and assistance.





% trigger a \newpage just before the given reference
% number - used to balance the columns on the last page
% adjust value as needed - may need to be readjusted if
% the document is modified later
%\IEEEtriggeratref{8}
% The "triggered" command can be changed if desired:
%\IEEEtriggercmd{\enlargethispage{-5in}}

% references section

% can use a bibliography generated by BibTeX as a .bbl file
% BibTeX documentation can be easily obtained at:
% http://www.ctan.org/tex-archive/biblio/bibtex/contrib/doc/
% The IEEEtran BibTeX style support page is at:
% http://www.michaelshell.org/tex/ieeetran/bibtex/
\bibliographystyle{IEEEtran}
% argument is your BibTeX string definitions and bibliography database(s)
\bibliography{IEEEabrv,refs}
%
% <OR> manually copy in the resultant .bbl file
% set second argument of \begin to the number of references
% (used to reserve space for the reference number labels box)
%\begin{thebibliography}{1}

%\bibitem{IEEEhowto:kopka}
%H.~Kopka and P.~W. Daly, \emph{A Guide to \LaTeX}, 3rd~ed.\hskip 1em plus
%  0.5em minus 0.4em\relax Harlow, England: Addison-Wesley, 1999.

%\end{thebibliography}




% that's all folks
\end{document}


