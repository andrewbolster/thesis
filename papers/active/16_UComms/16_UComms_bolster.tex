
%% bare_conf.tex
%% V1.3
%% 2007/01/11
%% by Michael Shell
%% See:
%% http://www.michaelshell.org/
%% for current contact information.
%%
%% This is a skeleton file demonstrating the use of IEEEtran.cls
%% (requires IEEEtran.cls version 1.7 or later) with an IEEE conference paper.
%%
%% Support sites:
%% http://www.michaelshell.org/tex/ieeetran/
%% http://www.ctan.org/tex-archive/macros/latex/contrib/IEEEtran/
%% and
%% http://www.ieee.org/

%%*************************************************************************
%% Legal Notice:
%% This code is offered as-is without any warranty either expressed or
%% implied; without even the implied warranty of MERCHANTABILITY or
%% FITNESS FOR A PARTICULAR PURPOSE! 
%% User assumes all risk.
%% In no event shall IEEE or any contributor to this code be liable for
%% any damages or losses, including, but not limited to, incidental,
%% consequential, or any other damages, resulting from the use or misuse
%% of any information contained here.
%%
%% All comments are the opinions of their respective authors and are not
%% necessarily endorsed by the IEEE.
%%
%% This work is distributed under the LaTeX Project Public License (LPPL)
%% ( http://www.latex-project.org/ ) version 1.3, and may be freely used,
%% distributed and modified. A copy of the LPPL, version 1.3, is included
%% in the base LaTeX documentation of all distributions of LaTeX released
%% 2003/12/01 or later.
%% Retain all contribution notices and credits.
%% ** Modified files should be clearly indicated as such, including  **
%% ** renaming them and changing author support contact information. **
%%
%% File list of work: IEEEtran.cls, IEEEtran_HOWTO.pdf, bare_adv.tex,
%%                    bare_conf.tex, bare_jrnl.tex, bare_jrnl_compsoc.tex
%%*************************************************************************

% *** Authors should verify (and, if needed, correct) their LaTeX system  ***
% *** with the testflow diagnostic prior to trusting their LaTeX platform ***
% *** with production work. IEEE's font choices can trigger bugs that do  ***
% *** not appear when using other class files.                            ***
% The testflow support page is at:
% http://www.michaelshell.org/tex/testflow/



% Note that the a4paper option is mainly intended so that authors in
% countries using A4 can easily print to A4 and see how their papers will
% look in print - the typesetting of the document will not typically be
% affected with changes in paper size (but the bottom and side margins will).
% Use the testflow package mentioned above to verify correct handling of
% both paper sizes by the user's LaTeX system.
%
% Also note that the "draftcls" or "draftclsnofoot", not "draft", option
% should be used if it is desired that the figures are to be displayed in
% draft mode.
%
\documentclass[conference]{IEEEtran}
% Add the compsoc option for Computer Society conferences.
%
% If IEEEtran.cls has not been installed into the LaTeX system files,
% manually specify the path to it like:
% \documentclass[conference]{../sty/IEEEtran}

% Some very useful LaTeX packages include:
% (uncomment the ones you want to load)


% *** MISC UTILITY PACKAGES ***
%
\usepackage{ifpdf}
% Heiko Oberdiek's ifpdf.sty is very useful if you need conditional
% compilation based on whether the output is pdf or dvi.
% usage:
% \ifpdf
%   % pdf code
% \else
%   % dvi code
% \fi
% The latest version of ifpdf.sty can be obtained from:
% http://www.ctan.org/tex-archive/macros/latex/contrib/oberdiek/
% Also, note that IEEEtran.cls V1.7 and later provides a builtin
% \ifCLASSINFOpdf conditional that works the same way.
% When switching from latex to pdflatex and vice-versa, the compiler may
% have to be run twice to clear warning/error messages.




% *** CITATION PACKAGES ***
%
%\usepackage{cite}
% cite.sty was written by Donald Arseneau
% V1.6 and later of IEEEtran pre-defines the format of the cite.sty package
% \cite{} output to follow that of IEEE. Loading the cite package will
% result in citation numbers being automatically sorted and properly
% "compressed/ranged". e.g., [1], [9], [2], [7], [5], [6] without using
% cite.sty will become [1], [2], [5]--[7], [9] using cite.sty. cite.sty's
% \cite will automatically add leading space, if needed. Use cite.sty's
% noadjust option (cite.sty V3.8 and later) if you want to turn this off.
% cite.sty is already installed on most LaTeX systems. Be sure and use
% version 4.0 (2003-05-27) and later if using hyperref.sty. cite.sty does
% not currently provide for hyperlinked citations.
% The latest version can be obtained at:
% http://www.ctan.org/tex-archive/macros/latex/contrib/cite/
% The documentation is contained in the cite.sty file itself.




% *** GRAPHICS RELATED PACKAGES ***
%
\ifpdf
  \usepackage[pdftex]{graphicx}
  % declare the path(s) where your graphic files are
  \graphicspath{{../pdf/}{../jpeg/}{img/}}
  % and their extensions so you won't have to specify these with
  % every instance of \includegraph .jpeg,.png}
\else
  % or other class option (dvipsone, dvipdf, if not using dvips). graphicx
  % will default to the driver specified in the system graphics.cfg if no
  % driver is specified.
  \usepackage[dvips]{graphicx}
  % declare the path(s) where your graphic files are
  \graphicspath{{../eps/}}
  % and their extensions so you won't have to specify these with
  % every instance of \includegraphics
  \DeclareGraphicsExtensions{.eps}
\fi% graphicx was written by David Carlisle and Sebastian Rahtz. It is
% required if you want graphics, photos, etc. graphicx.sty is already
% installed on most LaTeX systems. The latest version and documentation can
% be obtained at: 
% http://www.ctan.org/tex-archive/macros/latex/required/graphics/
% Another good source of documentation is "Using Imported Graphics in
% LaTeX2e" by Keith Reckdahl which can be found as epslatex.ps or
% epslatex.pdf at: http://www.ctan.org/tex-archive/info/
%
% latex, and pdflatex in dvi mode, support graphics in encapsulated
% postscript (.eps) format. pdflatex in pdf mode supports graphics
% in .pdf, .jpeg, .png and .mps (metapost) formats. Users should ensure
% that all non-photo figures use a vector format (.eps, .pdf, .mps) and
% not a bitmapped formats (.jpeg, .png). IEEE frowns on bitmapped formats
% which can result in "jaggedy"/blurry rendering of lines and letters as
% well as large increases in file sizes.
%
% You can find documentation about the pdfTeX application at:
% http://www.tug.org/applications/pdftex





% *** MATH PACKAGES ***
%
\usepackage[cmex10]{amsmath}
% A popular package from the American Mathematical Society that provides
% many useful and powerful commands for dealing with mathematics. If using
% it, be sure to load this package with the cmex10 option to ensure that
% only type 1 fonts will utilized at all point sizes. Without this option,
% it is possible that some math symbols, particularly those within
% footnotes, will be rendered in bitmap form which will result in a
% document that can not be IEEE Xplore compliant!
%
% Also, note that the amsmath package sets \interdisplaylinepenalty to 10000
% thus preventing page breaks from occurring within multiline equations. Use:
\interdisplaylinepenalty=2500
% after loading amsmath to restore such page breaks as IEEEtran.cls normally
% does. amsmath.sty is already installed on most LaTeX systems. The latest
% version and documentation can be obtained at:
% http://www.ctan.org/tex-archive/macros/latex/required/amslatex/math/

% ADDED BY BOLSTER
\usepackage{amssymb}




% *** SPECIALIZED LIST PACKAGES ***
%
%\usepackage{algorithmic}
% algorithmic.sty was written by Peter Williams and Rogerio Brito.
% This package provides an algorithmic environment fo describing algorithms.
% You can use the algorithmic environment in-text or within a figure
% environment to provide for a floating algorithm. Do NOT use the algorithm
% floating environment provided by algorithm.sty (by the same authors) or
% algorithm2e.sty (by Christophe Fiorio) as IEEE does not use dedicated
% algorithm float types and packages that provide these will not provide
% correct IEEE style captions. The latest version and documentation of
% algorithmic.sty can be obtained at:
% http://www.ctan.org/tex-archive/macros/latex/contrib/algorithms/
% There is also a support site at:
% http://algorithms.berlios.de/index.html
% Also of interest may be the (relatively newer and more customizable)
% algorithmicx.sty package by Szasz Janos:
% http://www.ctan.org/tex-archive/macros/latex/contrib/algorithmicx/




% *** ALIGNMENT PACKAGES ***
%
%\usepackage{array}
% Frank Mittelbach's and David Carlisle's array.sty patches and improves
% the standard LaTeX2e array and tabular environments to provide better
% appearance and additional user controls. As the default LaTeX2e table
% generation code is lacking to the point of almost being broken with
% respect to the quality of the end results, all users are strongly
% advised to use an enhanced (at the very least that provided by array.sty)
% set of table tools. array.sty is already installed on most systems. The
% latest version and documentation can be obtained at:
% http://www.ctan.org/tex-archive/macros/latex/required/tools/


\usepackage{mdwmath}
\usepackage{mdwtab}
% Also highly recommended is Mark Wooding's extremely powerful MDW tools,
% especially mdwmath.sty and mdwtab.sty which are used to format equations
% and tables, respectively. The MDWtools set is already installed on most
% LaTeX systems. The lastest version and documentation is available at:
% http://www.ctan.org/tex-archive/macros/latex/contrib/mdwtools/


% IEEEtran contains the IEEEeqnarray family of commands that can be used to
% generate multiline equations as well as matrices, tables, etc., of high
% quality.


%\usepackage{eqparbox}
% Also of notable interest is Scott Pakin's eqparbox package for creating
% (automatically sized) equal width boxes - aka "natural width parboxes".
% Available at:
% http://www.ctan.org/tex-archive/macros/latex/contrib/eqparbox/





% *** SUBFIGURE PACKAGES ***
%\usepackage[tight,footnotesize]{subfigure}
% subfigure.sty was written by Steven Douglas Cochran. This package makes it
% easy to put subfigures in your figures. e.g., "Figure 1a and 1b". For IEEE
% work, it is a good idea to load it with the tight package option to reduce
% the amount of white space around the subfigures. subfigure.sty is already
% installed on most LaTeX systems. The latest version and documentation can
% be obtained at:
% http://www.ctan.org/tex-archive/obsolete/macros/latex/contrib/subfigure/
% subfigure.sty has been superceeded by subfig.sty.



%\usepackage[caption=false]{caption}
%\usepackage[font=footnotesize]{subfig}
% subfig.sty, also written by Steven Douglas Cochran, is the modern
% replacement for subfigure.sty. However, subfig.sty requires and
% automatically loads Axel Sommerfeldt's caption.sty which will override
% IEEEtran.cls handling of captions and this will result in nonIEEE style
% figure/table captions. To prevent this problem, be sure and preload
% caption.sty with its "caption=false" package option. This is will preserve
% IEEEtran.cls handing of captions. Version 1.3 (2005/06/28) and later 
% (recommended due to many improvements over 1.2) of subfig.sty supports
% the caption=false option directly:
\usepackage[caption=false,font=footnotesize]{subfig}
%
% The latest version and documentation can be obtained at:
% http://www.ctan.org/tex-archive/macros/latex/contrib/subfig/
% The latest version and documentation of caption.sty can be obtained at:
% http://www.ctan.org/tex-archive/macros/latex/contrib/caption/




% *** FLOAT PACKAGES ***
%
%\usepackage{fixltx2e}
% fixltx2e, the successor to the earlier fix2col.sty, was written by
% Frank Mittelbach and David Carlisle. This package corrects a few problems
% in the LaTeX2e kernel, the most notable of which is that in current
% LaTeX2e releases, the ordering of single and double column floats is not
% guaranteed to be preserved. Thus, an unpatched LaTeX2e can allow a
% single column figure to be placed prior to an earlier double column
% figure. The latest version and documentation can be found at:
% http://www.ctan.org/tex-archive/macros/latex/base/



%\usepackage{stfloats}
% stfloats.sty was written by Sigitas Tolusis. This package gives LaTeX2e
% the ability to do double column floats at the bottom of the page as well
% as the top. (e.g., "\begin{figure*}[!b]" is not normally possible in
% LaTeX2e). It also provides a command:
%\fnbelowfloat
% to enable the placement of footnotes below bottom floats (the standard
% LaTeX2e kernel puts them above bottom floats). This is an invasive package
% which rewrites many portions of the LaTeX2e float routines. It may not work
% with other packages that modify the LaTeX2e float routines. The latest
% version and documentation can be obtained at:
% http://www.ctan.org/tex-archive/macros/latex/contrib/sttools/
% Documentation is contained in the stfloats.sty comments as well as in the
% presfull.pdf file. Do not use the stfloats baselinefloat ability as IEEE
% does not allow \baselineskip to stretch. Authors submitting work to the
% IEEE should note that IEEE rarely uses double column equations and
% that authors should try to avoid such use. Do not be tempted to use the
% cuted.sty or midfloat.sty packages (also by Sigitas Tolusis) as IEEE does
% not format its papers in such ways.





% *** PDF, URL AND HYPERLINK PACKAGES ***
%
\usepackage{url}
% url.sty was written by Donald Arseneau. It provides better support for
% handling and breaking URLs. url.sty is already installed on most LaTeX
% systems. The latest version can be obtained at:
% http://www.ctan.org/tex-archive/macros/latex/contrib/misc/
% Read the url.sty source comments for usage information. Basically,
% \url{my_url_here}.





% *** Do not adjust lengths that control margins, column widths, etc. ***
% *** Do not use packages that alter fonts (such as pslatex).         ***
% There should be no need to do such things with IEEEtran.cls V1.6 and later.
% (Unless specifically asked to do so by the journal or conference you plan
% to submit to, of course. )


% correct bad hyphenation here
\hyphenation{op-tical net-works semi-conduc-tor}

%%%%%%%%%%%%%%%%%%%%%%%%%%%%%%%%%%%
% BOLSTER PACKAGES FOR DEV PERIOD %
%%%%%%%%%%%%%%%%%%%%%%%%%%%%%%%%%%%

\usepackage{booktabs}


\begin{document}
%
% paper title
% can use linebreaks \\ within to get better formatting as desired
\title{Machine Learning Supported Metric Selection and Characterisation for Multi-Domain Trust in Autonomous Underwater MANETs}


% author names and affiliations
% use a multiple column layout for up to three different
% affiliations
\author{\IEEEauthorblockN{Andrew Bolster}
\IEEEauthorblockA{Department of Electrical Engineering and Electronics\\
University of Liverpool\\
Liverpool, UK\\
Email: bolster@liv.ac.uk}
\and
\IEEEauthorblockN{Alan Marshall}
\IEEEauthorblockA{Department of Electrical Engineering and Electronics\\
University of Liverpool\\
Liverpool, UK\\
Email: alan.marshall@liv.ac.uk}
}

% use for special paper notices
\IEEEspecialpapernotice{(Invited Paper for Underwater Communications Security session) Pages: 4, Deadline: 1/4}




% make the title area
\maketitle


\begin{abstract}
%\boldmath

The increasing sensing and communications capabilities of Autonomous Underwater Vehicles (AUVs) presents an opportunity to assess and establish trust based on an increasing range of available metrics. This increased information availability has been demonstrated to be beneficial in the detection and identification of single-node misbehaviour in Underwater Acoustic Networks (MANs) \cite{Bolster2015}.
We present an assistive Machine Learning based methodology to assess and characterise near-optimal trust metric sets and the weights of those metrics for use in Multiparmeter Trust Framework for MANETs (MTFM) \cite{Guo11}

\end{abstract}
% no keywords

% For peer review papers, you can put extra information on the cover
% page as needed:
% \ifCLASSOPTIONpeerreview
% \begin{center} \bfseries EDICS Category: 3-BBND \end{center}
% \fi
%
% For peerreview papers, this IEEEtran command inserts a page break and
% creates the second title. It will be ignored for other modes.
\IEEEpeerreviewmaketitle



\section{Introduction}
% no \IEEEPARstart

Trust in terrestrial MANETs has been an area of active research for many years; with implicitly limited resources in terms of power, mobility, and communications range, the elimination or isolation of malicious nodes within the decentralised MANET is an obvious driver for increasing overall network efficiency. 
However, as these decentralised networks expand beyond the terrestrial arena into aerial and underwater domains, these terrestrially-born efficiencies and security protocols must be assessed for their suitability to these new regions of operation.

With respect to the comparatively stable terrestrial RF environment, the underwater acoustic communications environment is unforgiving; generally static or stable assumptions about propagation delay and paths, frequency-based attenuation and minimal refraction simply don't apply to many marine communications channels.
This, coupled with the highly location, depth, weather, and flora and fauna dependent variability of these fundamental channel parameters make it difficult to make strong assumptions about if instantenous network performance is the fault of a misbehaving node or of a whale passing between nodes for instance.

Trust Management Frameworks (TMFs) provide information to assist the estimation of future states and actions of nodes within networks. 
This information is used to optimise the performance of a network against malicious, selfish or defective misbehaviour by one or more nodes. 
Existing research has demonstrated the advantages of implementing TMFs to 802.11 based MANETs in terms of preventing selfish operation and maintinaing throughput in the presence of malicious nodes \cite{Li2007,Buchegger2002}.

In this paper we build upon previous work \cite{Bolster2015} that demonstrated the use of a naieve Random Forest regression \cite{Breiman2001} to assess the relative importance of Communications Metrics in a simulated MAN, and extend the presented methodology with novel optimisation target functions and applying a range of Machine Learning techniques to investigate their suitability in Metric Assessment across both Physical and Communications Metrics.

This paper is laid out as follows: Sec.~\ref{sec:mtfm} outlines the operation and parameters of MTFM and summarises the comparison of MTFM and other classically terrestrial MANET TMFs.

\section{MTFM and TMF suitability in the Marine Communications Environment}\label{sec:mtfm}



\subsection{Terrestrial Trust Management Frameworks}

\subsubsection{Single Metric TMFs}

Various models and algorithms for describing trust and developing trust management in distributed systems, P2P communities or wireless networks have been considered.
\emph{Hermes Trust Establishment Framework} takes a Bayesian Beta function to model per-link Packet Loss Rate (PLR) over time, combining ``Trust'' and ``Confidence of Assessment'' into a single value \cite{Zouridaki2005}.
\emph{Objective Trust Management Framework} (OTMF) builds upon Hermes and distributes node observations across the network \cite{Li2008}, however does not appropriately combat multi-node-collusion in the network \cite{Cho2011}.
\emph{Trust-based Secure Routing} demonstrated an extension to Dynamic Source Routing (DSR), incorporating a Hidden Markov Model of sub-networks, reducing the efficacy of Byzantine attacks such as black-hole routing  \cite{Moe2008a}.
\emph{CONFIDANT} presented an approach using a probabilistic estimation of PLR, similar to OTMF, also introducing a topology aware weighting scheme and also weighting trust assessments based on historical experience of the reporter \cite{Buchegger2002}.
\emph{Fuzzy Trust-Based Filtering} uses Fuzzy Inference to adapt to malicious recommenders using conditional similarity to classify performance with overlapping fuzzy set membership, filtering assessments across a network \cite{Luo2008}. 

These TMFs can be generalised as single-value estimation based on a binary input state (success or failure of packet delivery) and generating a probabilistic estimation of the future states of that input. 

These single metric TMFs provide malicious actors with a significant advantage if their activity does not impact that metric.
In the case where the attacker can subvert the TMF, the metric under assessment by that TMF does not cover the threat mounted by the attacker.
This causes a significant negative effect on the efficiency of the network, as the TMF is assumed to have reduced the possible set of attacks when it has actually made it more advantageous to attack a different part of the networks operation.
An example of such a situation would be in a TMF focused on PLR where an attacker selectively delays packets going through it, reducing overall throughput but not dropping any packets.
Such behaviour would not be detected by the TMF.

\subsubsection{Multiparameter Trust for MANETs}

Guo et al. \cite{Guo11} demonstrated the ability of grey relational analysis (GRA) \cite{Zuo1995} to normalise and combine disparate traits of a communications link such as instantaneous throughput, received signal strength, etc. into a grey relational coefficient (GRC), or a ``trust vector'' in this instance.

Grey Theory performs cohort based normalization of metrics at runtime, providing a ``grade'' of trust compared to other observed nodes in that interval, while maintaining the ability to abstract trust values for decision support without requiring per-environment calibration or characterisation.
This presents a stark difference between the Grey and Probabilistic approaches.
Grey assessments are relative in both fairly and unfairly operating networks; all nodes receive mid-range trust assessments if there are no malicious actors as there is nothing ``bad'' to compare against, and variations in assessment will be primarily driven by topological and environmental factors.

The grey relational vector is given as
%
\begin{align}
  \label{eq:grc}
  \theta_{k,j}^t = \frac{\min_k|a_{k,j}^t - g_j^t| + \rho \max_k|a_{k,j}^t-g_j^t|}{|a_{k,j}^t-g_j^t| + \rho \max_k|a_{k,j}^t-g_j^t|} \\
  \phi_{k,j}^t = \frac{\min_k|a_{k,j}^t - b_j^t| + \rho \max_k|a_{k,j}^t-b_j^t|}{|a_{k,j}^t-b_j^t| + \rho \max_k|a_{k,j}^t-b_j^t|} \notag 
\end{align}
%
where $a_{k,j}^t$ is the value of an observed metric $x_j$ for a given node $k$ at time $t$, $\rho$ is a distinguishing coefficient set to $0.5$, $g$ and $b$ are respectively the ``good'' and ``bad'' reference metric sequences from $\{a_{k,j}^t k=1,2\dots K\}$, i.e. $g_j=\max_k({a_{k,j}^t})$,  $b_j=\min_k({a_{k,j}^t})$ (where each metric is selected to be monotonically positive for trust assessment, e.g. higher throughput is presumed to be always better). 

Weighting can be applied before generating a scalar value \eqref{eq:metric_weighting} allowing the detection and classification of misbehaviours.

%
\begin{equation}
  \label{eq:metric_weighting}
  [\theta_k^t, \phi_k^t] = \left[\sum_{j=0}^M h_j \theta_{k,j}^t,\sum_{j=0}^M h_j \phi_{k,j}^t \right]
\end{equation}
%
Where $H=[h_0\dots h_M]$ is a metric weighting vector such that $\sum h_j = 1$, and in unweighted case, $H=[\frac{1}{M},\frac{1}{M}\dots\frac{1}{M}]$.
$\theta$ and $\phi$ are then scaled to $[0,1]$ using the mapping $y = 1.5 x - 0.5$.
To minimise the uncertainties of belonging to either best ($g$) or worst ($b$) sequences in \eqref{eq:grc} the $[\theta,\phi]$ values are reduced into a scalar trust value by $T_k^t = ({1+{(\phi_k^t)^2}/{(\theta_k^t)^2}})^{-1}$ \cite{Hong2010}.
MTFM combines this GRA with a topology-aware weighting scheme \eqref{eq:networkeffects} and a fuzzy whitenization model \eqref{eq:whitenization}. 

There are three classes of topological trust relationship used; Direct, Recommendation, and Indirect.
Where an observing node $n_i$ assesses the trust of another target node, $n_j$; the Direct relationship is $n_i$'s own observations $n_j$'s behaviour.
In the Recommendation case, a node $n_k$ which shares Direct relationships with both $n_i$ and $n_j$, gives its assessment of $n_j$ to $n_i$.
In the Indirect case, similar to the Recommendation case, the recommender $n_k$ does not have a direct link with the observer $n_i$ but $n_k$ has a Direct link with the target node, $n_j$.
These relationships give node sets, $N_R$ and $N_I$ containing the nodes that have recommendation or indirect, relationships to the observing node respectively.
%
\begin{align}
  \label{eq:networkeffects}
  T_{i,j}^{MTFM}=&\frac{1}{2} \cdot \max_s\{f_s(T_{i,j})\} T_{i,j}\\ \notag
  +&\frac{1}{2} \frac{2|N_R| }{2|N_R| + |N_I|}\sum_{n \in N_R} \max_s\{f_s(T_{i,n})\} T_{i,n}\\ \notag
  +&\frac{1}{2} \frac{|N_I| }{2|N_R| + |N_I|}\sum_{n \in N_I} \max_s\{f_s(T_{i,n})\} T_{i,n} 
\end{align}

Where $T_{i,n}$ is the subjective trust assessment of $n_i$ by $n_n$, and $f_s = [ f_1,f_2, f_3]$ given as:

\begin{align}
  \label{eq:whitenization}
  f_1(x)&= -x+1\notag\\
  f_2(x)&= 
  \begin{cases}
    2x & \text{if }x\leq 0.5\\
    -2x+2 & \text{if }x>0.5
  \end{cases}\\
  f_3(x)&= x\notag
\end{align}
%


\subsubsection{Summary of Previous Work}
In \cite{Bolster2015}, Hermes trust establishment and OTMF were selected indicative single-metric TMFs to compare against MTFM, as Hermes captures the core operation of a pure single metric assessment methodology and OTMF provides a comparison that combines assessments from across nodes to develop trust opinions.

To fairly transition from Terrestrial to Marine spaces, a series of simulations were performed to establish a suitable optimal scaling rage for a sample scenario with six nodes performing a persistent survey operation. 
Both communications rate and physical separations were scaled to optimise throughput and minimise network saturation across a range of mobility models.
Subsequently, MTFM, Hermes and OTMF were applied to a range of control simulations in this new scaled range to assess their performance in detecting and identifying two modes of misbehaviour; Malicious Power Control (MPC) where an attacker raises it's transmission power for all nodes \emph{except} a particular target node, making that target node appear to be selfishly conserving energy to the rest of the network while the attacker itself appears to be operating fairly; and Selfish Target Selection (STS) where an ``attacker'' preferentially communicates with nodes near to itself to minimise energy use at the cost of network fairness and throughput.

This comparison demonstrated that while the performance of MTFM, Hermes and OTMF were all severely reduced in the marine environment compared to that in the terrestrial, Unweighted MTFM showed a consistent if small (~10\%) deviation in misbehaviour-cases and no such deviation in the Fair case.
When the individual metrics in MTFM were preferentially weighted, it was demonstrated that a ``Relevance Signature'' could be generated for detectable behaviours based on a Random Forest regression of multiple weighted assessments, but this method of characterisation was incomplete and did not produce generalised metric weights for use in real-time detection.

\section{Some clever section title}

The identification and establishment of weight-based ``filters'' for MTFM has classically been accomplished through the use of variations on 
the Analytic Hierarchy Process (AHP)\cite{Saaty2008}.
AHP attempts to systematically capture the relative metric performance in detecting alternative behaviours through the trust impact comparison of iteratively emphasising individual metrics.
However, similar applications of AHP has been criticised in the past \cite{Whitaker2007} for making potentially misleading or outright incorrect assumptions about the relative scaling and relationships between individual metrics, and this correctness being highly sensitive to the designer-imposed hierarchies used to generate ``optimal'' alternatives.

As such, we propose a purely computational methodology for filter-weight generation.

\subsection{The one where you talk about Mean T Deltas}

Before embarking on any optimisation, it is important to establish what analytical behaviour is being targeted; in this case this target is the identifiable exposure of a ``low'' trust value for a misbehaving node that is as distinct from other ``fair'' nodes as possible across a full (or multiple) simulation runs of a particular behaviour.

We characterise this ``objective function'' as $\Delta\overline T_{ix}$

\begin{equation}
  \Delta \overline T_{ix} = \frac{\sum_{j\neq x}\left( \overline{T_{i,j}}^{\forall t}\right)}{N-1} \label{eq:delta_t}
\end{equation}

Where $T_{i,j}$ is the resultant weighted trust value from ~\ref{eq:networkeffects} of node $j$ from the perspective of node $i$ and $N$ is the number of nodes in the current cohort.
The larger this $\Delta\overline T_{ix}$ value, the further that particular weight shows the misbehaving node as a ``Untrustworthy''.

Initially, we mirror the methodology of MTFM (i.e. emphasising single metrics independently) but take it to an extreme by instead fully weighting one metric at a time ($H_3 = {0,0,0,1,0,0}$ for example). 

When this extremely simple 



\section{Conclusion}
The conclusion goes here.




% conference papers do not normally have an appendix


% use section* for acknowledgement
\section*{Acknowledgment}


The Authors would like to thank the DSTL/DGA UK/FR PhD Programme for their support during this project, as well as NATO CMRE for their advice and assistance.

\bibliographystyle{IEEEtran}
% argument is your BibTeX string definitions and bibliography database(s)
\bibliography{IEEEabrv,refs}

% that's all folks
\end{document}


% An example of a floating figure using the graphicx package.
% Note that \label must occur AFTER (or within) \caption.
% For figures, \caption should occur after the \includegraphics.
% Note that IEEEtran v1.7 and later has special internal code that
% is designed to preserve the operation of \label within \caption
% even when the captionsoff option is in effect. However, because
% of issues like this, it may be the safest practice to put all your
% \label just after \caption rather than within \caption{}.
%
% Reminder: the "draftcls" or "draftclsnofoot", not "draft", class
% option should be used if it is desired that the figures are to be
% displayed while in draft mode.
%
%\begin{figure}[!t]
%\centering
%\includegraphics[width=2.5in]{myfigure}
% where an .eps filename suffix will be assumed under latex, 
% and a .pdf suffix will be assumed for pdflatex; or what has been declared
% via \DeclareGraphicsExtensions.
%\caption{Simulation Results}
%\label{fig_sim}
%\end{figure}

% Note that IEEE typically puts floats only at the top, even when this
% results in a large percentage of a column being occupied by floats.


% An example of a double column floating figure using two subfigures.
% (The subfig.sty package must be loaded for this to work.)
% The subfigure \label commands are set within each subfloat command, the
% \label for the overall figure must come after \caption.
% \hfil must be used as a separator to get equal spacing.
% The subfigure.sty package works much the same way, except \subfigure is
% used instead of \subfloat.
%
%\begin{figure*}[!t]
%\centerline{\subfloat[Case I]\includegraphics[width=2.5in]{subfigcase1}%
%\label{fig_first_case}}
%\hfil
%\subfloat[Case II]{\includegraphics[width=2.5in]{subfigcase2}%
%\label{fig_second_case}}}
%\caption{Simulation results}
%\label{fig_sim}
%\end{figure*}
%
% Note that often IEEE papers with subfigures do not employ subfigure
% captions (using the optional argument to \subfloat), but instead will
% reference/describe all of them (a), (b), etc., within the main caption.


% An example of a floating table. Note that, for IEEE style tables, the 
% \caption command should come BEFORE the table. Table text will default to
% \footnotesize as IEEE normally uses this smaller font for tables.
% The \label must come after \caption as always.
%
%\begin{table}[!t]
%% increase table row spacing, adjust to taste
%\renewcommand{\arraystretch}{1.3}
% if using array.sty, it might be a good idea to tweak the value of
% \extrarowheight as needed to properly center the text within the cells
%\caption{An Example of a Table}
%\label{table_example}
%\centering
%% Some packages, such as MDW tools, offer better commands for making tables
%% than the plain LaTeX2e tabular which is used here.
%\begin{tabular}{|c||c|}
%\hline
%One & Two\\
%\hline
%Three & Four\\
%\hline
%\end{tabular}
%\end{table}


% Note that IEEE does not put floats in the very first column - or typically
% anywhere on the first page for that matter. Also, in-text middle ("here")
% positioning is not used. Most IEEE journals/conferences use top floats
% exclusively. Note that, LaTeX2e, unlike IEEE journals/conferences, places
% footnotes above bottom floats. This can be corrected via the \fnbelowfloat
% command of the stfloats package.


