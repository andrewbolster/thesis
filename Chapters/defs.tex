%http://tex.stackexchange.com/questions/32867/tikz-rectangular-node-with-different-rounded-corners
\makeatletter
\pgfdeclareshape{roundedNode}{
	\inheritsavedanchors[from=rectangle] % this is nearly a rectangle
	\inheritanchorborder[from=rectangle]
	\inheritanchor[from=rectangle]{center}
	\inheritanchor[from=rectangle]{north}
	\inheritanchor[from=rectangle]{south}
	\inheritanchor[from=rectangle]{west}
	\inheritanchor[from=rectangle]{east}
	\backgroundpath{% this is new
		% store lower right in xa/ya and upper right in xb/yb
		\southwest \pgf@xa=\pgf@x \pgf@ya=\pgf@y
		\northeast \pgf@xb=\pgf@x \pgf@yb=\pgf@y
		% construct main path
		\pgfsetcornersarced{\pgfpoint{5pt}{5pt}}
		\pgfpathmoveto{\pgfpoint{\pgf@xa}{\pgf@ya}}
		\pgfpathlineto{\pgfpoint{\pgf@xa}{\pgf@yb}}
		\pgfpathlineto{\pgfpoint{\pgf@xb}{\pgf@yb}}
		\pgfsetcornersarced{\pgfpoint{5pt}{5pt}}
		\pgfpathlineto{\pgfpoint{\pgf@xb}{\pgf@ya}}
		\pgfpathclose
	}
}
\makeatother
\tikzset{
	block/.style = {draw, fill=white, rectangle, minimum height=3em, minimum width=3em, text width=8em, text centered},
	tmp/.style  = {coordinate}, 
	sum/.style= {draw, fill=white, circle, node distance=1cm},
	input/.style = {coordinate},
	output/.style= {coordinate},
	pinstyle/.style = {pin edge={to-,thin,black}
	},
	from/.style args={#1 to #2}{% without transformations
		above right={0cm of #1},% needs positioning library
		/utils/exec=\pgfpointdiff
		{\tikz@scan@one@point\pgfutil@firstofone(#1)\relax}
		{\tikz@scan@one@point\pgfutil@firstofone(#2)\relax},
		minimum width/.expanded=\the\pgf@x,
		minimum height/.expanded=\the\pgf@y},
	doc/.style={draw, shape=roundedNode, align=center},
	bluefill/.style={fill=blue, opacity=0.5, text opacity=1}
}

% USAGE: \cercle {center} {radius in cm} {begin degrees} {value of the arc} {line width} {color}
% \coordinate (OO) at (2.8,2.2);
% \cercle{OO}{0.5cm}{-20}{60}{1.0pt}{blue};
\newcommand{\cercle}[6]{
\node[circle,inner sep=0,minimum size={2*#2}](a) at (#1) {};
\draw[#6,line width=#5] (a.#3) arc (#3:{#3+#4}:#2);
}


% \SpecialCoor
\def\subsum{\mathit{\Sigma}}

\setcounter{secnumdepth}{2}
\setcounter{tocdepth}{2}


\newenvironment{myindentpar}[1]%
{\begin{list}{}%
         {\setlength{\leftmargin}{#1}}%
         \item[]%
}
{\end{list}}


\renewcommand*{\figureautorefname}{Fig.}
\renewcommand*{\chapterautorefname}{Chapter}
\renewcommand*{\sectionautorefname}{Section}
\renewcommand*{\subsectionautorefname}{Subsection}

\newcommand*{\fullref}[1]{\hyperref[{#1}]{\autoref*{#1} \nameref*{#1}}} % One single link

