\section{A Survey of Practical Issues in Underwater Networks \cite{Partan2006}}
\label{Partan2006}
\section{General Summary}
\section{A Survey of Practical Issues in Underwater Networks
 \citet*{Partan2006}}
\label{Partan2006_gen}
\begin{itemize}
  \item Predominantly Medium Access Control (MAC) considerations of marine
  networks, technological, and economical.
  \item Classifies (marine) networks into 4 (or 5, counting null network)
  regimens; 
  \begin{itemize}
    \item Null Networks (Nodes too distant and immobile for communication)
    \item Disruption Tolerant Networks (DTN, sparse but sufficiently
    mobile networks)
    \item Unpartitioned, multi-hop networks (overlapping chains of TDMA/CDMA 
    clusters with MACA or S-FAMA \cite{Molins2006})
    \item Single-Hop TDMA networks (fully connected network coverage, with
    limited contention)
    \item Dense Single-Hop Network (ok for CSMA, not much else, bandwidth
    contention)
  \end{itemize}
  \item Summarises marine channel characterise, generally summed up as 'bad'
  \begin{itemize}
    \item \textbf{Acoustic}: Experiences Propagation Delays and
    Doppler Effects\footnote{Speed of sound in water is 1500m/s (depth
    variant)}, as well as limited bandwidth due to frequency dependant
    attenuation. Additionally, high \acro{BER} due to Phase and Amplitude
    variations, requiring \acro{FEC}. See \cite{Catipovic1990}
    \item \textbf{RF}: Poor but possible (122 kHZ, 6-10m, 1-8kbits/sec
    )
    \item \textbf{Optical}: Very poor but possible in \textit{extremely} clear
    environments (490-500 nm, \textless 100m, several Mbits/s)
    \item \textbf{IR}: potential to use OTS IrDA TX/RX for low cost,
    short range links (1-2m, 57.6kbits/s)
  \end{itemize}
  \item Includes several Operational Examples (4.1, 5.1, 5.2, 5.5) that will be
  useful later.
  \item Introduces LBL\footnote{Long Baseline, an acoustic positioning method
  conceptually similar to GPS} localisation and potential contention between
  Navigational and data comms (5.2)
  \item Introduces a variety of DTN packet exchange techniques for mobile
  networks (5.3,5.4)
\end{itemize}

\subsection{Questions Raised}
\begin{itemize}
  \item It is not explained or justified what causes the indicated phase and
  amplitude variations that lead to high \acro{BER}. Possibly covered in
  \cite{Catipovic1990}.
  \item In one operational example (MCM, 4.1), a central gateway buoy was used
  as a uplink service and hub for a star-network of AUVs. Depending on how
  DSTL/DGA want to pursue this, this demonstrates a necessary compromise
  between centrality for power and communications economy, and decentralised security and
  reliability.
\end{itemize}

\section{Physical Layer; Channel and Technological Limitations}
\subsection{Acoustic Characteristics}
Covered in more depth in \cite{Catipovic1990}
\subsection{Radio Characteristics}
In general, \acro{RF} communications are heavily attenuated in salt water, but 
Long-wave radio can be used in short range links (\acro{122 kHZ}, 6-10m, 
1-8kbits/sec )
\subsection{Optical Characteristics}
Heavily scattered and absorbed in water, although blue-green wavelengths may be
used to short range (\lt100m) high bandwidth (several Mbits/s) connections in
particularly clear conditions. \cite{Partan2006}([10])

Also potentially viable (stated as 'considered) are the use of OTS IrDA
tranceivers (1-2m \@ 57.6kbits/s)
\section{MAC Layer Considerations}
\section{Node Mobility Implications inc. Energy Concerns}


