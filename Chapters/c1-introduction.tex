% !TeX spellcheck = en_GB
\def\ChapterTitle{Introduction} % Write in your own chapter title

\chapter{\ChapterTitle}
\label{Chapter\thechapter}
\lhead{Chapter \thechapter.
\emph{\nameref{Chapter\thechapter}}} % Write in your own chapter title to set the page header


\section{\glspl{manet}}\label{manets}

With the explosive growth in the use of mobile telephony and the increasing miniaturisation and efficiency gains of portable communications devices, the classical paradigm of a broadcast/receiver or server/client has given way to an increasing use of decentralised, ad-hoc networks that not only accommodate but take advantage of network mobility.

Whether these networks are decentralised cellular / RF / 802.11 WiFi networks for use in disaster relief areas \cite{Milliken2015} or biologically inspired wireless sensor networks for low-energy, low-maintenance environmental monitoring \cite{Nicholson2008}\cite{Selvakennedy2007}, \gls{manet} theory developed over the past 30 years has gone from it's first formal definition, emerging from DARPA's Packet Radio Network research \cite{Jubin1987}, to being an integral part of modern practical communications.

Minimally, a \gls{manet} consists of of a collection of mobile physical entities (nodes) that communicate cooperatively to collect, distribute, disseminate, and collate data and/or influence across an area.
In most cases \gls{manet} nodes incorporate bi-directional transceivers to send and receive data, however this bi-directionality is not a requirement (for example in the area of Wireless Sensor Networks \cite{Akyildiz2002}).
\glspl{manet} may utilise omnidirectional, static, or steerable communications antennae, and a selection of protocols such as WiFi, Bluetooth, GSM, UMTS, Optical or Acoustics, and may incorporate a range of mobilities across nodes, from static devices, terrestrial and marine surface platforms, and aerial and underwater platforms
A core characteristic of \glspl{manet} is the inclusion and integration of heterogeneous node collections, i.e where different nodes or groups of nodes in a network may have different capabilities in terms of propulsion, sensor apparatus, communications capability, etc.

\glspl{manet} may be totally independent with no external connections; include independent per-node communications backhauls (e.g. Cellular Modems in mobile phones as part of a Bluetooth Personal Area Network), or include static nodes that provide infrastructure based backhaul.
However, this multiplicity of variations and options presents several challenges to users and operators; the physical topology of \glspl{manet} can vary wildly over short periods of time.
A particular challenge to \gls{manet} operation is that given any node may operate as a routing / gateway node, if/when that node moves to a different region, network segments that had previously used that node as a path must renegotiate / re-establish their routes.
These situations, if not appropriately managed, lead to opportunities for subversion and selfishness.

The characteristics of \gls{manet}s as defined by Corson et al.\ are paraphrased in Table~\ref{tab:manet_characteristics}.

\begin{table}[h!]
  \hyphenpenalty=10000
\caption[Summary of Characteristics of \glspl{manet}]{Summary of Characteristics of \glspl{manet}\cite{Corson1999}}
\label{tab:manet_characteristics}
  \begin{tabularx}{\textwidth}{p{2cm}X}\toprule
    Dynamic Topologies & Nodes are free to move arbitrarily; thus, the typically multi-hop network topology may change randomly and rapidly at unpredictable times, and may consist of both bidirectional and unidirectional links.
\\
    Bandwidth Constrained, Varied Capacity & Wireless links will continue to have significantly lower capacity than their hardwired counterparts.
In addition, the realized throughput of wireless communications, after accounting for the effects of multiple access, fading, noise, and interference conditions, etc., is often much less than a radio's maximum transmission rate.
\par
One effect of the relatively low to moderate link capacities is that congestion is typically the norm rather than the exception, i.e.\  aggregate application demand will likely approach or exceed network capacity frequently.\\
    Energy Constrained Operation &  Some or all of the nodes in a \gls{manet} may rely on batteries or other exhaustible means for their energy.
For these nodes, the most important system design criteria for optimization may be energy conservation.\\
    Limited physical security & Mobile wireless networks are generally more prone to physical security threats than are fixed cable nets.
The increased possibility of eavesdropping, spoofing, and denial-of-service attacks should be carefully considered.\par
Existing link security techniques are often applied within wireless networks to reduce security threats.
As a benefit, the decentralized nature of network control in \glspl{manet} provides additional robustness against the single points of failure of more centralized approaches.\\\bottomrule
\end{tabularx}
\end{table}


\section{Node Density in \glspl{manet}}

One fundamental compromise in the operation of wireless \glspl{manet} is the trade-off between the number of hops required between source and destination nodes and the effective bandwidth available to the network overall\cite{Royer2001}.
This compromise is encapsulated in the relative density of a given network; that is, the number of nodes in a given node's one-hop locality, drawing direct links between wireless transmission strength / reception sensitivity, the environmental noise floor, environmental channel characteristics, the mobility of the nodes and the number of nodes deployed in a region.
\todo{Expand Node Density discussion to include examples of sparse, dense, long/deep, fully connected networks}


\section{Routing in \acrlong{manet}}

Given the decentralised nature of \gls{manet} operations, routing protocols are an active area of research. 
This research is classified according to the strategies used for discovering, monitoring and updating routes within the network, and are usually grouped into three classes; proactive (or Table Driven), reactive (or On Demand) and hybrid.
A summary of the generalised characteristics of these classes is shown in \autoref{tab:routing_categories}.
Additional, these classes can be further rearranged, combined or augmented based on assumptions made about the structure of the baseline topology, i.e. flat, hierarchical or geographic) or by assumed constraints of the available resources of the nodes within the network i.e. heterogeneity in power, mobility, or communications capabilities where resource-based heuristics are used as well as purely topological considerations\cite{Li2005}\cite{Gerla2002}.

\subsection{Proactive Routing}

In Proactive routing, protocols attempt to maintain a up-to-date, global topology awareness of the network, where every node knows how the best next-hop to contact any other node in the network.
This is extremely efficient for relatively small, static networks, with minimal storage and time requirements \cite{}.
When the network topology is significantly modified by a shift in topology, either due to a node ``dropping out'' or moving, route renegotiation and optimisation is extremely resource consuming, as this global state is converged upon in a distributed manner by nodes exchanging their local knowledge of the ``new'' topology.
The decomposition and updating of the node-knowledge of the network state, and the method of updating these state-tables, is the primary differentiator between proactive protocols, a selection of which are summarised in \autoref{tab:proactive_routing_protocols}.

\begin{table}\centering
  \caption[Selection of Proactive Routing Protocols]{Selection of Proactive Routing Protocols}
  \label{tab:proactive_routing_protocols}
  \begin{tabularx}{\textwidth}{p{1.25cm}|X}\toprule
    Name & Description \\ \midrule
    \gls{dsdv} & \acrlong{dsdv} is a loop free derivative of the Distributed Bellman-Ford algorithm where each node maintains two tables; one that attempts to maintain a globally accurate next-hop routing table for all destination nodes (the routing table) and a route advertisement table, monitoring routes that the node itself can provide. These tables are updated both periodically and opportunistically. Loop-free status is maintained by monitoring a monotonic ``sequence number'', which guarantees that if a long-loop returned packet is observed, it is discarded in favour of a route with a higher sequence number (i.e.\ newer route)~\cite{Perkins1994}.\\
    \gls{olsr} & \acrlong{olsr} reduces the traffic-overhead of truly distributed link-state exchange and monitoring by establishing a multipoint replaying strategy (MRP) where nodes select a subset of their one-hop network relay to retransmit their packets, based on the two-hop connectivity of the network, thereby reducing contention and overheads by reducing local re-transmitters. However, \gls{olsr} does not monitor link \emph{quality} beyond binary ``active/failed'' state which can lead to non-optimial MRP and route selection in wireless networks for instance.\\
    \gls{tbrpf} & \acrlong{tbrpf} consists of two main modules; the neighbour discovery (TND) module and the routing module; the TND uses differential updates to report only the \emph{changes} in the local topology. Further, instead of flooding the entire network with updates, \gls{tbrpf} selectively updates the relevant route updates to nodes that are on the minimum spanning tree route of of that update. Full topology updates are also used on a periodic, but occasional basis to maintain consistency and visibility. Given the use of differential updating, \gls{tbrpf} is more responsive and resilient in the face of dynamic mobile networks and incurs lower traffic overheads.\cite{Bellur1999}\\
    \bottomrule
  \end{tabularx}
\end{table}


\subsection{Reactive Routing}

In contract to Proactive Routing, Reactive (or ``on-demand'') routing establishes routing information when it is required, rather than in advance or periodically.
This route establishment is usually based on a request-response exchange where the node requesting routing information ``floods'' its local network with next-hop requests.
The structure of this flooding (and the context of any responses) are the main differentiators between protocols, as shown in \autoref{tab:reactive_routing_protocols}.
There are two main sub-classes of reactive routing; source routing and hop-by-hop routing where packet routing information is either totally planned in advance and encapsulated in the packet on transmission, or decided at each forwarding point respectively.
The on-demand nature of of route discovery can lead to significantly lower traffic than proactive routing protocols, but this is often a trade-off between lower average traffic and larger pre-transmission discovery delays. 
As such, reactive routing lends itself to low-traffic, delay tolerant, dynamic mobile applications as it does not require rediscovery after every \emph{topology} change, but only on transmission along a new or stale route.

\begin{table}\centering
  \caption[Selection of Reactive Routing Protocols]{Selection of Reactive Routing Protocols}
  \label{tab:reactive_routing_protocols}
  \begin{tabularx}{\textwidth}{p{1.25cm}|X}\toprule
    Name & Description \\ \midrule
    \gls{dsr} &  On-demand route formation when a transmitting node requests one. However, packets include full routing information instead of relying on the routing tables at each intermediate device\cite{Johnson1996}. Effective for small to medium, minimally mobile networks due to inclusion of route caching which reduces the number of route request discovery phases and associated congestion. Inneffective in large networks due to full-route packet overheads as network scale increases, and less than ideal for mobile networks due to cache-miss delays.\\
    \gls{aodv} & based on \gls{dsdv} and \gls{dsr}; uses beaconing and sequence numbering (\gls{dsdv}) as well as shortest path route discovery from \gls{dsr}, except that \gls{aodv} does not use full-path source routing, instead relying on intermediate-routing based only on a destination. This optimisation reduces overhead and is more resiliant to highly dynamic deployments at the cost of variable and potentially very long delays due to slow route construction or complete retransmissions due to link failure.\cite{Rai2010a}  \\
    \gls{roam} & uses internodal coordination along directed acyclic subgraphs, which is derived from the routers’ distance to destination. This operation is referred to as a “diffusing computation”. The advantage of this protocol is that it eliminates the search-to-infinity problem present in some of the on-demand routing protocols by stopping multiple flood searches when the required destination is no longer reachable. Another advantage is that each router maintains entries (in a route table) for destinations, which flow data packets through them (i.e. the router is a node which completes/or connects a router to the destination). This reduces significant amount of storage space and bandwidth needed to maintain an up-to-date routing table. Another novelty of \gls{roam} is that each time the distance of a router to a destination changes by more than a defined threshold, it broadcasts update messages to its neighbouring nodes, as described earlier. Although this has the benefit of increasing the network connectivity, in highly dynamic networks it may prevent nodes entering sleep mode to conserve power.\\ % REWORD THIS ITS DIRECTLY FROM \cite{Abolhasan2004}
    \gls{abr} & \acrlong{abr} extends classical source-routing by including a stability (``associativity'') heuristic of the long-term link state between mobile nodes, ensuring that the least-mobile nodes are preferentially used for routing. Further, this heuristic is applied outward from destination rather than from the source, selecting only the ``best'' route, reducing the liklihood of packet duplication in the mid-network. However this ``associativity'' measure requires periodic beaconing forcing all nodes to remain active. Finally, in the case where the ``best'' route fails through an in-the-air topology change, there is no in-network path recovery mechanism, and link discovery must be restarted\cite{Toh1997}.\\
    \gls{lar} & \acrlong{lar} incorporates location information (usually from \gls{gps}), and generates a heuristic based on either the distance from the current node \emph{towards} the destination location, or the distance from the current node \emph{away from} the original source, minimising and maximising this distance respectively.These methods limit control overheads and usually accurately determine the shortest path. However, in highly mobile networks this behaviour appears increasingly flood-like (similar to \gls{dsr} and \gls{aodv}), and the general requirement for highly accurate and timely positional information restricts the application of this protocol\\
    \gls{cbrp} & \acrlong{cbrp} uses a hierarchical clustering topology where each cluster has a cluster-head which coordinates routing within that cluster. As only cluster-heads coordinate routing across clusters, transmission overheads are minimised compared to other route distribution methods. However, the negotiation and maintence overheads and propogation delays associated with hierarchical clustering make the network susceptible to temporary routing loops as nodes may have inconsistent residual routing information during cluster re-negotiation\\
    
    \bottomrule
  \end{tabularx}
\todo{What other fundamental reactive routing strategies am I forgetting?}
\end{table}

\subsection{Hybrid Routing}

\todo{Write Hybrid Routing Blurb}

\begin{table}\centering
  \caption[Selection of Hybrid Routing Protocols]{Selection of Hybrid Routing Protocols}
  \label{tab:hybrid_routing_protocols}
  \begin{tabularx}{\textwidth}{p{1.25cm}|X}\toprule
    Name & Description \\ \midrule
    \gls{zrp} & A true-blend of Proactive and Reactive policies; \gls{zrp} draws ``Routing Zones'' around nodes based on hop-distance, within which routing is made proactively, providing immediate local routes, and reactive, on-demand routes outside this distance. This significantly reduces local overheads and delays by reducing the scope of potential routes as those nodes on the edge of the zone. The control of this boundary point is a significant challenge to optimise with respect to overall neetwork scale. \\
    \gls{dst} & Based on a combination of Hybrid Tree Flooding and Distributed Spanning Tree shuttling on tree based clusters where each cluster has a root node acting as a configuration leader. Routing updates are passed through direct neighbours and ``up'' the spanning trees under the root; this leads to a highly responsive and low over head routing policy in highly dynamic networks.\cite{Radhakrishnan1999}\\
    \gls{ddr} & A tree protocol similar to \gls{dst} except without the need for a root node; trees are constructed and maintined by periodic neighbour beaconing, where each node becomes the potential root of its own tree within the ``forest'' of the wider network. The construction of this forest follows six phases; neighbour election, forest construction, intra-tree clustering, inter-tree clustering, zone naming and zone partitioning. Each of these phases are executed based on information received in the beacon messages. One of the strengths of \gls{dst} is its lack of centralisation or a-priori structure requirement (i.e. root/clusterheads or static zone maps), however there is no equalisation method to balance the case where a gateway node that is a preferred neighbour to many subtrees becomes congested and represents a significant bottleneck, as the neighbour selection is predicated on graph connectivity alone, without taking maximum throughput into account. In variably connected networks this could potentially cause network-wide delays through mid-network packet drops.\\
    \gls{zhls} & Compared to \gls{zrp}, \gls{zhls} extendes the zoning concept to include some elements of \gls{lar}, by constructing hierarchial non-overlapping zones based on physical location as well as connectivity. This location management is purely decentralised, with no explicit ``zone-heads'', eliminating single-point-of-failure concerns and significatly reducing invalid-flooding overheads, as topology updates only carry towards zones where the information is relevant. Shifting topologies within zones are tolerated cleanly as the internal zone-map is flat rather than hierarchial, and does not require recomputation or re-location as long as the node stays within a given ``zone''. This static map also leads to a significant disadvantage in that this zone-map must be pre-set, so are inappropriate for fully-mobile applications or applications with dynamic geographic boundaries.\cite{Joa-Ng1999,Hamma2006} \\
    \gls{slurp} & With a similar hierarchial non-overlappung zone structure to \gls{zhls}, \gls{slurp} does away with global routing through a deterministic mapping of node identifiers to ``Home'' regions, such that any node attempting to communicate with a node, can directly calculate from what ``Home'' zone that node originated.As and when nodes leave their ``Home'' region, they feedback to that region their current location. Subscequently when that node is a destination for a packet, the routing query is automatically directed to the home region which can direct the source node as to the direction of its destination, upon which the source can start sending data towards the destination using a most forward with fixed radius (MFR) geographical forwarding algorithm. Once the data reaches the zone where the destination currently resides, source-routing is used internally to complete the route. This strategy works well for relatively static networks with some mobile nodes, or where node mobility is ``slow'', such as \gls{wsn}, however it still relies on pre-programmes static zone maps as per \gls{zhls}\cite{Woo2001}\\
    \bottomrule
  \end{tabularx}
\end{table}


\begin{table}\centering
  \caption[Comparison of Routing Strategy Classes]{Comparison of Routing Strategy Classes (from \citet{Abolhasan2004})}
  \label{tab:routing_categories}
  \begin{tabularx}{\textwidth}{p{2cm}|XXX}\toprule
    \diagbox[width=2cm, height=1.8cm]{Area}{Class} & Proactive & Reactive & Hybrid \\ \midrule
    Routing Structure & 
      Both flat and hierarchical structures are available &
      Mostly flat except \gls{cbrp} &
      Mostly hierarchical \\
    Route Availability &
      Always available if nodes are reachable &
      Determined when needed &
      Depends on the location of the destination \\
    Control Traffic Volume &
      Usually high, attempt at reduction is made. e.g. \gls{olsr}, \gls{tbrpf} &
      Lower than Global routing and further improved using \gls{gps}. e.g. \gls{lar} &
      Mostly lower than proactive and reactive \\
    Periodic Updating &
      Yes, some may be conditional e.g. \gls{star} &
      Not required, however some nodes may require periodic beacons. e.g. \glspl{abr} &
      Usually used within each zone or between gateway nodes \\
    Mobility Handling &
      Usually updates occur at fixed intervals. \gls{dream} alters periodic updates based on mobility &
      \gls{abr} uses localised broadcast queries, \gls{roam} uses threshold updates, \gls{aodv} routing uses local route discovery &
      Usually more than one path may be available. Single point of failures are reduces by working as a group\\
    Storage Requirements &
      High &
      Dependent on number of nodes kept or required; usually lower than proactive protocols &
      Usually depends on cluster or zone size; may become as large as proactive if clusters are big \\
    Delay Level &
      Short routes are predetermined &
      Higher than proactive &
      Short for destinations in the same zone/cluster as source. Inter-zone may be as large as Reactive protocols\\
    Scalability &
      Up to 100 nodes; \gls{ospf} and \gls{tbrpf} may scale higher &
      Source routing protocols; up to a few hundred nodes. Point-to-point may scale higher. Depends on level of traffic and levels of multihopping&
      Designed for up to or more than 1000 nodes \\
    \bottomrule
  \end{tabularx}
\end{table}



\section{\glspl{manet} in Harsh Environments}

As \acrfullpl{manet} grow beyond the terrestrial arena, their operation and the protocols designed around them must be reviewed to assess their suitability to different communications environments, ensuring their continued security, reliability, and performance.

The distributed and dynamic nature of \glspl{manet} mean that it is difficult to maintain an evidence based trust system such as \gls{ttp}, Certificate Authorities or using \gls{pki}. 
In both cases, there is the assumption of a run-time canonical source of trust, i.e. a ``Master'' node or Certifying Authority that can objectively coordinate the security and trust of the network.
This single-point-of-failure is antithetical to \gls{manet} architectures, and given the normally limited transmission, storage, battery and computational power of \gls{manet} nodes, the overheads of true \gls{ttp} or \gls{pki} architectures have been out of the realms of practicality for most applications.
Therefore, a distributed, collaborative system must be applied to these networks.
\footnote{ \citet{Zouridaki} has demonstrated an intriguing low-power Distributed CA based \gls{manet} architecture, however given the soon-to-be-discussed assumptions about capable attackers(~\autoref{sec:capable_attackers}), this ``decentralised'' approach is less than ideal}
Such distributed trust management frameworks aim to detect, identify, and mitigate the impacts of malicious actors by distributing per-node assessments and opinions to collectively self-police behaviour.
As such, \glspl{tmf} can be used to predict and reason on the future interactions between entities in a system.

\glspl{tmf} provide information to assist the estimation of future states and actions of nodes within \glspl{manet}.
This information is used to optimize the performance of a network against malicious, selfish, or defective misbehaviour by one or more nodes.
Previous research has established the advantages of implementing \glspl{tmf} in 802.11 based \glspl{manet}, particularly in terms of preventing selfish operation in collaborative systems \cite{Li2007}, and maintaining throughput in the presence of malicious actors \cite{Buchegger2002}

\section{Systems Approach to Trust and Trust Engineering}



\section{Trust Operation Against Capable Attackers}\label{sec:capable_attackers}

\todo{Trust operation against capable attackers}

\section{Contributions}

\todo{Contributions}

\section{Conclusion}

\todo{Conclusions including Layout}

\subsection{Layout}

\subsubsection{\autoref{Chapter2}}
In this chapter the current literature and research on the concepts, theory, and applications concerning Trust and Trust Management are explored, specifically leaning towards the applications of Trust within Autonomous \glspl{manet}.

In~\autoref{sec:trust_defs}, the abstract quantity of ``trust'' is explored,
In~\autoref{sec:trust_autonomy}, Autonomy and ``Trusted Operation'' of autonomous systems is investigated from a system architects and a system operators perspective.
In~\autoref{sec:trust_manets}, current use and applications of Trusted operation of \glspl{manet} is explored, including current \glspl{tmf}.

\subsubsection{\autoref{Chapter3}}
In \autoref{Chapter3}, the maritime context is investigated, particularly concerned with the mechanisms of maritime acoustic communications, including the opportunities and challeges of the marine acoustic channel and its modelling. 

\subsubsection{\autoref{Chapter4}}
In this chapter, the need for multi-metric trust assessment in \gls{uan} is demonstrated as an example of a harsh network environment.

The operation of a selection of traditional \gls{manet} \glspl{tmf} in this environment is investigated.
These challenges are characterised and results are presented that demonstrate a multi-metric approach to Trust greatly enhances the effectiveness of \glspl{tmf} in these environments.

In~\autoref{sec:initialsystemcharacterization} an experimental configuration for the marine space is established, and the scenarios and results presented in \cite{Guo11} are reviewed for comparison.
In~\autoref{sec:trustresultsanddiscussion} findings in trust establishment and malicious behaviour detection are presented and comparing with other current \glspl{tmf} (Hermes and \gls{otmf}) and the use of this multi-parameter approach to detecting malicious and selfish behaviour in autonomous marine networks is analysed.

The contributions of this chapter are a study on the comparative operation and performance of \glspl{tmf} in marine acoustic networks, and a review of metric suitability for \glspl{tmf} in marine environments, informing future metric selection for experimenters and theorists.
Finally, a methodology to assess the usefulness of metrics in discriminating against misbehaviours in such constrained, delay-tolerant networks is demonstrated.

Key parts of this chapter were presented at TrustCom-BigDataSE-ISPA 2015 as ``Single and Multi-Metric Trust Management Frameworks for use in Underwater Autonomous Networks.''\cite{Bolster2015}








%%%%%%%%%%%%%%%%%%%%%%%%%%%%%%%%%%%%%%%%%%%%%%%%%%%%%%%%%%%%%%%%%%%%%%%%%%%%%%%
