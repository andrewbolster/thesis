
%%%%%%%%%%%%%%%%%%%%%%% file adhocnow15_bolster.tex %%%%%%%%%%%%%%%
%[Ad Hoc Now 2015](http://www.netmode.ntua.gr/adhocnow2015/index.html)
%
%# Dates
%* Deadline: 7/2/15
%* Acceptance: 7/3/15
%* Due: 28/3/15
%* Dates: 29/6/15-2/7/15
%
%# Topics of Interest
%* Access Control 
%* Ad Hoc Networks of Autonomous Intelligent Systems 
%* Algorithmic Issues
%* Analytic Methods and Modeling for Performance Evaluation 
%* Ad Hoc Network Applications and Architectures 
%* Delay-Tolerant Networking
%* Distributed Algorithms for Ad Hoc Networks 
%* Energy Efficiency 
%* Geometric Graphs
%* Location Discovery and Management 
%* Mobility Handling and Utilization 
%* Wireless Mesh Networks
%* Big Data Inspired Data Sensing
%* Mobile Ad Hoc Computing Platforms
%* Systems and Testbeds 
%* Mobile Social Networking 
%* Quality-of-Service 
%* Routing Protocols (Unicast, Multicast, etc.) 
%* Secure Services and Protocols 
%* Sensor Networks 
%* Self-Configuration 
%* Service Discovery 
%* Timing Synchronization 
%* Vehicular Networks 
%* Wireless Internet
%* Processing and Networking Technologies Complexity and Computational Issues
%* Prototype systems and real-world deployment experiences
%
%# Proposed Area of Focus
%Re-present Bellas Chap 4 work with extensions to explain and explore differences between Ad Hoc and Marine.
%
%
%%%%%%%%%%%%%%%%%%%%%%%%%%%%%%%%%%%%%%%%%%%%%%%%%%%%%%%%%%%%%%%%%%%


\documentclass[runningheads,a4paper]{llncs}

\usepackage{amssymb}
\setcounter{tocdepth}{3}
\usepackage{graphicx}

\usepackage{url}
\newcommand{\keywords}[1]{\par\addvspace\baselineskip
\noindent\keywordname\enspace\ignorespaces#1}

\begin{document}

\mainmatter  % start of an individual contribution

% first the title is needed
\title{Trust Framework Operation in Autonomous Marine Communications Environments}

% a short form should be given in case it is too long for the running head
\titlerunning{Trust Framework Operation in Marine Communications Environments}

% the name(s) of the author(s) follow(s) next
%
% NB: Chinese authors should write their first names(s) in front of
% their surnames. This ensures that the names appear correctly in
% the running heads and the author index.
%
\author{Andrew Bolster%
\thanks{Please note that the LNCS Editorial assumes that all authors have used
the western naming convention, with given names preceding surnames. This determines
the structure of the names in the running heads and the author index.}
, Alan Marshall, Ji Guo}
%
\authorrunning{Trust Framework Operation in Marine Communications Environments}
% (feature abused for this document to repeat the title also on left hand pages)

% the affiliations are given next; don't give your e-mail address
% unless you accept that it will be published
\institute{Advanced Networks Research Group, \\Department of Electrical Engineering \& Electronics,\\
University of Liverpool, UK\\
\url{{andrew.bolster,alan.marshall}@liv.ac.uk}\\
\url{http://www.anrg.liv.ac.uk/}}

\toctitle{Lecture Notes in Computer Science}
\tocauthor{Authors' Instructions}
\maketitle


\begin{abstract}
  This paper presents an overview of trust assessment within classical ad-hoc networking environments (Terrestrial MANET) and presents a comparative study on the operation and performance of such trust frameworks between the terrestrial and underwater communications environments.
\keywords{ad-hoc, MANET, trust, marine, underwater}
\end{abstract}


\section{Introduction}

Trust Management Frameworks (TMFs) provide information regarding the estimated future states and operations of nodes within networks. They are used to optimize the performance of a system of systems (i.e. collections of autonomous, semi-autonomous, and/or human systems) in the face of malicious, selfish, or defective behavior by one or more nodes within such a system. 
Previous research has established the potential advantages of implementing distributed TMFs in mobile ad-hoc networks (MANETs)

\subsubsection*{Acknowledgments.} The heading should be treated as a
subsubsection heading and should not be assigned a number.

\section{The References Section}\label{references}

\bibliographystyle{amsplain}
\bibliography{refs}
% 
% \begin{thebibliography}{4}
% 
% \bibitem{jour} Smith, T.F., Waterman, M.S.: Identification of Common Molecular
% Subsequences. J. Mol. Biol. 147, 195--197 (1981)
% 
% \bibitem{lncschap} May, P., Ehrlich, H.C., Steinke, T.: ZIB Structure Prediction Pipeline:
% Composing a Complex Biological Workflow through Web Services. In: Nagel,
% W.E., Walter, W.V., Lehner, W. (eds.) Euro-Par 2006. LNCS, vol. 4128,
% pp. 1148--1158. Springer, Heidelberg (2006)
% 
% \bibitem{book} Foster, I., Kesselman, C.: The Grid: Blueprint for a New Computing
% Infrastructure. Morgan Kaufmann, San Francisco (1999)
% 
% \bibitem{proceeding1} Czajkowski, K., Fitzgerald, S., Foster, I., Kesselman, C.: Grid
% Information Services for Distributed Resource Sharing. In: 10th IEEE
% International Symposium on High Performance Distributed Computing, pp.
% 181--184. IEEE Press, New York (2001)
% 
% \bibitem{proceeding2} Foster, I., Kesselman, C., Nick, J., Tuecke, S.: The Physiology of the
% Grid: an Open Grid Services Architecture for Distributed Systems
% Integration. Technical report, Global Grid Forum (2002)
% 
% \bibitem{url} National Center for Biotechnology Information, \url{http://www.ncbi.nlm.nih.gov}
% 
% \end{thebibliography}


\begin{itemize}
\settowidth{\leftmargin}{{\Large$\square$}}\advance\leftmargin\labelsep
\itemsep8pt\relax
\renewcommand\labelitemi{{\lower1.5pt\hbox{\Large$\square$}}}

\item The final \LaTeX{} source files
\item A final PDF file
\item A copyright form, signed by one author on behalf of all of the
authors of the paper.
\item A readme giving the name and email address of the
corresponding author.
\end{itemize}
\end{document}
