
% ----------------------------------------------------------------------
%                   LATEX TEMPLATE FOR PhD THESIS
% ----------------------------------------------------------------------

% based on Harish Bhanderi's PhD/MPhil template, then Uni Cambridge
% http://www-h.eng.cam.ac.uk/help/tpl/textprocessing/ThesisStyle/
% corrected and extended in 2007 by Jakob Suckale, then MPI-CBG PhD programme
% and made available through OpenWetWare.org - the free biology wiki


%: Style file for Latex
% Most style definitions are in the external file PhDthesisPSnPDF.
% In this template package, it can be found in ./Latex/Classes/
% Move this to individual files
\documentclass[twoside,9pt,a4paper]{Latex/Classes/PhDthesisPSnPDF}


%: Macro file for Latex
% Macros help you summarise frequently repeated Latex commands.
% Here, they are placed in an external file /Latex/Macros/MacroFile1.tex
% An macro that you may use frequently is the figuremacro (see introduction.tex)
\include{Latex/Macros/MacroFile1}



%: ----------------------------------------------------------------------
%:                  TITLE PAGE: name, degree,..
% ----------------------------------------------------------------------
% below is to generate the title page with crest and author name

%if output to PDF then put the following in PDF header
\ifpdf  
    \pdfinfo { /Title  (An Investigation into Trust and Reputation Frameworks for Autonomous Underwater Vehicles)
               /Creator (TeX)
               /Producer (pdfTeX)
               /Author (Andrew Bolster abolster01@qub.ac.uk)
               /CreationDate (D:YYYYMMDDhhmmss)  %format D:YYYYMMDDhhmmss
               /ModDate (D:YYYYMMDDhhmm)
               /Subject (xyz)
               /Keywords (add, your, keywords, here) }
    \pdfcatalog { /PageMode (/UseOutlines)
                  /OpenAction (fitbh)  }
\fi


\title{An Investigation into Trust and Reputation Frameworks for Autonomous Underwater Vehicles}



% ----------------------------------------------------------------------
% The section below defines www links/email for author and institutions
% They will appear on the title page of the PDF and can be clicked
\ifpdf
  \author{\href{mailto:me@andrewbolster.info}{Andrew Bolster}}
%  \cityofbirth{born in XYZ} % uncomment this if your university requires this
%  % If city of birth is required, also uncomment 2 sections in PhDthesisPSnPDF
%  % Just search for the "city" and you'll find them.
  \collegeordept{\href{http://www.ecit.qub.ac.uk}{Institute of Electronics, Communications and Information Technology (ECIT)}}
  \university{\href{http://www.qub.ac.uk}{Queen's University Belfast (QUB)}}

  % The crest is a graphics file of the logo of your research institution.
  % Place it in ./frontmatter/figures and specify the width
  \crest{\includegraphics[width=4cm]{frontmatter/figures/crest.png}}
  
% If you are not creating a PDF then use the following. The default is PDF.
\else
  \author{Andrew Bolster}
%  \cityofbirth{born in XYZ}
  \collegeordept{Institute of Electronics, Communications and Information Technology (ECIT)}
  \university{Queen's University Belfast (QUB)}
  \crest{\includegraphics[width=4cm]{frontmatter/figures/crest.png}}
\fi

%\renewcommand{\submittedtext}{change the default text here if needed}
\degree{Three Week Report}
\degreedate{2011 October}


% ----------------------------------------------------------------------
       
% turn of those nasty overfull and underfull hboxes
\hbadness=10000
\hfuzz=50pt

%: --------------------------------------------------------------
%:                  FRONT MATTER: dedications, abstract,..
% --------------------------------------------------------------

\begin{document}

%\language{english}

% sets line spacing
\renewcommand\baselinestretch{1.2}
\baselineskip=18pt plus1pt


%: ----------------------- generate cover page ------------------------

\maketitle  % command to print the title page with above variables

%: ----------------------- abstract ------------------------

% Your institution may have specific regulations if you need an abstract and where it is to be placed in the document. The default here is just after title.

\include{frontmatter/twr_abstract}

% The original template provides and abstractseparate environment, if your institution requires them to be separate. I think it's easier to print the abstract from the complete thesis by restricting printing to the relevant page.
% \begin{abstractseparate}
%   As \glspl{auv} become more technically capable and economically feasible, they are being increasingly used in a great many areas of defence, commercial and environmental applications. 
These applications are tending towards using independent, autonomous, ad-hoc, collaborative behaviour of teams or fleets of these \gls{auv} platforms.
This convergence of research experiences in the \gls{uan} and \gls{manet} fields, along with the increasing \gls{loa} of such platforms, creates unique challenges to secure the operation and communication of these networks.

The question of security and reliability of operation in networked systems has usually been resolved by having a centralised coordinating agent to manage shared secrets and monitor for misbehaviour.
However, in the sparse, noisy and constrained communications environment of \glspl{uan}, the communications overheads and single-point-of-failure risk of this model is challenged (particularly when faced with capable attackers).

As such, more lightweight, distributed, experience based\footnote{rather than ``Evidence based'' in the case of shared keys, \gls{pki} etc.} systems of ``Trust'' have been proposed to dynamically model and evaluate the ``trustworthiness'' of nodes within a \gls{manet} across the network to prevent or isolate the impact of malicious, selfish, or faulty misbehaviour. 
Previously, these models have monitored actions purely within the communications domain. 
Moreover, the vast majority rely on only one type of observation (metric) to evaluate trust; successful packet forwarding.
In these cases, motivated actors may use this limited scope of observation to either perform unfairly without repercussions in other domains/metrics, or to make another, fair, node appear to be operating unfairly.

This thesis is primarily concerned with the use of terrestrial-\gls{manet} trust frameworks to the \gls{uan} space. 
Considering the massive theoretical and practical difference in the communications environment, these frameworks must be reassessed for suitability to the marine realm. 
We find that current single-metric \glspl{tmf} do not perform well in a best-case scaling of the marine network, due to sparse and noisy observation metrics, and while basic multi-metric communications-only frameworks perform better than their single-metric forms, this performance is still not at a reliable level. 
We propose, demonstrate (through simulation) and integrate the use of physical observational metrics for trust assessment, in tandem with metrics from the communications realm, improving the safety, security, reliability and integrity of autonomous \glspl{uan}.

Three main novelties are demonstrated in this work:
Trust evaluation using metrics from the physical domain (movement/distribution/etc.), demonstration of the failings of Communications-based Trust evaluation in sparse, noisy, delayful and non-linear \gls{uan} environments, and the deployment of trust assessment across multiple domains, e.g.\ the physical and communications domains.
The latter contribution includes the generation and optimisation of cross-domain metric composition or``synthetic domains'' as a performance improvement method.

% \end{abstractseparate}

%: ----------------------- contents ------------------------

\setcounter{secnumdepth}{3} % organisational level that receives a numbers
\setcounter{tocdepth}{3}    % print table of contents for level 3
\tableofcontents            % print the table of contents
% levels are: 0 - chapter, 1 - section, 2 - subsection, 3 - subsection

%: --------------------------------------------------------------
%:                  MAIN DOCUMENT SECTION
% --------------------------------------------------------------

% the main text starts here with the introduction, 1st chapter,...
\mainmatter

\renewcommand{\chaptername}{} % uncomment to print only "1" not "Chapter 1"


% --------------------------------------------------------------
%:                  REVIEW BODY: Intro, Lit summary, Lit Plan
% --------------------------------------------------------------
\chapter{Cursory Summary of Reviewed Literature}
% Literature Review summaries
\section{Currently Reviewed / Under Review Reading}
\begin{itemize}
  \item \citet*{Partan2006}
  \item \citet*{Fall2003}
  \item \citet*{Guo}
  \item \citet*{Liu2006}
\end{itemize}
\section{Per-Paper General Summaries}
\section{A Survey of Practical Issues in Underwater Networks
 \citet*{Partan2006}}
\label{Partan2006_gen}
\begin{itemize}
  \item Predominantly Medium Access Control (MAC) considerations of marine
  networks, technological, and economical.
  \item Classifies (marine) networks into 4 (or 5, counting null network)
  regimens; 
  \begin{itemize}
    \item Null Networks (Nodes too distant and immobile for communication)
    \item Disruption Tolerant Networks (DTN, sparse but sufficiently
    mobile networks)
    \item Unpartitioned, multi-hop networks (overlapping chains of TDMA/CDMA 
    clusters with MACA or S-FAMA \cite{Molins2006})
    \item Single-Hop TDMA networks (fully connected network coverage, with
    limited contention)
    \item Dense Single-Hop Network (ok for CSMA, not much else, bandwidth
    contention)
  \end{itemize}
  \item Summarises marine channel characterise, generally summed up as 'bad'
  \begin{itemize}
    \item \textbf{Acoustic}: Experiences Propagation Delays and
    Doppler Effects\footnote{Speed of sound in water is 1500m/s (depth
    variant)}, as well as limited bandwidth due to frequency dependant
    attenuation. Additionally, high \acro{BER} due to Phase and Amplitude
    variations, requiring \acro{FEC}. See \cite{Catipovic1990}
    \item \textbf{RF}: Poor but possible (122 kHZ, 6-10m, 1-8kbits/sec
    )
    \item \textbf{Optical}: Very poor but possible in \textit{extremely} clear
    environments (490-500 nm, \textless 100m, several Mbits/s)
    \item \textbf{IR}: potential to use OTS IrDA TX/RX for low cost,
    short range links (1-2m, 57.6kbits/s)
  \end{itemize}
  \item Includes several Operational Examples (4.1, 5.1, 5.2, 5.5) that will be
  useful later.
  \item Introduces LBL\footnote{Long Baseline, an acoustic positioning method
  conceptually similar to GPS} localisation and potential contention between
  Navigational and data comms (5.2)
  \item Introduces a variety of DTN packet exchange techniques for mobile
  networks (5.3,5.4)
\end{itemize}

\subsection{Questions Raised}
\begin{itemize}
  \item It is not explained or justified what causes the indicated phase and
  amplitude variations that lead to high \acro{BER}. Possibly covered in
  \cite{Catipovic1990}.
  \item In one operational example (MCM, 4.1), a central gateway buoy was used
  as a uplink service and hub for a star-network of AUVs. Depending on how
  DSTL/DGA want to pursue this, this demonstrates a necessary compromise
  between centrality for power and communications economy, and decentralised security and
  reliability.
\end{itemize}

\section{A Delay-Tolerant Network Architecture for
Challenged Internets \citet*{Fall2003}}
\label{Fall2003_gen}
\begin{itemize}
  \item This paper deals primarily with the description of Challenged
  network environments, but proposes a solution that is unsuitable for the
  desired application; i.e. centralised DTN Bundle Gateway interconnects
  between networks.
  \item Such an implementation \emph{could} be factored into the generation of a
  truly decentralised system.
  \item Investigates in some detail the characteristics and difficulties of
  Challenged Networks; summarised here.
  \begin{itemize}
    \item Path and Link Characteristics
    \begin{itemize}
      \item \emph{High-Latency, Low Data Rate}
      \item \emph{Disconnection} caused by fault (power loss, tx failure) or
      non-fault (motion or low-duty-cycle) sources. Motion can be predictable or
      opportunistic.
      \item \emph{Long Queuing Times} caused by multi-hop with unknown and
      dynamic topology rendering source-initiated retransmission
      extremely expensive, requiring long packet retention times to negate.
	\end{itemize}
    \item Network Architecture Concerns
    \begin{itemize}
      \item \emph{Interoperability Considerations}; individual CN's not designed
      to a standard 'stack', and thus may fail to implement 'abstractions \ldots
      supporting layered protocol families'.
      \item \emph{Security}; E2E\footnote{End-to-End} security, usually requiring challenge/key
      exchange, is undesirable due to the long latencies and delay prone nature
      of the networks. Additionally, its a waste of data to carry it all the way
      to the destination for authentication/access only to be denied.
	\end{itemize}
    \item End-System Characteristics
    \begin{itemize}
      \item \emph{Limited Longevity}: Message transit may outlive their
      source-node, due to power, strategic, or environmental considerations.
      Fall et all suggest that message reliability monitoring should be
      delegated to a currently (and hopefully, future) operating node.
      \item \emph{Low-Duty-Cycle Operation}: Highlighted in low-power/long-life
      networks where nodes are regularly 'listening' and/or recording data to be
      relayed, and periodically (but more rarely) transmit. The performance of
      this kind of operation and network is dependant on predictable paths
      (leading to efficient time-dependant path selection) and a-priori
      scheduling.
      \item \emph{Limited Resources}: Highlights the decisions to be made with
      regards to memory resources, factoring in RTT, expected retransmissions,
      and maintaining a store of in-transit data.
	\end{itemize}
  \end{itemize}
  \item Additionally, this paper covers the potential use of Proxy or Gateway
  agents at network interface points, E-Mail-like async messaging as
  solutions to the above, before detailing a bundle-gateway based DTN
  architecture based on regional and sub-regional addressing between challenged
  networks, with QoS-like behaviour based on Postal Service conventions (return
  receipt/delivery record/'handle with care')
  \item On the subject of path selection this paper refers to another 2003
  paper\cite{Alonso2003a} which should be reviewed, but is quite technical from
  a cursory skim, so should be delayed.
  \item Introduces the idea of Persistent and Non Persistent Gateways, and
  discussed the effect on network reliability and architecture between these
  nodes.
  \item Deals with network convergence; not entirely relevant to the project at
  hand.
  \item Highlights the congestion difficulties of challenged networks (4.9), eg
  'contacts may not arrive for some time in the future' and 'adopted' packets,
  where a node is delegated custody the ensure that packets reception, which
  cannot be deleted except under extreme conditions. Currently tackled using a
  shared priority queue, but this introduces exploitable behaviour i.e. priority
  inversion / head-of-line blocking.
\end{itemize}

\subsection{Questions Raised}
\begin{itemize}
  \item Fall et al state that the problem of \emph{Security} wastage (i.e.
  carrying excess authentication data to a destination only to be denied) remains an open
  problem on the Internet. I believe that within the general scope of this
  project, such an authentication scheme would be decentralised and
  N2N \footnote{Node-to-node} rather than E2E. This would be an example of
  collaborative delegation, as discussed below.
  \item On the subject of \emph{Limited Longevity}, Fall et al. suggest message
  delivery acknowledgement be delegated to a 'surviving' node. This implies a
  centralised approach which is counter to the project aims. One could envisage
  a collaborative delegation system whereby a subset of nodes within a fleet
  (i.e, the local neighbours N\{\ldots\} of node X upon X's transmission of
  message M), such that any of N can authoritatively accept a delivery
  acknowledgement (although personally I don't think ARQ is suitable for this
  application due to data overheads and network disruption). Doing so would also
  alleviate the issues raised in \emph{Limited Resources}, as once the
  transmission has been spread to its neighbours, N can realistically drop that
  information and assume it will makes its way across the network. This N2N
  approach is tangentially covered in (4.5)
\end{itemize}

\section{A Trust Management Framework for Detecting Malicious and Selfish Behaviour 
in Ad-Hoc Wireless Networks using Fuzzy Sets and Grey theory \citet*{Guo}}
\label{Guo}
\begin{itemize}
  \item Provides background on TMF
  \item References Policy Language, PKC\footnote{Public Key
  Cryptography}, Resurrecting Duckling Model and Distributed Trust Model
  \cite{Li2007}
  \item Raises need for TMF to mitigate selfishness
  \item Introduces Grey Theory From Deng Julong (Can't find any official
  citations for this, paper appears to be stuck in  Springerlink)
  \item Defines three types of trust relationship
  \begin{itemize}
    \item \emph{Direct}: Trust based on historical behaviour of node $B$ wrt
    node $A$.
    \item \emph{Indirect}: Trust transited through third-party entities, i.e
    $E$, $F$ wrt $B$ where neither communicate with $A$
    \item \emph{Recommendation}: a subjective trust transited through a common
    entity; i,e Trust($B\rightarrow C$) communicated to $A$ by $C$.
  \end{itemize}
  \item Highlights some of the potential attacks on TMF and their sources
  \cite{Sun2008}, \cite{Li2008}.
  \item Guo proposes a TMF leveraging the curve-fitting
  analytics of Grey Theory to allow in practical networks, multi-parametric
  based trust quantities, and to use this data and results to assess the type
  of selfish behaviour being exhibited by 'bad' nodes.
  \item Classical TMF only monitor single behaviours: probability of successful
  interactions (Bit Error Rate/Packet Loss Rate).
  The TMF suggested by Guo incorporates signal strength, data-rate and other
  physical factors in addition to PLR.
  \item The application of Grey Whitenisation and clustering allows for
  quantifiable Trustworthiness classification, i.e taking the multi-parametric
  measurements of behaviour and condensing this to a per-node wrt calculating
  node trust assessment.
  \item Includes worked simulation examples of a 6 node network
\end{itemize}

\subsection{Questions Raised}
\begin{itemize}
  \item Stated definitions of Direct, Indirect, and Recommendation trust do not
  suitably distinguish between Indirect/Recommendation. Hopefully after finding
  a copy of Julong this is be made clearer.
  \item Grey Theory seems mathematically intuitive but the fundamental question
    is the selection of \emph{distinguishing coefficient} values \cite{Cai2009}, and
  whether these values can be collaboratively 'learned' over time, almost like familial
  and social trust.
  \item Additionally, the derivation of Whitinisation functions is not at all
  clear.
  \item Generally: What is the implication for using this style of trust
  assessment against human operators, remotely or in physical nodes (eg
  mother-ships)? Will lapses in judgement be detrimentally held against an
  operator? Can a TMF as suggested be modified to accept human frailties (and
  should it!)?
\end{itemize}

\chapter{Intended Areas of Subsequent Research}
\section{General Direction}
While it is very early to say, I believe that the area of most potential gain is
to initially continue this level of generalist research \footnote{Within the
spheres of (Marine) Networks and Trust Management fields}.  Beyond that, the aim
is to study the practical operation of AUV/Marine Sensor Networks/Human
Operators in an effort to establish reliable physical, MAC, and behavioural level metrics in current
distributed marine networks that can be used with existing (and potentially, novel) trust
and reputation management frameworks.

Personally, I really like the idea of investigating the practical (and ethical)
issues surrounding the integration of human operators into multi-parametric
trust networks. This raises important questions such as 'If a fleet demonstrably
cannot trust a human operator, should it rebel?' etc. 

\section{Reading Lists}
\emph{Each section-list set in general order of prioritisation}
\subsection{Marine and Distributed Networks}
\begin{itemize}
  \item Underwater Acoustic Sensor Networks: Research Challenges
  \cite{Akyildiz2005}
  \item A QoS-Aware Underwater Optimisation Framework for Inter-vehicle
  Communications using Acoustic Directional Transducers \cite{Chen}
  \item Performance Limitations in underwater acoustic telemetry
  \cite{Catipovic1990}
  \item Slotted FAMA: a MAC protocol for underwater acoustic networks
  \cite{Molins2006}
  \item High-Rate Phase Coherent Acoustic Communications: A Review of a Decade
  of Research and a Perspective on Future Challenges \cite{Freitag2004}
\end{itemize}
\subsection{Trust and Reputation Management}
\begin{itemize}
  \item Trust Management in Distributed Systems \cite{Li2007}
  \item Information Theoretic Framework of Trust Modeling and Evaluation for Ad
  Hoc Networks \cite{Liu2006}
  \item The importance of trust between operator and AUV: Crossing the
  human/computer language barrier \cite{Johnson2007}
  \item Future Trust Management Frameworks for Mobile Ad Hoc Networks
  \cite{Li2008}
  \item Introduction to Grey System Theory \cite{Deng1989}
\end{itemize}

% --------------------------------------------------------------
%:                  BACK MATTER: appendices, refs,..
% --------------------------------------------------------------

% the back matter: appendix and references close the thesis


%: ----------------------- bibliography ------------------------

% The section below defines how references are listed and formatted
% The default below is 2 columns, small font, complete author names.
% Entries are also linked back to the page number in the text and to external URL if provided in the BibTex file.

% PhDbiblio-url2 = names small caps, title bold & hyperlinked, link to page 
\begin{multicols}{2} % \begin{multicols}{ # columns}[ header text][ space]
\begin{tiny} % tiny(5) < scriptsize(7) < footnotesize(8) < small (9)

%\bibliographystyle{Latex/Classes/PhDbiblio-url2} % Title is link if provided
\renewcommand{\bibname}{References} % changes the header; default: Bibliography

\bibliography{library} % adjust this to fit your BibTex file

\end{tiny}
\end{multicols}

% --------------------------------------------------------------
% Various bibliography styles exit. Replace above style as desired.

% in-text refs: (1) (1; 2)
% ref list: alphabetical; author(s) in small caps; initials last name; page(s)
%\bibliographystyle{Latex/Classes/PhDbiblio-case} % title forced lower case
%\bibliographystyle{Latex/Classes/PhDbiblio-bold} % title as in bibtex but bold
%\bibliographystyle{Latex/Classes/PhDbiblio-url} % bold + www link if provided

%\bibliographystyle{Latex/Classes/jmb} % calls style file jmb.bst
% in-text refs: author (year) without brackets
% ref list: alphabetical; author(s) in normal font; last name, initials; page(s)

\bibliographystyle{plainnat} % calls style file plainnat.bst
% in-text refs: author (year) without brackets
% (this works with package natbib)


% --------------------------------------------------------------

% according to Dresden med fac summary has to be at the end
%As \glspl{auv} become more technically capable and economically feasible, they are being increasingly used in a great many areas of defence, commercial and environmental applications. 
These applications are tending towards using independent, autonomous, ad-hoc, collaborative behaviour of teams or fleets of these \gls{auv} platforms.
This convergence of research experiences in the \gls{uan} and \gls{manet} fields, along with the increasing \gls{loa} of such platforms, creates unique challenges to secure the operation and communication of these networks.

The question of security and reliability of operation in networked systems has usually been resolved by having a centralised coordinating agent to manage shared secrets and monitor for misbehaviour.
However, in the sparse, noisy and constrained communications environment of \glspl{uan}, the communications overheads and single-point-of-failure risk of this model is challenged (particularly when faced with capable attackers).

As such, more lightweight, distributed, experience based\footnote{rather than ``Evidence based'' in the case of shared keys, \gls{pki} etc.} systems of ``Trust'' have been proposed to dynamically model and evaluate the ``trustworthiness'' of nodes within a \gls{manet} across the network to prevent or isolate the impact of malicious, selfish, or faulty misbehaviour. 
Previously, these models have monitored actions purely within the communications domain. 
Moreover, the vast majority rely on only one type of observation (metric) to evaluate trust; successful packet forwarding.
In these cases, motivated actors may use this limited scope of observation to either perform unfairly without repercussions in other domains/metrics, or to make another, fair, node appear to be operating unfairly.

This thesis is primarily concerned with the use of terrestrial-\gls{manet} trust frameworks to the \gls{uan} space. 
Considering the massive theoretical and practical difference in the communications environment, these frameworks must be reassessed for suitability to the marine realm. 
We find that current single-metric \glspl{tmf} do not perform well in a best-case scaling of the marine network, due to sparse and noisy observation metrics, and while basic multi-metric communications-only frameworks perform better than their single-metric forms, this performance is still not at a reliable level. 
We propose, demonstrate (through simulation) and integrate the use of physical observational metrics for trust assessment, in tandem with metrics from the communications realm, improving the safety, security, reliability and integrity of autonomous \glspl{uan}.

Three main novelties are demonstrated in this work:
Trust evaluation using metrics from the physical domain (movement/distribution/etc.), demonstration of the failings of Communications-based Trust evaluation in sparse, noisy, delayful and non-linear \gls{uan} environments, and the deployment of trust assessment across multiple domains, e.g.\ the physical and communications domains.
The latter contribution includes the generation and optimisation of cross-domain metric composition or``synthetic domains'' as a performance improvement method.


%: Declaration of originality
\include{backmatter/declaration}



\end{document}
