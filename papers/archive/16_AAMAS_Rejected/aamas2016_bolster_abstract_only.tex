% This is "aamas2016_sample.tex", a revised version of aamas2015_sample.tex
% This file should be compiled with "aamas2016.cls"
% This example file demonstrates the use of the 'aamas2015.cls'
% LaTeX2e document class file. It is intended for those submitting
% articles to the AAMAS-2016 conference. This file is based on
% the sig-alternate.tex example file.
% The 'sig-alternate.cls' file of ACM will produce a similar-looking,
% albeit, 'tighter' paper resulting in, invariably, fewer pages
% than the original ACM style.
%
% ----------------------------------------------------------------------------------------------------------------
% This .tex file (and associated .cls ) produces:
%       1) The Permission Statement
%       2) The Conference (location) Info information
%       3) The Copyright Line with AAMAS data
%       4) NO page numbers
%
% as against the acm_proc_article-sp.cls file which
% DOES NOT produce 1) through 3) above.
%
% Using 'aamas2015.cls' you don't have control
% from within the source .tex file, over both the CopyrightYear
% (defaulted to 20XX) and the IFAAMAS Copyright Data
% (defaulted to X-XXXXX-XX-X/XX/XX).
% These information will be overwritten by fixed AAMAS 2015  information
% in the style files - it is NOT as you are used to with ACM style files.
%
% ---------------------------------------------------------------------------------------------------------------
% This .tex source is an example which *does* use
% the .bib file (from which the .bbl file is produced).
% REMEMBER HOWEVER: After having produced the .bbl file,
% and prior to final submission, you *NEED* to 'insert'
% your .bbl file into your source .tex file so as to provide
% ONE 'self-contained' source file.
%


\documentclass{aamas2016}

\graphicspath{{../pdf/}{../jpeg/}{img/}{../../../Figures/}{../../../../../Dropbox/Thesis_Figures/}}

% if you are using PDF LaTeX and you cannot find a way for producing
% letter, the following explicit settings may help

\pdfpagewidth=8.5truein
\pdfpageheight=11truein

\begin{document}

% In the original styles from ACM, you would have needed to
% add meta-info here. This is not necessary for AAMAS 2015  as
% the complete copyright information is generated by the cls-files.


\title{A Multi-Domain Trust Framework for Autonomous Mobile Networks in Harsh Environments}

% AUTHORS


% For initial submission, do not give author names, but the
% tracking number, instead, as the review process is blind.

% You need the command \numberofauthors to handle the 'placement
% and alignment' of the authors beneath the title.
%
% For aesthetic reasons, we recommend 'three authors at a time'
% i.e. three 'name/affiliation blocks' be placed beneath the title.
%
% NOTE: You are NOT restricted in how many 'rows' of
% "name/affiliations" may appear. We just ask that you restrict
% the number of 'columns' to three.
%
% Because of the available 'opening page real-estate'
% we ask you to refrain from putting more than six authors
% (two rows with three columns) beneath the article title.
% More than six makes the first-page appear very cluttered indeed.
%
% Use the \alignauthor commands to handle the names
% and affiliations for an 'aesthetic maximum' of six authors.
% Add names, affiliations, addresses for
% the seventh etc. author(s) as the argument for the
% \additionalauthors command.
% These 'additional authors' will be output/set for you
% without further effort on your part as the last section in
% the body of your article BEFORE References or any Appendices.

%\numberofauthors{8} %  in this sample file, there are a *total*
% of EIGHT authors. SIX appear on the 'first-page' (for formatting
% reasons) and the remaining two appear in the \additionalauthors section.
%

\numberofauthors{2}

\author{
% You can go ahead and credit any number of authors here,
% e.g. one 'row of three' or two rows (consisting of one row of three
% and a second row of one, two or three).
%
% The command \alignauthor (no curly braces needed) should
% precede each author name, affiliation/snail-mail address and
% e-mail address. Additionally, tag each line of
% affiliation/address with \affaddr, and tag the
% e-mail address with \email.
% 1st. author
\alignauthor
Andrew Bolster\\
       \affaddr{Department of Electrical Engineering and Electronics}\\
       \affaddr{University of Liverpool}\\
       \affaddr{United Kingdom}\\
       \email{andrew.bolster@liv.ac.uk}
% 2nd. author
\alignauthor
Alan Marshall\\
       \affaddr{Department of Electrical Engineering and Electronics}\\
       \affaddr{University of Liverpool}\\
       \affaddr{United Kingdom}\\
       \email{alan.marshall@liv.ac.uk}
% 3rd. author
%\alignauthor Lars Th{\o}rv{\"a}ld\titlenote{This author is the one who did all the really hard work.}\\
%       \affaddr{The Th{\o}rv{\"a}ld Group}\\
%       \affaddr{1 Th{\o}rv{\"a}ld Circle}\\
%       \affaddr{Hekla, Iceland}\\
%       \email{larst@affiliation.org}
}

%\and  % use '\and' if you need 'another row' of author names

% 4th. author
%\alignauthor Lawrence P. Leipuner\\
%       \affaddr{Brookhaven Laboratories}\\
%       \affaddr{Brookhaven National Lab}\\
%       \affaddr{P.O. Box 5000}\\
%       \email{lleipuner@researchlabs.org}

% 5th. author
%\alignauthor Sean Fogarty\\
%       \affaddr{NASA Ames Research Center}\\
%       \affaddr{Moffett Field}\\
%       \affaddr{California 94035}\\
%       \email{fogartys@amesres.org}

% 6th. author
%\alignauthor Charles Palmer\\
%       \affaddr{Palmer Research Laboratories}\\
%      \affaddr{8600 Datapoint Drive}\\
%       \affaddr{San Antonio, Texas 78229}\\
%       \email{cpalmer@prl.com}

%\and

%% 7th. author
%\alignauthor Lawrence P. Leipuner\\
%       \affaddr{Brookhaven Laboratories}\\
%       \affaddr{Brookhaven National Lab}\\
%       \affaddr{P.O. Box 5000}\\
%       \email{lleipuner@researchlabs.org}

%% 8th. author
%\alignauthor Sean Fogarty\\
%       \affaddr{NASA Ames Research Center}\\
%       \affaddr{Moffett Field}\\
%       \affaddr{California 94035}\\
%       \email{fogartys@amesres.org}

%% 9th. author
%\alignauthor Charles Palmer\\
%       \affaddr{Palmer Research Laboratories}\\
%       \affaddr{8600 Datapoint Drive}\\
%       \affaddr{San Antonio, Texas 78229}\\
%       \email{cpalmer@prl.com}

%}

%% There's nothing stopping you putting the seventh, eighth, etc.
%% author on the opening page (as the 'third row') but we ask,
%% for aesthetic reasons that you place these 'additional authors'
%% in the \additional authors block, viz.
%\additionalauthors{Additional authors: John Smith (The Th{\o}rv{\"a}ld Group,
%email: {\texttt{jsmith@affiliation.org}}) and Julius P.~Kumquat
%(The Kumquat Consortium, email: {\texttt{jpkumquat@consortium.net}}).}
%\date{30 July 1999}
%% Just remember to make sure that the TOTAL number of authors
%% is the number that will appear on the first page PLUS the
%% number that will appear in the \additionalauthors section.

\maketitle

\begin{abstract}

Trust Management Frameworks (TMFs) are being used to improve the efficiency, security, and reliability of decentralized and distributed autonomous systems. 
Techniques have been developed for high-speed, uncontended environments such as terrestrial 802.11 MANETs. 
However, these do not perform well in sparse / harsh environments such as those found in Underwater Acoustic Networks (UANs), where network nodes experience significant and variable delays, comparatively low data rates, large contention periods, and considerable multi-path artefacts.\cite{Bolster2015} 

In such sparse networks, trust establishment based on statistical observations of success/failure events become unstable and ineffective in detecting or identifying misbehaviours. 
Additionally, these methodologies focus solely on the communications activities of entities and do not incorporate information from other domains, such as physical mobility. 

In this paper we demonstrate the use and operation of a multi-domain trust management framework (MD-TMF) using UANs as an exemplar application. 
%We demonstrate a methodology that applies Grey Sequence operations and Grey Generators (conceptually analogous to Sequential Bayesian Filtering) to provide continuous trust assessment in a sparse, asynchronous metric space across multiple domains of trust.
We present a methodology for assessing the performance of varying metric sets in detection and differentiation of a range of communications and physical misbehaviours, demonstrating that by utilising information from multiple domains, trust assessment is more accurate in identifying misbehaviour than in single-domain assessment.


\end{abstract}

% Note that the category section should be completed after reference to the ACM Computing Classification Scheme available at
% http://www.acm.org/about/class/1998/.

\category{H.4}{Information Systems Applications}{Miscellaneous}

%A category including the fourth, optional field follows...
%\category{D.2.8}{Software Engineering}{Metrics}[complexity measures, performance measures]

%General terms should be selected from the following 16 terms: Algorithms, Management, Measurement, Documentation, Performance, Design, Economics, Reliability, Experimentation, Security, Human Factors, Standardization, Languages, Theory, Legal Aspects, Verification.

\terms{Algorithms Management Performance Reliability Security}

%Keywords are your own choice of terms you would like the paper to be indexed by.

\keywords{MANET, Underwater, Simulation, Trust}


\end{document}
